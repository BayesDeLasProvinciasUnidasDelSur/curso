\documentclass[a4paper,10pt]{article}
\usepackage[utf8]{inputenc}
\usepackage[spanish]{babel}
\usepackage{fullpage}
\usepackage{cite}
\usepackage[utf8]{inputenc}
\usepackage{a4wide}
\usepackage{url}
\usepackage{graphicx}
\usepackage{caption}
\usepackage{float} % para que los gr\'aficos se queden en su lugar con [H]
\usepackage{subcaption}
\usepackage{wrapfig}
\usepackage{color}
\usepackage{amsmath} %para escribir funci\'on partida , matrices
\usepackage{amsthm} %para numerar definciones y teoremas
\usepackage[hidelinks]{hyperref} % para inlcuir links dentro del texto
\usepackage{tabu} 
\usepackage{comment}
\usepackage{amsfonts} % \mathbb{N} -> conjunto de los n\'umeros naturales  
\usepackage{enumerate}
\usepackage{listings}
\usepackage[colorinlistoftodos, textsize=small]{todonotes} % Para poner notas en el medio del texto!! No olvidar hacer. 
\usepackage{framed} % Para encuadrar texto. \begin{framed}
\usepackage{csquotes} % Para citar texto \begin{displayquote}
\usepackage{epigraph} % Epigrafe  \epigraph{texto}{\textit{autor}}
\usepackage{authblk}
\usepackage{titlesec}
\usepackage{varioref}
\usepackage{bm} % \bm{\alpha} bold greek symbol
\usepackage{pdfpages} % \includepdf
\usepackage[makeroom]{cancel} % \cancel{} \bcancel{} etc
\usepackage{wrapfig} % \begin{wrapfigure} Pone figura al lado del texto
\usepackage{tikz}
\usepackage{algorithm}


\usepackage{paracol}

\newcommand{\citel}[1]{\cite{#1}\label{#1}}
\newcommand\hfrac[2]{\genfrac{}{}{0pt}{}{#1}{#2}} %\frac{}{} sin la linea del medio

\theoremstyle{definition}
\newtheorem{definition}{Definition}[section]
\newtheorem{theorem}{Theorem}[section]
\newtheorem{proposition}{Proposition}[section]

%http://latexcolor.com/
\definecolor{azul}{rgb}{0.36, 0.54, 0.66}
\definecolor{rojo}{rgb}{0.7, 0.2, 0.116}
\definecolor{rojopiso}{rgb}{0.8, 0.25, 0.17}
\definecolor{verdeingles}{rgb}{0.12, 0.5, 0.17}
\definecolor{ubuntu}{rgb}{0.44, 0.16, 0.39}
\definecolor{debian}{rgb}{0.84, 0.04, 0.33}

\definecolor{dkgreen}{rgb}{0,0.6,0}
\definecolor{gray}{rgb}{0.5,0.5,0.5}
\definecolor{mauve}{rgb}{0.58,0,0.82}

\lstset{
  language=Python,
  aboveskip=3mm,
  belowskip=3mm,
  showstringspaces=true,
  columns=flexible,
  basicstyle={\small\ttfamily},
  numbers=none,
  numberstyle=\tiny\color{gray},
  keywordstyle=\color{blue},
  commentstyle=\color{dkgreen},
  stringstyle=\color{mauve},
  breaklines=true,
  breakatwhitespace=true,
  tabsize=4
}

% tikzlibrary.code.tex
%
% Copyright 2010-2011 by Laura Dietz
% Copyright 2012 by Jaakko Luttinen
%
% This file may be distributed and/or modified
%
% 1. under the LaTeX Project Public License and/or
% 2. under the GNU General Public License.
%
% See the files LICENSE_LPPL and LICENSE_GPL for more details.

% Load other libraries
\usetikzlibrary{shapes}
\usetikzlibrary{fit}
\usetikzlibrary{chains}
\usetikzlibrary{arrows}

% Latent node
\tikzstyle{latent} = [circle,fill=white,draw=black,inner sep=1pt,
minimum size=20pt, font=\fontsize{10}{10}\selectfont, node distance=1]
% Observed node
\tikzstyle{obs} = [latent,fill=gray!25]
% Invisible node
\tikzstyle{invisible} = [latent,minimum size=0pt,color=white, opacity=0, node distance=0]
% Constant node
\tikzstyle{const} = [rectangle, inner sep=0pt, node distance=0.1]
%state
\tikzstyle{estado} = [latent,minimum size=8pt,node distance=0.4]
%action
\tikzstyle{accion} =[latent,circle,minimum size=5pt,fill=black,node distance=0.4]


% Factor node
\tikzstyle{factor} = [rectangle, fill=black,minimum size=10pt, draw=black, inner
sep=0pt, node distance=1]
% Deterministic node
\tikzstyle{det} = [latent, rectangle]

% Plate node
\tikzstyle{plate} = [draw, rectangle, rounded corners, fit=#1]
% Invisible wrapper node
\tikzstyle{wrap} = [inner sep=0pt, fit=#1]
% Gate
\tikzstyle{gate} = [draw, rectangle, dashed, fit=#1]

% Caption node
\tikzstyle{caption} = [font=\footnotesize, node distance=0] %
\tikzstyle{plate caption} = [caption, node distance=0, inner sep=0pt,
below left=5pt and 0pt of #1.south east] %
\tikzstyle{factor caption} = [caption] %
\tikzstyle{every label} += [caption] %

\tikzset{>={triangle 45}}

%\pgfdeclarelayer{b}
%\pgfdeclarelayer{f}
%\pgfsetlayers{b,main,f}

% \factoredge [options] {inputs} {factors} {outputs}
\newcommand{\factoredge}[4][]{ %
  % Connect all nodes #2 to all nodes #4 via all factors #3.
  \foreach \f in {#3} { %
    \foreach \x in {#2} { %
      \path (\x) edge[-,#1] (\f) ; %
      %\draw[-,#1] (\x) edge[-] (\f) ; %
    } ;
    \foreach \y in {#4} { %
      \path (\f) edge[->,#1] (\y) ; %
      %\draw[->,#1] (\f) -- (\y) ; %
    } ;
  } ;
}

% \edge [options] {inputs} {outputs}
\newcommand{\edge}[3][]{ %
  % Connect all nodes #2 to all nodes #3.
  \foreach \x in {#2} { %
    \foreach \y in {#3} { %
      \path (\x) edge [->,#1] (\y) ;%
      %\draw[->,#1] (\x) -- (\y) ;%
    } ;
  } ;
}

% \factor [options] {name} {caption} {inputs} {outputs}
\newcommand{\factor}[5][]{ %
  % Draw the factor node. Use alias to allow empty names.
  \node[factor, label={[name=#2-caption]#3}, name=#2, #1,
  alias=#2-alias] {} ; %
  % Connect all inputs to outputs via this factor
  \factoredge {#4} {#2-alias} {#5} ; %
}

% \plate [options] {name} {fitlist} {caption}
\newcommand{\plate}[4][]{ %
  \node[wrap=#3] (#2-wrap) {}; %
  \node[plate caption=#2-wrap] (#2-caption) {#4}; %
  \node[plate=(#2-wrap)(#2-caption), #1] (#2) {}; %
}

% \gate [options] {name} {fitlist} {inputs}
\newcommand{\gate}[4][]{ %
  \node[gate=#3, name=#2, #1, alias=#2-alias] {}; %
  \foreach \x in {#4} { %
    \draw [-*,thick] (\x) -- (#2-alias); %
  } ;%
}

% \vgate {name} {fitlist-left} {caption-left} {fitlist-right}
% {caption-right} {inputs}
\newcommand{\vgate}[6]{ %
  % Wrap the left and right parts
  \node[wrap=#2] (#1-left) {}; %
  \node[wrap=#4] (#1-right) {}; %
  % Draw the gate
  \node[gate=(#1-left)(#1-right)] (#1) {}; %
  % Add captions
  \node[caption, below left=of #1.north ] (#1-left-caption)
  {#3}; %
  \node[caption, below right=of #1.north ] (#1-right-caption)
  {#5}; %
  % Draw middle separation
  \draw [-, dashed] (#1.north) -- (#1.south); %
  % Draw inputs
  \foreach \x in {#6} { %
    \draw [-*,thick] (\x) -- (#1); %
  } ;%
}

% \hgate {name} {fitlist-top} {caption-top} {fitlist-bottom}
% {caption-bottom} {inputs}
\newcommand{\hgate}[6]{ %
  % Wrap the left and right parts
  \node[wrap=#2] (#1-top) {}; %
  \node[wrap=#4] (#1-bottom) {}; %
  % Draw the gate
  \node[gate=(#1-top)(#1-bottom)] (#1) {}; %
  % Add captions
  \node[caption, above right=of #1.west ] (#1-top-caption)
  {#3}; %
  \node[caption, below right=of #1.west ] (#1-bottom-caption)
  {#5}; %
  % Draw middle separation
  \draw [-, dashed] (#1.west) -- (#1.east); %
  % Draw inputs
  \foreach \x in {#6} { %
    \draw [-*,thick] (\x) -- (#1); %
  } ;%
}



%opening
\title{Bayesian Linear Regression}
\author{}

\begin{document}

\maketitle

\begin{abstract}

\end{abstract}

\section{Links}

\url{http://krasserm.github.io/2019/02/23/bayesian-linear-regression/}


Linear regression models share the property of being linear in their parameters but not necessarily in their input variables.
Using non-linear basis functions of input variables, linear models are able model arbitrary non-linearities from input variables to targets.  
Polynomial regression is such an example of basis functions.
A linear regression model $y(\bm{x},\bm{w})$

\begin{equation}
y(\bm{x},\bm{w}) = \sum_{i=0}^{M-1} w_i \phi_i(\bm{x}) = \bm{w}^T \bm{\phi}(\bm{x})
\end{equation}

Where $\Phi$ is the basis function and $M$ the total number of parameters $w_j$.
Here, we use the convention $\phi_0(\bm{x})=1$.
The simplest form of linear regression models are also linear functions of their input variables i.e. the set of basis functions in this case is the identity $\bm{\phi}(\bm{x})=x$. 

The target variable $t$ of an observation $\bm{x}$ is given by a deterministic function $y(\bm{x},\bm{w})$ plus additive random noise.

\begin{equation}
 t = y(\bm{x},\bm{w}) + \epsilon
\end{equation}

where $\epsilon$ is a zero mean Gaussian random variable with precision (inverse variance) $\beta$, $\epsilon \sim N(0,\beta^{-1})$.
So the probabilistic model of the target random variable,

\begin{equation}
p(t | \bm{x}, \bm{w}, \beta) = N(t | y(\bm{x},\bm{w}), \beta^{-1}) = N(t | \bm{w}^T \bm{\Phi}(\bm{x}) , \beta^{-1})
\end{equation}

Since we asume i.i.d., the joint conditional probability of $\bm{y}$

\begin{equation}
p(\bm{t} | \bm{x}, \bm{w}, \beta) = \prod_{i=1}^n N(t_i | \bm{w}^T \bm{\Phi}(\bm{x}_i) , \beta^{-1})
\end{equation}


Maximizing the log likelihood gives the maximum likelihood estimate of parameters $\bm{\beta}$.

\begin{equation}\label{eq:maximum_likelihood}
 \begin{split}
   \text{log } p(\bm{t} | \bm{x}, \bm{w}, \beta) & = \sum_{i=1}^{n} \text{log } N(t_i | \bm{w}^T \bm{\phi}(\bm{x}_i), \sigma)  \\
  & =  \sum_{i=1}^{n} \text{log }  \frac{\sqrt{\beta} }{\sqrt{2\pi}} e^{\frac{-(t_i - \bm{w}^T\bm{\phi}(\bm{x}_i))^2}{2\beta^{-1}} } = \sum_{i=1}^{n} \text{log } \frac{\sqrt{\beta} }{\sqrt{2\pi}} + \sum_{i=1}^{n} \text{log } e^{\frac{-(t_i - \bm{w}^T\bm{\phi}(\bm{x}_i))^2}{2\beta^{-1}} } \\
  & = n \text{log } \frac{\sqrt{\beta} }{\sqrt{2\pi}} + \sum_{i=1}^{n} \text{log } e^{\frac{-(t_i - \bm{w}^T\bm{\phi}(\bm{x}_i))^2}{2\beta^{-1}} } = n \text{log } \frac{\sqrt{\beta} }{\sqrt{2\pi}} + \sum_{i=1}^{n}  \frac{-(t_i - \bm{w}^T\bm{\phi}(\bm{x}_i))^2}{2\beta^{-1}} \\
   &  = n \text{ log } \sqrt{\beta} - n \text{ log } \sqrt{2\pi} - \frac{\beta}{2} \sum_{i=1}^{n}  (t_i - \bm{w}^T\bm{\phi}(\bm{x}_i))^2   \\
  & \propto  - \sum_{i=1}^{n}  (t_i - \bm{w}^T\bm{\phi}(\bm{x}_i))^2 = ||\bm{t}-\bm{\Phi}\bm{w}||^2
 \end{split}
\end{equation}

Maximizing the likelihood is equivalent to minimizing the sum-of-squares error function.
Matrix $\bm{\Phi}$ is called the \emph{design matrix}

\begin{equation}
 \bm{\Phi} =
  \begin{bmatrix}
    \phi_0(\bm{x}_1) & \phi_1(\bm{x}_1) & \dots & \phi_{M-1}(\bm{x}_1)\\
    \vdots & \vdots & \ddots & \vdots \\
    \phi_0(\bm{x}_N) & \phi_1(\bm{x}_N) & \dots & \phi_{M-1}(\bm{x}_N)
  \end{bmatrix}
\end{equation}

Notar que cada basis function $\phi_j(\cdot)$ recibe el vector-input completo $\bm{x}_i$.
Cuando trabajamos en una dimensi\'on, el vector $\bm{x}_i$ es un escalar.



\todo[inline]{Resolver $\bm{w}_{ML}$ y $\beta_{ML}$}
% Solving for $\bm{w}$ we obtain
% 
% \begin{equation}
%  \bm{w}_{ML} = ()
% \end{equation}





\section{Bayesian}

Maximum likelihood estimation can lead to severe over-fitting if complex models (e.g. polynomial regression models of high order) are fit to datasets of limited size.
Common approachs to prevent over-fitting is to add a regularization term to the error function or to perform corss-validation.

\begin{framed}
Bayesian treatment of linear regression avoid over-fitting problem of maximum likelihood, and which will also leads to automatic methods of determining model complexity using the training data alone.
\end{framed}

For a Bayesian treatment of linear regression we need a prior probability distribution over model parameters $w$.

\begin{equation}
 p(\bm{w}) = N(\bm{w}|\bm{m}_0)
\end{equation}

Due to the choice of a conjugate Gaussian prior, the posterior will also be Gaussian\todo{Demostrar Eq (2.116) Bishop}.
We can evaluate this distribution by the usal procedure of completing the square in the exponential, and then finding the normalization coefficient using the standard result for normalized Gaussian.
The work for deriving the general result is at equation 2.116 at the Bishops book.

\begin{equation}
 p(\bm{w}|\bm{t}) = N(\bm{w}|\bm{m}_N, \bm{S}_N)
\end{equation}

where 

\begin{equation}
 \bm{m}_N = \bm{S}_N (\bm{S}_0^{-1} \bm{m}_0 + \beta \bm{\Phi}^T \bm{t})
\end{equation}

\begin{equation}
 \bm{S}_N^{-1} = \bm{S}_0^{-1} + \beta \bm{\Phi}^T\bm{\Phi}
\end{equation}

For simplicity, we consider a zero-mean isotropic Gaussian governed by a single precision paramater $\alpha$ so that

\begin{equation}
 p(\bm{w}|\bm{t}) = N(\bm{w}|\bm{0}, \alpha^{-1} \bm{I})
\end{equation}

and the corresponding posterior distribution over $\bm{w}$ is then 


\begin{equation}
 \bm{m}_N = \beta  \bm{S}_N\bm{\Phi}^T \bm{t}
\end{equation}

\begin{equation}
 \bm{S}_N^{-1} = \alpha^{-1} \bm{I} + \beta \bm{\Phi}^T\bm{\Phi}
\end{equation}



To obtain an analytical solution we will treat $\beta$ as a known constant.
Note that in supervised learning problems such as regression we are not seeking to model the distribution of the input variables, so we will treat the input $\bm{x}$ as a known constant.

\begin{figure}[H]
    \centering
    \tikz{

    \node[latent, fill=black!100, minimum size=2pt] (x) {} ; %
    \node[const, right=of x] (c_x) {$\bm{x}_n$};
    \node[latent, fill=black!20, yshift=-1.5cm] (t) {$t_n$} ; %
    \node[latent, fill=black!100, yshift=-1.5cm , xshift=-2cm,minimum size=2pt] (beta)
    {} ; %
    \node[const, above=of beta] (c_beta) {$\beta$};
    \node[latent, fill=black!0, yshift=-1.5cm, xshift=2cm] (w) {$\bm{w}$};
    \node[latent, fill=black!100, xshift=2cm, minimum size=2pt] (alpha) {} ; %
    \node[const, right=of alpha] (c_alpha) {$\alpha$};
    
    \edge {x,beta,w} {t};
    \edge {alpha} {w};
    
    \node[inv, fill=black!0, minimum size=0pt, xshift=-0.45cm] (data_inv) {} ; %
    \plate {data} {(t)(x)(c_x)(data_inv)} {$N$}; %
      
    }  
\end{figure}







\end{document}
