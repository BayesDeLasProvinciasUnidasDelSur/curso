\documentclass[shownotes,aspectratio=169]{beamer}

\input{../../auxiliar/tex/diapo_encabezado.tex}
% tikzlibrary.code.tex
%
% Copyright 2010-2011 by Laura Dietz
% Copyright 2012 by Jaakko Luttinen
%
% This file may be distributed and/or modified
%
% 1. under the LaTeX Project Public License and/or
% 2. under the GNU General Public License.
%
% See the files LICENSE_LPPL and LICENSE_GPL for more details.

% Load other libraries
\usetikzlibrary{shapes}
\usetikzlibrary{fit}
\usetikzlibrary{chains}
\usetikzlibrary{arrows}

% Latent node
\tikzstyle{latent} = [circle,fill=white,draw=black,inner sep=1pt,
minimum size=20pt, font=\fontsize{10}{10}\selectfont, node distance=1]
% Observed node
\tikzstyle{obs} = [latent,fill=gray!25]
% Invisible node
\tikzstyle{invisible} = [latent,minimum size=0pt,color=white, opacity=0, node distance=0]
% Constant node
\tikzstyle{const} = [rectangle, inner sep=0pt, node distance=0.1]
%state
\tikzstyle{estado} = [latent,minimum size=8pt,node distance=0.4]
%action
\tikzstyle{accion} =[latent,circle,minimum size=5pt,fill=black,node distance=0.4]


% Factor node
\tikzstyle{factor} = [rectangle, fill=black,minimum size=10pt, draw=black, inner
sep=0pt, node distance=1]
% Deterministic node
\tikzstyle{det} = [latent, rectangle]

% Plate node
\tikzstyle{plate} = [draw, rectangle, rounded corners, fit=#1]
% Invisible wrapper node
\tikzstyle{wrap} = [inner sep=0pt, fit=#1]
% Gate
\tikzstyle{gate} = [draw, rectangle, dashed, fit=#1]

% Caption node
\tikzstyle{caption} = [font=\footnotesize, node distance=0] %
\tikzstyle{plate caption} = [caption, node distance=0, inner sep=0pt,
below left=5pt and 0pt of #1.south east] %
\tikzstyle{factor caption} = [caption] %
\tikzstyle{every label} += [caption] %

\tikzset{>={triangle 45}}

%\pgfdeclarelayer{b}
%\pgfdeclarelayer{f}
%\pgfsetlayers{b,main,f}

% \factoredge [options] {inputs} {factors} {outputs}
\newcommand{\factoredge}[4][]{ %
  % Connect all nodes #2 to all nodes #4 via all factors #3.
  \foreach \f in {#3} { %
    \foreach \x in {#2} { %
      \path (\x) edge[-,#1] (\f) ; %
      %\draw[-,#1] (\x) edge[-] (\f) ; %
    } ;
    \foreach \y in {#4} { %
      \path (\f) edge[->,#1] (\y) ; %
      %\draw[->,#1] (\f) -- (\y) ; %
    } ;
  } ;
}

% \edge [options] {inputs} {outputs}
\newcommand{\edge}[3][]{ %
  % Connect all nodes #2 to all nodes #3.
  \foreach \x in {#2} { %
    \foreach \y in {#3} { %
      \path (\x) edge [->,#1] (\y) ;%
      %\draw[->,#1] (\x) -- (\y) ;%
    } ;
  } ;
}

% \factor [options] {name} {caption} {inputs} {outputs}
\newcommand{\factor}[5][]{ %
  % Draw the factor node. Use alias to allow empty names.
  \node[factor, label={[name=#2-caption]#3}, name=#2, #1,
  alias=#2-alias] {} ; %
  % Connect all inputs to outputs via this factor
  \factoredge {#4} {#2-alias} {#5} ; %
}

% \plate [options] {name} {fitlist} {caption}
\newcommand{\plate}[4][]{ %
  \node[wrap=#3] (#2-wrap) {}; %
  \node[plate caption=#2-wrap] (#2-caption) {#4}; %
  \node[plate=(#2-wrap)(#2-caption), #1] (#2) {}; %
}

% \gate [options] {name} {fitlist} {inputs}
\newcommand{\gate}[4][]{ %
  \node[gate=#3, name=#2, #1, alias=#2-alias] {}; %
  \foreach \x in {#4} { %
    \draw [-*,thick] (\x) -- (#2-alias); %
  } ;%
}

% \vgate {name} {fitlist-left} {caption-left} {fitlist-right}
% {caption-right} {inputs}
\newcommand{\vgate}[6]{ %
  % Wrap the left and right parts
  \node[wrap=#2] (#1-left) {}; %
  \node[wrap=#4] (#1-right) {}; %
  % Draw the gate
  \node[gate=(#1-left)(#1-right)] (#1) {}; %
  % Add captions
  \node[caption, below left=of #1.north ] (#1-left-caption)
  {#3}; %
  \node[caption, below right=of #1.north ] (#1-right-caption)
  {#5}; %
  % Draw middle separation
  \draw [-, dashed] (#1.north) -- (#1.south); %
  % Draw inputs
  \foreach \x in {#6} { %
    \draw [-*,thick] (\x) -- (#1); %
  } ;%
}

% \hgate {name} {fitlist-top} {caption-top} {fitlist-bottom}
% {caption-bottom} {inputs}
\newcommand{\hgate}[6]{ %
  % Wrap the left and right parts
  \node[wrap=#2] (#1-top) {}; %
  \node[wrap=#4] (#1-bottom) {}; %
  % Draw the gate
  \node[gate=(#1-top)(#1-bottom)] (#1) {}; %
  % Add captions
  \node[caption, above right=of #1.west ] (#1-top-caption)
  {#3}; %
  \node[caption, below right=of #1.west ] (#1-bottom-caption)
  {#5}; %
  % Draw middle separation
  \draw [-, dashed] (#1.west) -- (#1.east); %
  % Draw inputs
  \foreach \x in {#6} { %
    \draw [-*,thick] (\x) -- (#1); %
  } ;%
}


 \mode<presentation>
 {
 %   \usetheme{Madrid}      % or try Darmstadt, Madrid, Warsaw, ...
 %   \usecolortheme{default} % or try albatross, beaver, crane, ...
 %   \usefonttheme{serif}  % or try serif, structurebold, ...
  \usetheme{Antibes}
  \setbeamertemplate{navigation symbols}{}
 }
 
\usepackage{todonotes}
\setbeameroption{show notes}

\newif\ifen
\newif\ifes
\newcommand{\en}[1]{\ifen#1\fi}
\newcommand{\es}[1]{\ifes#1\fi}
\estrue

%\title[Bayes del Sur]{}

\begin{document}

\color{black!85}
\large

 
%\setbeamercolor{background canvas}{bg=gray!15}

\begin{frame}[plain,noframenumbering]
 
 \begin{textblock}{90}(03,05)
 \centering \huge  \textcolor{black!40}{Creencias, datos y sorpresas}
\end{textblock}

 \begin{textblock}{47}(113,74)
\centering \Large  \textcolor{white!55}{Ergodicidad}
\end{textblock}

 %\vspace{2cm}brown
%\maketitle
\Wider[2cm]{
\includegraphics[width=1\textwidth]{../../auxiliar/static/peligro_predador}
}
\end{frame}


\begin{frame}[plain]
 \begin{textblock}{160}(0,4)
  \centering \LARGE Monedas
 \end{textblock}

\vspace{1cm}

\includegraphics[width=1\textwidth]{../../auxiliar/static/plata-potosi}
 
\end{frame}

\begin{frame}[plain]
 \begin{textblock}{160}(0,4)
  \centering \Large The Unfinished game (1654) \\ \normalsize Pascal, Fermat and the letters
 \end{textblock}
 \vspace{1cm}

Dos personas apuestan quién ganará tres lanzamientos de una moneda

\vspace{0.5cm}

Cada uno tira su moneda. El que obtiene más caras gana.

\vspace{0.5cm}

Antes de terminar se ven obligados a escapar. 

\vspace{0.5cm}

¿Cuál es la forma justa de repartir las apuestas?
 
\end{frame}

\begin{frame}[plain]
 \begin{textblock}{160}(0,4)
  \centering \Large Valor esperado
 \end{textblock}
 \vspace{1cm}
 
 \centering
 El promedio de todos los $N$ resultados posibles $x_i$ 
 
 \begin{align*}
  \frac{1}{N} \sum_i^N x_i
 \end{align*}

 % TODO: Pregunta práctica. Qué diferencia tiene este criterio respecto de dividir las creencias en partes iguales. (En el caso de la moneda se reduce a dividir las creencias en partes iguales, pero en otros casos no). Mostrar ejemplo y calcular uno y otro valor.
 
\end{frame}

\begin{frame}[plain]
 \begin{textblock}{160}(0,4)
  \centering \Large Paradoja de San Petersburgo (1712)
 \end{textblock}
 \vspace{1cm}

 \centering
 Apuesta que paga $2^{n-1}$, donde $n$ son caras consecutivas de una moneda

 \vspace{0.5cm}
 
 ¿Cuánto pagarías para entrar?
 
 \pause
 \begin{align*}
 \sum_i^{\infty} \left(\frac{1}{2} \right)^n 2^{n-1} = \frac{1}{2} \cdot \infty
 \end{align*}
 
\end{frame}

\begin{frame}[plain]
 \begin{textblock}{160}(0,4)
  \centering \Large Teoría de la utilidad esperada (1738)
 \end{textblock}


 \begin{align*}
  \frac{1}{N} \sum_i^N u(x_i)
 \end{align*}
 
 \end{frame}

 

 
 \begin{frame}[plain]
 \begin{textblock}{160}(0,4)
  \centering \Large Psicologización de la economía
\end{textblock}
\centering
\vspace{1cm}
 
\includegraphics[width=1\textwidth]{../../auxiliar/static/risk-aversion}

 \end{frame}


\begin{frame}[plain]
 \begin{textblock}{160}(0,4)
  \centering \Large >Apostar o no apostar?
 \end{textblock}
 \vspace{1cm}
 
 \begin{align*}
 \Delta \,r = 
      \begin{cases*}
       (+0.5) & \ \ \text{Cara} \\
       (-0.4) & \ \ \text{Seca}
    \end{cases*}
 \end{align*}

 \pause
 \vspace{1cm}
 \centering
 
 Función de utilidad neutral
 \begin{align*}
  0.5\cdot \frac{1}{2} - 0.4 \frac{1}{2} = 0.05
 \end{align*} 
 
 
\end{frame}


\begin{frame}[plain]
 \begin{textblock}{160}(0,4)
  \centering \Large Resultado temporal
 \end{textblock}
 

 \centering
\vspace{1cm}
 
\includegraphics[width=0.66\textwidth]{figures/simple_gamble.pdf}

\end{frame}

\begin{frame}[plain]
 \begin{textblock}{160}(0,4)
  \centering \Large Promedio temporal
 \end{textblock}
 \vspace{1cm}
 
 
 \centering
\includegraphics[width=1\textwidth]{../../auxiliar/static/ergodicity}

\end{frame}

\begin{frame}[plain]
 \begin{textblock}{160}(0,4)
  \centering \Large Ergodicidad
 \end{textblock}


 \begin{align*}
  \lim_{N\rightarrow \infty} \frac{1}{N} \sum_i^N Y_i(t*) = \lim_{T\rightarrow \infty} \frac{1}{T} \sum_t^T Y_{i*}(t)
 \end{align*}

\end{frame}

\begin{frame}[plain]
 \begin{textblock}{160}(0,4)
  \centering \LARGE Monedas
 \end{textblock}

 \vspace{1cm}

\includegraphics[width=0.9\textwidth]{../../auxiliar/static/monedas} 
 
\end{frame}

\begin{frame}[plain]
 \begin{textblock}{160}(0,4)
  \centering \LARGE Galton board
 \end{textblock}

 \vspace{1cm}

 \centering
\includegraphics[width=0.33\textwidth]{../../auxiliar/static/galton_board.png} 
 
\end{frame}

\begin{frame}[plain]
 \begin{textblock}{160}(0,4)
  \centering \LARGE El triangulo de Yanghui
 \end{textblock}

\vspace{1cm}

\centering
\includegraphics[width=0.42\textwidth]{../../auxiliar/static/triangleYanghui}

\end{frame}

\begin{frame}[plain]
 \begin{textblock}{160}(0,4)
  \centering \LARGE El triangulo de Yanghui
 \end{textblock}

\vspace{1cm}

\centering
\includegraphics[width=0.6\textwidth]{../../auxiliar/static/triangleCombinatorio}

 
\end{frame}


\begin{frame}[plain]
 \begin{textblock}{160}(0,4)
  \centering \LARGE Monedas
 \end{textblock}

 \begin{textblock}{160}(20,8)
\includegraphics[width=0.75\textwidth]{../../auxiliar/static/moneda}  
 \end{textblock}

\end{frame}

\begin{frame}[plain]
 \begin{textblock}{160}(0,4)
  \centering \LARGE Monedas
 \end{textblock}

\vspace{1cm}

\includegraphics[width=1\textwidth]{../../auxiliar/static/plata-potosi}
 
\end{frame}


\begin{frame}[plain]
 \begin{textblock}{160}(0,4)
  \centering \LARGE Monedas \\ \normalsize Polya Urn
 \end{textblock}

 \vspace{1cm}
\centering
\includegraphics[width=0.5\textwidth]{../../auxiliar/static/polya-urn}

 \end{frame}


 \begin{frame}[plain]
 \begin{textblock}{160}(0,4)
  \centering \LARGE Monedas \\ \normalsize Polya Urn
 \end{textblock}

 \vspace{1cm}
\centering
\includegraphics[width=0.75\textwidth]{../../auxiliar/static/polya-urn-exp}
 
 \end{frame}

 \begin{frame}[plain]
 \begin{textblock}{160}(0,4)
  \centering \LARGE Monedas
 \end{textblock}

\vspace{1cm}

\includegraphics[width=1\textwidth]{../../auxiliar/static/plata-potosi}
 
\end{frame}
 
 
\begin{frame}[plain]
 \begin{textblock}{160}(0,4)
  \centering \Large Resultado temporal \\ \normalsize Individual
 \end{textblock}
 

 \centering
\vspace{1cm}
 
\includegraphics[width=0.66\textwidth]{figures/simple_gamble.pdf}

\end{frame}


\begin{frame}[plain]
 \begin{textblock}{160}(0,4)
  \centering \Large Proceso multiplicativo
 \end{textblock}
 \vspace{1cm}
 
 
\begin{align*}
 V_N & = V_0 (1+r)^N = V_0 (1-l)^L (1+xl)^W \\
\end{align*}

\vspace{-1cm}
 \begin{align*}
V_N & := \text{Capital at time } N \\
V_0 & := \text{initial capital} \\
r & := \text{the exponentiation growth (continuous) rate} \\
L &:= \text{number of losses} \\
W &:= \text{number of wins} \\
l & := \text{capital invested (proportion)} \\
x & := \text{odd}-1, \text{la ganancia neta de odd, los odd como probability$^{-1}$}
\end{align*} 
 
\end{frame}



\begin{frame}[plain]
 \begin{textblock}{160}(0,4)
  \centering \Large \textbf{Evoluci\'on}: secuencia de tasas de reproducción y supervivencia 
 \end{textblock}
\vspace{1.33cm}
 
 \centering
 \includegraphics[width=0.85\textwidth]{../../auxiliar/static/peligro_predador}

 
\end{frame}

\begin{frame}[plain]
 \begin{textblock}{160}(0,4)
  \centering \Large Solución temporal cooperativa \\ \normalsize Cooperar: la puerta a los multiversos
 \end{textblock}
 
 \centering
\vspace{1cm}
 
\includegraphics[width=0.66\textwidth]{figures/simple_gamble_incesto.pdf}

\end{frame}


\begin{frame}[plain]
 \begin{textblock}{160}(0,4)
  \centering \Large Solución temporal individual \\ \normalsize Apuesta parcial
 \end{textblock}

 \centering
\vspace{1cm}
 
\includegraphics[width=0.9\textwidth]{../../auxiliar/static/kelly}

 
\end{frame}

\begin{frame}[plain]
 \begin{textblock}{160}(0,4)
  \centering \Large Kelly criterion
 \end{textblock}
\vspace{1.3cm}
 
Tomo logaritmo de ambos lados
\begin{align*}
 \log V_0 (1+r)^N = \log V_0 (1-l)^L (1+xl)^W \\
\end{align*}

\pause

Simplifico
\begin{align*}
 N \log (1+r) &= L \log (1-l) + W \log (1+xl) \\
 \log (1+r) &= \frac{L}{N} \log (1-l) + \frac{W}{N} \log (1+xl) \\
\end{align*}

\pause

En el límite $N \rightarrow \infty$
\begin{align*}
 \log (1+r) &= p \log (1-l) + (1-p) \log (1+xl) \\
\end{align*}

\end{frame}

\begin{frame}[plain]
\begin{textblock}{160}(0,4)
  \centering \Large Kelly criterion
 \end{textblock}
\vspace{1.3cm}
 

Podemos maximizar $r$ encontrnado su punto crítico en $l$ (es cóncava).
\begin{align*}
0 &= \frac{\delta}{\delta l} p \log (1-l) + (1-p) \log (1+xl) \\
\end{align*}

\pause

Resolviendo la derivada
\begin{align*}
0 = - \frac{x (p+l-1) + p}{(1-l)(xl+1)}
\end{align*}

\pause

Despejando
\begin{align*}
 l = \frac{x(1 - p) - p}{x}
\end{align*}
 
\end{frame}


 
\begin{frame}[plain]
\centering
  \includegraphics[width=0.35\textwidth]{../../auxiliar/static/pachacuteckoricancha.jpg}
\end{frame}






\end{document}



