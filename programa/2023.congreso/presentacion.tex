\newif\ifen
\newif\ifes
\newif\iffr
\newcommand{\fr}[1]{\iffr#1 \fi}
\newcommand{\En}[1]{\ifen#1\fi}
\newcommand{\Es}[1]{\ifes#1\fi}
\estrue
\documentclass[shownotes,aspectratio=169]{beamer}

\input{../../auxiliar/tex/diapo_encabezado.tex}
% tikzlibrary.code.tex
%
% Copyright 2010-2011 by Laura Dietz
% Copyright 2012 by Jaakko Luttinen
%
% This file may be distributed and/or modified
%
% 1. under the LaTeX Project Public License and/or
% 2. under the GNU General Public License.
%
% See the files LICENSE_LPPL and LICENSE_GPL for more details.

% Load other libraries
\usetikzlibrary{shapes}
\usetikzlibrary{fit}
\usetikzlibrary{chains}
\usetikzlibrary{arrows}

% Latent node
\tikzstyle{latent} = [circle,fill=white,draw=black,inner sep=1pt,
minimum size=20pt, font=\fontsize{10}{10}\selectfont, node distance=1]
% Observed node
\tikzstyle{obs} = [latent,fill=gray!25]
% Invisible node
\tikzstyle{invisible} = [latent,minimum size=0pt,color=white, opacity=0, node distance=0]
% Constant node
\tikzstyle{const} = [rectangle, inner sep=0pt, node distance=0.1]
%state
\tikzstyle{estado} = [latent,minimum size=8pt,node distance=0.4]
%action
\tikzstyle{accion} =[latent,circle,minimum size=5pt,fill=black,node distance=0.4]


% Factor node
\tikzstyle{factor} = [rectangle, fill=black,minimum size=10pt, draw=black, inner
sep=0pt, node distance=1]
% Deterministic node
\tikzstyle{det} = [latent, rectangle]

% Plate node
\tikzstyle{plate} = [draw, rectangle, rounded corners, fit=#1]
% Invisible wrapper node
\tikzstyle{wrap} = [inner sep=0pt, fit=#1]
% Gate
\tikzstyle{gate} = [draw, rectangle, dashed, fit=#1]

% Caption node
\tikzstyle{caption} = [font=\footnotesize, node distance=0] %
\tikzstyle{plate caption} = [caption, node distance=0, inner sep=0pt,
below left=5pt and 0pt of #1.south east] %
\tikzstyle{factor caption} = [caption] %
\tikzstyle{every label} += [caption] %

\tikzset{>={triangle 45}}

%\pgfdeclarelayer{b}
%\pgfdeclarelayer{f}
%\pgfsetlayers{b,main,f}

% \factoredge [options] {inputs} {factors} {outputs}
\newcommand{\factoredge}[4][]{ %
  % Connect all nodes #2 to all nodes #4 via all factors #3.
  \foreach \f in {#3} { %
    \foreach \x in {#2} { %
      \path (\x) edge[-,#1] (\f) ; %
      %\draw[-,#1] (\x) edge[-] (\f) ; %
    } ;
    \foreach \y in {#4} { %
      \path (\f) edge[->,#1] (\y) ; %
      %\draw[->,#1] (\f) -- (\y) ; %
    } ;
  } ;
}

% \edge [options] {inputs} {outputs}
\newcommand{\edge}[3][]{ %
  % Connect all nodes #2 to all nodes #3.
  \foreach \x in {#2} { %
    \foreach \y in {#3} { %
      \path (\x) edge [->,#1] (\y) ;%
      %\draw[->,#1] (\x) -- (\y) ;%
    } ;
  } ;
}

% \factor [options] {name} {caption} {inputs} {outputs}
\newcommand{\factor}[5][]{ %
  % Draw the factor node. Use alias to allow empty names.
  \node[factor, label={[name=#2-caption]#3}, name=#2, #1,
  alias=#2-alias] {} ; %
  % Connect all inputs to outputs via this factor
  \factoredge {#4} {#2-alias} {#5} ; %
}

% \plate [options] {name} {fitlist} {caption}
\newcommand{\plate}[4][]{ %
  \node[wrap=#3] (#2-wrap) {}; %
  \node[plate caption=#2-wrap] (#2-caption) {#4}; %
  \node[plate=(#2-wrap)(#2-caption), #1] (#2) {}; %
}

% \gate [options] {name} {fitlist} {inputs}
\newcommand{\gate}[4][]{ %
  \node[gate=#3, name=#2, #1, alias=#2-alias] {}; %
  \foreach \x in {#4} { %
    \draw [-*,thick] (\x) -- (#2-alias); %
  } ;%
}

% \vgate {name} {fitlist-left} {caption-left} {fitlist-right}
% {caption-right} {inputs}
\newcommand{\vgate}[6]{ %
  % Wrap the left and right parts
  \node[wrap=#2] (#1-left) {}; %
  \node[wrap=#4] (#1-right) {}; %
  % Draw the gate
  \node[gate=(#1-left)(#1-right)] (#1) {}; %
  % Add captions
  \node[caption, below left=of #1.north ] (#1-left-caption)
  {#3}; %
  \node[caption, below right=of #1.north ] (#1-right-caption)
  {#5}; %
  % Draw middle separation
  \draw [-, dashed] (#1.north) -- (#1.south); %
  % Draw inputs
  \foreach \x in {#6} { %
    \draw [-*,thick] (\x) -- (#1); %
  } ;%
}

% \hgate {name} {fitlist-top} {caption-top} {fitlist-bottom}
% {caption-bottom} {inputs}
\newcommand{\hgate}[6]{ %
  % Wrap the left and right parts
  \node[wrap=#2] (#1-top) {}; %
  \node[wrap=#4] (#1-bottom) {}; %
  % Draw the gate
  \node[gate=(#1-top)(#1-bottom)] (#1) {}; %
  % Add captions
  \node[caption, above right=of #1.west ] (#1-top-caption)
  {#3}; %
  \node[caption, below right=of #1.west ] (#1-bottom-caption)
  {#5}; %
  % Draw middle separation
  \draw [-, dashed] (#1.west) -- (#1.east); %
  % Draw inputs
  \foreach \x in {#6} { %
    \draw [-*,thick] (\x) -- (#1); %
  } ;%
}


 \mode<presentation>
 {
 %   \usetheme{Madrid}      % or try Darmstadt, Madrid, Warsaw, ...
 %   \usecolortheme{default} % or try albatross, beaver, crane, ...
 %   \usefonttheme{serif}  % or try serif, structurebold, ...
  \usetheme{Antibes}
  \setbeamertemplate{navigation symbols}{}
 }
\estrue
\usepackage{todonotes}
\setbeameroption{show notes}

\newcommand{\gray}{\color{black!55}}

\usepackage{listings}
\lstset{
  aboveskip=3mm,
  belowskip=3mm,
  showstringspaces=true,
  columns=flexible,
  basicstyle={\ttfamily},
  breaklines=true,
  breakatwhitespace=true,
  tabsize=4,
  showlines=true
}


\begin{document}

\color{black!85}
\large

\begin{frame}[plain,noframenumbering]


\begin{textblock}{160}(0,0)
\includegraphics[width=1\textwidth]{../../auxiliar/static/deforestacion}
\end{textblock}

\begin{textblock}{80}(18,9)
\textcolor{black!15}{\fontsize{44}{55}\selectfont Verdades}
\end{textblock}

\begin{textblock}{47}(85,70)
\centering \textcolor{black!15}{{\fontsize{52}{65}\selectfont Empíricas}}
\end{textblock}

\begin{textblock}{80}(100,28)
\LARGE  \textcolor{black!15}{\rotatebox[origin=tr]{-3}{\scalebox{9}{\scalebox{1}[-1]{$p$}}}}
\end{textblock}

\begin{textblock}{80}(66,43)
\LARGE  \textcolor{black!15}{\scalebox{6}{$=$}}
\end{textblock}

\begin{textblock}{80}(36,29)
\LARGE  \textcolor{black!15}{\scalebox{9}{$p$}}
\end{textblock}

%
%
% \begin{textblock}{160}(01,81)
% \footnotesize \textcolor{black!5}{\textbf{\small Seminario ``Acuerdos intersubjetivos''\\
% Comunidad Bayesiana Plurinacional} \\}
% \end{textblock}

\end{frame}



\begin{frame}[plain]
\begin{textblock}{160}(00,04)
\centering
\LARGE Verdad
\end{textblock}
\vspace{1.5cm} \large

\centering

 La ciencia tiene pretensión de verdad, de alcanzar\\

\textbf{acuerdos intersubjetivos con validez universal}

\vspace{0.7cm}

\pause

 \large Ciencias formales  \\
 \large  Sistemas axiomáticos sin incertidumbre\\

 \vspace{0.3cm}

  \pause

 \large Ciencias con datos  \\
\large Sistemas abiertos con incertidumbre

\pause
\vspace{0.6cm}

\Large

¿Cuál es la verdad en \\ contextos de incertidumbre?
%
% \pause
% \vspace{0.2cm}
%
%
% Sí. Podemos evitar mentir.

\end{frame}


\begin{frame}[plain]
\begin{textblock}{160}(00,04)
\centering
\LARGE ¿Todo vale lo mismo?\\
\end{textblock}
\vspace{1cm} \large


\only<2->{
\begin{textblock}{50}(3,26) \centering
\includegraphics[width=1\textwidth, page={6}]{../../auxiliar/static/sidewalk_bubblegum_1997_1}
\end{textblock}}
 \only<3->{
\begin{textblock}{50}(55,26) \centering
\includegraphics[width=1\textwidth, page={6}]{../../auxiliar/static/sidewalk_bubblegum_1997_2}
\end{textblock}}
% \only<4>{
% \begin{textblock}{50}(107,20) \centering
% \includegraphics[width=1\textwidth, page={6}]{../../auxiliar/static/sidewalk_bubblegum_1997_3}
% \end{textblock}}
\only<4->{
\begin{textblock}{50}(107,26) \centering
\includegraphics[width=1\textwidth, page={6}]{../../auxiliar/static/sidewalk_bubblegum_1997_4}
\end{textblock}}

\end{frame}

\begin{frame}[plain]
\begin{textblock}{160}(0,4) \centering
\LARGE Sabemos no mentir \\
\end{textblock}
\vspace{2cm}



\Large

\centering

$\bullet$ No afirmar más de lo que sabemos \pause

$\bullet$ Sin dejar de decir todo lo que sí sabemos

\pause \centering \vspace{1cm}

\Large

\textbf{¿Cómo exactamente?}


\end{frame}






%
% \begin{frame}[plain]
%  \begin{textblock}{160}(0,4)
%  \centering \LARGE Incertidumbre \\
%  \Large Libre albedrío
% \end{textblock}
% \vspace{1.5cm}
% \centering
%
% \only<2-3>{
% \begin{textblock}{75}(75,28)
% Detrás de una caja hay un regalo \\[0.3cm]
%
%  \scalebox{1}{
% \tikz{ %
%          \node[factor, minimum size=1cm] (p1) {} ;
%          \node[factor, minimum size=1cm, xshift=1.5cm] (p2) {} ;
%          \node[factor, minimum size=1cm, xshift=3cm] (p3) {} ;
%
%
%          \node[const, above=of p1, yshift=0.1cm] (np1) {\Large $?$};
%          \node[const, above=of p2, yshift=0.1cm] (np2) {\Large $?$};
%          \node[const, above=of p3, yshift=0.1cm] (np3) {\Large $?$};
%          }
% }
% \end{textblock}
% }
%
%
% \begin{textblock}{75}(05,18)
%  \centering
%  \begin{figure}[H]
% \centering
%     \scalebox{1.2}{
%  \tikz{ %
%         \node[estado] (s) {};
%         \node[const, above=of s] {$s$};
%         \node[accion, below=of s, xshift=-1cm] (a1) {} ; %
% 	\node[accion, below=of s, xshift=0cm] (a2) {} ; %
% 	\node[accion, below=of s, xshift=1cm] (a3) {} ; %
% 	\node[const, right=of a3] {$a$};
%         \edge[-] {s} {a1,a2,a3};
%
% 	\node[estado, below=of a1,xshift=0cm] (s1a) {}; %
%
% 	\node[estado, below=of a2,xshift=0cm] (s2a) {}; %
%
% 	\node[estado, below=of a3,xshift=0cm] (s3a) {}; %
%
% 	\node[const, right=of s3a] {$s^{\prime}$};
% 	\edge[-] {a1} {s1a};
% 	\edge[-] {a2} {s2a};
% 	\edge[-] {a3} {s3a};
%         }
%
%     }
% \end{figure}
% \normalsize \textbf{Universos paralelos} \ \ \ \   \\ $s$: estados, $a$: acciones \ \ \ \
% \end{textblock}
%
% \only<3>{
% \begin{textblock}{160}(0,70)
% \centering \Large
% Funci\'ones de probabilidad permiten expresar nuestra incertidumbre
%  \begin{align*}
% P(a = 1)  = \, ? \ \ \  P(a = 2) = \,? \ \ \  P(a = 3) = \,?
%  \end{align*}
% \end{textblock}
% }
%
% \end{frame}
%

\begin{frame}[plain]
 \begin{textblock}{160}(0,4)
 \centering \LARGE \onslide<3->{Distribución de creencias \\}
 %\Large \only<5>{\textbf{Primer acuerdo intersubjetivo!}}
\end{textblock}
\vspace{1.5cm}
\centering


\only<1>{
\begin{textblock}{160}(0,62)
\Large Detrás de una de estas caja hay un regalo. \\[0.1cm]

\large ¿Dónde está el regalo?
\end{textblock}
}

\only<1>{
\begin{textblock}{160}(0,28)
 \scalebox{1.1}{
\tikz{ %
         \node[factor, minimum size=1cm] (p1) {} ;
         \node[factor, minimum size=1cm, xshift=1.5cm] (p2) {} ;
         \node[factor, minimum size=1cm, xshift=3cm] (p3) {} ;


         \node[const, above=of p1, yshift=0.1cm] (np1) {\Large $?$};
         \node[const, above=of p2, yshift=0.1cm] (np2) {\Large $?$};
         \node[const, above=of p3, yshift=0.1cm] (np3) {\Large $?$};
         }
}
\end{textblock}
}

\only<2>{
\begin{textblock}{160}(0,28)
 \scalebox{1.1}{
\tikz{ %
         \node[factor, minimum size=1cm] (p1) {} ;
         \node[factor, minimum size=1cm, xshift=1.5cm] (p2) {} ;
         \node[factor, minimum size=1cm, xshift=3cm] (p3) {} ;


         \node[const, above=of p1, yshift=0.125cm] (np1) {\Large $0$};
         \node[const, above=of p2, yshift=0.125cm] (np2) {\Large $1$};
         \node[const, above=of p3, yshift=0.125cm] (np3) {\Large $0$};
         }
}
\end{textblock}
}

\only<3>{
\begin{textblock}{160}(0,28)
 \scalebox{1.1}{
\tikz{ %
         \node[factor, minimum size=1cm] (p1) {} ;
         \node[factor, minimum size=1cm, xshift=1.5cm] (p2) {} ;
         \node[factor, minimum size=1cm, xshift=3cm] (p3) {} ;


         \node[const, above=of p1, yshift=-0.05cm] (np1) {\Large $1/10$};
         \node[const, above=of p2, yshift=-0.05cm] (np2) {\Large $8/10$};
         \node[const, above=of p3, yshift=-0.05cm] (np3) {\Large $1/10$};
         }
}
\end{textblock}
}


\only<4->{
\begin{textblock}{160}(0,28)
 \scalebox{1.1}{
\tikz{ %
         \node[factor, minimum size=1cm] (p1) {} ;
         \node[factor, minimum size=1cm, xshift=1.5cm] (p2) {} ;
         \node[factor, minimum size=1cm, xshift=3cm] (p3) {} ;


         \node[const, above=of p1, yshift=-0.05cm] (np1) {\Large $1/3$};
         \node[const, above=of p2, yshift=-0.05cm] (np2) {\Large $1/3$};
         \node[const, above=of p3, yshift=-0.05cm] (np3) {\Large $1/3$};
         }
}
\end{textblock}
}

\only<5->{
\begin{textblock}{140}(10,64)   \centering \Large
Acuerdo intersubjetivo\\[0.1cm]
\large 1. Máximizamos incertidumbre  \\
\large 2. Dada la información disponible

\end{textblock}
}

\end{frame}

\begin{frame}[plain]
\begin{textblock}{160}(0,4)
\centering \Large ¿Cómo podemos dar continuidad a los acuerdos intersubjetivos? \\
\large \onslide<2>{\textbf{Las reglas de la probabilidad}}
\end{textblock}
\onslide<2>{
\vspace{1.5cm}



\begin{columns}[t]
\begin{column}{0.5\textwidth}
 \centering

 \vspace{0.4cm}

\textbf{Regla de la suma}


\begin{equation*}
 P(r) = \sum_j P(r,s_j)
\end{equation*}



 \footnotesize
Predecimos con la contribución\\
de todas las hipótesis.

 \end{column}
 \begin{column}{0.5\textwidth}

\centering

 \vspace{0.4cm}

\textbf{Regla del producto}

\begin{equation*}
 P(r|s)  = \frac{P(r,s)}{P(s)}
\end{equation*}

\vspace{0.1cm}

\footnotesize
Preservamos la creencia previa que \\
sigue siendo compatible con el dato

\end{column}
\end{columns}
}
\end{frame}
%
% \begin{frame}[plain]
% \begin{textblock}{160}(0,4)
% \centering \Large Ejemplo: Monty Hall
% \end{textblock}
%
% \begin{textblock}{150}(0,18)
% \begin{figure}[ht!]
%  \centering
%   \begin{subfigure}[b]{0.3\textwidth}
%   \centering
%   \tikz{
%
%     \node[latent] (d) {\includegraphics[width=0.10\textwidth]{../../auxiliar/static/dedo.png}} ;
%     \node[const,below=of d] (nd) {\En{Hint}\Es{Pista}: $s \neq r$, $s \neq c$  } ;
%
%     \node[latent, above=of d,yshift=-1.2cm, xshift=-1.3cm] (r) {\includegraphics[width=0.12\textwidth]{../../auxiliar/static/regalo.png}} ;
%     \node[const,above=of r] (nr) {\phantom{g}\En{Gift}\Es{Regalo}: $r$\phantom{g}} ;
%
%     \node[latent, fill=black!30, above=of d,yshift=-1.2cm, xshift=1.3cm] (c) {\includegraphics[width=0.12\textwidth]{../../auxiliar/static/cerradura.png}} ;
%     \node[const,above=of c] (nc) {\phantom{g}\En{Election}\Es{Elección}: $c_1$\phantom{g}} ;
%
%     \edge {r,c} {d};
%
%          \node[factor, minimum size=0.8cm, xshift=-1.3cm, yshift=2.5cm] (p1) {\includegraphics[width=0.075\textwidth]{../../auxiliar/static/cerradura.png}} ;
%          \node[det, minimum size=0.8cm, yshift=2.5cm] (p2) {\includegraphics[width=0.085\textwidth]{../../auxiliar/static/dedo.png}} ;
%          \node[factor, minimum size=0.8cm, xshift=1.3cm, yshift=2.5cm] (p3) {} ;
%          \node[const, above=of p1, yshift=.05cm] (fp1) {$?$};
%          \node[const, above=of p2, yshift=.05cm] (fp2) {$?$};
%          \node[const, above=of p3, yshift=.05cm] (fp3) {$?$};
%          \node[const, below=of p2, yshift=-.10cm, xshift=0.3cm] (dedo) {};
%
%   }
%   \label{fig:modelo_causal}
%   \end{subfigure}
%   \onslide<2->{
%  \begin{subfigure}[b]{0.32\textwidth}
% \centering
% \tikz{
% \node[latent, draw=white, yshift=0.7cm] (b0) {$1$};
% \node[latent,below=of b0,yshift=0.9cm, xshift=-1.5cm] (r1) {$r_1$};
% {\color{black}\node[latent,draw=black,below=of b0,yshift=0.9cm] (r2) {$r_2$}; }
% \node[latent,below=of b0,yshift=0.9cm, xshift=1.5cm] (r3) {$r_3$};
% \node[latent, below=of r1, draw=white, yshift=0.7cm] (bc11) {$\frac{1}{3}$};
% {\color{black}\node[latent, below=of r2, draw=white, yshift=0.7cm] (bc12) {$\frac{1}{3}$};}
% \node[latent, below=of r3, draw=white, yshift=0.7cm] (bc13) {$\frac{1}{3}$};
% \node[latent,below=of bc11,yshift=0.9cm, xshift=-0.5cm] (r1d2) {$s_2$};
% {\color{black}\node[latent,draw=black,below=of bc11,yshift=0.9cm, xshift=0.5cm] (r1d3) {$s_3$};}
% {\color{black}\node[latent, draw=black,below=of bc12,yshift=0.9cm] (r2d3) {$s_3$};}
% \node[latent,below=of bc13,yshift=0.9cm] (r3d2) {$s_2$};
%
% \node[latent,below=of r1d2,yshift=0.9cm,draw=white] (br1d2) {$\frac{1}{3}\frac{1}{2}$};
% {\color{black}\node[latent,below=of r1d3,yshift=0.9cm, draw=white] (br1d3) {$\frac{1}{3}\frac{1}{2}$};}s
% {\color{black}\node[latent,below=of r2d3,yshift=0.9cm,draw=white] (br2d3) {$\frac{1}{3}$};}
% \node[latent,below=of r3d2,yshift=0.9cm,draw=white] (br3d2) {$\frac{1}{3}$};
% \edge[-] {b0} {r1,r3};
% \edge[-,draw=black] {b0} {r2};
% \edge[-] {r1} {bc11};
% \edge[-,draw=black] {r2} {bc12};
% \edge[-] {r3} {bc13};
% \edge[-] {bc11} {r1d2};
% \edge[-,draw=black] {bc11} {r1d3};
% \edge[-,draw=black] {bc12} {r2d3};
% \edge[-] {bc13} {r3d2};
% \edge[-] {r1d2} {br1d2};
% \edge[-,draw=black] {r1d3} {br1d3};
% \edge[-,draw=black] {r2d3} {br2d3};
% \edge[-] {r3d2} {br3d2};
% }
% \label{fig:caminos_montyhall_compatibles}
% \end{subfigure}
% }
% \onslide<3>{
% \begin{subfigure}[b]{0.32\textwidth}
% \phantom{$P(s_j)$\hspace{0.2cm}g$P(r_i, s_j)$\phantom{g}\hspace{0.8cm}$P(s_j)$} \\
% \centering
%   \begin{tabular}{|c|c|c|c||c|} \hline  \setlength\tabcolsep{0.4cm}
%  & \, $r_1$ \, &  \, $r_2$ \, & \, $r_3$ \, &  \\ \hline
%   $\gray s_1$ & $\gray0$ & $\gray0$ & $\gray0$ &   $\gray 0$ \\ \hline
%   $\bm{s_2}$ & $\bm{1/6}$ & $\bm{0}$ & $\bm{1/3}$ &  $\bm{1/2}$ \\  \hline
%   $\gray s_3$ & $\gray1/6$ & $\gray1/3$ & $\gray0$ & $\gray1/2$ \\ \hline
%   \end{tabular}
%
%   \phantom{--}\\[0.1cm]
%
%   \tikz{
%
%          \node[factor, minimum size=0.8cm, xshift=-1.3cm] (p1) {\includegraphics[width=0.075\textwidth]{../../auxiliar/static/cerradura.png}} ;
%          \node[det, minimum size=0.8cm] (p2) {\includegraphics[width=0.085\textwidth]{../../auxiliar/static/dedo.png}} ;
%          \node[factor, minimum size=0.8cm, xshift=1.3cm] (p3) {} ;
%          \node[const, below=of p1, yshift=-.025cm] (fp1) {$1/3$};
%          \node[const, below=of p2, yshift=-.025cm] (fp2) {$0$};
%          \node[const, below=of p3, yshift=-.025cm] (fp3) {$2/3$};
%          \node[const, above=of p2, yshift=.045cm] (title) {$P(r_i | s_2)=P(r_i,s_2)/P(s_2)$};
%
%          \node[const, below=of p2, yshift=-.10cm, xshift=0.3cm] (dedo) {};
%
%   }
% \label{fig:f3}
% \end{subfigure}
% }
% \label{fig:monty_hall}
% \end{figure}
% \end{textblock}
%
% \only<3>{
% \begin{textblock}{160}(105,20)
% \phantom{$P(s_j)$\hspace{0.2cm}g}$P(r_i, s_j)$\phantom{g}\hspace{0.8cm}$P(s_j)$
% \end{textblock}
% }
% \end{frame}


\begin{frame}[plain]
\begin{textblock}{160}(0,4)
\centering \LARGE Enfoque bayesiano \\
\Large Aplicación estricta de las reglas de la probabilidad
\end{textblock}



\begin{textblock}{160}(0,28) \centering
\Large El problema histórico: \\[0.1cm]

\large
La complejidad computacional de evaluar \\
\textbf{todo el espacio de hipótesis}.

\end{textblock}


\begin{textblock}{120}(20,56)
\only<2->{$\bullet$ Siglo 19: Física estadística}

\only<3->{$\bullet$ Siglo 20: Enfoque frecuentista}

\only<4->{$\bullet$ Siglo 21: Por primera vez comienza a ser posible aplicar las reglas de la probabilidad de forma estricta en todos los campos de la ciencia}
\end{textblock}



\end{frame}


\begin{frame}[plain]
\begin{textblock}{160}(0,4)
\centering  \Large Regresi\'on lineal (bayesiana)
\end{textblock}

\vspace{1.5cm}

\begin{equation*}
\begin{split}
y & = \beta_0 + \beta_1 x + \varepsilon \\[0.2cm]
p(\varepsilon) &= \N(\varepsilon\,|\, 0, c^2) \\[0.6cm]
\onslide<2->{
p(\beta_0) &= \N(\beta_0 \,|\, 0, \sigma_{0}^2) \\[0cm]
p(\beta_1) &= \N(\beta_1 \,|\, 0, \sigma_{1}^2) \\}
\end{split}
\end{equation*}

\end{frame}


\begin{frame}[plain]

\Wider[-3cm]{
 \begin{figure}
\begin{subfigure}[t]{0.32\textwidth}
\onslide<3->{\caption*{Verosimilitud}}
\end{subfigure}
\begin{subfigure}[t]{0.32\textwidth}
\caption*{Priori\onslide<3->{/Posteriori}}
\includegraphics[width=\textwidth]{../../clases/5-regresionLineal/figures/pdf/linearRegression_posterior_0.pdf}
\end{subfigure}
\begin{subfigure}[t]{0.32\textwidth}
\onslide<2->{
\caption*{Espacio de datos}
\includegraphics[width=\textwidth]{../../clases/5-regresionLineal/figures/pdf/linearRegression_dataSpace_0.pdf}}
\end{subfigure}


\begin{subfigure}[c]{0.32\textwidth}
\onslide<3->{\includegraphics[width=\textwidth]{../../clases/5-regresionLineal/figures/pdf/linearRegression_likelihood_1.pdf}}
\end{subfigure}
\begin{subfigure}[c]{0.32\textwidth}
\onslide<3->{\includegraphics[width=\textwidth]{../../clases/5-regresionLineal/figures/pdf/linearRegression_posterior_1.pdf}}
\end{subfigure}
\begin{subfigure}[c]{0.32\textwidth}
\onslide<3->{\includegraphics[width=\textwidth]{../../clases/5-regresionLineal/figures/pdf/linearRegression_dataSpace_1.pdf}}
\end{subfigure}

\begin{subfigure}[c]{0.32\textwidth}
\onslide<4->{\includegraphics[width=\textwidth]{../../clases/5-regresionLineal/figures/pdf/linearRegression_likelihood_2.pdf}}
\end{subfigure}
\begin{subfigure}[c]{0.32\textwidth}
\onslide<4->{\includegraphics[width=\textwidth]{../../clases/5-regresionLineal/figures/pdf/linearRegression_posterior_2.pdf}}
\end{subfigure}
\begin{subfigure}[c]{0.32\textwidth}
\onslide<4->{\includegraphics[width=\textwidth]{../../clases/5-regresionLineal/figures/pdf/linearRegression_dataSpace_2.pdf}}
\end{subfigure}

\end{figure}
}
\end{frame}


\begin{frame}[plain]
\begin{textblock}{160}(0,4)
\centering  \Large Regresi\'on lineal (bayesiana)
\end{textblock}

\only<2->{
\begin{textblock}{160}(0,18)
\begin{equation*}
y = \beta_0 + \beta_1 x + \beta_2 x^2 + \dots
\end{equation*}
\end{textblock}
}


\begin{textblock}{80}(0,28)
 \centering
 \onslide<1->{Funci\'on objetivo}
\end{textblock}

\begin{textblock}{80}(80,28)
 \centering
 \onslide<2>{Modelos polinomiales}
\end{textblock}

\begin{textblock}{160}(0,32)
     \centering
       \onslide<1->{\includegraphics[width=0.45\textwidth]{../../clases/5-regresionLineal/figures/pdf/model_selection_true_and_sample} }
       \onslide<2>{\includegraphics[width=0.45\textwidth]{../../clases/5-regresionLineal/figures/pdf/model_selection_MAP_non-informative} }
\end{textblock}

\end{frame}

\begin{frame}[plain]
\begin{textblock}{160}(0,4)
\centering  \Large Regresi\'on lineal (bayesiana) \\
\large Evaluación de modelo
\end{textblock}



\begin{textblock}{160}(0,18)
\begin{equation*}
y = \beta_0 + \beta_1 x + \beta_2 x^2 + \dots
\end{equation*}
\end{textblock}

\begin{textblock}{80}(0,28)
 \centering
 M\'axima verosimilitud
\end{textblock}

\only<2->{
\begin{textblock}{80}(80,28)
\centering
Aplicación estricta de la probabilidad
\end{textblock}
}

\begin{textblock}{160}(0,29)
     \centering
  \begin{figure}[H]
     \centering
      \begin{subfigure}[b]{0.45\textwidth}
       \includegraphics[width=1\textwidth]{../../clases/5-regresionLineal/figures/pdf/model_selection_maxLikelihood}
     \end{subfigure}
     \onslide<2->{
    \begin{subfigure}[b]{0.45\textwidth}
       \includegraphics[width=1\textwidth]{../../clases/5-regresionLineal/figures/pdf/model_selection_evidence}
     \end{subfigure}
    }
\end{figure}
\end{textblock}

\end{frame}



\begin{frame}[plain]
\begin{textblock}{160}(0,4)
\centering \LARGE La función de costo epistémica \\
\end{textblock}


\begin{textblock}{160}(0,20)
\begin{equation*}
\underbrace{P(\text{Modelo},\text{\En{Data}\Es{Datos}})}_{\hfrac{\text{\footnotesize\En{Initial belief compatible}\Es{Creencia compatible }}}{\text{\footnotesize \En{with the data}\Es{con los datos}}}} = \underbrace{P(\text{Modelo})}_{\hfrac{\text{\footnotesize\En{Initial intersubjective}\Es{Acuerdo intersubjetivo}}}{\text{\footnotesize\En{agreement}\Es{inicial}}}} \underbrace{P(\text{data}_1 |\text{Modelo})}_{\text{\footnotesize Predic\En{tion}\Es{ción} 1}} \, \underbrace{P(\text{data}_2 | \text{data}_1 , \text{Modelo})}_{\text{\footnotesize Predic\En{tion}\Es{ción} 2}} \dots
\end{equation*}
\end{textblock}

\only<2->{
\begin{textblock}{140}(10,50)\centering
Un 0 (cero) en la secuencia hace falsa la hipótesis para siempre \\

\only<3->{\textbf{Ventaja a favor de las variantes que reducen fluctuaciones}}
\end{textblock}
}


\only<4->{
\begin{textblock}{140}(10,66)\centering
\begin{equation*}
P(\text{data}_1|\text{Modelo}) = \only<5->{\phantom}{\sum_{\text{hipótesis}}}  P(\text{data}_1| \text{hipótesis}, \text{Modelo}) \only<5->{\phantom}{P(\text{hipótesis} | \text{Modelo})}
\end{equation*}
\end{textblock}
}


\end{frame}



\begin{frame}[plain]
\begin{textblock}{160}(0,4)
\centering \LARGE La función de costo evolutiva \\
\end{textblock}

\begin{textblock}{160}(0,18) \centering
La selección de las formas de vida también es por\\

secuencia de tasas de supervivencia y reproducción
\end{textblock}



\only<2-3>{
\begin{textblock}{160}(0,34) \centering
El modelo estándar de evolución es equivalente al \textbf{Teorema de Bayes}!

\onslide<3>{
\begin{equation*} \footnotesize
\overbrace{P\Big(\hfrac{\text{Hipótesis o}}{\text{Forma de vida}}  \Big| \text{Datos}, \hfrac{\text{Modelo}}{\text{Causal}} \Big)}^{\hfrac{\text{\scriptsize Nueva proporción}}{\text{\scriptsize de la variante}}} = \frac{ \overbrace{P\Big(\text{Datos},  \Big|  \hfrac{\text{Hipótesis o}}{\text{Forma de vida}}  , \hfrac{\text{Modelo}}{\text{Causal}} \Big)}^{\hfrac{\text{\scriptsize Adaptabilidad de la}}{\text{\scriptsize variante a la realidad}}} \overbrace{P\Big(\hfrac{\text{Hipótesis o}}{\text{Forma de vida}} \Big|  \hfrac{\text{Modelo}}{\text{Causal}} \Big)}^{\hfrac{\text{\scriptsize Vieja proporción}}{\text{\scriptsize de la variante}}}}{\underbrace{P\Big(\text{Datos},  \hfrac{\text{\small Modelo}}{\text{\small Causal}} \Big)}_{\hfrac{\text{\scriptsize Proporción}}{\text{\scriptsize sobreviviente}}}}
\end{equation*}
}
\end{textblock}
}


\only<4>{
\begin{textblock}{160}(0,36) \centering

Transiciones evolutivas mayores

\vspace{0.5cm}

\begin{figure}[ht!]
    \centering
  \scalebox{1.2}{
  \tikz{
      \node[accion] (i1) {} ;
      \node[accion, yshift=0.6cm, xshift=0.4cm] (i2) {} ;
      \node[accion, yshift=0.6cm, xshift=-0.4cm] (i3) {} ;
      \node[const, yshift=0.3cm, xshift=0.4cm] (i) {};

      \node[const, yshift=-0.8cm] (ni) {$\hfrac{\text{Individuos}}{\text{solitarios}}$};

      \node[const, yshift=1.2cm, xshift=1.5cm] (m1) {$\hfrac{\text{Formación}}{\text{de grupos}}$};

      \node[const, right=of i, xshift=2cm] (c) {};
      \node[accion, below=of c, yshift=0.35cm, xshift=0.4cm] (c1) {} ;
      \node[accion, above=of c, yshift=-0.35cm, xshift=0.6cm] (c2) {} ;
      \node[accion, above=of c, yshift=-0.35cm, xshift=0.2cm] (c3) {} ;
      \node[const, right=of c, xshift=0.6cm] (cc) {};

      \node[const, right=of ni, xshift=1.3cm] (nc) {$\hfrac{\text{Grupos}}{\text{cooperativos}}$};

      \node[const, right=of m1, xshift=1.2cm] (m2) {$\hfrac{\text{Transición}}{\text{mayor}}$};

      \node[const, right=of cc, xshift=2cm] (t) {};
      \node[accion, below=of t, yshift=0.35cm, xshift=0.4cm] (t1) {} ;
      \node[accion, above=of t, yshift=-0.35cm, xshift=0.6cm] (t2) {} ;
      \node[accion, above=of t, yshift=-0.35cm, xshift=0.2cm] (t3) {} ;

      \node[const, right=of nc, xshift=1.1cm] (nt) {$\hfrac{\text{Unidad de}}{\text{nivel superior}}$};

      \edge {i} {c};
      \edge {cc} {t};

      \plate {transition} {(t1)(t2)(t3)} {}; %
      }
  }
  \end{figure}
\end{textblock}
}


\end{frame}



\begin{frame}[plain]
\begin{textblock}{160}(0,04) \centering
  \LARGE Las apuestas como ejemplo\\
  %\Large Otro proceso multiplicativo
\end{textblock}
\vspace{1.2cm}

\begin{textblock}{140}(10,16)
\begin{multicols*}{2}
\noindent
\large Casa de apuestas,\\[0.2cm]

\normalsize
$\bullet$ Pagos por \texttt{Cara}: $Q_c= 3$

$\bullet$ Pagos por \texttt{Seca}: $Q_s= 1.2$


\columnbreak

\onslide<2->{
\noindent
\large Las apuestas,\\[0.2cm]

\normalsize
$\bullet$ Proporción de recursos a \texttt{Cara}: $b$

$\bullet$ Proporción de recursos a \texttt{Seca}: $1-b$
}

\end{multicols*}
\end{textblock}
%
% \only<3-8>{
% \begin{textblock}{140}(10,46)
%  \begin{equation*}
%  \onslide<6->{(} \omega_0 \, \onslide<4->{\phantom}{b} \,  \onslide<5->{\phantom}{Q_c} \onslide<6->{)} \, \onslide<7->{\phantom}{(1-b)} \, \onslide<8->{\phantom}{Q_s}
% \end{equation*}
% \end{textblock}
% }
%
%
% \only<9>{
% \begin{textblock}{140}(10,37)
% \begin{equation*}
% \begin{split}
% \omega_2(b) & = \underbrace{\omega_0 \, \overbrace{b \,  Q_c}^{\text{Cara}}}_{\omega_1(b)} \, \overbrace{(1-b) \, Q_s}^{\text{Seca}}
% \end{split}
% \end{equation*}
% \end{textblock}
% }
%
%
% \only<10->{
% \begin{textblock}{140}(10,41)
% \begin{equation*}
% \begin{split}
% \omega_T(b) & = \omega_0 \,  (b \,  {Q_c})^{n_c}  \,  ((1-b) \, {Q_s})^{n_s}
% \end{split}
% \end{equation*}
% \end{textblock}
% }
%
%
% \only<11->{
% \begin{textblock}{140}(10,54)
% \begin{equation*}
% \begin{split}
% & \frac{\omega_T(b)}{\omega_T(d)} = \frac{\omega_0 \,  (b \,  \cancel{Q_c})^{n_c}  \,  ((1-b) \, \bcancel{Q_s})^{n_s}}{\omega_0 \,   (d \,  \cancel{Q_c})^{n_c}  \,  ((1-d) \, \bcancel{Q_s})^{n_s}}
% \end{split}
% \end{equation*}
% \end{textblock}
% }
%
% \only<3->{
% \begin{textblock}{140}(10,42)
% \begin{equation*}
% \begin{split}
% \underset{b}{\text{arg max}} \phantom{T} \omega_T(b) \ = \onslide<3>{?}\onslide<4->{f}
% \end{split}
% \end{equation*}
% \end{textblock}
% }
%
%
% \only<5->{
% \begin{textblock}{140}(10,57) \centering
% ¡No importa lo que pague la casa de apuestas!
% \end{textblock}
% }
%
% \only<6->{
% \begin{textblock}{140}(10,70) \centering
% \Large La propiedad epistémica: \\ \large
%
% La apuesta óptima diversifica en la misma proporción que la frecuencia\\
% \end{textblock}
% }


\only<3>{
\begin{textblock}{160}(0,44) \centering
\includegraphics[width=0.4\linewidth]{figuras/pdf/ergodicity_individual_trayectories_phantom.pdf}
\end{textblock}
}
\only<4>{
\begin{textblock}{160}(0,44) \centering
\includegraphics[width=0.4\linewidth]{figuras/pdf/ergodicity_individual_trayectories.pdf}
\end{textblock}
}

\only<3->{
\begin{textblock}{160}(115,70)
Jugando individual
\end{textblock}
}

\only<4->{
\begin{textblock}{160}(115,44)
Cooperando
\end{textblock}
}


\end{frame}




\begin{frame}[plain]
\begin{textblock}{160}(0,04) \centering
  \LARGE La propiedad mayor\\
\end{textblock}


\begin{textblock}{140}(05,23) \centering
¿Y si dejo de aportar al fondo común y sigo recibiendo sus beneficios?

\vspace{0.8cm}

\only<2>{\includegraphics[width=0.5\linewidth]{figuras/pdf/ergodicity_desertion0.pdf}}\only<3>{\includegraphics[width=0.5\linewidth]{figuras/pdf/ergodicity_desertion1.pdf}}\only<4>{\includegraphics[width=0.5\linewidth]{figuras/pdf/ergodicity_desertion.pdf}}
\end{textblock}


\end{frame}


\begin{frame}[plain]
\begin{textblock}{160}(0,04) \centering
  \LARGE La historia de la vida\\
\end{textblock}
\centering
\vspace{1cm}


\textbf{A largo plazo solo sobreviven las variantes que \\ reducen fluctuaciones por diversificación y cooperación.}

\vspace{1cm}


\onslide<2->{Nuestra propia vida depende de varios niveles de cooperación}

\vspace{0.3cm}

\begin{figure}[ht!]
\centering
 \begin{subfigure}[b]{0.25\textwidth} \centering
  \onslide<3->{\includegraphics[width=\linewidth]{../../auxiliar/static//cloroplastos.jpg}
  \caption*{Células eucariotas}}
  \end{subfigure}
 \begin{subfigure}[b]{0.23\textwidth} \centering
  \onslide<4->{\includegraphics[width=\linewidth]{../../auxiliar/static//fotosintesis.jpg}
  \caption*{Organismos}
  }
  \end{subfigure}
  \begin{subfigure}[b]{0.235\textwidth} \centering
 \onslide<5->{\includegraphics[width=\linewidth]{../../auxiliar/static//hormigas2.jpg}
 \caption*{Sociedades}
 }
  \end{subfigure}
 \begin{subfigure}[b]{0.235\textwidth} \centering
 \onslide<6->{\includegraphics[width=\linewidth]{../../auxiliar/static//tsimane2.jpg}
 \caption*{Ecosistemas}
 }
  \end{subfigure}
\end{figure}


\end{frame}



\begin{frame}[plain,noframenumbering]
\centering \vspace{0.5cm}
\includegraphics[width=1\textwidth]{../../auxiliar/static/CBP.png}
\end{frame}

%
% \begin{frame}[plain]
% \begin{textblock}{96}(0,6.5)\centering
% {\transparent{0.9}\includegraphics[width=0.8\textwidth]{../../auxiliar/static/inti.png}}
% \end{textblock}
%
% \begin{textblock}{160}(96,5.5)
% \includegraphics[width=0.35\textwidth]{../../auxiliar/static/pachacuteckoricancha}
% \end{textblock}
% \end{frame}





\end{document}



