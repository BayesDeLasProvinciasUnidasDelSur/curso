\documentclass[shownotes,aspectratio=169]{beamer}

\input{../../auxiliar/tex/diapo_encabezado.tex}
\input{../../auxiliar/tex/tikzlibrarybayesnet.code.tex}
 \mode<presentation>
 {
 %   \usetheme{Madrid}      % or try Darmstadt, Madrid, Warsaw, ...
 %   \usecolortheme{default} % or try albatross, beaver, crane, ...
 %   \usefonttheme{serif}  % or try serif, structurebold, ...
  \usetheme{Antibes}
  \setbeamertemplate{navigation symbols}{}
 }
\usetikzlibrary{decorations.text}
\usepackage{rotating}
\usepackage{transparent}

\usepackage{todonotes}
\setbeameroption{show notes}

\newcounter{capitulo}
\setcounter{capitulo}{1}
\newcommand{\unidad}{\thecapitulo \stepcounter{capitulo}}


\estrue

%\title[Bayes del Sur]{}

\begin{document}

\color{black!85}
\large

% \begin{frame}[plain,noframenumbering]
%
%
% \begin{textblock}{160}(0,6) \centering
% \includegraphics[width=0.45\textwidth]{../../auxiliar/static/CBP}
% \end{textblock}
%
%
% % \begin{textblock}{160}(0,24) \centering
% % \LARGE \textcolor{black!85}{\fontsize{22}{0}\selectfont \textbf{Congreso Bayesiano Plurinacional}}
% % \end{textblock}
% % \begin{textblock}{160}(0,6) \centering
% % \LARGE  \textcolor{black!85}{\rotatebox[origin=tr]{0}{\scalebox{4}{\scalebox{1}[-1]{$p$}}}}
% % \end{textblock}
% % \begin{textblock}{160}(0,6) \centering
% % \LARGE \textcolor{black!85}{\scalebox{4}{$p$}}
% % \end{textblock}
% % \begin{textblock}{160}(0,6) \centering
% % \LARGE \textcolor{black!85}{\scalebox{3.7}{C} \hspace{2.4cm} }
% % \end{textblock}
% % \begin{textblock}{160}(0,6) \centering
% % \LARGE \textcolor{black!85}{ \hspace{2.5cm} \scalebox{3.7}{P} }
% % \end{textblock}
%
% \begin{textblock}{160}(0,40)\centering
% \hspace{3.6cm} \LARGE  \textcolor{black!85}{\rotatebox[origin=tr]{-3}{\scalebox{6}{\scalebox{1}[-1]{$p$}}}}
% \end{textblock}
%
% \begin{textblock}{160}(0,49) \centering
% \LARGE  \textcolor{black!85}{\scalebox{4}{$=$}}
% \end{textblock}
%
% \begin{textblock}{160}(0,40)\centering
% \hspace{-3.8cm} \LARGE  \textcolor{black!85}{\scalebox{6}{$p$}}
% \end{textblock}
%
% \begin{textblock}{160}(0,73) \centering \Large \textcolor{black!75}{\textbf{
% Del 4 al 5 de agosto 2023 \\
% La Banda, Santiago del Estero, Argentina \\[0.1cm]}}
%
% \normalsize \texttt{bayesdelsur@gmail.com}
% \end{textblock}
%
% \end{frame}


\begin{frame}[plain,noframenumbering]


\begin{textblock}{160}(0,0)
\includegraphics[width=1\textwidth]{../../auxiliar/static/deforestacion}
\end{textblock}

\begin{textblock}{80}(18,9)
\textcolor{black!15}{\fontsize{44}{55}\selectfont Verdades}
\end{textblock}

\begin{textblock}{47}(85,70)
\centering \textcolor{black!15}{{\fontsize{52}{65}\selectfont Empíricas}}
\end{textblock}

\begin{textblock}{80}(100,28)
\LARGE  \textcolor{black!15}{\rotatebox[origin=tr]{-3}{\scalebox{9}{\scalebox{1}[-1]{$p$}}}}
\end{textblock}

\begin{textblock}{80}(66,43)
\LARGE  \textcolor{black!15}{\scalebox{6}{$=$}}
\end{textblock}

\begin{textblock}{80}(36,29)
\LARGE  \textcolor{black!15}{\scalebox{9}{$p$}}
\end{textblock}

\vspace{2cm}
\maketitle



\begin{textblock}{160}(01,81)
\footnotesize \textcolor{black!5}{\textbf{Curso Verdades Empíricas. \\
Congreso Bayesiano Plurinacional 2023} \\}
\end{textblock}

\end{frame}


\begin{frame}[plain,noframenumbering]

\begin{textblock}{160}(01,03)\centering
\textcolor{black!85}{{\large
\large Curso \textbf{Verdades empíricas} \\[-0.1cm] \footnotesize Congreso Bayesiano Plurinacional 2023}} \\ \scriptsize \texttt{bayesdelsur@gmail.com.ar} - \texttt{bayesdelsur.com.ar}
\end{textblock}



\begin{textblock}{140}(10,19)

\normalsize
 No se requiere ningún tipo de formación previa para participar, pero el curso se aprovecha mejor teniendo conocimientos mínimos de programación.

\vspace{0.3cm}


\normalsize \textbf{Causalidad} \\[0.1cm] \footnotesize
\ \ $1$. Modelos gráficos e inferencia \\
\ \ $2$. Inferencia causal\\

 \vspace{0.3cm}

\normalsize \textbf{Series de tiempo} \\[0.1cm] \footnotesize
\ \ $3$. Sorpresa: el problema de la comunicación con la realidad \\
\ \ $4$. Modelos de historia completa. \\

\vspace{0.3cm}

\normalsize \textbf{Toma de decisiones} \\[0.1cm] \footnotesize
\ \ $5$. La función de costo epistémico-evolutiva \\
\ \ $6$. Competencia ``Apuestas de vida'' \\

\end{textblock}

\end{frame}


\begin{frame}[plain,noframenumbering]

\begin{textblock}{160}(00,04)\centering
\textcolor{black!85}{\Large Objetivos}
\end{textblock}

\begin{textblock}{140}(10,16)

\normalsize

\parbox{14cm}{En las últimas décadas se han desarrollado una gran cantidad de algoritmo de aprendizaje automático.
Bajo este marco, cada nuevo problema se ``resuelve'' con alguno de los algoritmos ya existentes.
Si bien este flujo de trabajo ha sido exitoso para muchas tareas, tiene la desventaja de ser inflexible a la hora de considerar las especificidades propias de cada problema.
Por ejemplo, en él se dificulta la inferencia causal.

\vspace{0.3cm}

En este curso introduciremos un enfoque ``basado en modelos''.
A lo largo de esas mismas décadas también se fueron desarrollando técnicas generales, que permiten crear modelos a medida del problema, de forma sencilla.
Con ellas podemos: expresar de forma gráfica las relaciones causales entre las variables; descomponer las reglas de la probabilidad como mensajes entre los nodos de la red causal; y delegar la inferencia a los lenguajes de programación probabilística.
Además de ser flexible, hace uso óptimo de la información debido a que está basado en la aplicación estricta de la probabilidad.
}
\end{textblock}
\end{frame}


\begin{frame}[plain,noframenumbering]

\centering \LARGE
Taller 1. Causalidad.

\end{frame}



\begin{frame}[plain,noframenumbering]
\begin{textblock}{160}(0,43)
\includegraphics[width=1\textwidth]{../../auxiliar/static/modelosGraficos}
\end{textblock}


\begin{textblock}{160}(4,4)
\LARGE \textcolor{black!85}{\fontsize{22}{0}\selectfont \textbf{Modelos gráficos e inferencia}}
\end{textblock}
% \begin{textblock}{160}(4,12)
% \LARGE \textcolor{black!85}{\fontsize{22}{0}\selectfont \textbf{algoritmos de inferencia}}
% \end{textblock}


\begin{textblock}{55}[0,0](72,23)
\begin{turn}{0}
\parbox{10cm}{\sloppy\setlength\parfillskip{0pt}
\textcolor{black!85}{Unidad \unidad} \\
\small\textcolor{black!85}{Acuerdos intersubjetivos en contextos de incertidumbre.} \\
\small\textcolor{black!85}{Especificación gráfica de modelos causales. Evaluación} \\
\small\textcolor{black!85}{de modelos causales. La emergencia del sobreajuste y el} \\
\small\textcolor{black!85}{balance natural de las reglas de la probabilidad.} \\
}
\end{turn}
\end{textblock}

\end{frame}


\begin{frame}[plain,noframenumbering]

\begin{textblock}{160}(0,0)
\includegraphics[width=1\textwidth]{../../auxiliar/static/peligro_predador}
\end{textblock}

\begin{textblock}{160}(127,67)
\LARGE \textcolor{black!5}{\fontsize{22}{0}\selectfont \textbf{Inferencia  \\[-0.1cm] \hspace{0.5cm} causal}}
\end{textblock}

\begin{textblock}{55}(2,3)
\begin{turn}{0}
\parbox{15cm}{\small
\textcolor{black!95}{Los niveles del razonamiento causal. Flujos de inferencia en}\\
\textcolor{black!95}{modelos causales. Efecto de las intervenciones en modelos} \\
\textcolor{black!95}{causales. Conclusiones causales a partir de observaciones.} \\
\textcolor{black!95}{Identificación de modelo causal mediante intervenciones.} \\
\normalsize\textcolor{black!95}{Unidad \unidad} \\
}
\end{turn}
\end{textblock}

\end{frame}



\begin{frame}[plain,noframenumbering]

\centering \LARGE
Taller 1. Series temporales.

\end{frame}




\begin{frame}[plain,noframenumbering]

\begin{textblock}{160}(0,0)
\includegraphics[width=1\textwidth]{../../auxiliar/static/fuego}
\end{textblock}

\begin{textblock}{160}(4,26)
\LARGE \textcolor{black!5}{\fontsize{22}{0}\selectfont \textbf{Sorpresa: el problema}}
\end{textblock}
\begin{textblock}{160}(4,34)
\LARGE \textcolor{black!5}{\fontsize{22}{0}\selectfont \textbf{de la comunicación}}
\end{textblock}
\begin{textblock}{160}(4,42)
\LARGE \textcolor{black!5}{\fontsize{22}{0}\selectfont \textbf{con la realidad}}
\end{textblock}
% \begin{textblock}{160}(3,82)
% \LARGE \textcolor{black!15}{\fontsize{22}{0}\selectfont \textbf{3}}
% \end{textblock}


\begin{textblock}{55}[0,0](88,25)
\begin{turn}{0}
\parbox{7cm}{\sloppy\setlength\parfillskip{0pt}
\textcolor{black!0}{Unidad \unidad} \\
\small\textcolor{black!5}{\hspace{-0.3cm}La estructura invariante del dato empírico.} \\
\small\textcolor{black!5}{\hspace{-0.3cm}Construcción de sistemas de comunicación con}\\
\small\textcolor{black!5}{\hspace{-0.4cm}la realidad. Tasa de información. Evaluación de} \\
\small\textcolor{black!5}{\hspace{-0.6cm}sistemas alternativos por su tasa de sorpresa.} \\
}
\end{turn}
\end{textblock}


\end{frame}



\begin{frame}[plain,noframenumbering]
\begin{textblock}{160}(0,-4.3) \centering
\includegraphics[width=1\textwidth]{../../auxiliar/static/antartic}
\end{textblock}

\begin{textblock}{160}(0,0) \centering
\tikz{
\node[det, fill=black,draw=black] (k) {\textcolor{black}{--------------------------------------------------------------------------------------------------------------------------------------}} ;
}
\end{textblock}

\begin{textblock}{160}(5,0)
\tikz{
\node[det, fill=black,draw=black,text width=0.01cm] (k) {\textcolor{black}{--------------------------------------------------------------------------------------------------------------------------------------}} ;
}
\end{textblock}


\begin{textblock}{160}(0,4) \centering
\LARGE \hspace{1cm} \textcolor{black!20}{\fontsize{22}{0}\selectfont \textbf{Series de tiempo}}
\end{textblock}


\begin{textblock}{55}[0,1](8,70)
\begin{turn}{90}
\parbox{6cm}{\footnotesize
\textcolor{black!10}{Millones de km$^2$ de hielo Antártico}}
\end{turn}
\end{textblock}


\begin{textblock}{160}(20,63)
\textcolor{black!10}{Unidad \unidad \\ \small
Redes bayesianas de historia completa. \\
El problema de usar el posterior como prior del siguiente evento\\
El algoritmo de inferencia por loopy belief propagation. \\
Consideraciones de inferencia causal en series temporales. \\
}
\end{textblock}


\end{frame}
%
% \begin{frame}[plain,noframenumbering]
% \begin{textblock}{160}(-5,0) \centering
% \includegraphics[width=1.05\textwidth]{../../auxiliar/static/pajarosTrayectorias}
% \end{textblock}
% \begin{textblock}{160}(4,20)
% \LARGE \textcolor{black!6}{\fontsize{22}{0}\selectfont \textbf{Aproximaciones}}
% \end{textblock}
% \begin{textblock}{160}(14,27)
% \LARGE \textcolor{black!6}{\fontsize{22}{0}\selectfont \textbf{por exploración}}
% \end{textblock}
%
%
% \begin{textblock}{65}(90,40)
% \textcolor{black!5}{ \hfill Capítulo \unidad \\ \small
% \hfill Métodos de aproximación para \\
% \hfill modelos causales intratables: \\
% \hfill Markov chain Monte Carlo.\\
% \hfill MCMC. HMC. \\
% }
% \end{textblock}
%
%
% \end{frame}
%
% \begin{frame}[plain,noframenumbering]
% \begin{textblock}{160}(0,0) \centering
% \includegraphics[width=1.2\textwidth]{../../auxiliar/static/ppls}
% \end{textblock}
% \begin{textblock}{160}(8,8)
% \LARGE \textcolor{black!15}{\fontsize{22}{0}\selectfont \textbf{Programación \\ probabilistica \\}}
% \end{textblock}
% % \begin{textblock}{160}(94,23)
% % \LARGE \textcolor{black!16}{\fontsize{22}{0}\selectfont \textbf{probabilistica}}
% % \end{textblock}
%
% \begin{textblock}{160}(70,30)
% \normalsize
% \textcolor{black!25}{
% \tikz{
% \node[det, fill=black!60,draw=black!25] (k) {\textcolor{black!5}{$k_i$}} ;
% \node[latent, fill opacity=0, draw=black!25, text opacity=1, above=of k, xshift=-1cm] (p) {\textcolor{black!5}{$p$}};
% \node[det, fill opacity=0, draw=black!25, text opacity=1, above=of k, xshift=1cm] (n) {\textcolor{black!5}{$n$}};
% \edge {p,n} {k};
% \plate[inner sep=0.3cm, xshift=0cm, yshift=0.12cm] {intentos} {(k)} {$i$}
% \node[const, right=of n, xshift=0.3cm] (np) {$p \sim \text{Beta}(1,1)$};
% \node[const, right=of n, xshift=0.3cm, yshift=-1cm] (np) {$n \sim \text{Categorical}(N_\text{max})$};
% \node[const, right=of n, xshift=0.3cm, yshift=-2cm] (np) {$k_i \sim \text{Binomial}(p,n)$};
% }
% }
% \end{textblock}
%
% \begin{textblock}{160}(20,74)
% \textcolor{black!15}{Capítulo \unidad \\ \small
% Implementación de modelos usando lenguajes de programación probabilística. \\
% Verificación visual de buen funcionamiento de las aproximaciones.  \\
% }
% \end{textblock}
%
%
% \end{frame}



\begin{frame}[plain,noframenumbering]

\centering \LARGE
Taller 3. Toma de decisiones.

\end{frame}


\begin{frame}[plain,noframenumbering]

% \begin{textblock}{160}(0,0)
% \includegraphics[width=1.18\textwidth]{../../aux/static/fotosintesis}
% \end{textblock}
\begin{textblock}{160}(0,-15)
\includegraphics[width=1\textwidth]{../../auxiliar/static/tsimane}
\end{textblock}


% VERSION 2
\begin{textblock}{160}(6,36)
\LARGE \rotatebox[origin=tr]{18}{\textcolor{black!95}{\fontsize{22}{0}\selectfont \textbf{La función}}}
\end{textblock}
\begin{textblock}{160}(41,32)
\LARGE \rotatebox[origin=tr]{23}{\textcolor{black!95}{\fontsize{22}{0}\selectfont \textbf{de}}}
\end{textblock}
\begin{textblock}{160}(50.5,23)
\LARGE \rotatebox[origin=tr]{28}{\textcolor{black!95}{\fontsize{22}{0}\selectfont \textbf{costo}}}
\end{textblock}
\begin{textblock}{160}(68,5.3)
\LARGE \rotatebox[origin=tr]{26}{\textcolor{black!95}{\fontsize{22}{0}\selectfont \textbf{epistémico}}}
\end{textblock}
\begin{textblock}{160}(104,5.5)
\LARGE \rotatebox[origin=tr]{8}{\textcolor{black!95}{\fontsize{22}{0}\selectfont \textbf{-}}}
\end{textblock}
\begin{textblock}{160}(110,3)
\LARGE \rotatebox[origin=tr]{-14}{\textcolor{black!95}{\fontsize{22}{0}\selectfont \textbf{evolutivo}}}
\end{textblock}

%
% \begin{textblock}{55}[0,0](119,22)
% \begin{turn}{-57}
% \parbox{7cm}{\sloppy\setlength\parfillskip{0pt}
% \textcolor{black!0}{\ \ \ \ \ Unidad \unidad} \\
% \small\textcolor{black!5}{\hspace{-0.15cm} Apuestas óptimas.} \\
% \small\textcolor{black!5}{\hspace{-0.85cm} Ventajas a favor de la:} \\
% \small\textcolor{black!5}{\hspace{-1.45cm} Diversificación (propiedad epistémica)}\\
% \small\textcolor{black!5}{\hspace{-1.7cm} Cooperación (propiedad evolutiva)}\\
% \small\textcolor{black!5}{ \hspace{-1.75cm}Especialización (propiedad de especiación)} \\
% \small\textcolor{black!5}{\hspace{-2cm} Heterogeniedad (propiedad ecológica).\\ }}
% \end{turn}
% \end{textblock}
%

\end{frame}


\begin{frame}[plain,noframenumbering]
% \begin{textblock}{160}(0,-80)  \centering
% \includegraphics[width=1\textwidth]{../../aux/static/galton_box}
% \end{textblock}

\begin{textblock}{160}(0,11)  \centering
\includegraphics[width=0.40\textwidth]{../../auxiliar/static/treeOfLife-liviano}
\end{textblock}

\begin{textblock}{160}(0,3) \centering
\LARGE \textcolor{black!85}{\rotatebox[origin=tr]{0}{\fontsize{22}{0}\selectfont \textbf{Apuestas de vida}}}
\end{textblock}
% % <
% \begin{textblock}{160}(0,3) \centering
% \begin{tikzpicture}
%   \node (Start) at (2.8,0) {};
%   \node (End) at (-2.8,0) {};
%   \draw [decorate,decoration={text along path,text align=center,text={sssssssssssssssssssssssssssssssssss|\bf\fontsize{22}{22}\selectfont|Distribuciones de creencias},text color=black!85 }] (End) to [bend left=45] (Start);
% \end{tikzpicture}
% \end{textblock}

%  \begin{textblock}{160}(0,3) \centering
% \tikz{
% \node[factor, xshift=-3cm, opacity=0] (a) {} ;
% \node[factor, xshift=3cm, opacity=0] (b) {} ;
% \path[draw, -, fill=black!50,sloped,draw opacity=0] (a) edge[bend left=45,draw=black!50] node[color=black!75] {\scriptsize  \texttt{lhood\_lose\_tb}} (b);
% }
% \end{textblock}


\begin{textblock}{55}(75,39)
\textcolor{black!85}{\normalsize El árbol de la vida \\
\fontsize{2}{0}\selectfont Synthesis of phylogeny and taxonomy into a comprehensive tree of life \\}
\end{textblock}


\begin{textblock}{55}(3,81)
\textcolor{black!85}{Unidad \unidad}
\end{textblock}

\begin{textblock}{55}(25,81.3)
\begin{turn}{0}
\parbox{15cm}{\small \textcolor{black!85}{Presentación de una competencia de inferecia con apuestas e intercambio de recursos.}
}
\end{turn}
\end{textblock}

\end{frame}

%
%
% \begin{frame}[plain,noframenumbering]
% \begin{textblock}{160}(0,0)  \centering
% \includegraphics[width=1\textwidth]{../../auxiliar/static/reciprocidad}
% \end{textblock}
%
% \begin{textblock}{160}(0,3) \centering
% \LARGE \textcolor{black!15}{\rotatebox[origin=tr]{0}{\fontsize{22}{0}\selectfont \textbf{Tiempo de colaborar}}}
% \end{textblock}
%
% \begin{textblock}{55}(126,82)
% \begin{turn}{-5}
% \textcolor{black!5}{Capítulo \unidad}
% \end{turn}
% \end{textblock}
%
% \begin{textblock}{55}(115,84)
% \begin{turn}{-5}
% \small \textcolor{black!5}{Tecnologías de reciprocidad}
% \end{turn}
% \end{textblock}
%
% \end{frame}

%
% \begin{frame}[plain,noframenumbering]
%
% \begin{textblock}{96}(0,6.5)\centering
% {\transparent{0.9}\includegraphics[width=0.8\textwidth]{../../auxiliar/static/inti.png}}
% \end{textblock}
%
% \begin{textblock}{160}(96,5.5)
% \includegraphics[width=0.35\textwidth]{../../auxiliar/static/pachacuteckoricancha}
% \end{textblock}
%
% \end{frame}


\end{document}
