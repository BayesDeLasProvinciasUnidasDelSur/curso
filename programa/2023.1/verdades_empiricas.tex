\documentclass[shownotes,aspectratio=169]{beamer}

\input{../../auxiliar/tex/diapo_encabezado.tex}
% tikzlibrary.code.tex
%
% Copyright 2010-2011 by Laura Dietz
% Copyright 2012 by Jaakko Luttinen
%
% This file may be distributed and/or modified
%
% 1. under the LaTeX Project Public License and/or
% 2. under the GNU General Public License.
%
% See the files LICENSE_LPPL and LICENSE_GPL for more details.

% Load other libraries
\usetikzlibrary{shapes}
\usetikzlibrary{fit}
\usetikzlibrary{chains}
\usetikzlibrary{arrows}

% Latent node
\tikzstyle{latent} = [circle,fill=white,draw=black,inner sep=1pt,
minimum size=20pt, font=\fontsize{10}{10}\selectfont, node distance=1]
% Observed node
\tikzstyle{obs} = [latent,fill=gray!25]
% Invisible node
\tikzstyle{invisible} = [latent,minimum size=0pt,color=white, opacity=0, node distance=0]
% Constant node
\tikzstyle{const} = [rectangle, inner sep=0pt, node distance=0.1]
%state
\tikzstyle{estado} = [latent,minimum size=8pt,node distance=0.4]
%action
\tikzstyle{accion} =[latent,circle,minimum size=5pt,fill=black,node distance=0.4]


% Factor node
\tikzstyle{factor} = [rectangle, fill=black,minimum size=10pt, draw=black, inner
sep=0pt, node distance=1]
% Deterministic node
\tikzstyle{det} = [latent, rectangle]

% Plate node
\tikzstyle{plate} = [draw, rectangle, rounded corners, fit=#1]
% Invisible wrapper node
\tikzstyle{wrap} = [inner sep=0pt, fit=#1]
% Gate
\tikzstyle{gate} = [draw, rectangle, dashed, fit=#1]

% Caption node
\tikzstyle{caption} = [font=\footnotesize, node distance=0] %
\tikzstyle{plate caption} = [caption, node distance=0, inner sep=0pt,
below left=5pt and 0pt of #1.south east] %
\tikzstyle{factor caption} = [caption] %
\tikzstyle{every label} += [caption] %

\tikzset{>={triangle 45}}

%\pgfdeclarelayer{b}
%\pgfdeclarelayer{f}
%\pgfsetlayers{b,main,f}

% \factoredge [options] {inputs} {factors} {outputs}
\newcommand{\factoredge}[4][]{ %
  % Connect all nodes #2 to all nodes #4 via all factors #3.
  \foreach \f in {#3} { %
    \foreach \x in {#2} { %
      \path (\x) edge[-,#1] (\f) ; %
      %\draw[-,#1] (\x) edge[-] (\f) ; %
    } ;
    \foreach \y in {#4} { %
      \path (\f) edge[->,#1] (\y) ; %
      %\draw[->,#1] (\f) -- (\y) ; %
    } ;
  } ;
}

% \edge [options] {inputs} {outputs}
\newcommand{\edge}[3][]{ %
  % Connect all nodes #2 to all nodes #3.
  \foreach \x in {#2} { %
    \foreach \y in {#3} { %
      \path (\x) edge [->,#1] (\y) ;%
      %\draw[->,#1] (\x) -- (\y) ;%
    } ;
  } ;
}

% \factor [options] {name} {caption} {inputs} {outputs}
\newcommand{\factor}[5][]{ %
  % Draw the factor node. Use alias to allow empty names.
  \node[factor, label={[name=#2-caption]#3}, name=#2, #1,
  alias=#2-alias] {} ; %
  % Connect all inputs to outputs via this factor
  \factoredge {#4} {#2-alias} {#5} ; %
}

% \plate [options] {name} {fitlist} {caption}
\newcommand{\plate}[4][]{ %
  \node[wrap=#3] (#2-wrap) {}; %
  \node[plate caption=#2-wrap] (#2-caption) {#4}; %
  \node[plate=(#2-wrap)(#2-caption), #1] (#2) {}; %
}

% \gate [options] {name} {fitlist} {inputs}
\newcommand{\gate}[4][]{ %
  \node[gate=#3, name=#2, #1, alias=#2-alias] {}; %
  \foreach \x in {#4} { %
    \draw [-*,thick] (\x) -- (#2-alias); %
  } ;%
}

% \vgate {name} {fitlist-left} {caption-left} {fitlist-right}
% {caption-right} {inputs}
\newcommand{\vgate}[6]{ %
  % Wrap the left and right parts
  \node[wrap=#2] (#1-left) {}; %
  \node[wrap=#4] (#1-right) {}; %
  % Draw the gate
  \node[gate=(#1-left)(#1-right)] (#1) {}; %
  % Add captions
  \node[caption, below left=of #1.north ] (#1-left-caption)
  {#3}; %
  \node[caption, below right=of #1.north ] (#1-right-caption)
  {#5}; %
  % Draw middle separation
  \draw [-, dashed] (#1.north) -- (#1.south); %
  % Draw inputs
  \foreach \x in {#6} { %
    \draw [-*,thick] (\x) -- (#1); %
  } ;%
}

% \hgate {name} {fitlist-top} {caption-top} {fitlist-bottom}
% {caption-bottom} {inputs}
\newcommand{\hgate}[6]{ %
  % Wrap the left and right parts
  \node[wrap=#2] (#1-top) {}; %
  \node[wrap=#4] (#1-bottom) {}; %
  % Draw the gate
  \node[gate=(#1-top)(#1-bottom)] (#1) {}; %
  % Add captions
  \node[caption, above right=of #1.west ] (#1-top-caption)
  {#3}; %
  \node[caption, below right=of #1.west ] (#1-bottom-caption)
  {#5}; %
  % Draw middle separation
  \draw [-, dashed] (#1.west) -- (#1.east); %
  % Draw inputs
  \foreach \x in {#6} { %
    \draw [-*,thick] (\x) -- (#1); %
  } ;%
}


 \mode<presentation>
 {
 %   \usetheme{Madrid}      % or try Darmstadt, Madrid, Warsaw, ...
 %   \usecolortheme{default} % or try albatross, beaver, crane, ...
 %   \usefonttheme{serif}  % or try serif, structurebold, ...
  \usetheme{Antibes}
  \setbeamertemplate{navigation symbols}{}
 }
\usetikzlibrary{decorations.text}
\usepackage{rotating}
\usepackage{transparent}

\usepackage{todonotes}
\setbeameroption{show notes}

\newcounter{capitulo}
\setcounter{capitulo}{1}
\newcommand{\unidad}{\thecapitulo \stepcounter{capitulo}}


\estrue

%\title[Bayes del Sur]{}

\begin{document}

\color{black!85}
\large

% \begin{frame}[plain,noframenumbering]
%
%
% \begin{textblock}{160}(0,6) \centering
% \includegraphics[width=0.45\textwidth]{../../auxiliar/static/CBP}
% \end{textblock}
%
%
% % \begin{textblock}{160}(0,24) \centering
% % \LARGE \textcolor{black!85}{\fontsize{22}{0}\selectfont \textbf{Congreso Bayesiano Plurinacional}}
% % \end{textblock}
% % \begin{textblock}{160}(0,6) \centering
% % \LARGE  \textcolor{black!85}{\rotatebox[origin=tr]{0}{\scalebox{4}{\scalebox{1}[-1]{$p$}}}}
% % \end{textblock}
% % \begin{textblock}{160}(0,6) \centering
% % \LARGE \textcolor{black!85}{\scalebox{4}{$p$}}
% % \end{textblock}
% % \begin{textblock}{160}(0,6) \centering
% % \LARGE \textcolor{black!85}{\scalebox{3.7}{C} \hspace{2.4cm} }
% % \end{textblock}
% % \begin{textblock}{160}(0,6) \centering
% % \LARGE \textcolor{black!85}{ \hspace{2.5cm} \scalebox{3.7}{P} }
% % \end{textblock}
%
% \begin{textblock}{160}(0,40)\centering
% \hspace{3.6cm} \LARGE  \textcolor{black!85}{\rotatebox[origin=tr]{-3}{\scalebox{6}{\scalebox{1}[-1]{$p$}}}}
% \end{textblock}
%
% \begin{textblock}{160}(0,49) \centering
% \LARGE  \textcolor{black!85}{\scalebox{4}{$=$}}
% \end{textblock}
%
% \begin{textblock}{160}(0,40)\centering
% \hspace{-3.8cm} \LARGE  \textcolor{black!85}{\scalebox{6}{$p$}}
% \end{textblock}
%
% \begin{textblock}{160}(0,73) \centering \Large \textcolor{black!75}{\textbf{
% Del 4 al 5 de agosto 2023 \\
% La Banda, Santiago del Estero, Argentina \\[0.1cm]}}
%
% \normalsize \texttt{bayesdelsur@gmail.com}
% \end{textblock}
%
% \end{frame}


\begin{frame}[plain,noframenumbering]


\begin{textblock}{160}(0,0)
\includegraphics[width=1\textwidth]{../../auxiliar/static/deforestacion}
\end{textblock}

\begin{textblock}{80}(18,9)
\textcolor{black!15}{\fontsize{44}{55}\selectfont Verdades}
\end{textblock}

\begin{textblock}{47}(85,70)
\centering \textcolor{black!15}{{\fontsize{52}{65}\selectfont Empíricas}}
\end{textblock}

\begin{textblock}{80}(100,28)
\LARGE  \textcolor{black!15}{\rotatebox[origin=tr]{-3}{\scalebox{9}{\scalebox{1}[-1]{$p$}}}}
\end{textblock}

\begin{textblock}{80}(66,43)
\LARGE  \textcolor{black!15}{\scalebox{6}{$=$}}
\end{textblock}

\begin{textblock}{80}(36,29)
\LARGE  \textcolor{black!15}{\scalebox{9}{$p$}}
\end{textblock}

\vspace{2cm}
\maketitle



\begin{textblock}{160}(01,81)
\footnotesize \textcolor{black!5}{\textbf{Curso Verdades Empíricas. \\
Congreso Bayesiano Plurinacional 2023} \\}
\end{textblock}

\end{frame}


\begin{frame}[plain,noframenumbering]

\begin{textblock}{160}(01,03)\centering
\textcolor{black!85}{{\large
\large Curso \textbf{Verdades empíricas} \\[-0.1cm] \footnotesize Congreso Bayesiano Plurinacional 2023}} \\ \scriptsize \texttt{bayesdelsur@gmail.com.ar} - \texttt{bayesdelsur.com.ar}
\end{textblock}



\begin{textblock}{140}(10,19)

\normalsize
 No se requiere ningún tipo de formación previa para participar, pero el curso se aprovecha mejor teniendo conocimientos mínimos de programación.

\vspace{0.3cm}

\normalsize \textbf{Fundamentos}. Jueves 3 de agosto, de 9hs a 12hs. \\[0.1cm] \footnotesize
\ \ $1$. Principios interculturales de acuerdos intersubjetivos (Teoría de la probabilidad) \\
\ \ $2$. Sorpresa: el problema de la comunicación con la realidad (Teoría de la información) \\
\ \ $3$. Modelos gráficos y algoritmos de inferencia (Introducción metodología)\\

 \vspace{0.3cm}

\normalsize \textbf{Metodologías}. Viernes 4 de Agosto, de 9hs a 12hs.\scriptsize  \\[0.1cm] \footnotesize
\ \ $4$. Fundamentos e inferencia causal \\
\ \ $5$. Evaluación de modelos causales \\
\ \ $6$. Series de tiempo \\

\vspace{0.3cm}

\normalsize \textbf{Toma de decisiones}. Sábado 5 de agosto, de 9hs a 12hs.\scriptsize  \\[0.1cm] \footnotesize
\ \ $7$. La función de costo epistémico-evolutiva\\
\ \ $8$. Competencia ``Apuestas de vida'' \\
\ \ $9$. Tiempo para colaborar

\end{textblock}

\end{frame}


\begin{frame}[plain,noframenumbering]

\begin{textblock}{160}(00,04)\centering
\textcolor{black!85}{\Large Objetivos}
\end{textblock}

\begin{textblock}{140}(10,16)

\normalsize

\parbox{14cm}{En las últimas décadas se han desarrollado una gran cantidad de algoritmo de aprendizaje automático.
Bajo este marco, cada nuevo problema se ``resuelve'' con alguno de los algoritmos ya existentes.
Si bien este flujo de trabajo ha sido exitoso para muchas tareas, tiene la desventaja de ser inflexible a la hora de considerar las especificidades propias de cada problema.
Por ejemplo, en él se dificulta la inferencia causal.

\vspace{0.3cm}

En este curso introduciremos un enfoque ``basado en modelos''.
A lo largo de esas mismas décadas también se fueron desarrollando técnicas generales, que permiten crear modelos a medida del problema, de forma sencilla.
Con ellas podemos: expresar de forma gráfica las relaciones causales entre las variables; descomponer las reglas de la probabilidad como mensajes entre los nodos de la red causal; y delegar la inferencia a los lenguajes de programación probabilística.
Además de ser flexible, hace uso óptimo de la información debido a que está basado en la aplicación estricta de la probabilidad.
}
\end{textblock}
\end{frame}


\begin{frame}[plain,noframenumbering]

\centering \LARGE
Día 1. Fundamentos.

\end{frame}


\begin{frame}[plain,noframenumbering]
\begin{textblock}{170}(-9,0)
\rotatebox[origin=tr]{90}{\includegraphics[width=0.53\textwidth]{../../auxiliar/static/egipto3.jpeg}}
\end{textblock}

\begin{textblock}{160}(16,9)
\LARGE \textcolor{black!5}{\fontsize{22}{0}\selectfont \textbf{Principios interculturales}}
\end{textblock}
\begin{textblock}{160}(22,18)
\LARGE \textcolor{black!5}{\fontsize{22}{0}\selectfont \textbf{de acuerdos intersubjetivos}}
\end{textblock}


\begin{textblock}{55}(64,33)
\begin{turn}{33}
\parbox{6cm}{
\textcolor{black!5}{\hspace{-0.3cm}Capítulo \unidad} \\
\small\textcolor{black!5}{\hspace{-0.1cm}Definición de verdad en contextos de}\\
\small\textcolor{black!5}{\hspace{0.07cm}incertidumbre. Principios de optima-} \\
\small\textcolor{black!5}{\hspace{0.25cm}lidad y coherencia. Las reglas de} \\ \small\textcolor{black!5}{\hspace{0.4cm}la probabilidad. Teorema de} \\
\small\textcolor{black!5}{\hspace{0.36cm}Bayes, verosimilitud y evidencia.} \\
\small\textcolor{black!5}{\hspace{0.46cm}Evaluación de modelos.} \\
}
\end{turn}
\end{textblock}


\end{frame}



\begin{frame}[plain,noframenumbering]

\begin{textblock}{160}(0,0)
\includegraphics[width=1\textwidth]{../../auxiliar/static/fuego}
\end{textblock}

\begin{textblock}{160}(4,26)
\LARGE \textcolor{black!5}{\fontsize{22}{0}\selectfont \textbf{Sorpresa: el problema}}
\end{textblock}
\begin{textblock}{160}(4,34)
\LARGE \textcolor{black!5}{\fontsize{22}{0}\selectfont \textbf{de la comunicación}}
\end{textblock}
\begin{textblock}{160}(4,42)
\LARGE \textcolor{black!5}{\fontsize{22}{0}\selectfont \textbf{con la realidad}}
\end{textblock}
% \begin{textblock}{160}(3,82)
% \LARGE \textcolor{black!15}{\fontsize{22}{0}\selectfont \textbf{3}}
% \end{textblock}



\begin{textblock}{55}[0,0](88,25)
\begin{turn}{0}
\parbox{7cm}{\sloppy\setlength\parfillskip{0pt}
\textcolor{black!0}{Capítulo \unidad} \\
\small\textcolor{black!5}{\hspace{-0.05cm}La estructura invariante del dato empírico:} \\
\small\textcolor{black!5}{\hspace{0cm}fuente, realidad causal, señal, canal, percep-} \\ \small\textcolor{black!5}{\hspace{-0.05cm}ción, modelo causal, estimación. Medida de} \\
\small\textcolor{black!5}{\hspace{-0.3cm}la información y máxima entropía. Ejemplos} \\
\small\textcolor{black!5}{\hspace{-0.45cm}de física estadística y codificación. Divergencia} \\
\small\textcolor{black!5}{\hspace{-0.45cm}entre distribuciones de probabilidad.} \\
}
\end{turn}
\end{textblock}

\end{frame}

\begin{frame}[plain,noframenumbering]
\begin{textblock}{160}(0,43)
\includegraphics[width=1\textwidth]{../../auxiliar/static/modelosGraficos}
\end{textblock}


\begin{textblock}{160}(4,4)
\LARGE \textcolor{black!85}{\fontsize{22}{0}\selectfont \textbf{Modelos gráficos y}}
\end{textblock}
\begin{textblock}{160}(4,12)
\LARGE \textcolor{black!85}{\fontsize{22}{0}\selectfont \textbf{algoritmos de inferencia}}
\end{textblock}


\begin{textblock}{55}[0,0](72,23)
\begin{turn}{0}
\parbox{10cm}{\sloppy\setlength\parfillskip{0pt}
\textcolor{black!85}{Capítulo \unidad} \\
\small\textcolor{black!85}{Especificación gráfica de modelos causales. Algoritmos } \\
\small\textcolor{black!85}{de inferencia: conjugado, variacional y muestreo.} \\
\small\textcolor{black!85}{Ejemplos: la solución de Bayes, modelo de habilidad de} \\
\small\textcolor{black!85}{la industria del video juego y evaluación de test Covid.}\\
}
\end{turn}
\end{textblock}

\end{frame}

\begin{frame}[plain,noframenumbering]

\centering \LARGE
Día 2. Metodologías.

\end{frame}

%
% \begin{frame}[plain,noframenumbering]
%
% \begin{textblock}{160}(0,0)
% \includegraphics[width=1.01\textwidth]{../../auxiliar/static/bali-channel}
% \end{textblock}
%
%
% \begin{textblock}{160}(99,68)
% \LARGE \textcolor{black!95}{\rotatebox[origin=tr]{10}{\fontsize{22}{0}\selectfont \textbf{Flujos de}}}
% \end{textblock}
%
% \begin{textblock}{160}(103,76)
% \LARGE \textcolor{black!95}{\rotatebox[origin=tr]{12}{\fontsize{22}{0}\selectfont \textbf{inferencia}}}
% \end{textblock}
%
%
%
% \begin{textblock}{55}(48,20)
% \begin{turn}{0}
% \parbox{15cm}{\textcolor{black!5}{\hspace{0.3cm} Capítulo \unidad} \\
% \small \textcolor{black!5}{\hspace{0.7cm} Flujos de \hspace{0.6cm} inferencia} \\
% \small \textcolor{black!5}{\hspace{0.3cm} en modelos \hspace{0.7cm} causales.} \\}
% \end{turn}
% \end{textblock}
%
%
%
% \end{frame}


\begin{frame}[plain,noframenumbering]

\begin{textblock}{160}(0,0)
\includegraphics[width=1\textwidth]{../../auxiliar/static/peligro_predador}
\end{textblock}

\begin{textblock}{160}(127,67)
\LARGE \textcolor{black!5}{\fontsize{22}{0}\selectfont \textbf{Inferencia  \\[-0.1cm] \hspace{0.5cm} causal}}
\end{textblock}

\begin{textblock}{55}(2,3)
\begin{turn}{0}
\parbox{15cm}{\small
\textcolor{black!95}{Algoritmo de inferencia por pasaje de mensajes. Flujos de}\\
\textcolor{black!95}{inferencia en modelos causales. Efecto de las intervenciones}\\
\textcolor{black!95}{sobre los modelos causales. Conclusiones causales a partir} \\
\textcolor{black!95}{datos observables. Buenos y malos controles.} \\
\normalsize\textcolor{black!95}{Capítulo \unidad} \\
}
\end{turn}
\end{textblock}


\end{frame}

\begin{frame}[plain,noframenumbering]
\begin{textblock}{160}(0,15) \centering
\includegraphics[width=0.8\textwidth]{../../auxiliar/static/biomassBarOn.png}
\end{textblock}

\begin{textblock}{160}(0,3) \centering
\LARGE \textcolor{black!90}{\fontsize{22}{0}\selectfont \textbf{Evaluación de modelos causales}}
\end{textblock}

\begin{textblock}{160}(35,24)
\textcolor{black!95}{\small Biomasa de la vida \\
\fontsize{2}{0}\selectfont \hspace{0.05cm} Bar-on et al. The biomass distribution on Earth (2018) \\}
\end{textblock}

\begin{textblock}{160}(16,15)
\LARGE \textcolor{black!0}{\fontsize{1200}{1200}\selectfont $\bm{\bullet}$ }
\end{textblock}
\begin{textblock}{160}(16,17)
\LARGE \textcolor{black!0}{\fontsize{1200}{1200}\selectfont $\bm{\bullet}$ }
\end{textblock}

\begin{textblock}{160}(77,15)
\LARGE \textcolor{black!0}{\fontsize{1200}{1200}\selectfont $\bm{\bullet}$ }
\end{textblock}
\begin{textblock}{160}(78,17)
\LARGE \textcolor{black!0}{\fontsize{1200}{1200}\selectfont $\bm{\bullet}$ }
\end{textblock}


\begin{textblock}{160}(0,71) \centering
\textcolor{black!95}{Capítulo \unidad \\ \small
Evaluación de modelos causales con datos observacionales e intervenciones. \\
La emergencia del sobreajuste (\emph{overfitting}) en los enfoques que seleccionan una única hipótesis. \\
El balance natural de la evaluación de modelo por integración del espacio de hipótesis (evidencia). \\
%La forma correcta de evaluar modelo. Ejemplo: regresión líneal bayesiana.\\
}
\end{textblock}


\end{frame}
%
% \begin{frame}[plain,noframenumbering]
%
% \begin{textblock}{160}[0,0](0,-47.5)
% \includegraphics[width=1\textwidth]{../../auxiliar/static/raices}
% \end{textblock}
%
% % \begin{textblock}{160}[0,1](0,80)
% % \includegraphics[width=0.19\textwidth]{../../auxiliar/static/aproximacion1}
% % \hspace{-0.1cm}
% % \includegraphics[width=0.19\textwidth]{../../auxiliar/static/aproximacion2}
% % \hspace{-0.1cm}
% % \includegraphics[width=0.19\textwidth]{../../auxiliar/static/aproximacion3}
% % \hspace{-0.1cm}
% % \includegraphics[width=0.19\textwidth]{../../auxiliar/static/aproximacion4}
% % \hspace{-0.1cm}
% % \includegraphics[width=0.19\textwidth]{../../auxiliar/static/aproximacion5}
% % \end{textblock}
%
% \begin{textblock}{160}(107,3)
% \LARGE \textcolor{black!65}{\fontsize{22}{0}\selectfont \textbf{Aproximaciones \\ analíticas}}
% \end{textblock}
%
%
%
% \begin{textblock}{160}(2,2)
% \textcolor{black!80}{Capítulo \unidad \\ \small
% Métodos eficientes de aproximación: \\
% expectation propagation y variational \\
% inference. Ejemplo: estimación de habi-\\ \hspace{0.8cm} lidad en la industria del video juego. \\
% }
% \end{textblock}
%
% \end{frame}



\begin{frame}[plain,noframenumbering]
\begin{textblock}{160}(0,-4.3) \centering
\includegraphics[width=1\textwidth]{../../auxiliar/static/antartic}
\end{textblock}

\begin{textblock}{160}(0,0) \centering
\tikz{
\node[det, fill=black,draw=black] (k) {\textcolor{black}{--------------------------------------------------------------------------------------------------------------------------------------}} ;
}
\end{textblock}

\begin{textblock}{160}(5,0)
\tikz{
\node[det, fill=black,draw=black,text width=0.01cm] (k) {\textcolor{black}{--------------------------------------------------------------------------------------------------------------------------------------}} ;
}
\end{textblock}


\begin{textblock}{160}(0,4) \centering
\LARGE \hspace{1cm} \textcolor{black!20}{\fontsize{22}{0}\selectfont \textbf{Series de tiempo}}
\end{textblock}


\begin{textblock}{55}[0,1](8,70)
\begin{turn}{90}
\parbox{6cm}{\footnotesize
\textcolor{black!10}{Millones de km$^2$ de hielo Antártico}}
\end{turn}
\end{textblock}


\begin{textblock}{160}(20,63)
\textcolor{black!10}{Capítulo \unidad \\ \small
Filtrado y smoothing en series de tiempo.\\
El algoritmo de convergencia por loopy belief propagation. \\
Ejemplos: modelo de estimación de habilidad estado-del-arte, \\
e inferencia causal en series temporales. \\
}
\end{textblock}


\end{frame}
%
% \begin{frame}[plain,noframenumbering]
% \begin{textblock}{160}(-5,0) \centering
% \includegraphics[width=1.05\textwidth]{../../auxiliar/static/pajarosTrayectorias}
% \end{textblock}
% \begin{textblock}{160}(4,20)
% \LARGE \textcolor{black!6}{\fontsize{22}{0}\selectfont \textbf{Aproximaciones}}
% \end{textblock}
% \begin{textblock}{160}(14,27)
% \LARGE \textcolor{black!6}{\fontsize{22}{0}\selectfont \textbf{por exploración}}
% \end{textblock}
%
%
% \begin{textblock}{65}(90,40)
% \textcolor{black!5}{ \hfill Capítulo \unidad \\ \small
% \hfill Métodos de aproximación para \\
% \hfill modelos causales intratables: \\
% \hfill Markov chain Monte Carlo.\\
% \hfill MCMC. HMC. \\
% }
% \end{textblock}
%
%
% \end{frame}
%
% \begin{frame}[plain,noframenumbering]
% \begin{textblock}{160}(0,0) \centering
% \includegraphics[width=1.2\textwidth]{../../auxiliar/static/ppls}
% \end{textblock}
% \begin{textblock}{160}(8,8)
% \LARGE \textcolor{black!15}{\fontsize{22}{0}\selectfont \textbf{Programación \\ probabilistica \\}}
% \end{textblock}
% % \begin{textblock}{160}(94,23)
% % \LARGE \textcolor{black!16}{\fontsize{22}{0}\selectfont \textbf{probabilistica}}
% % \end{textblock}
%
% \begin{textblock}{160}(70,30)
% \normalsize
% \textcolor{black!25}{
% \tikz{
% \node[det, fill=black!60,draw=black!25] (k) {\textcolor{black!5}{$k_i$}} ;
% \node[latent, fill opacity=0, draw=black!25, text opacity=1, above=of k, xshift=-1cm] (p) {\textcolor{black!5}{$p$}};
% \node[det, fill opacity=0, draw=black!25, text opacity=1, above=of k, xshift=1cm] (n) {\textcolor{black!5}{$n$}};
% \edge {p,n} {k};
% \plate[inner sep=0.3cm, xshift=0cm, yshift=0.12cm] {intentos} {(k)} {$i$}
% \node[const, right=of n, xshift=0.3cm] (np) {$p \sim \text{Beta}(1,1)$};
% \node[const, right=of n, xshift=0.3cm, yshift=-1cm] (np) {$n \sim \text{Categorical}(N_\text{max})$};
% \node[const, right=of n, xshift=0.3cm, yshift=-2cm] (np) {$k_i \sim \text{Binomial}(p,n)$};
% }
% }
% \end{textblock}
%
% \begin{textblock}{160}(20,74)
% \textcolor{black!15}{Capítulo \unidad \\ \small
% Implementación de modelos usando lenguajes de programación probabilística. \\
% Verificación visual de buen funcionamiento de las aproximaciones.  \\
% }
% \end{textblock}
%
%
% \end{frame}



\begin{frame}[plain,noframenumbering]

\centering \LARGE
Día 3. Toma de decisiones.

\end{frame}


\begin{frame}[plain,noframenumbering]

% \begin{textblock}{160}(0,0)
% \includegraphics[width=1.18\textwidth]{../../aux/static/fotosintesis}
% \end{textblock}
\begin{textblock}{160}(0,-15)
\includegraphics[width=1\textwidth]{../../auxiliar/static/tsimane}
\end{textblock}


% VERSION 2
\begin{textblock}{160}(6,36)
\LARGE \rotatebox[origin=tr]{18}{\textcolor{black!95}{\fontsize{22}{0}\selectfont \textbf{La función}}}
\end{textblock}
\begin{textblock}{160}(41,32)
\LARGE \rotatebox[origin=tr]{23}{\textcolor{black!95}{\fontsize{22}{0}\selectfont \textbf{de}}}
\end{textblock}
\begin{textblock}{160}(50.5,23)
\LARGE \rotatebox[origin=tr]{28}{\textcolor{black!95}{\fontsize{22}{0}\selectfont \textbf{costo}}}
\end{textblock}
\begin{textblock}{160}(68,5.3)
\LARGE \rotatebox[origin=tr]{26}{\textcolor{black!95}{\fontsize{22}{0}\selectfont \textbf{epistémico}}}
\end{textblock}
\begin{textblock}{160}(104,5.5)
\LARGE \rotatebox[origin=tr]{8}{\textcolor{black!95}{\fontsize{22}{0}\selectfont \textbf{-}}}
\end{textblock}
\begin{textblock}{160}(110,3)
\LARGE \rotatebox[origin=tr]{-14}{\textcolor{black!95}{\fontsize{22}{0}\selectfont \textbf{evolutiva}}}
\end{textblock}


\begin{textblock}{55}[0,0](119,22)
\begin{turn}{-57}
\parbox{7cm}{\sloppy\setlength\parfillskip{0pt}
\textcolor{black!0}{\ \ \ \ \ Capítulo \unidad} \\
\small\textcolor{black!5}{\hspace{-0.15cm} Apuestas óptimas.} \\
\small\textcolor{black!5}{\hspace{-0.85cm} Ventajas a favor de la:} \\
\small\textcolor{black!5}{\hspace{-1.45cm} Diversificación (propiedad epistémica)}\\
\small\textcolor{black!5}{\hspace{-1.7cm} Cooperación (propiedad evolutiva)}\\
\small\textcolor{black!5}{ \hspace{-1.75cm}Especialización (propiedad de especiación)} \\
\small\textcolor{black!5}{\hspace{-2cm} Heterogeniedad (propiedad ecológica).\\ }}
\end{turn}
\end{textblock}


\end{frame}


\begin{frame}[plain,noframenumbering]
% \begin{textblock}{160}(0,-80)  \centering
% \includegraphics[width=1\textwidth]{../../aux/static/galton_box}
% \end{textblock}

\begin{textblock}{160}(0,11)  \centering
\includegraphics[width=0.40\textwidth]{../../auxiliar/static/treeOfLife-liviano}
\end{textblock}

\begin{textblock}{160}(0,3) \centering
\LARGE \textcolor{black!85}{\rotatebox[origin=tr]{0}{\fontsize{22}{0}\selectfont \textbf{Apuestas de vida}}}
\end{textblock}
% % <
% \begin{textblock}{160}(0,3) \centering
% \begin{tikzpicture}
%   \node (Start) at (2.8,0) {};
%   \node (End) at (-2.8,0) {};
%   \draw [decorate,decoration={text along path,text align=center,text={sssssssssssssssssssssssssssssssssss|\bf\fontsize{22}{22}\selectfont|Distribuciones de creencias},text color=black!85 }] (End) to [bend left=45] (Start);
% \end{tikzpicture}
% \end{textblock}

%  \begin{textblock}{160}(0,3) \centering
% \tikz{
% \node[factor, xshift=-3cm, opacity=0] (a) {} ;
% \node[factor, xshift=3cm, opacity=0] (b) {} ;
% \path[draw, -, fill=black!50,sloped,draw opacity=0] (a) edge[bend left=45,draw=black!50] node[color=black!75] {\scriptsize  \texttt{lhood\_lose\_tb}} (b);
% }
% \end{textblock}


\begin{textblock}{55}(75,39)
\textcolor{black!85}{\normalsize El árbol de la vida \\
\fontsize{2}{0}\selectfont Synthesis of phylogeny and taxonomy into a comprehensive tree of life \\}
\end{textblock}


\begin{textblock}{55}(3,81)
\textcolor{black!85}{Capítulo \unidad}
\end{textblock}

\begin{textblock}{55}(25,81.3)
\begin{turn}{0}
\parbox{15cm}{\small \textcolor{black!85}{Presentación de una competencia de inferecia con apuestas e intercambio de recursos.}
}
\end{turn}
\end{textblock}

\end{frame}



\begin{frame}[plain,noframenumbering]
\begin{textblock}{160}(0,0)  \centering
\includegraphics[width=1\textwidth]{../../auxiliar/static/reciprocidad}
\end{textblock}

\begin{textblock}{160}(0,3) \centering
\LARGE \textcolor{black!15}{\rotatebox[origin=tr]{0}{\fontsize{22}{0}\selectfont \textbf{Tecnologías de reciprocidad}}}
\end{textblock}

\begin{textblock}{55}(122,82)
\begin{turn}{-5}
\textcolor{black!5}{Capítulo \unidad}
\end{turn}
\end{textblock}

\begin{textblock}{55}(119,85)
\begin{turn}{-5}
\small \textcolor{black!5}{Tiempo para colaborar}
\end{turn}
\end{textblock}

\end{frame}

%
% \begin{frame}[plain,noframenumbering]
%
% \begin{textblock}{96}(0,6.5)\centering
% {\transparent{0.9}\includegraphics[width=0.8\textwidth]{../../auxiliar/static/inti.png}}
% \end{textblock}
%
% \begin{textblock}{160}(96,5.5)
% \includegraphics[width=0.35\textwidth]{../../auxiliar/static/pachacuteckoricancha}
% \end{textblock}
%
% \end{frame}


\end{document}
