\documentclass[10pt]{article}
\input{../auxiliar/tex/encabezado.tex}
\input{../auxiliar/tex/tikzlibrarybayesnet.code.tex}

\newif\ifen
\newif\ifes
\newcommand{\en}[1]{\ifen#1\fi}
\newcommand{\es}[1]{\ifes#1\fi}
\entrue

\title{\huge Inferencia Bayesiana para la construcción de acuerdos intersubjetivos en contextos de incertidumbre  \\  \LARGE Programa}

\author{Gustavo Landfried$^{1,2}$}
\affil{\small 1. Bayes de las Provincias Unidas del Sur }
\affil{\vspace{-0.2cm}\small 2. Laboratorio Pacha Pampas}
\affil[]{Correspondencia: \url{bayesdelsur@gmail.com}}

\begin{document}

\maketitle

\section{Objetivos}

A diferencia de las ciencias formales (matemáticas) que validan sus proposiciones dentro de sistemas axiomáticos cerrados, las ciencias empíricas (desde la física hasta las ciencias sociales) deben validar sus proposiciones en sistemas abiertos que por definición contienen siempre algún grado de incertidumbre. ¿Podemos alcanzar “verdades” si es inevitable decir “no sé”?. Sí. La aplicación estricta de la teoría de la probabilidad (inferencia Bayesiana) garantiza los acuerdos intersubjetivos en contextos de incertidumbre, fundamento de las verdades empíricas. Sin embargo, su adopción se vio históricamente limitada debido al alto costo computacional asociado. A diferencia del enfoque frecuentista de la probabilidad que selecciona una única hipótesis, la inferencia Bayesiana se ve obligada a actualizar las creencias de cada una de las hipótesis de forma óptima (maximizando la incertidumbre) dada la evidencia empírica y formal (datos y modelos). Si bien en las últimas décadas las limitaciones computacionales han sido superadas en gran medida gracias al desarrollo de métodos eficientes de aproximación, la inercia histórica es ahora su limitación principal. Este curso tiene por objetivo promover la adopción de la inferencia Bayesiana como método general para la construcción de acuerdos intersubjetivos en contextos de incertidumbre a través del estudio de sus conceptos fundamentales (principios y propiedades de los procesos de selección probabilística) y a través de la resolución de problemas concretos (modelos gráficos, inferencia causal, lenguajes de programación probabilística). La inferencia Bayesiana no solo ha mostrado ser la lógica más exitosa en la era de la inteligencia artificial, es un criterio general para la resolución de cualquier problema empírico de forma óptima, sea en la industria, la academia o el poder judicial.

\section{Unidades}

\begin{enumerate}

\item \textbf{Principios interculturales de acuerdos intersubjetivos}
\vspace{-0.15cm}
\begin{description}
\item[Teórica:] Principios de razón suficiente, integridad, indiferencia y coherencia. Las reglas de la probabilidad. El teorema de Bayes. Interpretación de la verosimilitud y la evidencia. Selección de modelos causales alternativos. 
\item[Práctica:] Juegos de apuestas en grupos. 
\end{description}

\vspace{0.1cm}
\item \textbf{Función de costo epistémico-evolutiva}
\vspace{-0.15cm}
\begin{description}
\item[Teórica:] Naturaleza multiplicativa de los procesos de selección probabilística y evolutiva. Consecuencias de las fluctuaciones en la tasa de crecimiento. Las distribución de biomasa en la tierra. La ventaja a favor de las variantes que reducen las fluctuaciones: diversificación individual (propiedad epistémica), cooperación (propiedad evolutiva mayor), especialización (propiedad meta-epistémica), coexistencia (propiedad ecológica).
\item[Práctica:] Análisis de las apuestas. Polya Urn. 
\end{description}

% Parrafo

\vspace{0.1cm}
\item \textbf{Máxima incertidumbre}
\vspace{-0.15cm}
\begin{description}
\item[Teórica:] Entopía. Distribuciones de probabilidad exponencial. Distribuciones conjugadas. Beta-Binomial. Dirichlet-multinomial.
\item[Práctica:] Probabilidad de la probabilidad. Gases. Distribución de la riqueza. Variables pseudo-aleatorias.
\end{description}

% Parrafo

\vspace{0.1cm}
\item \textbf{Modelos gráficos y pasaje de mensajes}
\vspace{-0.15cm}
\begin{description}
\item[Teórica:] Variables observadas y ocultas. Modelos digidos y no dirigidos. Gráficos de factores. Algoritmo suma-producto. 
\item[Práctica:] Flujo de inferencia (\emph{d-separation}) en el Modelo Alarma-Terremoto
\end{description}

% Parrafo

\vspace{0.1cm}
\item \textbf{Inferencia causal}
\vspace{-0.15cm}
\begin{description}
\item[Teórica:] Intervenciones sobre modelos causales (\emph{do-operator}) y su correlato en modelos gráficos. Criterios para sacar conclusiones causales de datos observacionales. Contrafácticos. 
\item[Práctica:] Buenos y malos controles.
\end{description}

% Parrafo

\vspace{0.1cm}
\item \textbf{Selección de modelo}
\vspace{-0.15cm}
\begin{description}
\item[Teórica:] Modelos gaussianos multidimensionales. Regresión lineal bayesiana. El enfoque frecuentista y el sobreajuste \emph{overfitting}. Evidencia. Media 
\item[Práctica:] Buenos y malos controles.
\end{description}

% Parrafo

\vspace{0.1cm}
\item \textbf{Procesos Gaussianos}
\vspace{-0.15cm}
\begin{description}
\item[Teórica:]
\item[Práctica:]
\end{description}

% Parrafo

\vspace{0.1cm}
\item \textbf{Aproximaciones por cadenas de Markov}
\vspace{-0.15cm}
\begin{description}
\item[Teórica:]
\item[Práctica:]
\end{description}

% Parrafo

\vspace{0.1cm}
\item \textbf{Aproximaciones Variacionales}
\vspace{-0.15cm}
\begin{description}
\item[Teórica:]
\item[Práctica:]
\end{description}

% Parrafo

\vspace{0.1cm}
\item \textbf{Base empírica metodológica}
\vspace{-0.15cm}
\begin{description}
\item[Teórica:]
\item[Práctica:]
\end{description}

\end{enumerate}
% 
% 
% |0 | Procesos multiplicativos | 23/03/2022 (revisión 06/04/2022) | Peters [Cooperación](https://researchers.one/articles/19.03.00004) [ergodicidad](https://www.nature.com/articles/s41567-019-0732-0) |[Presentación](https://github.com/BayesDeLasProvinciasUnidasDelSur/curso/releases/download/2022.1/teorica0.pdf) | Apuestas | 13/04/2022  | [Kelly 1956](https://www.princeton.edu/~wbialek/rome/refs/kelly_56.pdf) |[Soluciones](https://github.com/BayesDeLasProvinciasUnidasDelSur/curso/releases/download/2022.1/practica0.pdf) |
% |1 | Inferencia Bayesiana | 30/03/2022 | [Bayesian methods: General background](http://citeseerx.ist.psu.edu/viewdoc/download;jsessionid=E3CAC8BC04D114B9FA346D29DF78A692?doi=10.1.1.41.1055&rep=rep1&type=pdf) | [Presentación](https://github.com/BayesDeLasProvinciasUnidasDelSur/curso/releases/download/2022.1/teorica1.pdf) | Selección de Modelo | Suspendida | | |
% |2 | Máxima entropía | Suspendida | Cap 1 Bishop - Física estadísitica [Desde](https://www.youtube.com/watch?v=vdSWMIh2o_E&t=0s) [Hasta](https://www.youtube.com/watch?v=37kRnZxJImA)  | |  Gases, distribución de la riqueza (combinatoria) | Suspendida | Ejemplos en [Desde](https://www.youtube.com/watch?v=vdSWMIh2o_E&t=0s) - [Hasta](https://www.youtube.com/watch?v=37kRnZxJImA) | |
% | 3 | Algoritmo Suma-Producto | 20/04/2022 | Cap 8 Bishop | [Presentación](https://github.com/BayesDeLasProvinciasUnidasDelSur/curso/releases/download/2022.1/teorica-sumprod.pdf) | D-separation en el Modelo Alarma-Terremoto | 27/04/2022 | Cap 8 Bishop | |
% | 4 | Inferencia Causal | 04/05/2022 | Pearl [A premier](http://gen.lib.rus.ec/)  | | [Buenos y malos controles](https://papers.ssrn.com/sol3/Delivery.cfm/SSRN_ID4062645_code4146131.pdf?abstractid=3689437&mirid=1) | 11/05/2022 | | |
% | 5 | Distribuciones continuas conjugadas | - | Cap 2 Bishop | | Regresión lineal | - | Cap 3 Bishop | |
% | 6 | Cadenas de Markov MCMC, HMC| - | ¿Cap 11 Bishop? | | Ejemplos varios | - | [Stan](https://mc-stan.org/docs/2_29/stan-users-guide-2_29.pdf) | | 
% | 7 | Aproximaciones Variacionales | - | Cap 10 Bishop y [Deep Bayes](https://www.youtube.com/watch?v=xH1mBw3tb_c&list=PLe5rNUydzV9QHe8VDStpU0o8Yp63OecdW&index=4&t=0s) | | Filtro de Kalman (TrueSkill exacto y aproximado) |  - | [TrueSkill](https://papers.nips.cc/paper/3079-trueskilltm-a-bayesian-skill-rating-system) | |
% | 8 | Series temporales | - | Cap 13 Bishop | | Smoothing de Kalman (TrueSkill Through Time) | - | [TrueSkill Through Time](https://papers.nips.cc/paper/3331-trueskill-through-time-revisiting-the-history-of-chess) | |
% | 9 | Procesos Gaussianos | - | Cap 6 Bishop | | ¿KickScore? |  | [Gaussain process with pytroch](https://gpytorch.ai/) ||
% | - | - | -| - | - | - | - | - | - |
% | ¿? | ¿Estructura invariante del dato de base empírica metodológica? | ¿? | [Juan Samaja Parte 3](https://ens9004-infd.mendoza.edu.ar/sitio/upload/12-%20SAMAJA,%20J.%20-%20LIBRO%20-%20Epistemologia%20y%20metodologia.pdf) - [Klimovsky Base empírica]() | | ¿? | ¿? | | |
% 


{\scriptsize
\bibliographystyle{../auxiliar/biblio/plos2015}
\bibliography{../auxiliar/biblio/biblio_notUrl.bib}
}

\end{document}
