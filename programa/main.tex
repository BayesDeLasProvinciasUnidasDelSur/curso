\documentclass[10pt]{article}
\input{../auxiliar/tex/encabezado.tex}
\input{../auxiliar/tex/tikzlibrarybayesnet.code.tex}

\newif\ifen
\newif\ifes
\newcommand{\en}[1]{\ifen#1\fi}
\newcommand{\es}[1]{\ifes#1\fi}
\entrue

\title{\huge Inferencia bayesiana.  \\  \LARGE Programa}

\author{Gustavo Landfried$^{1,2}$}
\affil{\small 1. Universidad de Buenos Aires. Facultad de Ciencias Exactas y Naturales. Departamento de Computaci\'on. Buenos Aires, Argentina}
\affil{\small 2. Bayes de las Provincias Unidas del Sur.}
\affil[]{Correspondencia: \url{bayesdelsur@gmail.com}}

\begin{document}

\maketitle

\begin{enumerate}
\item Coexistencia

\begin{enumerate}
\item Intro

La ciencia oficial (Grecia $\rightarrow$ Europa $\rightarrow$ Inteligencia Artificial).
La vida como fuente de conocimiento del conocimiento empírico.
Las transiciones evolutivas mayores.

\item Conocimiento empírico. Cooperaciones. 

Distirbución de la biomassa.
Fotosíntesis, fuente de la energia vital.
La plantas cooperando a través de los gases de atmósfera.
Cooperación plantas-hongos.
La necesidad de humildad de la ciencia (ningún laboratorio tiene el conocimiento empírico para desarrollar fotosíntesis).
La cooperación entre plantas-humanos (agricultura).
El éxito de la agricultura como cooperación plantas-humanos.
Cosmovisiones de reciprocidad con la madre tierra, y sus lógicas paraconsistentes.

\item Conocimiento cultural. Cooperaciones.

La crianza cooperativa y la transicón cultural.
El éxito de la cooperación cultural (por crianza cooperativa): transición de conocimiento cultural y adaptación a todos los nichos ecológicos.
El éxito de la cooperación cultural (por cooperación con plantas): aumento de la población y la acelaración de las innovaciones tecnológicas y científicas.
Ejemplos, China, Arabes, Polinesia, Incas. Intercambios entre ellos.
El \textbf{problema de la reciprocidad pascal-fermat}: ¿Cuál es el valor justo de la reciprocidad en contextos de incertidumbre?

\item Rupturas de reciprocidad: pérdida de diversidad cultural y biodiversidad.

Tasmania, el ejemplo del aislamiento y la pérdida de innovaciones culturales.
La involución cultural de la edad media como consecuencia de la perdida de diversidad cultural previa, durante el imerior Romano.
El giro colonial-moderno (migración fedual, pandemia americana y la plata como moneda oficial China), como acceso a nueva diversidad cultural y época de innovaciones culturales.
Las guerras Chinas contra el narco-estado británico (guerras del opio), comienzo de la centralidad europa y la era de los genocidios y pérdida de diversidad cultural.
La pérdida de biodiversidad como consecuencia de la pérdida de las insitituciones culturales locales bien adaptadas a los nichos ecológicos.

\item La ventaja evolutiva de la cooperación y la especialización.

¿Por qué la ruptura del principio de la reciprocidad tiene como consecuencia la pérdida de conocimiento empírico?
El crecimiento de los linajes como procesos multiplicativos.
La media aritmética, cuando la población tiende a infinito.
La media geométrica, cuando el tiempo tiene a infinito.
La no-ergocidad de los procesos multiplicativos.
El aumento de la tasa de crecimiento a través de la cooperación.
Las ecuaciones de la nueva variable aleatoria ``fondo común''.
El efecto negativo que produce la ruptura de la cooperación para aquellos que intentan sacar provecho.
La ventaja de la especialización una vez emerge la cooperación.

\item \textbf{Princpio de coexistencia}

Las casas de apuestas ofrecen pagos a cada uno de las hipótesis.
Los pagos actualizan nuestra riqueza siguiendo un proceo multiplicativo.
Hemos visto que en los procesos multiplicativos existe una ventaja a favor de la cooperación y la especialización.
Supongamos que la casa de apuestas paga $Q_C$ por Cara y $Q_S$ por Secas.
Supongamos que se conoce la probabilidad exacta de la moneda $p$, la cual está sesgada a favor de Cara, $p>0.5$.
¿Existe un conjunto de pagos que garantice la coexistencia en el tiempo entre la casa de apuestas y una población de apostadoras?
¿Es decir, no será que una población cooperadora siempre puede romper cualquier pago que ofrezca la casa de apuestas a través de la especialización, poniendo más recursos en Cara que en Seca de los que indica la probabilidad?

\end{enumerate}

\item Principios.

\begin{enumerate}

\item Verdades formales y empíricas.

¿Cuales son las verdades empíricas?

\item Principio de reciprocidad

\item Principio de razón suficiente

\item Principio de indiferencia

\item Principio de coherencia.

Ejemplo de las cajas y la pista sin incluir elección de puerta. En este caso, hay una tentación de volver a aplicar el principio de indiferencia después de ver el dato. Y en este caso particular los valores coincidirían. Pero no es así en general. Después vamos a ver un caso donde no ocurre (Monty Hall) 

\item Las probabilidad bayesiana como garantía de los acuerdos intersubjetivos, verdades empíricas.

\item Las reglas de la probabilidad

Esto es todo. No hay nada más. De ahora en adelante nos vamos a dedicar a definir modelos causales y a resolver la inferencia usando las reglas de la probabilidad.

\item Primer ejemplo. Monty Hall.

Computo del posterior aplicando los 4 principios.

\item El teorema de Bayes (corolario de Laplace).

Definición de la verosimilitud.
Interpretación de la verosimilitud.
El uso de la verosimilitud para derivar el teorema de bayes.
Los modelos detrás del teorema de bayes.
La interpretación de la evidencia.

\item Selección de modelos

Probabilidad de los modelos dado los datos.
Código: generación de datos con el modelo causal monty hall
Código: computo de la probabilidad de los modelos alternativos vistos hoy
Bayes Factor, con mismo prior se reduce a evidencias.
La interpretación cooperativa de la evidencia: la evidencia escrita como producto, la evidencia escrita como suma.





\end{enumerate}



\end{enumerate}


{\scriptsize
\bibliographystyle{../auxiliar/biblio/plos2015}
\bibliography{../auxiliar/biblio/biblio_notUrl.bib}
}

\end{document}
