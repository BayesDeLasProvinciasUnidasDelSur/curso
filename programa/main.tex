\documentclass[10pt]{article}
\usepackage[spanish]{babel}
\usepackage{fullpage}
\usepackage{cite}
\usepackage[utf8]{inputenc}
\usepackage{a4wide}
\usepackage{url}
\usepackage{graphicx}
\usepackage{caption}
\usepackage{float} % para que los gr\'aficos se queden en su lugar con [H]
\usepackage{subcaption}
\usepackage{wrapfig}
\usepackage{color}
\usepackage{amsmath} %para escribir funci\'on partida , matrices
\usepackage{amsthm} %para numerar definciones y teoremas
\usepackage[hidelinks]{hyperref} % para inlcuir links dentro del texto
\usepackage{tabu} 
\usepackage{comment}
\usepackage{amsfonts} % \mathbb{N} -> conjunto de los n\'umeros naturales  
\usepackage{enumerate}
\usepackage{listings}
\usepackage[colorinlistoftodos, textsize=small]{todonotes} % Para poner notas en el medio del texto!! No olvidar hacer. 
\usepackage{framed} % Para encuadrar texto. \begin{framed}
\usepackage{csquotes} % Para citar texto \begin{displayquote}
\usepackage{epigraph} % Epigrafe  \epigraph{texto}{\textit{autor}}
\usepackage{authblk}
\usepackage{titlesec}
\usepackage{varioref}
\usepackage{bm} % \bm{\alpha} bold greek symbol
\usepackage{pdfpages} % \includepdf
\usepackage[makeroom]{cancel} % \cancel{} \bcancel{} etc
\usepackage{wrapfig} % \begin{wrapfigure} Pone figura al lado del texto
\usepackage{tikz}
\usepackage{algorithm}


\usepackage{paracol}

\newcommand{\citel}[1]{\cite{#1}\label{#1}}
\newcommand\hfrac[2]{\genfrac{}{}{0pt}{}{#1}{#2}} %\frac{}{} sin la linea del medio

\theoremstyle{definition}
\newtheorem{definition}{Definition}[section]
\newtheorem{theorem}{Theorem}[section]
\newtheorem{proposition}{Proposition}[section]

%http://latexcolor.com/
\definecolor{azul}{rgb}{0.36, 0.54, 0.66}
\definecolor{rojo}{rgb}{0.7, 0.2, 0.116}
\definecolor{rojopiso}{rgb}{0.8, 0.25, 0.17}
\definecolor{verdeingles}{rgb}{0.12, 0.5, 0.17}
\definecolor{ubuntu}{rgb}{0.44, 0.16, 0.39}
\definecolor{debian}{rgb}{0.84, 0.04, 0.33}

\definecolor{dkgreen}{rgb}{0,0.6,0}
\definecolor{gray}{rgb}{0.5,0.5,0.5}
\definecolor{mauve}{rgb}{0.58,0,0.82}

\lstset{
  language=Python,
  aboveskip=3mm,
  belowskip=3mm,
  showstringspaces=true,
  columns=flexible,
  basicstyle={\small\ttfamily},
  numbers=none,
  numberstyle=\tiny\color{gray},
  keywordstyle=\color{blue},
  commentstyle=\color{dkgreen},
  stringstyle=\color{mauve},
  breaklines=true,
  breakatwhitespace=true,
  tabsize=4
}

% tikzlibrary.code.tex
%
% Copyright 2010-2011 by Laura Dietz
% Copyright 2012 by Jaakko Luttinen
%
% This file may be distributed and/or modified
%
% 1. under the LaTeX Project Public License and/or
% 2. under the GNU General Public License.
%
% See the files LICENSE_LPPL and LICENSE_GPL for more details.

% Load other libraries
\usetikzlibrary{shapes}
\usetikzlibrary{fit}
\usetikzlibrary{chains}
\usetikzlibrary{arrows}

% Latent node
\tikzstyle{latent} = [circle,fill=white,draw=black,inner sep=1pt,
minimum size=20pt, font=\fontsize{10}{10}\selectfont, node distance=1]
% Observed node
\tikzstyle{obs} = [latent,fill=gray!25]
% Invisible node
\tikzstyle{invisible} = [latent,minimum size=0pt,color=white, opacity=0, node distance=0]
% Constant node
\tikzstyle{const} = [rectangle, inner sep=0pt, node distance=0.1]
%state
\tikzstyle{estado} = [latent,minimum size=8pt,node distance=0.4]
%action
\tikzstyle{accion} =[latent,circle,minimum size=5pt,fill=black,node distance=0.4]


% Factor node
\tikzstyle{factor} = [rectangle, fill=black,minimum size=10pt, draw=black, inner
sep=0pt, node distance=1]
% Deterministic node
\tikzstyle{det} = [latent, rectangle]

% Plate node
\tikzstyle{plate} = [draw, rectangle, rounded corners, fit=#1]
% Invisible wrapper node
\tikzstyle{wrap} = [inner sep=0pt, fit=#1]
% Gate
\tikzstyle{gate} = [draw, rectangle, dashed, fit=#1]

% Caption node
\tikzstyle{caption} = [font=\footnotesize, node distance=0] %
\tikzstyle{plate caption} = [caption, node distance=0, inner sep=0pt,
below left=5pt and 0pt of #1.south east] %
\tikzstyle{factor caption} = [caption] %
\tikzstyle{every label} += [caption] %

\tikzset{>={triangle 45}}

%\pgfdeclarelayer{b}
%\pgfdeclarelayer{f}
%\pgfsetlayers{b,main,f}

% \factoredge [options] {inputs} {factors} {outputs}
\newcommand{\factoredge}[4][]{ %
  % Connect all nodes #2 to all nodes #4 via all factors #3.
  \foreach \f in {#3} { %
    \foreach \x in {#2} { %
      \path (\x) edge[-,#1] (\f) ; %
      %\draw[-,#1] (\x) edge[-] (\f) ; %
    } ;
    \foreach \y in {#4} { %
      \path (\f) edge[->,#1] (\y) ; %
      %\draw[->,#1] (\f) -- (\y) ; %
    } ;
  } ;
}

% \edge [options] {inputs} {outputs}
\newcommand{\edge}[3][]{ %
  % Connect all nodes #2 to all nodes #3.
  \foreach \x in {#2} { %
    \foreach \y in {#3} { %
      \path (\x) edge [->,#1] (\y) ;%
      %\draw[->,#1] (\x) -- (\y) ;%
    } ;
  } ;
}

% \factor [options] {name} {caption} {inputs} {outputs}
\newcommand{\factor}[5][]{ %
  % Draw the factor node. Use alias to allow empty names.
  \node[factor, label={[name=#2-caption]#3}, name=#2, #1,
  alias=#2-alias] {} ; %
  % Connect all inputs to outputs via this factor
  \factoredge {#4} {#2-alias} {#5} ; %
}

% \plate [options] {name} {fitlist} {caption}
\newcommand{\plate}[4][]{ %
  \node[wrap=#3] (#2-wrap) {}; %
  \node[plate caption=#2-wrap] (#2-caption) {#4}; %
  \node[plate=(#2-wrap)(#2-caption), #1] (#2) {}; %
}

% \gate [options] {name} {fitlist} {inputs}
\newcommand{\gate}[4][]{ %
  \node[gate=#3, name=#2, #1, alias=#2-alias] {}; %
  \foreach \x in {#4} { %
    \draw [-*,thick] (\x) -- (#2-alias); %
  } ;%
}

% \vgate {name} {fitlist-left} {caption-left} {fitlist-right}
% {caption-right} {inputs}
\newcommand{\vgate}[6]{ %
  % Wrap the left and right parts
  \node[wrap=#2] (#1-left) {}; %
  \node[wrap=#4] (#1-right) {}; %
  % Draw the gate
  \node[gate=(#1-left)(#1-right)] (#1) {}; %
  % Add captions
  \node[caption, below left=of #1.north ] (#1-left-caption)
  {#3}; %
  \node[caption, below right=of #1.north ] (#1-right-caption)
  {#5}; %
  % Draw middle separation
  \draw [-, dashed] (#1.north) -- (#1.south); %
  % Draw inputs
  \foreach \x in {#6} { %
    \draw [-*,thick] (\x) -- (#1); %
  } ;%
}

% \hgate {name} {fitlist-top} {caption-top} {fitlist-bottom}
% {caption-bottom} {inputs}
\newcommand{\hgate}[6]{ %
  % Wrap the left and right parts
  \node[wrap=#2] (#1-top) {}; %
  \node[wrap=#4] (#1-bottom) {}; %
  % Draw the gate
  \node[gate=(#1-top)(#1-bottom)] (#1) {}; %
  % Add captions
  \node[caption, above right=of #1.west ] (#1-top-caption)
  {#3}; %
  \node[caption, below right=of #1.west ] (#1-bottom-caption)
  {#5}; %
  % Draw middle separation
  \draw [-, dashed] (#1.west) -- (#1.east); %
  % Draw inputs
  \foreach \x in {#6} { %
    \draw [-*,thick] (\x) -- (#1); %
  } ;%
}



\newif\ifen
\newif\ifes
\newcommand{\en}[1]{\ifen#1\fi}
\newcommand{\es}[1]{\ifes#1\fi}
\entrue

\title{\huge Inferencia bayesiana.  \\  \LARGE Programa}

\author{Gustavo Landfried$^{1,2}$}
\affil{\small 1. Bayes de las Provincias Unidas del Sur }
\affil{\vspace{-0.2cm}\small 2. Laboratorio Pacha Pampas}
\affil[]{Correspondencia: \url{bayesdelsur@gmail.com}}

\begin{document}

\maketitle

\begin{enumerate}

\item \textbf{Principios interculturales de acuerdos intersubjetivos}
\vspace{-0.15cm}
\begin{description}
\item[Teórica:] Principios de razón suficiente, integridad, indiferencia y coherencia. Las reglas de la probabilidad. El teorema de Bayes. Interpretación de la verosimilitud y la evidencia. Selección de modelos causales alternativos. 
\item[Práctica:] Juegos de apuestas en grupos. 
\end{description}

\vspace{0.1cm}
\item \textbf{Función de costo epistémico-evolutiva}
\vspace{-0.15cm}
\begin{description}
\item[Teórica:] Naturaleza multiplicativa de los procesos de selección probabilística y evolutiva. Consecuencias de las fluctuaciones en la tasa de crecimiento. Las distribución de biomasa en la tierra. La ventaja a favor de las variantes que reducen las fluctuaciones: diversificación individual (propiedad epistémica), cooperación (propiedad evolutiva mayor), especialización (propiedad meta-epistémica), coexistencia (propiedad ecológica).
\item[Práctica:] Análisis de las apuestas. Polya Urn. 
\end{description}

% Parrafo

\vspace{0.1cm}
\item \textbf{Máxima incertidumbre}
\vspace{-0.15cm}
\begin{description}
\item[Teórica:] Entopía. Distribuciones de probabilidad exponencial. Distribuciones conjugadas. 
\item[Práctica:] Probabilidad de la probabilidad. Gases. Distribución de la riqueza. Variables pseudo-aleatorias.
\end{description}

% Parrafo

\vspace{0.1cm}
\item \textbf{Modelos gráficos y pasaje de mensajes}
\vspace{-0.15cm}
\begin{description}
\item[Teórica:] Variables observadas y ocultas. Modelos digidos y no dirigidos. Gráficos de factores. Algoritmo suma-producto. 
\item[Práctica:] Flujo de inferencia (\emph{d-separation}) en el Modelo Alarma-Terremoto
\end{description}

% Parrafo

\vspace{0.1cm}
\item \textbf{Inferencia causal}
\vspace{-0.15cm}
\begin{description}
\item[Teórica:] Intervenciones sobre modelos causales (\emph{do-operator}) y su correlato en modelos gráficos. Criterios para sacar conclusiones causales de datos observacionales. Contrafácticos. 
\item[Práctica:] Buenos y malos controles.
\end{description}

% Parrafo

\vspace{0.1cm}
\item \textbf{Selección de modelo}
\vspace{-0.15cm}
\begin{description}
\item[Teórica:] Modelos gaussianos multidimensionales. Regresión lineal bayesiana. El enfoque frecuentista y el sobreajuste \emph{overfitting}. Evidencia. Media 
\item[Práctica:] Buenos y malos controles.
\end{description}

% Parrafo

\vspace{0.1cm}
\item \textbf{Procesos Gaussianos}
\vspace{-0.15cm}
\begin{description}
\item[Teórica:]
\item[Práctica:]
\end{description}

% Parrafo

\vspace{0.1cm}
\item \textbf{Aproximaciones por cadenas de Markov}
\vspace{-0.15cm}
\begin{description}
\item[Teórica:]
\item[Práctica:]
\end{description}

% Parrafo

\vspace{0.1cm}
\item \textbf{Aproximaciones Variacionales}
\vspace{-0.15cm}
\begin{description}
\item[Teórica:]
\item[Práctica:]
\end{description}

% Parrafo

\vspace{0.1cm}
\item \textbf{Base empírica metodológica}
\vspace{-0.15cm}
\begin{description}
\item[Teórica:]
\item[Práctica:]
\end{description}

\end{enumerate}
% 
% 
% |0 | Procesos multiplicativos | 23/03/2022 (revisión 06/04/2022) | Peters [Cooperación](https://researchers.one/articles/19.03.00004) [ergodicidad](https://www.nature.com/articles/s41567-019-0732-0) |[Presentación](https://github.com/BayesDeLasProvinciasUnidasDelSur/curso/releases/download/2022.1/teorica0.pdf) | Apuestas | 13/04/2022  | [Kelly 1956](https://www.princeton.edu/~wbialek/rome/refs/kelly_56.pdf) |[Soluciones](https://github.com/BayesDeLasProvinciasUnidasDelSur/curso/releases/download/2022.1/practica0.pdf) |
% |1 | Inferencia Bayesiana | 30/03/2022 | [Bayesian methods: General background](http://citeseerx.ist.psu.edu/viewdoc/download;jsessionid=E3CAC8BC04D114B9FA346D29DF78A692?doi=10.1.1.41.1055&rep=rep1&type=pdf) | [Presentación](https://github.com/BayesDeLasProvinciasUnidasDelSur/curso/releases/download/2022.1/teorica1.pdf) | Selección de Modelo | Suspendida | | |
% |2 | Máxima entropía | Suspendida | Cap 1 Bishop - Física estadísitica [Desde](https://www.youtube.com/watch?v=vdSWMIh2o_E&t=0s) [Hasta](https://www.youtube.com/watch?v=37kRnZxJImA)  | |  Gases, distribución de la riqueza (combinatoria) | Suspendida | Ejemplos en [Desde](https://www.youtube.com/watch?v=vdSWMIh2o_E&t=0s) - [Hasta](https://www.youtube.com/watch?v=37kRnZxJImA) | |
% | 3 | Algoritmo Suma-Producto | 20/04/2022 | Cap 8 Bishop | [Presentación](https://github.com/BayesDeLasProvinciasUnidasDelSur/curso/releases/download/2022.1/teorica-sumprod.pdf) | D-separation en el Modelo Alarma-Terremoto | 27/04/2022 | Cap 8 Bishop | |
% | 4 | Inferencia Causal | 04/05/2022 | Pearl [A premier](http://gen.lib.rus.ec/)  | | [Buenos y malos controles](https://papers.ssrn.com/sol3/Delivery.cfm/SSRN_ID4062645_code4146131.pdf?abstractid=3689437&mirid=1) | 11/05/2022 | | |
% | 5 | Distribuciones continuas conjugadas | - | Cap 2 Bishop | | Regresión lineal | - | Cap 3 Bishop | |
% | 6 | Cadenas de Markov MCMC, HMC| - | ¿Cap 11 Bishop? | | Ejemplos varios | - | [Stan](https://mc-stan.org/docs/2_29/stan-users-guide-2_29.pdf) | | 
% | 7 | Aproximaciones Variacionales | - | Cap 10 Bishop y [Deep Bayes](https://www.youtube.com/watch?v=xH1mBw3tb_c&list=PLe5rNUydzV9QHe8VDStpU0o8Yp63OecdW&index=4&t=0s) | | Filtro de Kalman (TrueSkill exacto y aproximado) |  - | [TrueSkill](https://papers.nips.cc/paper/3079-trueskilltm-a-bayesian-skill-rating-system) | |
% | 8 | Series temporales | - | Cap 13 Bishop | | Smoothing de Kalman (TrueSkill Through Time) | - | [TrueSkill Through Time](https://papers.nips.cc/paper/3331-trueskill-through-time-revisiting-the-history-of-chess) | |
% | 9 | Procesos Gaussianos | - | Cap 6 Bishop | | ¿KickScore? |  | [Gaussain process with pytroch](https://gpytorch.ai/) ||
% | - | - | -| - | - | - | - | - | - |
% | ¿? | ¿Estructura invariante del dato de base empírica metodológica? | ¿? | [Juan Samaja Parte 3](https://ens9004-infd.mendoza.edu.ar/sitio/upload/12-%20SAMAJA,%20J.%20-%20LIBRO%20-%20Epistemologia%20y%20metodologia.pdf) - [Klimovsky Base empírica]() | | ¿? | ¿? | | |
% 


{\scriptsize
\bibliographystyle{../auxiliar/biblio/plos2015}
\bibliography{../auxiliar/biblio/biblio_notUrl.bib}
}

\end{document}
