\documentclass[10pt]{article}
\usepackage[spanish]{babel}
\usepackage{fullpage}
\usepackage{cite}
\usepackage[utf8]{inputenc}
\usepackage{a4wide}
\usepackage{url}
\usepackage{graphicx}
\usepackage{caption}
\usepackage{float} % para que los gr\'aficos se queden en su lugar con [H]
\usepackage{subcaption}
\usepackage{wrapfig}
\usepackage{color}
\usepackage{amsmath} %para escribir funci\'on partida , matrices
\usepackage{amsthm} %para numerar definciones y teoremas
\usepackage[hidelinks]{hyperref} % para inlcuir links dentro del texto
\usepackage{tabu} 
\usepackage{comment}
\usepackage{amsfonts} % \mathbb{N} -> conjunto de los n\'umeros naturales  
\usepackage{enumerate}
\usepackage{listings}
\usepackage[colorinlistoftodos, textsize=small]{todonotes} % Para poner notas en el medio del texto!! No olvidar hacer. 
\usepackage{framed} % Para encuadrar texto. \begin{framed}
\usepackage{csquotes} % Para citar texto \begin{displayquote}
\usepackage{epigraph} % Epigrafe  \epigraph{texto}{\textit{autor}}
\usepackage{authblk}
\usepackage{titlesec}
\usepackage{varioref}
\usepackage{bm} % \bm{\alpha} bold greek symbol
\usepackage{pdfpages} % \includepdf
\usepackage[makeroom]{cancel} % \cancel{} \bcancel{} etc
\usepackage{wrapfig} % \begin{wrapfigure} Pone figura al lado del texto
\usepackage{tikz}
\usepackage{algorithm}


\usepackage{paracol}

\newcommand{\citel}[1]{\cite{#1}\label{#1}}
\newcommand\hfrac[2]{\genfrac{}{}{0pt}{}{#1}{#2}} %\frac{}{} sin la linea del medio

\theoremstyle{definition}
\newtheorem{definition}{Definition}[section]
\newtheorem{theorem}{Theorem}[section]
\newtheorem{proposition}{Proposition}[section]

%http://latexcolor.com/
\definecolor{azul}{rgb}{0.36, 0.54, 0.66}
\definecolor{rojo}{rgb}{0.7, 0.2, 0.116}
\definecolor{rojopiso}{rgb}{0.8, 0.25, 0.17}
\definecolor{verdeingles}{rgb}{0.12, 0.5, 0.17}
\definecolor{ubuntu}{rgb}{0.44, 0.16, 0.39}
\definecolor{debian}{rgb}{0.84, 0.04, 0.33}

\definecolor{dkgreen}{rgb}{0,0.6,0}
\definecolor{gray}{rgb}{0.5,0.5,0.5}
\definecolor{mauve}{rgb}{0.58,0,0.82}

\lstset{
  language=Python,
  aboveskip=3mm,
  belowskip=3mm,
  showstringspaces=true,
  columns=flexible,
  basicstyle={\small\ttfamily},
  numbers=none,
  numberstyle=\tiny\color{gray},
  keywordstyle=\color{blue},
  commentstyle=\color{dkgreen},
  stringstyle=\color{mauve},
  breaklines=true,
  breakatwhitespace=true,
  tabsize=4
}

% tikzlibrary.code.tex
%
% Copyright 2010-2011 by Laura Dietz
% Copyright 2012 by Jaakko Luttinen
%
% This file may be distributed and/or modified
%
% 1. under the LaTeX Project Public License and/or
% 2. under the GNU General Public License.
%
% See the files LICENSE_LPPL and LICENSE_GPL for more details.

% Load other libraries
\usetikzlibrary{shapes}
\usetikzlibrary{fit}
\usetikzlibrary{chains}
\usetikzlibrary{arrows}

% Latent node
\tikzstyle{latent} = [circle,fill=white,draw=black,inner sep=1pt,
minimum size=20pt, font=\fontsize{10}{10}\selectfont, node distance=1]
% Observed node
\tikzstyle{obs} = [latent,fill=gray!25]
% Invisible node
\tikzstyle{invisible} = [latent,minimum size=0pt,color=white, opacity=0, node distance=0]
% Constant node
\tikzstyle{const} = [rectangle, inner sep=0pt, node distance=0.1]
%state
\tikzstyle{estado} = [latent,minimum size=8pt,node distance=0.4]
%action
\tikzstyle{accion} =[latent,circle,minimum size=5pt,fill=black,node distance=0.4]


% Factor node
\tikzstyle{factor} = [rectangle, fill=black,minimum size=10pt, draw=black, inner
sep=0pt, node distance=1]
% Deterministic node
\tikzstyle{det} = [latent, rectangle]

% Plate node
\tikzstyle{plate} = [draw, rectangle, rounded corners, fit=#1]
% Invisible wrapper node
\tikzstyle{wrap} = [inner sep=0pt, fit=#1]
% Gate
\tikzstyle{gate} = [draw, rectangle, dashed, fit=#1]

% Caption node
\tikzstyle{caption} = [font=\footnotesize, node distance=0] %
\tikzstyle{plate caption} = [caption, node distance=0, inner sep=0pt,
below left=5pt and 0pt of #1.south east] %
\tikzstyle{factor caption} = [caption] %
\tikzstyle{every label} += [caption] %

\tikzset{>={triangle 45}}

%\pgfdeclarelayer{b}
%\pgfdeclarelayer{f}
%\pgfsetlayers{b,main,f}

% \factoredge [options] {inputs} {factors} {outputs}
\newcommand{\factoredge}[4][]{ %
  % Connect all nodes #2 to all nodes #4 via all factors #3.
  \foreach \f in {#3} { %
    \foreach \x in {#2} { %
      \path (\x) edge[-,#1] (\f) ; %
      %\draw[-,#1] (\x) edge[-] (\f) ; %
    } ;
    \foreach \y in {#4} { %
      \path (\f) edge[->,#1] (\y) ; %
      %\draw[->,#1] (\f) -- (\y) ; %
    } ;
  } ;
}

% \edge [options] {inputs} {outputs}
\newcommand{\edge}[3][]{ %
  % Connect all nodes #2 to all nodes #3.
  \foreach \x in {#2} { %
    \foreach \y in {#3} { %
      \path (\x) edge [->,#1] (\y) ;%
      %\draw[->,#1] (\x) -- (\y) ;%
    } ;
  } ;
}

% \factor [options] {name} {caption} {inputs} {outputs}
\newcommand{\factor}[5][]{ %
  % Draw the factor node. Use alias to allow empty names.
  \node[factor, label={[name=#2-caption]#3}, name=#2, #1,
  alias=#2-alias] {} ; %
  % Connect all inputs to outputs via this factor
  \factoredge {#4} {#2-alias} {#5} ; %
}

% \plate [options] {name} {fitlist} {caption}
\newcommand{\plate}[4][]{ %
  \node[wrap=#3] (#2-wrap) {}; %
  \node[plate caption=#2-wrap] (#2-caption) {#4}; %
  \node[plate=(#2-wrap)(#2-caption), #1] (#2) {}; %
}

% \gate [options] {name} {fitlist} {inputs}
\newcommand{\gate}[4][]{ %
  \node[gate=#3, name=#2, #1, alias=#2-alias] {}; %
  \foreach \x in {#4} { %
    \draw [-*,thick] (\x) -- (#2-alias); %
  } ;%
}

% \vgate {name} {fitlist-left} {caption-left} {fitlist-right}
% {caption-right} {inputs}
\newcommand{\vgate}[6]{ %
  % Wrap the left and right parts
  \node[wrap=#2] (#1-left) {}; %
  \node[wrap=#4] (#1-right) {}; %
  % Draw the gate
  \node[gate=(#1-left)(#1-right)] (#1) {}; %
  % Add captions
  \node[caption, below left=of #1.north ] (#1-left-caption)
  {#3}; %
  \node[caption, below right=of #1.north ] (#1-right-caption)
  {#5}; %
  % Draw middle separation
  \draw [-, dashed] (#1.north) -- (#1.south); %
  % Draw inputs
  \foreach \x in {#6} { %
    \draw [-*,thick] (\x) -- (#1); %
  } ;%
}

% \hgate {name} {fitlist-top} {caption-top} {fitlist-bottom}
% {caption-bottom} {inputs}
\newcommand{\hgate}[6]{ %
  % Wrap the left and right parts
  \node[wrap=#2] (#1-top) {}; %
  \node[wrap=#4] (#1-bottom) {}; %
  % Draw the gate
  \node[gate=(#1-top)(#1-bottom)] (#1) {}; %
  % Add captions
  \node[caption, above right=of #1.west ] (#1-top-caption)
  {#3}; %
  \node[caption, below right=of #1.west ] (#1-bottom-caption)
  {#5}; %
  % Draw middle separation
  \draw [-, dashed] (#1.west) -- (#1.east); %
  % Draw inputs
  \foreach \x in {#6} { %
    \draw [-*,thick] (\x) -- (#1); %
  } ;%
}



\newif\ifen
\newif\ifes
\newcommand{\en}[1]{\ifen#1\fi}
\newcommand{\es}[1]{\ifes#1\fi}
\entrue

\title{\huge Inferencia Bayesiana: construcción de acuerdos intersubjetivos en contextos de incertidumbre  \\[0.4cm]  \LARGE Programa}

\author{Docente a cargo: Gustavo Landfried$^{1,2}$}
\affil{\small 1. Bayes de las Provincias Unidas del Sur }
\affil{\vspace{-0.2cm}\small 2. Laboratorio Pacha Pampas}
\affil[]{Correspondencia: \texttt{glandfried@dc.uba.ar}, \texttt{bayesdelsur@gmail.com}}

\begin{document}

\maketitle

\section{Objetivos}

A diferencia de las ciencias formales (matemáticas) que validan sus proposiciones dentro de sistemas axiomáticos cerrados, las ciencias empíricas (desde la física hasta las ciencias sociales) deben validar sus proposiciones en sistemas abiertos que por definición contienen siempre algún grado de incertidumbre. ¿Es posible alcanzar ``verdades'' si es inevitable decir ``no sé''?. Sí. La aplicación estricta de la teoría de la probabilidad (inferencia Bayesiana) garantiza los acuerdos intersubjetivos en contextos de incertidumbre, fundamento de las verdades empíricas. Sin embargo, su adopción se vio históricamente limitada debido al alto costo computacional asociado. A diferencia del enfoque frecuentista de la probabilidad que selecciona una única hipótesis, la inferencia Bayesiana se ve obligada a actualizar las creencias de cada una de las hipótesis de forma óptima (maximizando la incertidumbre) dada la evidencia empírica y formal (datos y modelos). Si bien en las últimas décadas las limitaciones computacionales han sido superadas en gran medida gracias al desarrollo de métodos eficientes de aproximación, la inercia histórica es ahora su limitación principal. Este curso tiene por objetivo promover la adopción de la inferencia Bayesiana como método general para la construcción de acuerdos intersubjetivos en contextos de incertidumbre a través del estudio de sus conceptos fundamentales (principios y propiedades de los procesos de selección probabilística) y a través de la resolución de problemas concretos (modelos gráficos, inferencia causal, lenguajes de programación probabilística). La inferencia Bayesiana no solo ha mostrado ser la lógica más exitosa en la era de la inteligencia artificial, es un criterio general para la resolución de cualquier problema empírico especialmente en las industrias 4.0, la academia y el poder judicial.

\section{Unidades}

\begin{enumerate}

\item \textbf{Principios interculturales de acuerdos intersubjetivos}
\vspace{-0.15cm}
\begin{description}
\item[Teórica:] Principios de razón suficiente, integridad, indiferencia y coherencia. Las reglas de la probabilidad. El teorema de Bayes. Interpretación de la verosimilitud y la evidencia. Introducción a la selección de modelos causales alternativos.
\item[Práctica:] Juegos de apuestas en grupos. Implementación del juego de apuestas en un lenguaje de programación.
\end{description}

\vspace{0.1cm}
\item \textbf{Función de costo epistémico-evolutiva}
\vspace{-0.15cm}
\begin{description}
\item[Teórica:] Naturaleza multiplicativa de los procesos de selección probabilística y evolutiva. Consecuencias de las fluctuaciones en la tasa de crecimiento. La ventaja a favor de las variantes que reducen las fluctuaciones: diversificación individual (propiedad epistémica), cooperación (propiedad evolutiva mayor), especialización (propiedad meta-epistémica), coexistencia (propiedad ecológica). Las distribución de biomasa en la tierra.
\item[Práctica:] Simulación de un proceso multiplicativo, aproximación numérica del promedio de los estados y del promedio temporal. Análisis matemáticas de las apuestas.
\end{description}

% Parrafo

\vspace{0.1cm}
\item \textbf{Máxima incertidumbre}
\vspace{-0.15cm}
\begin{description}
\item[Teórica:] Divergencia KL. Entopía. Distribuciones de probabilidad exponencial. Distribuciones conjugadas. Beta-Binomial. Dirichlet-multinomial. Polya Urn.
\item[Práctica:] Probabilidad de la probabilidad. Gases. Distribución de la riqueza. Polya Urn. Variables pseudo-aleatorias.
\end{description}

% Parrafo

\vspace{0.1cm}
\item \textbf{Modelos gráficos y algoritmo de inferencia}
\vspace{-0.15cm}
\begin{description}
\item[Teórica:] Variables observadas y ocultas. Modelos digidos y no dirigidos. Gráficos de factores. Algoritmo suma-producto. 
\item[Práctica:] Flujo de inferencia (\emph{d-separation}) en el Modelo Alarma-Terremoto.
\end{description}

% Parrafo

\vspace{0.1cm}
\item \textbf{Inferencia causal}
\vspace{-0.15cm}
\begin{description}
\item[Teórica:] Intervenciones sobre modelos causales (\emph{do-operator}) y su correlato en modelos gráficos. Criterios para sacar conclusiones causales de datos observacionales. Contrafácticos. 
\item[Práctica:] Buenos y malos controles.
\end{description}

% Parrafo

\vspace{0.1cm}
\item \textbf{Selección de modelo}
\vspace{-0.15cm}
\begin{description}
\item[Teórica:] Modelos gaussianos multidimensionales. Regresión lineal bayesiana. El enfoque frecuentista y el sobreajuste \emph{overfitting}. Evidencia. Media 
\item[Práctica:] Buenos y malos controles.
\end{description}

% Parrafo

\vspace{0.1cm}
\item \textbf{Aproximaciones Variacionales}
\vspace{-0.15cm}
\begin{description}
\item[Teórica:] Modelos de estimación de habilidad de la industria del video juego. Modelo Elo. Modelo TrueSkill. Aproximación por \emph{expectation propagation}. Modelo bayesiano de recomendación de Netflix. Aproximación por \emph{variational inference}.
\item[Práctica:] Implementación del Modelo Elo y el modelo TrueSkill. Estimación de habilidad y comparación.
\end{description}

% Parrafo

\vspace{0.1cm}
\item \textbf{Datos secuenciales y series de tiempo}
\vspace{-0.15cm}
\begin{description}
\item[Teórica:] El problema de usar el posterior como prior del siguiente evento en el modelo de estimación de habilidad. Los modelos gráficos de historia completa. El algoritmos de pasaje de mensajes iterativos y su convergencia.
\item[Práctica:] Implementación de modelos secuenciales sencillos. Verificación del problema. Implementación de su solución.
\end{description}

% Parrafo

\vspace{0.1cm}
\item \textbf{Aproximaciones por cadenas de Markov}
\vspace{-0.15cm}
\begin{description}
\item[Teórica:] Algoritmos de muestreo básicos. Métodos de aceptación y rechazo. Importance sampling. Cadenas de markov. El algoritmo Metrópolis-Hasting. Gibbs Sampling.
\item[Práctica:] Implementación del algoritmo Metrópolis-Hasting
\end{description}

% Parrafo

\vspace{0.1cm}
\item \textbf{Lenguajes de programación probabilística}
\vspace{-0.15cm}
\begin{description}
\item[Teórica:] Implementación de modelos usando PPLs. Verificaciones de buen funcionamiento. 
\item[Práctica:] Implementación de varios modelos usando PPLs.
\end{description}

% Parrafo

\vspace{0.1cm}
\item \textbf{Base empírica metodológica}
\vspace{-0.15cm}
\begin{description}
\item[Teórica:] Concepto de base empírica. La estructura invariante del dato. El lugar de las hipótesis en el dato. La hipótesis indicadora universal. Procedimiento de determinación. Datos T-teóricos y datos de base empírica.
\item[Práctica:] Debate. Operacionalización de funciones proposicionales. El problema del acuerdo y su solución.


\end{description}

\end{enumerate}

\nocite{jaynes1984-bayesianBackground, mcelreath2020-rethinking, bishop2006-PRML, pearl2009-causality, cinelli2021-crashCourse, stan-userGuide, martin2021-BMCP, samaja1999-epistemologiaMetodologia }

{\bibliographystyle{../auxiliar/biblio/plos2015}
\bibliography{../auxiliar/biblio/biblio_notUrl.bib}
}

\end{document}
