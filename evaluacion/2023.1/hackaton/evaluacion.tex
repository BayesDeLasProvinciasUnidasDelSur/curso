\newif\ifen
\newif\ifes
\newif\iffr
\newcommand{\en}[1]{\ifen#1 \fi}
\newcommand{\es}[1]{\ifes#1 \fi}
\newcommand{\fr}[1]{\iffr#1 \fi}
\newcommand{\En}[1]{\ifen#1\fi}
\newcommand{\Es}[1]{\ifes#1\fi}
\estrue
\documentclass[a4paper,10pt]{article}
\en{\usepackage[english]{babel}}
\es{\usepackage[spanish,es-noquoting]{babel}}

\usepackage[utf8]{inputenc}
\input{../../../auxiliar/tex/encabezado.tex}
\input{../../../auxiliar/tex/tikzlibrarybayesnet.code.tex}
\usepackage{amsmath} %para escribir funci\'on partida , matrices, hfrac
%
\usepackage{textpos}
\setlength{\TPHorizModule}{1mm}
\setlength{\TPVertModule}{1mm}

\captionsetup[figure]{labelsep=period} % xq Figura termine con punto,
\newcommand{\gray}{\color{black!65}}

\usepackage{siunitx}
\usepackage{fullpage}


%opening

\title{\vspace{-2cm}
\en{Competition: ``Bets of life''}\es{Competencia ``Apuestas de vida''}\\}
\author{
\includegraphics[width=0.46\linewidth]{../../../auxiliar/static/BP.png} \\ \large
%\En{Bayesian methods labs}\Es{Laboratorios de Métodos Bayesianos}
}
%\date{\en{January 30, 2023}\es{30 de enero del 2023}}
\date{}
\begin{document}
%
\maketitle

\section{\En{Dates}\Es{Fechas}}

\begin{description} \setlength\itemsep{0cm}
\item[\En{Beginning of registrations}\Es{Inicio de inscripciones}:]
\en{June 21st (Andean New Year)}%
\es{21 de Junio (Año Nuevo Andino)}%
\item[\En{End of registrations and beginning of submissions}\Es{Fin de inscripciones e inicio de entregas}:]
\en{August 1st (Pachamama Day)}%
\es{1 de Agosto (Día de la Pachamama)}%
\item[\En{End of submissions}\Es{Fin de entregas}:]
\en{October 12th (Cultural Diversity Day)}%
\es{12 de Octubre (Día de la diversidad cultural)}%
\item[\En{Prize}\Es{Premio}:]
\en{November 4th (Unity of the Latin American Peoples Day)}%
\es{4 de Noviembre (Día de la Unidad de los Pueblos Latinoamericanos)}%
\end{description}

\section{\En{Competition}\Es{Competencia}}


\en{The game is a problem of inference with bets and exchange of resources.}%
\es{El juego es un problema de inferencia con apuestas e intercambio de recursos.}%
%
\en{Registration and submission will be done through a telegram bot \texttt{MetodosBayesianosBot}.}%
\es{La inscripción y la entrega se realizará a través del bot de telegram \texttt{@MetodosBayesianosBot}.}%

\paragraph{\En{Prize}\Es{Premio}.}

\en{Each individual $i$ start with one resource unit, $\omega_i = 1$.}%
\es{Cada persona $i$ comienzan con una unidad de recursos, $\omega_i = 1$.}%
\en{A prize, equivalent to \num{150000} Argentine pesos, will be distributed according to the proportion of resources that each person has over the total at the end of the betting process, $\omega_i / \sum_i \omega_i$.}%
%\footnote{The payment of the prize is made by international transfer to an IBAN account in one of the following currencies: ARG - USD - EUR - CHF - GBP - JPY - AUD - CAD - SEK - HKD - NOK - DKK - AED - SGD. The exchange between these currencies is free of charge. A fixed cost of 4 dollars plus the additional cost included by the banks involved in the transfer must be deducted from the transfer.}.
\es{Un premio, equivalente a \num{150000} pesos argentinos, se repartirá siguiendo la proporción de recursos que cada persona tienen sobre el total al finalizar el proceso de apuestas, $\omega_i / \sum_i \omega_i$.}%
%\footnote{El pago del premio se realiza por transferencia internacional a una cuenta IBAN en alguna de las siguientes monedas: ARG - USD - EUR - CHF - GBP - JPY - AUD - CAD - SEK - HKD - NOK - DKK - AED - SGD. El cambio entre estas monedas se realiza sin costo. A la transferencia se le debe descontar un costo fijo de 4 dólares más el costo adicional que incluyan los bancos intervinientes en la transferencia.}.}%
%
\en{The award does not become effective if no individual achieves growth rate greater than 11\% per time step.}%
\es{El premio no se entrega si ninguna persona logra una tasa de crecimiento mayor al 11\% por paso temporal.}%(1.7\% para grupos cooperativ´os de 3 personas)


\paragraph{\En{Registration: inference}\Es{Inscripción: inferencia}.}

\en{The problem is a four-gate Monty Hall, in which the gift position is generated with a bias that has a cycle of 365 time steps (See details in section~\ref{sec:inferencia}).}%
\es{El problema es un Monty Hall de cuatro puertas, en el que la posición del regalo se genera con un sesgo que tiene un ciclo de 365 pasos temporales (Ver detalles en sección~\ref{sec:inferencia}).}%
%
\en{When registering each person will receive an array of length 2190 representing the position of the gifts in consecutive time steps.}%
\es{Al inscribirse, cada persona recibirá un array de longitud 2190 que representa la posición de los regalos en pasos temporales consecutivos.}%
%
\en{The objective of the inference is to estimate the position of the gift.}%
\es{El objetivo de la inferencia es estimar la posición del regalo.}%
%
%\en{Before receiving it, individuals need to choose the starting point within the cycle at which they want the array they receive to begin (a number between 1 and 365).}%
%\es{Antes de recibirlo, las personas deben elegir el punto de inicio del ciclo en el que quieren que empiece el array que reciben (un número entre 1 y 365).}%
%
%\en{An individual identifier will be created at this stage.}%
%\es{Un identificador individual se creará en esta etapa.}%

\paragraph{\En{Submission: interventions, bets and reciprocity}\Es{Entrega: intervenciones, apuestas y reciprocidad}.}

\en{The goal is to maximize the growth rate of their own resources betting on the position of the gift over 1095 time steps.}%
\es{El objetivo es maximizar la tasa de crecimiento de sus propios recursos apostando sobre la posición del regalo en los siguientes 1095 pasos temporales.}%
%
\en{Mother nature offers a payoff $q=2.75$ for each hypothesis $h$, ``the gift is located in the $h$ position''.}%
\es{La madre naturaleza ofrece un pago $q=2.75$ por cada hipótesis $h$, ``el regalo se encuentra en la posición $h$''.}%
%
\en{At each time step individuals are obliged to bet all their resources, distributing $b_h$ proportions among the $h$ hypotheses, such that $\sum_h b_h = 1$.}%
\es{En cada paso temporal las personas están obligadas a apostar todos sus recursos, distribuyendo proporciones $b_h$ entre las hipótesis $h$, tal que $\sum_h b_h = 1$.}%
%
\en{If at time step $t$ the hypothesis $h$ is true and $b_h$ is the proportion of the resources bet on that hypothesis, then the resources are updated as $\omega_{t+1} = \omega_t b_h q$.}%
\es{Si en el paso temporal $t$ la hipótesis $h$ es verdadera y $b_h$ es la proporción de los recursos apostados a esa hipótesis, entonces los recursos se actualiza como $\omega_{t+1} = \omega_t b_h q$.}%
%
\en{Following Monty Hall's idea, before betting the person must intervene by choosing a position in order to receive as a clue a different position that does not have the gift (see details in section~\ref{sec:inferencia}).}%
\es{Siguiendo la idea del Monty Hall, antes de apostar la persona deben intervenir eligiendo una posición para recibir como pista una posición distinta que no tiene el regalo (Ver detalles en sección~\ref{sec:inferencia}).}%
%
\en{In addition, between time steps, individuals can give and receive resources.}%
\es{Además, entre pasos temporales, las personas podrán dar y recibir recursos.}%
%
\en{The details of the submission are detailed in the section~\ref{sec:apuestas}.}
\es{Los detalles de la entega se detallan en la sección~\ref{sec:apuestas}.}

\section{\En{Inference problem}\Es{Problema de inferencia}} \label{sec:inferencia}

\en{Monty Hall is one of the most popular probabilistic games.}%
\es{El Monty Hall es uno de los juegos probabilísticos más conocidos.}%
%
\en{In the original problem there are three doors.}%
\es{En el problema original hay tres puertas.}%
%
\en{Behind one of them, a gift is hidden.}%
\es{Detrás de una de ellas se esconde un regalo.}%
%
\en{The individual wins the gift if they choose the correct door.}%
\es{La persona se lleva el regalo si elije la puerta correcta.}%
%
\en{The interesting thing is that once the person chooses the door, someone who knows where the gift is opens a different door that has nothing in it.}%
\es{Lo interesante es que una vez que la personas elije la puerta, alguien que conoce donde está el regalo abre una puerta distinta que no tiene nada.}%
%
\en{This information can be used to update the previous belief about the position of the gift.}%
\es{Esta información puede ser usada para actualizar la creencia previa sobre la posición del regalo.}%
%
\en{To do this, it is necessary to understand the causal model that generates the hint.}%
\es{Para ello es necesario comprender el modelo causal que genera la pista.}%

% Parrafo

\en{The figure~\ref{fig:causal model} shows (at top) the question we want to answer and (at bottom) the causal model with which we are going to answer it.}%
\es{En la figura~\ref{fig:modelo_causal} se muestran (arriba) la pregunta que queremos responder y (abajo) el modelo causal con el que la vamos a responder.}%
%
\en{What is the position of the gift after we have chosen door 1 and we were shown that there is nothing behind door 2?}%
\es{¿Cuál es la posición del regalo luego de haber elegido la puerta 1 y que nos mostraron que no hay nada detrás de la puerta 2?}%
%
\en{The causal model ensures that the hint $s$ cannot point to the chosen door $c$ (with the lock) nor to the door where the gift $r$ is located (the hidden hypothesis): $s\neq c$ and $s \neq r$.}%
\es{El modelo causal asegura que la pista $s$ no puede señalar la puerta elegida $c$ (con la cerradura) ni la puerta donde se encuentra el regalo $r$ (la hipótesis oculta): $s\neq c$ y $s\neq r$.}%
%
\en{If we divide the belief equally at each bifurcation of the possible parallel universes given the causal model and our choice (figure~\ref{fig:caminos_montyhall_compatibles}), we get a joint prior belief about the position of the gift and the hint (figura~\ref{fig:f3} at top).}%
\es{Si dividimos la creencia en partes iguales por cada bifurcación de los universos paralelos dado el modelo causal y la elección (figura~\ref{fig:caminos_montyhall_compatibles}), obtenemos una creencia conjunta a priori sobre la posición del regalo y la pista (figura~\ref{fig:f3} arriba).}%
%
\en{The new belief (figure~\ref{fig:f3} below) is nothing more than the initial joint belief (table) that is still compatible with the data $s_2$ (dark row), expressed as 100\%.}%
\es{La nueva creencia (figura~\ref{fig:f3} abajo) no es más que la creencia inicial conjunta (tabla) que sigue siendo compatible con el dato $s_2$ (renglón oscuro), expresada como 100\%}%


\begin{figure}[ht!]
 \centering
  \begin{subfigure}[b]{0.3\textwidth}
  \centering
  \tikz{

    \node[latent] (d) {\includegraphics[width=0.10\textwidth]{../../../auxiliar/static/dedo.png}} ;
    \node[const,below=of d] (nd) {\En{Hint}\Es{Pista}: $s \neq r$, $s \neq c$  } ;

    \node[latent, above=of d,yshift=-1.2cm, xshift=-1.3cm] (r) {\includegraphics[width=0.12\textwidth]{../../../auxiliar/static/regalo.png}} ;
    \node[const,above=of r] (nr) {\phantom{g}\En{Gift}\Es{Regalo}: $r$\phantom{g}} ;

    \node[latent, fill=black!30, above=of d,yshift=-1.2cm, xshift=1.3cm] (c) {\includegraphics[width=0.12\textwidth]{../../../auxiliar/static/cerradura.png}} ;
    \node[const,above=of c] (nc) {\phantom{g}\En{Election}\Es{Elección}: $c_1$\phantom{g}} ;

    \edge {r,c} {d};

         \node[factor, minimum size=0.8cm, xshift=-1.3cm, yshift=2.5cm] (p1) {\includegraphics[width=0.075\textwidth]{../../../auxiliar/static/cerradura.png}} ;
         \node[det, minimum size=0.8cm, yshift=2.5cm] (p2) {\includegraphics[width=0.085\textwidth]{../../../auxiliar/static/dedo.png}} ;
         \node[factor, minimum size=0.8cm, xshift=1.3cm, yshift=2.5cm] (p3) {} ;
         \node[const, above=of p1, yshift=.05cm] (fp1) {$?$};
         \node[const, above=of p2, yshift=.05cm] (fp2) {$?$};
         \node[const, above=of p3, yshift=.05cm] (fp3) {$?$};
         \node[const, below=of p2, yshift=-.10cm, xshift=0.3cm] (dedo) {};

  }
  \caption{}
  \label{fig:modelo_causal}
  \end{subfigure}
 \begin{subfigure}[b]{0.32\textwidth}
\centering
\tikz{
\node[latent, draw=white, yshift=0.7cm] (b0) {$1$};
\node[latent,below=of b0,yshift=0.9cm, xshift=-1.5cm] (r1) {$r_1$};
{\color{black}\node[latent,draw=black,below=of b0,yshift=0.9cm] (r2) {$r_2$}; }
\node[latent,below=of b0,yshift=0.9cm, xshift=1.5cm] (r3) {$r_3$};
\node[latent, below=of r1, draw=white, yshift=0.7cm] (bc11) {$\frac{1}{3}$};
{\color{black}\node[latent, below=of r2, draw=white, yshift=0.7cm] (bc12) {$\frac{1}{3}$};}
\node[latent, below=of r3, draw=white, yshift=0.7cm] (bc13) {$\frac{1}{3}$};
\node[latent,below=of bc11,yshift=0.9cm, xshift=-0.5cm] (r1d2) {$s_2$};
{\color{black}\node[latent,draw=black,below=of bc11,yshift=0.9cm, xshift=0.5cm] (r1d3) {$s_3$};}
{\color{black}\node[latent, draw=black,below=of bc12,yshift=0.9cm] (r2d3) {$s_3$};}
\node[latent,below=of bc13,yshift=0.9cm] (r3d2) {$s_2$};

\node[latent,below=of r1d2,yshift=0.9cm,draw=white] (br1d2) {$\frac{1}{3}\frac{1}{2}$};
{\color{black}\node[latent,below=of r1d3,yshift=0.9cm, draw=white] (br1d3) {$\frac{1}{3}\frac{1}{2}$};}s
{\color{black}\node[latent,below=of r2d3,yshift=0.9cm,draw=white] (br2d3) {$\frac{1}{3}$};}
\node[latent,below=of r3d2,yshift=0.9cm,draw=white] (br3d2) {$\frac{1}{3}$};
\edge[-] {b0} {r1,r3};
\edge[-,draw=black] {b0} {r2};
\edge[-] {r1} {bc11};
\edge[-,draw=black] {r2} {bc12};
\edge[-] {r3} {bc13};
\edge[-] {bc11} {r1d2};
\edge[-,draw=black] {bc11} {r1d3};
\edge[-,draw=black] {bc12} {r2d3};
\edge[-] {bc13} {r3d2};
\edge[-] {r1d2} {br1d2};
\edge[-,draw=black] {r1d3} {br1d3};
\edge[-,draw=black] {r2d3} {br2d3};
\edge[-] {r3d2} {br3d2};
}
\caption{}
\label{fig:caminos_montyhall_compatibles}
\end{subfigure}
\
\begin{subfigure}[b]{0.34\textwidth}
\phantom{$P(s_j)$\hspace{0.7cm}g}$P(r_i, s_j)$\phantom{g}\hspace{0.9cm}$P(s_j)$
\centering
  \begin{tabular}{|c|c|c|c||c|} \hline  \setlength\tabcolsep{0.4cm}
 & \, $r_1$ \, &  \, $r_2$ \, & \, $r_3$ \, &  \\ \hline
  $\gray s_1$ & $\gray0$ & $\gray0$ & $\gray0$ &   $\gray 0$ \\ \hline
  $\bm{s_2}$ & $\bm{1/6}$ & $\bm{0}$ & $\bm{1/3}$ &  $\bm{1/2}$ \\  \hline
  $\gray s_3$ & $\gray1/6$ & $\gray1/3$ & $\gray0$ & $\gray1/2$ \\ \hline
  \end{tabular}

  \phantom{--}\\[0.1cm]

  \tikz{

         \node[factor, minimum size=0.8cm, xshift=-1.3cm] (p1) {\includegraphics[width=0.075\textwidth]{../../../auxiliar/static/cerradura.png}} ;
         \node[det, minimum size=0.8cm] (p2) {\includegraphics[width=0.085\textwidth]{../../../auxiliar/static/dedo.png}} ;
         \node[factor, minimum size=0.8cm, xshift=1.3cm] (p3) {} ;
         \node[const, below=of p1, yshift=-.025cm] (fp1) {$1/3$};
         \node[const, below=of p2, yshift=-.025cm] (fp2) {$0$};
         \node[const, below=of p3, yshift=-.025cm] (fp3) {$2/3$};
         \node[const, above=of p2, yshift=.045cm] (title) {$P(r_i | s_2)=P(r_i,s_2)/P(s_2)$};

         \node[const, below=of p2, yshift=-.10cm, xshift=0.3cm] (dedo) {};

  }

\caption{}
\label{fig:f3}
\end{subfigure}
\caption{
\en{Original Monty Hall problem}%
\es{El problema Monty Hall original}%
}
\label{fig:monty_hall}
\end{figure}

\en{The inference problem for this competition is a four-gate extended Monty Hall, in which the gift position is generated with a bias that has a cycle of 365 time steps.}%
\es{El problema de inferencia para esta competencia es una Monty Hall extendido, de cuatro puertas, en el que la posición del regalo se genera con un sesgo que tiene un ciclo de 365 pasos temporales.}%
%
\en{The first goal of inference is to use the data provided during registration, which contains the position of the gifts in 2190 consecutive time steps, to estimate the bias with which the gifts are hidden over time.}%
\es{El primer objetivo de la inferencia es usar los datos que se entregan durante la inscripción, que contiene la posición de los regalos en 2190 pasos temporales consecutivos, para estimar el sesgo con el que se esconden los regalos en el tiempo.}%
%
\en{The second objective is to determine, at each time step, which box should be chosen initially in order to receive the hint.}%
\es{El segundo objetivo es determinar, en cada paso temporal, qué caja conviene elegir inicialmente para recibir la pista.}%
%
\en{The third objective is to pre-compute the belief about the position of the gift for each of the different possible hints.}%
\es{El tercer objetivo es pre-computar la creencia sobre la posición del regalo para cada una de las diferentes posibles pistas.}%
%
\en{This information will be crucial for the betting process that is carried out with the submission of the \texttt{csv} files through the telegram bot \texttt{MetodosBayesianosBot}.  }%
\es{Esta información va a ser clave para el proceso de apuestas que se realiza con la entrega de los archivos \texttt{csv} a través del bot de telegram \texttt{MetodosBayesianosBot}.}%


\section{\En{Submission: interventions, bets and reciprocity}\Es{Entrega: intervenciones, apuestas y reciprocidad}.} \label{sec:apuestas}

\en{The submission consists of two \texttt{csv} files.}%
\es{La entrega consiste de dos archivos \texttt{csv}.}%
%
\en{In addition, you will be asked to choose at which point in the cycle you want to start (a number between 1 and 365).}%
\es{Además, se le pedirá que elija en qué punto del ciclo quieren comenzar (un número entre 1 y 365).}%
%
\en{The files are,}%
\es{Los archivos son,}%
%
\vspace{-0.1cm}
\begin{description}\setlength\itemsep{-0.05cm}
\item[\texttt{apuestas-id.csv}:]
\en{For each time step, the choice of the door and the bets that would be placed for each possible hint (details in subsection).}
\es{Para cada paso temporal, la elección de la puerta y las apuestas que se realizaría para cada posible pista (detalles en subsección).}%
\item[\texttt{cooperacion-id.csv}:]
\en{Resource sharing policy with other members of the competition (details in subsection).}%
\es{La política de intercambio de recursos con otros miembros de la competencia (detalles en subsección).}%
\end{description}
%
\en{If the identifier received during registration is the number 9999, then the files should be named \texttt{apuestas-9999.csv} and \texttt{cooperacion-9999.csv}.}%
\es{Si el identificador recibido durante la inscripción es el número 9999, entonces los archivos deben llamarse \texttt{apuestas-9999.csv} y \texttt{cooperacion-9999.csv}.}%


\subsection{\texttt{apuestas-id.csv}}

\en{The choice of the door and the potential bets that would be placed at each time step must be structured in a \texttt{csv} with 17 columns and 1095 rows.}%
\es{La elección de la puerta y las potenciales apuestas que se realizaría en cada paso temporal debe estar estructurada en un \texttt{csv} con 17 columnas y 1095 filas.}%
%
\en{The rows represent the time steps, up to 3 periods of length 365 (the 1095 rows in total).}%
\es{Las filas representan los pasos temporales, hasta completar 3 períodos de longitud 365 (las 1095 filas en total).}%
%
\en{Column 0 represents the selected door at each time step.}%
\es{La columna 0 representa la puerta elegida en cada paso temporal.}%
%
\en{The remaining 16 columns are divided into 4 blocks representing 4 mutually exclusive cases (the door indicated by the hint).}%
\es{Las 16 columnas restantes se dividen en 4 bloques que representan 4 casos mutuamente excluyentes (la puerta que indicada por la pista).}%
%
\en{The first block, columns 1 to 4, contains the bets that would be placed at each time step on doors 1 to 4 if the hint received is door 1 (column 1 must contain a 0 because you already know that the gift is not at door 1).}%
\es{El primer bloque, columnas 1 a 4, contienen las apuestas que se haría en cada paso temporal a las puertas 1 a 4 si la pista recibida fuera la puerta 1 (la columna 1 debe contener un 0 debido a que ya se sabe que el regalo no está en la puerta 1).}%
%
\en{The second block, columns 5 to 8, contain the bets that would be placed on doors 1 to 4 if the hint received is door 2 (column 6 must contain a 0 because you already know that the gift is not at door 2).}%
\es{El segundo bloque, columnas 5 a 8, contienen las apuestas
que se haría a las puertas 1 a 4 si la pista recibida fuera la puerta 2 (la columna 6 debe contener un 0 debido a que ya se sabe que el regalo no está en la puerta 2).}%
%
\en{The third block, columns 9 to 12, contains the bets on doors 1 to 4 if the hint received is door 3 (column 11 must contain a 0 because you already know that the gift is not at door 3).}%
\es{El tercer bloque, columnas 9 a 12, contienen las apuestas a las puertas 1 a 4 si la pista recibida fuera la puerta 3 (la columna 11 debe contener un 0 debido a que ya se sabe que el regalo no está en la puerta 3).}%
%
\en{The fourth block, columns 13 to 16, contains the bets on doors 1 to 4 if the hint received is door 4 (column 16 must contain a 0 because you already know that the gift is not at door 4).}%
\es{El cuarto bloque, columnas 13 a 16, contienen las apuestas a las puertas 1 a 4 si la pista recibida fuera la puerta 4 (la columna 16 debe contener un 0 debido a que ya se sabe que el regalo no está en la puerta 4).}%
%
\en{In addition, if column 0 (the chosen door) is X, we expect the block X to contain only zeros because hint X can never be generated.}%
\es{Además, si la columna 0 (puerta elegida) es X, esperamos el bloque X contenga únicamente ceros debido a que la pista X jamás podrá generarse.}%
%
\en{In the other three blocks, the bets must always add up to 1, as all resources must be bet at each time step.}%
\es{En los otros tres bloques, las apuestas deben sumar siempre 1 pues en cada paso temporal se deben apostar todos los recursos.}%
%
\en{Then, the CSV should have the following structure.}%
\es{Por ejemplo, el CSV debería tener la siguiente estructura.}%
%
\begin{table}[ht] \footnotesize
\begin{tabular}{c||c|cccc|cccc|cccc|cccc}
 & 0 & 1 & 2 & 3 & 4 & 5 & 6 & 7 & 8 & 9 & 10 & 11 & 12 & 13 & 14 & 15 & 16 \\ \hline \hline
 0 & 2 & 0 & 0.33 & 0.33 & 0.34 & 0 & 0 & 0 & 0 & 0.33 & 0.33 & 0 & 0.34 & 0.33 & 0.33 & 0.34 & 0    \\
 1 & 1 & 0  & 0 & 0 & 0 & 0.33 & 0 & 0.33 & 0.34 & 0.33 & 0.33 & 0 & 0.34 & 0.33 & 0.33 & 0.34 & 0    \\
    &  &  &  &  &  &  &  &  &  &  &  &  &  &  &  &  &     \\
 1094 & 3 & 0 & 0.33 & 0.33 & 0.34 & 0.33 & 0 & 0.33 & 0.34 & 0 & 0 & 0 & 0 & 0.33 & 0.33 & 0.34 & 0
 \end{tabular}
\end{table}
%
\en{The \texttt{csv} must not contain any column or row names.}%
\es{El \texttt{csv} no debe contener nombre de las columnas ni de las filas.}%
%
\en{The columns and rows are numbered here for reference purposes only.}%
\es{Aquí numeramos las columnas y filas solo a modo de referencia.}%
%
\en{This way of structuring the \texttt{csv} makes columns 1, 6, 11, and 16 always containing 0 and also always one of the blocks containing 0.}%
\es{Esta forma de estructurar el \texttt{csv} hace que las columnas 1, 6, 11, y 16 contengan siempre 0 y que además siempre uno de los bloques contenga 0.}%
%
\en{While it is easy to imagine more compact structures, this redundant structure allows to verify the consistency of the file.}%
\es{Si bien es fácil imaginar estructuras más compactas, esta estructura redundante permite verificar la consistencia del archivo.}%
%
\en{If these constraints are not met, the \texttt{csv} will be rejected.}%
\es{Si no se cumplen estas restricciones, el \texttt{csv} será rechazado.}%

\subsection{\texttt{cooperacion-id.csv}}

\en{The resource sharing policy with other members of the competition should be structured in a CSV with three columns.}%
\es{La política de intercambio de recursos con otros miembros de la competencia debe estar estructurada en un CSV con tres columnas.}%
%
\en{Column 0 should contain the identifiers of the persons to whom resources are provided.}%
\es{La columna 0 debe contener los identificadores de las personas a la que se le entregan recursos.}%
%
\en{Columns 1 and 2 together represent a fraction (numerator and denominator respectively) indicating the proportion of resources given to that person at each time step.}%
\es{La columna 1 y 2 representan en conjunto una fracción (numerador y denominador respectivamente) con la que se indica la proporción de los recursos que se le entregan a esa persona en cada paso temporal.}%
%
\en{For example, if one person gives 1/3 of the resources to individuals 24, 3 and 11, the file should have the following structure.}%
\es{Por ejemplo, si una persona le entrega 1/3 de los recursos a las personas 24, 3 y 11, el archivo debe tener la siguiente estructura.}%


\begin{table}[ht]\centering
\begin{tabular}{ccc}
0 & 1 & 2   \\ \hline \hline
24 & 1 & 3   \\
3  & 1 & 3   \\
11 & 1 & 3
\end{tabular}
\end{table}

\en{The \texttt{csv} must not contain any column or row names.}%
\es{El \texttt{csv} no debe contener nombre de las columnas ni de las filas.}%
\en{The proportions expressed with columns 1 and 2 must all be positive, and all together must add up to a maximum of 1 and a minimum of 0.}%
\es{Las proporciones expresadas con las columnas 1 y 2 deben ser todas positivas y en conjunto deben sumar como máximo 1 y como mínimo 0.}%
%
\en{Column 0 must contain registered identifiers, and must not contain repeated numbers.}%
\es{La columna 0 debe contener identificadores de personas inscriptas, y no debe contener números repetidos.}%

\paragraph{\en{Help}\es{Ayuda}}
\en{Read ``Properties of the epistemic-evolutionary cost function'' [1] and cooperate.}%
\es{Lee ``Propiedades de la función de costo epistémico-evolutiva'' [1] y cooperá.}%
%
% \vspace{0.3cm}


\noindent
%\en{[1] \url{https://metodosbayesianos.github.io/archivos/2023/properties.pdf}}%
%\es{[1] \url{https://metodosbayesianos.github.io/archivos/2023/propiedades.pdf}}%
[1] \url{https://metodosbayesianos.github.io/archivos/2023/propiedades.pdf}
\end{document}

