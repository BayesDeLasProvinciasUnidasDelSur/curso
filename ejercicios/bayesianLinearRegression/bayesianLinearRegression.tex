\documentclass[a4paper,10pt]{article}
\usepackage[utf8]{inputenc}
\usepackage[spanish]{babel}
\usepackage{fullpage}
\usepackage{cite}
\usepackage[utf8]{inputenc}
\usepackage{a4wide}
\usepackage{url}
\usepackage{graphicx}
\usepackage{caption}
\usepackage{float} % para que los gr\'aficos se queden en su lugar con [H]
\usepackage{subcaption}
\usepackage{wrapfig}
\usepackage{color}
\usepackage{amsmath} %para escribir funci\'on partida , matrices
\usepackage{amsthm} %para numerar definciones y teoremas
\usepackage[hidelinks]{hyperref} % para inlcuir links dentro del texto
\usepackage{tabu} 
\usepackage{comment}
\usepackage{amsfonts} % \mathbb{N} -> conjunto de los n\'umeros naturales  
\usepackage{enumerate}
\usepackage{listings}
\usepackage[colorinlistoftodos, textsize=small]{todonotes} % Para poner notas en el medio del texto!! No olvidar hacer. 
\usepackage{framed} % Para encuadrar texto. \begin{framed}
\usepackage{csquotes} % Para citar texto \begin{displayquote}
\usepackage{epigraph} % Epigrafe  \epigraph{texto}{\textit{autor}}
\usepackage{authblk}
\usepackage{titlesec}
\usepackage{varioref}
\usepackage{bm} % \bm{\alpha} bold greek symbol
\usepackage{pdfpages} % \includepdf
\usepackage[makeroom]{cancel} % \cancel{} \bcancel{} etc
\usepackage{wrapfig} % \begin{wrapfigure} Pone figura al lado del texto
\usepackage{tikz}
\usepackage{algorithm}


\usepackage{paracol}

\newcommand{\citel}[1]{\cite{#1}\label{#1}}
\newcommand\hfrac[2]{\genfrac{}{}{0pt}{}{#1}{#2}} %\frac{}{} sin la linea del medio

\theoremstyle{definition}
\newtheorem{definition}{Definition}[section]
\newtheorem{theorem}{Theorem}[section]
\newtheorem{proposition}{Proposition}[section]

%http://latexcolor.com/
\definecolor{azul}{rgb}{0.36, 0.54, 0.66}
\definecolor{rojo}{rgb}{0.7, 0.2, 0.116}
\definecolor{rojopiso}{rgb}{0.8, 0.25, 0.17}
\definecolor{verdeingles}{rgb}{0.12, 0.5, 0.17}
\definecolor{ubuntu}{rgb}{0.44, 0.16, 0.39}
\definecolor{debian}{rgb}{0.84, 0.04, 0.33}

\definecolor{dkgreen}{rgb}{0,0.6,0}
\definecolor{gray}{rgb}{0.5,0.5,0.5}
\definecolor{mauve}{rgb}{0.58,0,0.82}

\lstset{
  language=Python,
  aboveskip=3mm,
  belowskip=3mm,
  showstringspaces=true,
  columns=flexible,
  basicstyle={\small\ttfamily},
  numbers=none,
  numberstyle=\tiny\color{gray},
  keywordstyle=\color{blue},
  commentstyle=\color{dkgreen},
  stringstyle=\color{mauve},
  breaklines=true,
  breakatwhitespace=true,
  tabsize=4
}



%opening
\title{Bayesian Linear Regression}
\author{}

\begin{document}

\maketitle

\begin{abstract}

\end{abstract}

\section{Links}

\url{http://krasserm.github.io/2019/02/23/bayesian-linear-regression/}

\section{Linear basis function }

Linear regression models share the property of being linear in their parameters but not necessarily in their input variables. 
Using non-linear basis functions of input variables, linear models are able model arbitrary non-linearities from input variables to targets.  
Polynomial regression is such an example of basis functions.
A linear regression model $\text{linear}(\bm{x},\bm{\beta})$

\begin{equation}
\texttt{linear}(\bm{x},\bm{\beta}) = \sum_{i=0}^{M-1} \beta_i \Phi_i(\bm{x}) = \bm{\beta}^T \bm{\Phi}(\bm{x})
\end{equation}

Where $\Phi$ is the basis function and $M$ the total number of parameters.

Then, the conditional distribution of $y$ $p(y | \bm{x}, \bm{\beta}, \sigma)$ can therefore be written as

\begin{equation}
p(\bm{y}_i | \bm{x}_i, \bm{\beta}, \sigma) = N(\bm{y}_i | \bm{\beta}^T \bm{\Phi}(\bm{x}_i) , \sigma)
\end{equation}

Since we asume i.i.d., the joint conditional probability of $\bm{y}$

\begin{equation}
p(\bm{y} | \bm{x}, \bm{\beta}, \sigma) = \prod_{i=1}^{N} N(\bm{y}_i | \bm{\beta}^T \bm{\Phi}(\bm{x}_i), \sigma)
\end{equation}

Maximizing the log likelihood (= minimizing the sum-of-squares error function) gives the maximum likelihood estimate of parameters $\bm{\beta}$.

\begin{equation}
 \begin{split}
  l(\bm{y}, \bm{x},\bm{\beta},\sigma) & = \text{log } L(\bm{y}, \bm{x},\bm{\beta},\sigma) = \text{log } \prod_{i=1}^{n} p(\bm{y}_i | \bm{x}_i, \bm{\beta}, \sigma)\\
  & =  \sum_{i=1}^{n}  \text{log } p(\bm{y}_i | \bm{x}_i, \bm{\beta}, \sigma) = \sum_{i=1}^{n} \text{log } N(\bm{y}_i | \bm{\beta}^T \bm{\Phi}(\bm{x}_i), \sigma)  \\
  & =  \sum_{i=1}^{n} \text{log }  \frac{1}{\sqrt{2\pi}\sigma} e^{\frac{-(\bm{y}_i - \bm{\beta}^T\bm{\Phi}(\bm{x}_i))^2}{2\sigma^2} } = \sum_{i=1}^{n} \text{log } \frac{1}{\sqrt{2\pi}\sigma} + \sum_{i=1}^{n} \text{log } e^{\frac{-(\bm{y}_i - \bm{\beta}^T\bm{\Phi}(\bm{x}_i))^2}{2\sigma^2} } \\
  & = n \text{log } \frac{1}{\sqrt{2\pi}\sigma} + \sum_{i=1}^{n} \text{log } e^{\frac{-(\bm{y}_i - \bm{\beta}^T\bm{\Phi}(\bm{x}_i))^2}{2\sigma^2} } = n \text{log } \frac{1}{\sqrt{2\pi}\sigma} + \sum_{i=1}^{n}  \frac{-(\bm{y}_i - \bm{\beta}^T\bm{\Phi}(\bm{x}_i))^2}{2\sigma^2} \\
  &  = -\frac{n}{2} (\text{log } 2\pi\sigma^2) - \frac{1}{2\sigma^2} \sum_{i=1}^{n}  (\bm{y}_i - \bm{\beta}^T\bm{\Phi}(\bm{x}_i))^2 \\
  & \propto  \sum_{i=1}^{n}  (\bm{y}_i - \bm{\beta}^T\bm{\Phi}(\bm{x}_i))^2
 \end{split}
\end{equation}



\end{document}
