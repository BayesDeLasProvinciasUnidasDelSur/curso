\documentclass[a4paper,10pt]{article}
\usepackage[utf8]{inputenc}
\usepackage[spanish]{babel}
\usepackage{fullpage}
\usepackage{cite}
\usepackage[utf8]{inputenc}
\usepackage{a4wide}
\usepackage{url}
\usepackage{graphicx}
\usepackage{caption}
\usepackage{float} % para que los gr\'aficos se queden en su lugar con [H]
\usepackage{subcaption}
\usepackage{wrapfig}
\usepackage{color}
\usepackage{amsmath} %para escribir funci\'on partida , matrices
\usepackage{amsthm} %para numerar definciones y teoremas
\usepackage[hidelinks]{hyperref} % para inlcuir links dentro del texto
\usepackage{tabu} 
\usepackage{comment}
\usepackage{amsfonts} % \mathbb{N} -> conjunto de los n\'umeros naturales  
\usepackage{enumerate}
\usepackage{listings}
\usepackage[colorinlistoftodos, textsize=small]{todonotes} % Para poner notas en el medio del texto!! No olvidar hacer. 
\usepackage{framed} % Para encuadrar texto. \begin{framed}
\usepackage{csquotes} % Para citar texto \begin{displayquote}
\usepackage{epigraph} % Epigrafe  \epigraph{texto}{\textit{autor}}
\usepackage{authblk}
\usepackage{titlesec}
\usepackage{varioref}
\usepackage{bm} % \bm{\alpha} bold greek symbol
\usepackage{pdfpages} % \includepdf
\usepackage[makeroom]{cancel} % \cancel{} \bcancel{} etc
\usepackage{wrapfig} % \begin{wrapfigure} Pone figura al lado del texto
\usepackage{tikz}
\usepackage{algorithm}


\usepackage{paracol}

\newcommand{\citel}[1]{\cite{#1}\label{#1}}
\newcommand\hfrac[2]{\genfrac{}{}{0pt}{}{#1}{#2}} %\frac{}{} sin la linea del medio

\theoremstyle{definition}
\newtheorem{definition}{Definition}[section]
\newtheorem{theorem}{Theorem}[section]
\newtheorem{proposition}{Proposition}[section]

%http://latexcolor.com/
\definecolor{azul}{rgb}{0.36, 0.54, 0.66}
\definecolor{rojo}{rgb}{0.7, 0.2, 0.116}
\definecolor{rojopiso}{rgb}{0.8, 0.25, 0.17}
\definecolor{verdeingles}{rgb}{0.12, 0.5, 0.17}
\definecolor{ubuntu}{rgb}{0.44, 0.16, 0.39}
\definecolor{debian}{rgb}{0.84, 0.04, 0.33}

\definecolor{dkgreen}{rgb}{0,0.6,0}
\definecolor{gray}{rgb}{0.5,0.5,0.5}
\definecolor{mauve}{rgb}{0.58,0,0.82}

\lstset{
  language=Python,
  aboveskip=3mm,
  belowskip=3mm,
  showstringspaces=true,
  columns=flexible,
  basicstyle={\small\ttfamily},
  numbers=none,
  numberstyle=\tiny\color{gray},
  keywordstyle=\color{blue},
  commentstyle=\color{dkgreen},
  stringstyle=\color{mauve},
  breaklines=true,
  breakatwhitespace=true,
  tabsize=4
}

% tikzlibrary.code.tex
%
% Copyright 2010-2011 by Laura Dietz
% Copyright 2012 by Jaakko Luttinen
%
% This file may be distributed and/or modified
%
% 1. under the LaTeX Project Public License and/or
% 2. under the GNU General Public License.
%
% See the files LICENSE_LPPL and LICENSE_GPL for more details.

% Load other libraries
\usetikzlibrary{shapes}
\usetikzlibrary{fit}
\usetikzlibrary{chains}
\usetikzlibrary{arrows}

% Latent node
\tikzstyle{latent} = [circle,fill=white,draw=black,inner sep=1pt,
minimum size=20pt, font=\fontsize{10}{10}\selectfont, node distance=1]
% Observed node
\tikzstyle{obs} = [latent,fill=gray!25]
% Invisible node
\tikzstyle{invisible} = [latent,minimum size=0pt,color=white, opacity=0, node distance=0]
% Constant node
\tikzstyle{const} = [rectangle, inner sep=0pt, node distance=0.1]
%state
\tikzstyle{estado} = [latent,minimum size=8pt,node distance=0.4]
%action
\tikzstyle{accion} =[latent,circle,minimum size=5pt,fill=black,node distance=0.4]


% Factor node
\tikzstyle{factor} = [rectangle, fill=black,minimum size=10pt, draw=black, inner
sep=0pt, node distance=1]
% Deterministic node
\tikzstyle{det} = [latent, rectangle]

% Plate node
\tikzstyle{plate} = [draw, rectangle, rounded corners, fit=#1]
% Invisible wrapper node
\tikzstyle{wrap} = [inner sep=0pt, fit=#1]
% Gate
\tikzstyle{gate} = [draw, rectangle, dashed, fit=#1]

% Caption node
\tikzstyle{caption} = [font=\footnotesize, node distance=0] %
\tikzstyle{plate caption} = [caption, node distance=0, inner sep=0pt,
below left=5pt and 0pt of #1.south east] %
\tikzstyle{factor caption} = [caption] %
\tikzstyle{every label} += [caption] %

\tikzset{>={triangle 45}}

%\pgfdeclarelayer{b}
%\pgfdeclarelayer{f}
%\pgfsetlayers{b,main,f}

% \factoredge [options] {inputs} {factors} {outputs}
\newcommand{\factoredge}[4][]{ %
  % Connect all nodes #2 to all nodes #4 via all factors #3.
  \foreach \f in {#3} { %
    \foreach \x in {#2} { %
      \path (\x) edge[-,#1] (\f) ; %
      %\draw[-,#1] (\x) edge[-] (\f) ; %
    } ;
    \foreach \y in {#4} { %
      \path (\f) edge[->,#1] (\y) ; %
      %\draw[->,#1] (\f) -- (\y) ; %
    } ;
  } ;
}

% \edge [options] {inputs} {outputs}
\newcommand{\edge}[3][]{ %
  % Connect all nodes #2 to all nodes #3.
  \foreach \x in {#2} { %
    \foreach \y in {#3} { %
      \path (\x) edge [->,#1] (\y) ;%
      %\draw[->,#1] (\x) -- (\y) ;%
    } ;
  } ;
}

% \factor [options] {name} {caption} {inputs} {outputs}
\newcommand{\factor}[5][]{ %
  % Draw the factor node. Use alias to allow empty names.
  \node[factor, label={[name=#2-caption]#3}, name=#2, #1,
  alias=#2-alias] {} ; %
  % Connect all inputs to outputs via this factor
  \factoredge {#4} {#2-alias} {#5} ; %
}

% \plate [options] {name} {fitlist} {caption}
\newcommand{\plate}[4][]{ %
  \node[wrap=#3] (#2-wrap) {}; %
  \node[plate caption=#2-wrap] (#2-caption) {#4}; %
  \node[plate=(#2-wrap)(#2-caption), #1] (#2) {}; %
}

% \gate [options] {name} {fitlist} {inputs}
\newcommand{\gate}[4][]{ %
  \node[gate=#3, name=#2, #1, alias=#2-alias] {}; %
  \foreach \x in {#4} { %
    \draw [-*,thick] (\x) -- (#2-alias); %
  } ;%
}

% \vgate {name} {fitlist-left} {caption-left} {fitlist-right}
% {caption-right} {inputs}
\newcommand{\vgate}[6]{ %
  % Wrap the left and right parts
  \node[wrap=#2] (#1-left) {}; %
  \node[wrap=#4] (#1-right) {}; %
  % Draw the gate
  \node[gate=(#1-left)(#1-right)] (#1) {}; %
  % Add captions
  \node[caption, below left=of #1.north ] (#1-left-caption)
  {#3}; %
  \node[caption, below right=of #1.north ] (#1-right-caption)
  {#5}; %
  % Draw middle separation
  \draw [-, dashed] (#1.north) -- (#1.south); %
  % Draw inputs
  \foreach \x in {#6} { %
    \draw [-*,thick] (\x) -- (#1); %
  } ;%
}

% \hgate {name} {fitlist-top} {caption-top} {fitlist-bottom}
% {caption-bottom} {inputs}
\newcommand{\hgate}[6]{ %
  % Wrap the left and right parts
  \node[wrap=#2] (#1-top) {}; %
  \node[wrap=#4] (#1-bottom) {}; %
  % Draw the gate
  \node[gate=(#1-top)(#1-bottom)] (#1) {}; %
  % Add captions
  \node[caption, above right=of #1.west ] (#1-top-caption)
  {#3}; %
  \node[caption, below right=of #1.west ] (#1-bottom-caption)
  {#5}; %
  % Draw middle separation
  \draw [-, dashed] (#1.west) -- (#1.east); %
  % Draw inputs
  \foreach \x in {#6} { %
    \draw [-*,thick] (\x) -- (#1); %
  } ;%
}


%opening

\title{Baysian skill estimator}
\author{Gustavo Landfried}

\begin{document}

\maketitle

\section{Computo anal\'itico de la distribuci\'on posterior} \label{sec:computoAnilitico}


Para calcular la posterior utilizamos un algoritmo gen\'erico de pasaje de mensajes para \emph{factor graphs} llamado \emph{sum-product algorithm}~\cite{kschischang2001-factorGraphsAndTheSumProductAlgorithm}.
Permite resolver funciones marginales eficientemente mediante el uso de la forma en la cual la funci\'on global se factoriza en el producto de funciones locales simples, cada una dependiente de una subconjunto de variables.

\subsection{Sum-product algorithm}

Un \emph{factor graph} es una grafo bipartito, (relaciones entre nodos variable $v$ y nodos factor $f$).
Los ejes del factor graph representan la relaci\'on matem\'atica ``nodo $v$ es argumento de nodo $f$''.
La estructura del grafo codifica la factorizaci\'on de la funci\'on global.
Pero adem\'as, cuando un grafo de factorizaci\'on (\emph{factor graph}) no contiene ciclos, el grafo codifica tambi\'en las expresiones aritm\'eticas mediante las cuales se puede computar las marginales asociadas a la funci\'on global.

En el caso de \texttt{TrueSkill}, la funci\'on representada por el factor graph es la distribuci\'on conjunta $p(\bm{s},\bm{p},\bm{t}|\bm{o},A)$ (Fig.\ref{factorGraph_trueskill}). 

\begin{figure}[H]
  \centering
  \scalebox{.9}{\tikz{ %
        
      
        \node[factor] (fr) {} ;
        \node[const, above=of fr] (nfr) {$f_r$}; %
	\node[const, above=of nfr] (dfr) {\large $\mathbb{I}(d_j>0)$}; %
        \node[latent, left=of fr] (d) {$d_j$} ; %
        \node[factor, left=of d] (fd) {} ;
        \node[const, above=of fd] (nfd) {$f_d$}; %
        \node[const, above=of nfd] (dfd) {\large $\mathbb{I}(d_j=t_{o_j} - t_{o_{j+1}})$}; %
        \node[const, below=of d,yshift=-0.15cm] (j) {\footnotesize con $o:=$ ordenamiento observado};
        
        \node[latent, left=of fd,xshift=-0.9cm] (t) {$t_e$} ; %
        \node[factor, left=of t] (ft) {} ;
        \node[const, above=of ft] (nft) {$f_t$}; %
        \node[const, above=of nft,xshift=0.5cm] (dft) {\large $\mathbb{I}(t_e = \sum_{i \in A_e} p_i)$}; %
        
        \node[latent, left=of ft] (p) {$p_i$} ; %
        \node[factor, left=of p] (fp) {} ;
        \node[const, above=of fp] (nfp) {$f_p$}; %
        \node[const, above=of nfp] (dfp) {\large $N(p_i;s_i,\beta^2)$}; %
 
        \node[latent, left=of fp] (s) {$s_i$} ; %
        \node[factor, left=of s] (fs) {} ;
        \node[const, above=of fs] (nfs) {$f_s$}; %
        \node[const, above=of nfs] (dfs) {\large $N(s_i;\mu_i,\sigma^2)$}; %
         
        \edge[-] {d} {fr};
	\edge[-] {fd} {d};
        \edge[-] {fd} {t};
        \edge[-] {t} {ft};
        \edge[-] {ft} {p};
        \edge[-] {p} {fp};
        \edge[-] {fp} {s};
        \edge[-] {s} {fs};
	
        \plate {personas} {(p)(s)(fs)(nfs)(dfp)(dfs)} {$i \in A_e$}; %
        \node[invisible, below=of ft, yshift=-0.6cm] (inv_below_e) {};
	\node[invisible, above=of ft, yshift=1.1cm] (inv_above_e) {};
	\plate {equipos} {(personas) (t)(ft)(dft) (inv_above_e) (inv_below_e)} {$  \text{con $A$ partici\'on de jugadores }$  \hspace{3cm} $0 < e \leq |A|$}; %
	\node[invisible, below=of fr, yshift=-0.6cm] (inv_below) {};
	\node[invisible, above=of fr, yshift=1.1cm] (inv_above) {};
	\plate {comparaciones} {(fd) (dfd) (d) (fr) (dfr) (inv_below) (inv_above)} {$0 < j < |A|$}
	
	
	%\node[const, right= of r, xshift=1.2cm ,yshift=-2.1cm] (result-dist) {$r_{ab} \sim B\left(\Phi\left(\frac{\mu_a - \mu_b}{\sqrt{\beta_a^2+\beta_b^2}}\right)\right)$} ; %
	      
        }}
  \caption{\small Grafo bipartito de la factorizaci\'on (\emph{factor graph}) del modelo \texttt{Trueskill}}
  \label{factorGraph_trueskill}
\end{figure}

\paragraph{The Sum-Product Update Rule} El mensaje enviado desde un v\'ertice $v$ a trav\'es de un eje $e$ es el producto de la funci\'on local de $v$ (funci\'on indicadora si $v$ es un nodo variable) con todos los mensajes recibidos en $v$ a trav\'es de ejes \emph{distintos} de $e$, integrados para la variable asociada con $e$.

\vspace{0.3cm}

Sea $m_{x \rightarrow f}(x)$ el mensaje enviado por el nodo variable $x$ al nodo factor $f$, y $m_{f \rightarrow x}(x)$ el mensaje enviado por un nodo factor $f$ a un nodo variable $x$.
Sea $n(v)$ el conjunto de nodos vecinos al nodo $v$.
Luego, los mensajes pueden ser expresados del siguiente modo.
\begin{equation}\label{eq:m_v_f} 
m_{x \rightarrow f}(x) = \prod_{h \in n(x) \setminus \{f\} } m_{h \rightarrow x}(x)
\end{equation}
\begin{equation}\label{eq:m_f_v}  
m_{f \rightarrow x}(x) = \int \cdots \int \Big( f(\bm{x}) \prod_{h \in n(f) \setminus \{x\} } m_{h \rightarrow f}(h) \Big) \,  d\bm{x}_{\setminus x}
\end{equation}

donde $\bm{x} = \text{arg}(f)$ es el conjunto de argumentos de la funci\'on $f$. 
Luego, para calcular una marginal cualquiera,
\begin{equation}\label{eq:marginal}
g_i(x_i) = \prod_{h \in n(x_i)} m_{h \rightarrow x_i}
\end{equation}

\subsection{Propiedades}\label{sec:propiedades}

Las siguientes tres propiedades, junto con las reglas del \emph{sum-product algorithm}, es lo \'unico que se necesita para calcular la posterior anal\'itica del modelo bayesiano.

\subsubsection*{Multiplicaci\'on de normales}  
\begin{equation}\label{eq:multiplicacion_normales}
\begin{split}
 \int_{-\infty}^{\infty} N(x|\mu_x,\sigma_x^2)N(x|\mu_y,\sigma_y^2) \, dx  &  \overset{*}{=} \int_{-\infty}^{\infty}  \underbrace{N(\mu_x|\mu_y,\sigma_x^2+\sigma_y^2)}_{\text{constante}} N(x|\mu_{*},\sigma_{*}^2) dx \\
 & = N(\mu_x|\mu_y,\sigma_x^2+\sigma_y^2) \underbrace{\int_{-\infty}^{\infty}  N(x|\mu_{*},\sigma_{*}^2) dx}_{1} \\
 & = N(\mu_x|\mu_y,\sigma_x^2+\sigma_y^2) 
\end{split}
\end{equation}

Donde la igualdad destacada ($\overset{*}{=}$) se demuestra en la secci\'on~\ref{multiplicacion_normales} anexa.

\subsubsection*{Integrales con funci\'on indicadora}
\begin{equation}\label{eq:integral_con_indicadora} 
\begin{split}
 \int_{-\infty}^{\infty}  \int_{-\infty}^{\infty}  \mathbb{I}(x=h(y,z)) f(x) g(y)\, dx\, dy &=  \int_{-\infty}^{\infty} \int_{h(y,z)}^{h(y,z)} f(h(y,z)) g(y)\, dx\, dy\\
 & = \int_{-\infty}^{\infty} f(h(y,z)) g(y) dy 
 %& \propto \int f(h(y,z)) g(y) dy 
\end{split}
\end{equation}

\subsubsection*{Simetr\'ia de normales}
\begin{equation}\label{eq:simetria}
 N(x|\mu,\sigma^2) = N(\mu|x,\sigma^2) = N(-\mu|-x,\sigma^2) = N(-x|-\mu,\sigma^2) 
\end{equation}

\subsubsection*{Derivada de la acumulada normal}
\begin{equation}\label{eq:phi_norm}
 \frac{\partial}{\partial x} \Phi(x|\mu,\sigma^2) = N(x|\mu,\sigma^2)
\end{equation}

\subsubsection*{Distribuci\'on normal estandarizada}
\begin{equation}\label{eq:estandarizar}
 X \sim N(\mu,\sigma^2) \Rightarrow \frac{X-\mu}{\sigma} \sim N(0,1)
\end{equation}



\subsection{Ejemplo 2 vs 2}

\begin{figure}[H]
  \centering
  \scalebox{.75}{
\documentclass[10pt]{standalone}
\usepackage[utf8]{inputenc}
\usepackage{tikz}

% tikzlibrary.code.tex
%
% Copyright 2010-2011 by Laura Dietz
% Copyright 2012 by Jaakko Luttinen
%
% This file may be distributed and/or modified
%
% 1. under the LaTeX Project Public License and/or
% 2. under the GNU General Public License.
%
% See the files LICENSE_LPPL and LICENSE_GPL for more details.

% Load other libraries
\usetikzlibrary{shapes}
\usetikzlibrary{fit}
\usetikzlibrary{chains}
\usetikzlibrary{arrows}

% Latent node
\tikzstyle{latent} = [circle,fill=white,draw=black,inner sep=1pt,
minimum size=20pt, font=\fontsize{10}{10}\selectfont, node distance=1]
% Observed node
\tikzstyle{obs} = [latent,fill=gray!25]
% Invisible node
\tikzstyle{invisible} = [latent,minimum size=0pt,color=white, opacity=0, node distance=0]
% Constant node
\tikzstyle{const} = [rectangle, inner sep=0pt, node distance=0.1]
%state
\tikzstyle{estado} = [latent,minimum size=8pt,node distance=0.4]
%action
\tikzstyle{accion} =[latent,circle,minimum size=5pt,fill=black,node distance=0.4]


% Factor node
\tikzstyle{factor} = [rectangle, fill=black,minimum size=10pt, draw=black, inner
sep=0pt, node distance=1]
% Deterministic node
\tikzstyle{det} = [latent, rectangle]

% Plate node
\tikzstyle{plate} = [draw, rectangle, rounded corners, fit=#1]
% Invisible wrapper node
\tikzstyle{wrap} = [inner sep=0pt, fit=#1]
% Gate
\tikzstyle{gate} = [draw, rectangle, dashed, fit=#1]

% Caption node
\tikzstyle{caption} = [font=\footnotesize, node distance=0] %
\tikzstyle{plate caption} = [caption, node distance=0, inner sep=0pt,
below left=5pt and 0pt of #1.south east] %
\tikzstyle{factor caption} = [caption] %
\tikzstyle{every label} += [caption] %

\tikzset{>={triangle 45}}

%\pgfdeclarelayer{b}
%\pgfdeclarelayer{f}
%\pgfsetlayers{b,main,f}

% \factoredge [options] {inputs} {factors} {outputs}
\newcommand{\factoredge}[4][]{ %
  % Connect all nodes #2 to all nodes #4 via all factors #3.
  \foreach \f in {#3} { %
    \foreach \x in {#2} { %
      \path (\x) edge[-,#1] (\f) ; %
      %\draw[-,#1] (\x) edge[-] (\f) ; %
    } ;
    \foreach \y in {#4} { %
      \path (\f) edge[->,#1] (\y) ; %
      %\draw[->,#1] (\f) -- (\y) ; %
    } ;
  } ;
}

% \edge [options] {inputs} {outputs}
\newcommand{\edge}[3][]{ %
  % Connect all nodes #2 to all nodes #3.
  \foreach \x in {#2} { %
    \foreach \y in {#3} { %
      \path (\x) edge [->,#1] (\y) ;%
      %\draw[->,#1] (\x) -- (\y) ;%
    } ;
  } ;
}

% \factor [options] {name} {caption} {inputs} {outputs}
\newcommand{\factor}[5][]{ %
  % Draw the factor node. Use alias to allow empty names.
  \node[factor, label={[name=#2-caption]#3}, name=#2, #1,
  alias=#2-alias] {} ; %
  % Connect all inputs to outputs via this factor
  \factoredge {#4} {#2-alias} {#5} ; %
}

% \plate [options] {name} {fitlist} {caption}
\newcommand{\plate}[4][]{ %
  \node[wrap=#3] (#2-wrap) {}; %
  \node[plate caption=#2-wrap] (#2-caption) {#4}; %
  \node[plate=(#2-wrap)(#2-caption), #1] (#2) {}; %
}

% \gate [options] {name} {fitlist} {inputs}
\newcommand{\gate}[4][]{ %
  \node[gate=#3, name=#2, #1, alias=#2-alias] {}; %
  \foreach \x in {#4} { %
    \draw [-*,thick] (\x) -- (#2-alias); %
  } ;%
}

% \vgate {name} {fitlist-left} {caption-left} {fitlist-right}
% {caption-right} {inputs}
\newcommand{\vgate}[6]{ %
  % Wrap the left and right parts
  \node[wrap=#2] (#1-left) {}; %
  \node[wrap=#4] (#1-right) {}; %
  % Draw the gate
  \node[gate=(#1-left)(#1-right)] (#1) {}; %
  % Add captions
  \node[caption, below left=of #1.north ] (#1-left-caption)
  {#3}; %
  \node[caption, below right=of #1.north ] (#1-right-caption)
  {#5}; %
  % Draw middle separation
  \draw [-, dashed] (#1.north) -- (#1.south); %
  % Draw inputs
  \foreach \x in {#6} { %
    \draw [-*,thick] (\x) -- (#1); %
  } ;%
}

% \hgate {name} {fitlist-top} {caption-top} {fitlist-bottom}
% {caption-bottom} {inputs}
\newcommand{\hgate}[6]{ %
  % Wrap the left and right parts
  \node[wrap=#2] (#1-top) {}; %
  \node[wrap=#4] (#1-bottom) {}; %
  % Draw the gate
  \node[gate=(#1-top)(#1-bottom)] (#1) {}; %
  % Add captions
  \node[caption, above right=of #1.west ] (#1-top-caption)
  {#3}; %
  \node[caption, below right=of #1.west ] (#1-bottom-caption)
  {#5}; %
  % Draw middle separation
  \draw [-, dashed] (#1.west) -- (#1.east); %
  % Draw inputs
  \foreach \x in {#6} { %
    \draw [-*,thick] (\x) -- (#1); %
  } ;%
}


\usepackage[spanish]{babel}
\usepackage{fullpage}
\usepackage{cite}
\usepackage[utf8]{inputenc}
\usepackage{a4wide}
\usepackage{url}
\usepackage{graphicx}
\usepackage{caption}
\usepackage{float} % para que los gr\'aficos se queden en su lugar con [H]
\usepackage{subcaption}
\usepackage{wrapfig}
\usepackage{color}
\usepackage{amsmath} %para escribir funci\'on partida , matrices
\usepackage{amsthm} %para numerar definciones y teoremas
\usepackage[hidelinks]{hyperref} % para inlcuir links dentro del texto
\usepackage{tabu} 
\usepackage{comment}
\usepackage{amsfonts} % \mathbb{N} -> conjunto de los n\'umeros naturales  
\usepackage{enumerate}
\usepackage{listings}
\usepackage[colorinlistoftodos, textsize=small]{todonotes} % Para poner notas en el medio del texto!! No olvidar hacer. 
\usepackage{framed} % Para encuadrar texto. \begin{framed}
\usepackage{csquotes} % Para citar texto \begin{displayquote}
\usepackage{epigraph} % Epigrafe  \epigraph{texto}{\textit{autor}}
\usepackage{authblk}
\usepackage{titlesec}
\usepackage{varioref}
\usepackage{bm} % \bm{\alpha} bold greek symbol
\usepackage{pdfpages} % \includepdf
\usepackage[makeroom]{cancel} % \cancel{} \bcancel{} etc
\usepackage{wrapfig} % \begin{wrapfigure} Pone figura al lado del texto
\usepackage{tikz}
\usepackage{algorithm}


\usepackage{paracol}

\newcommand{\citel}[1]{\cite{#1}\label{#1}}
\newcommand\hfrac[2]{\genfrac{}{}{0pt}{}{#1}{#2}} %\frac{}{} sin la linea del medio

\theoremstyle{definition}
\newtheorem{definition}{Definition}[section]
\newtheorem{theorem}{Theorem}[section]
\newtheorem{proposition}{Proposition}[section]

%http://latexcolor.com/
\definecolor{azul}{rgb}{0.36, 0.54, 0.66}
\definecolor{rojo}{rgb}{0.7, 0.2, 0.116}
\definecolor{rojopiso}{rgb}{0.8, 0.25, 0.17}
\definecolor{verdeingles}{rgb}{0.12, 0.5, 0.17}
\definecolor{ubuntu}{rgb}{0.44, 0.16, 0.39}
\definecolor{debian}{rgb}{0.84, 0.04, 0.33}

\definecolor{dkgreen}{rgb}{0,0.6,0}
\definecolor{gray}{rgb}{0.5,0.5,0.5}
\definecolor{mauve}{rgb}{0.58,0,0.82}

\lstset{
  language=Python,
  aboveskip=3mm,
  belowskip=3mm,
  showstringspaces=true,
  columns=flexible,
  basicstyle={\small\ttfamily},
  numbers=none,
  numberstyle=\tiny\color{gray},
  keywordstyle=\color{blue},
  commentstyle=\color{dkgreen},
  stringstyle=\color{mauve},
  breaklines=true,
  breakatwhitespace=true,
  tabsize=4
}

\makeatletter
\newcommand{\vast}{\bBigg@{2.5}}
\newcommand{\Vast}{\bBigg@{14.5}}
\makeatother


\tikzstyle{arrow}= [thick, ->, >=stealth]
\usetikzlibrary{shapes.geometric, arrows}
\tikzstyle{operator}=[circle, radius= 0.2 cm, text centered, draw=black]


\usepackage{helvet}
\renewcommand{\familydefault}{\sfdefault}
\begin{document}
 
\begin{tikzpicture}

  [
    ->,
    >=stealth',
    auto,node distance=3cm,
    thick,
    main node/.style={circle, draw, font=\sffamily\Large\bfseries}
    ]
        
        \node[det, fill=black!20] (r) {$r_1$};
      
        \node[factor, above=of r] (fr) {} ;
        \node[const, right=of fr] (nfr) {$f_{r_1}$}; %
	
	\node[latent, above=of fr] (d) {$d_1$} ; %
        \node[factor, above=of d] (fd) {} ;
        \node[const, above=of fd] (nfd) {$f_{d_1}$}; %
	
        
        \node[latent, left=of fd] (ta) {$t_a$} ; %
        \node[factor, left=of ta] (fta) {} ;
        \node[const, above=of fta] (nfta) {$f_{t_a}$}; %
        
        
        
        \node[latent, left=of fta,yshift=1cm] (p1) {$p_1$} ; %
        \node[factor, left=of p1] (fp1) {} ;
        \node[const, above=of fp1] (nfp1) {$f_{p_1}$}; %
        
        \node[latent, left=of fp1] (s1) {$s_1$} ; %
        \node[factor, left=of s1] (fs1) {} ;
	\node[const, above=of fs1] (nfs1) {$f_{s_1}$}; %
     
        \node[latent, left=of fta,yshift=-1cm] (p2) {$p_2$} ; %
        \node[factor, left=of p2] (fp2) {} ;
        \node[const, above=of fp2] (nfp2) {$f_{p_2}$}; %
        
        \node[latent, left=of fp2] (s2) {$s_2$} ; %
        \node[factor, left=of s2] (fs2) {} ;
	\node[const, above=of fs2] (nfs2) {$f_{s_2}$}; %
        
            
        \node[latent, right=of fd] (tb) {$t_b$} ; %
        \node[factor, right=of tb] (ftb) {} ;
        \node[const, above=of ftb] (nftb) {$f_{t_b}$}; %
        
        \node[latent, right=of ftb,yshift=1cm] (p3) {$p_3$} ; %
        \node[factor, right=of p3] (fp3) {} ;
        \node[const, above=of fp3] (nfp3) {$f_{p_3}$}; %
        
        \node[latent, right=of fp3] (s3) {$s_3$} ; %
        \node[factor, right=of s3] (fs3) {} ;
	\node[const, above=of fs3] (nfs3) {$f_{s_3}$}; %
     
        \node[latent, right=of ftb,yshift=-1cm] (p4) {$p_4$} ; %
        \node[factor, right=of p4] (fp4) {} ;
        \node[const, above=of fp4] (nfp4) {$f_{p_4}$}; %
        
        \node[latent, right=of fp4] (s4) {$s_4$} ; %
        \node[factor, right=of s4] (fs4) {} ;
	\node[const, above=of fs4] (nfs4) {$f_{s_4}$}; %
     
        \node[invisible, xshift=10cm] (i) {};
     
        \edge[-] {r} {fr};
        \edge[-] {fr} {d};
	\edge[-] {d} {fd};
	
        \edge[-] {fd} {ta};
        \edge[-] {ta} {fta};
        \edge[-] {fta} {p1};
        \edge[-] {p1} {fp1};
        \edge[-] {fp1} {s1};
        \edge[-] {s1} {fs1};
        \edge[-] {fta} {p2};
        \edge[-] {p2} {fp2};
        \edge[-] {fp2} {s2};
        \edge[-] {s2} {fs2};
        	
	\edge[-] {fd} {tb};
        \edge[-] {tb} {ftb};
        \edge[-] {ftb} {p3};
        \edge[-] {p3} {fp3};
        \edge[-] {fp3} {s3};
        \edge[-] {s3} {fs3};
        \edge[-] {ftb} {p4};
        \edge[-] {p4} {fp4};
        \edge[-] {fp4} {s4};
        \edge[-] {s4} {fs4};
        
%         \path (fs1) edge[bend left, arrow] node [right] {} (s1);
%         \path (fs2) edge[bend left, arrow] node [right] {} (s2);	      
%         \path (fs3) edge[bend right, arrow] node [left] {} (s3);
%         \path (fs4) edge[bend right, arrow] node [left] {} (s4);
\end{tikzpicture}

        
\end{document}

}
  \caption{\small Grafo bipartito de factorizaci\'on del modelo \texttt{TrueSkill} (Caso 2 vs 2)}
  \label{modelo_trueskill_2vs2}
\end{figure}


\subsubsection{Mensajes descendentes}

\paragraph{$\bm{m_{f_s \rightarrow s}(s)}:$}

\begin{equation}\label{eq:m_fs_s}
\begin{split}
 m_{f_{s_i} \rightarrow s_i}(s_i) & \overset{\hfrac{\text{eq}}{\ref{eq:m_f_v}}}{=} \int \dots \int f_{s_i}(\bm{x}) \prod_{h \in n(f_{s_i}) \setminus \{s_i\} } m_{h \rightarrow f_{s_i}}(h) d\bm{x}_{\setminus \{s_i\}}  \\
& \overset{\hfrac{\text{fig}}{\ref{modelo_trueskill_2vs2}}}{=} \int \dots \int N(s_i| \mu_i, \sigma_i^2) d\bm{x}_{\setminus \{s_i\} } \\
& \underset{*}{\overset{\hfrac{\text{eq}}{\ref{eq:m_f_v}}}{=}} N(s_i| \mu_i, \sigma_i^2)
\end{split}
\end{equation}

Notar que la igualdad señalada $\overset{*}{=}$ vale por definici\'on de la notaci\'on resumen (Eq.~\ref{eq:m_f_v}),

\begin{equation*}
\begin{split}
\bm{x}_{\setminus \{s_i\} } & = \text{arg}(f) \setminus \{s_i\} \\
&= \text{arg}(N(s_i| \mu_i, \sigma_i^2)) \setminus \{s_i\} \\
&= \{s_i\} \setminus \{s_i\} \\
& = \emptyset
\end{split}
\end{equation*}



\paragraph{$\bm{m_{s \rightarrow f_p}(s)}:$}

\begin{equation}\label{eq:m_s_fp}
 m_{s_i \rightarrow f_{p_i}}(s_i) \overset{\hfrac{\text{eq}}{\ref{eq:m_v_f}}}{=} \prod_{g \in n(s_i) \setminus  \{f_{p_i} \}} m_{g \rightarrow s_i} (s_i) \overset{\hfrac{\text{fig}}{\ref{modelo_trueskill_2vs2}}}{=} m_{f_{s_i} \rightarrow s_i}(s_i) \overset{\hfrac{\text{eq}}{\ref{eq:m_fs_s}}}{=}   N(s_i| \mu_i, \sigma_i^2)
\end{equation}



\paragraph{$\bm{m_{f_p \rightarrow p}(p)}:$}

\begin{equation}\label{eq:m_fp_p}
\begin{split}
 m_{f_{p_i} \rightarrow p_i}(p_i) & \overset{\hfrac{\text{eq}}{\ref{eq:m_f_v}}}{=} \int \dots \int f_{p_i}(\bm{x}) \prod_{h \in n(f_{p_i}) \setminus \{p_i\} } m_{h \rightarrow f_{p_i}}(h) d\bm{x}_{\setminus \{p_i\} }  \\
 & \overset{\hfrac{\text{fig}}{\ref{modelo_trueskill_2vs2}}}{\underset{\hfrac{\text{eq}}{\ref{eq:m_s_fp}}}{=}} \int \dots \int N(p_i| s_i, \beta^2) N(s_i| \mu_i, \sigma_i^2) d\bm{x}_{\setminus \{p_i\} } \\[0.3cm]
 & \overset{\hfrac{\text{eq}}{\ref{eq:m_f_v}}}{=} \int N(p_i| s_i, \beta^2) N(s_i| \mu_i, \sigma_i^2) ds_i \\
 & \overset{\hfrac{\text{eq}}{\ref{eq:simetria}}}{=} \int N(s_i|p_i, \beta^2) N(s_i| \mu_i, \sigma_i^2) ds_i \\ 
& \overset{\hfrac{\text{eq}}{\ref{eq:multiplicacion_normales}}}{=} N(p_i|\mu_i,\beta^2 + \sigma_i^2)
\end{split}
\end{equation}

\paragraph{$\bm{m_{p \rightarrow f_t}(p)}:$}

\begin{equation}\label{eq:m_p_ft}
\begin{split}
 m_{p_i \rightarrow f_{t_e}}(p_i) & \overset{\hfrac{\text{eq}}{\ref{eq:m_v_f}}}{=} \prod_{g \in n(p_i) \setminus  \{f_{t_e} \}} m_{g \rightarrow p_i} (p_i) \\ 
 & \overset{\hfrac{\text{fig}}{\ref{modelo_trueskill_2vs2}}}{=} m_{f_{p_i} \rightarrow p_i}(p_i) \overset{\hfrac{\text{eq}}{\ref{eq:m_fp_p}}}{=} N(p_i|\mu_i,\beta^2 + \sigma_i^2)
\end{split}
\end{equation}

\paragraph{$\bm{m_{f_t \rightarrow t}(t)}:$}

\begin{equation}
\begin{split}
 m_{f_{t_e} \rightarrow t_e}(t_e) & \overset{\hfrac{\text{eq}}{\ref{eq:m_f_v}}}{=} \int \dots \int f_{t_e}(\bm{x}) \prod_{h \in n(f_{t_e}) \setminus \{t_e\} } m_{h \rightarrow f_{t_e}}(h) d\bm{x}_{\setminus \{t_e\} }  \\
 & \overset{\hfrac{\text{fig}}{\ref{modelo_trueskill_2vs2}}}{\underset{\hfrac{\text{eq}}{\ref{eq:m_p_ft}}}{=}} \int \dots \int \mathbb{I}(t_e = p_i + p_j) N(p_i|\mu_i,\beta^2 + \sigma_i^2)N(p_j|\mu_j,\beta^2 + \sigma_j^2) d\bm{x}_{\setminus \{t_e\} }\\[0.3cm]
 & \overset{\hfrac{\text{eq}}{\ref{eq:m_f_v}}}{=} \iint \mathbb{I}(t_e = p_i + p_j) N(p_i|\mu_i,\beta^2 + \sigma_i^2)N(p_j|\mu_j,\beta^2 + \sigma_j^2) dp_idp_j \\
 & \overset{\hfrac{\text{eq}}{\ref{eq:integral_con_indicadora}}}{=} \int N(p_i|\mu_i,\beta^2 + \sigma_i^2) N(t_e - p_i|\mu_j,\beta^2 + \sigma_j^2) dp_i   \\
 & \overset{\hfrac{\text{eq}}{\ref{eq:simetria}}}{=} \int N(p_i|\mu_i,\beta^2 + \sigma_i^2) N(p_i|t_e - \mu_j,\beta^2 + \sigma_j^2) dp_i \\
 & \overset{\hfrac{\text{eq}}{\ref{eq:multiplicacion_normales}}}{=} N(t_e|\mu_i+\mu_j,2\beta^2 + \sigma_i^2 + \sigma_j^2)
\end{split}
\end{equation}

\vspace{0.3cm}

General (por inducci\'on)
\begin{equation}\label{eq:m_ft_t}
 m_{f_{t_e} \rightarrow t_e}(t_e) =  N \Big( t_e | \underbrace{\sum_{i\in A_e } \mu_i}_{\hfrac{\text{Habilidad}}{\text{de equipo}} \ \mu_e}, \underbrace{\sum_{i \in A_e} \beta^2 + \sigma_i^2}_{\hfrac{\text{Varianza}}{\text{de equipo}} \ \sigma_e^2} \Big) = N(t_e | \mu_e, \sigma_e^2)
\end{equation}

Ver anexo~\ref{suma_normales_induccion}, la secci\'on sobre suma de $n$ normales.

\paragraph{$\bm{m_{t \rightarrow f_d}(t)}:$}

\begin{equation}\label{eq:m_t_fd}
\begin{split}
m_{t_e \rightarrow f_{d_{k}}}(d_{k}) & \overset{\hfrac{\text{eq}}{\ref{eq:m_v_f}}}{=} \prod_{g \in n(t_e) \setminus  \{f_{d_{k}} \}} m_{g \rightarrow t_e} (t_e) \\[0.3cm] 
 & \overset{\hfrac{\text{fig}}{\ref{modelo_trueskill_2vs2}}}{=} m_{f_{t_e} \rightarrow t_e}(t_e) \overset{\hfrac{\text{eq}}{\ref{eq:m_ft_t}}}{=} N(t_e| \sum_{i \in A_e} \mu_i, \sum_{i \in A_e} \beta^2 + \sigma_i^2) \overset{\hfrac{\text{eq}}{\ref{eq:m_ft_t}}}{=} N(t_e|\mu_e,\sigma_e^2)
\end{split}
\end{equation}

\paragraph{$\bm{m_{f_d \rightarrow d}(d)}:$}

\begin{equation}
 \begin{split}
  m_{f_{d_1} \rightarrow d_1}(d_1) & \overset{\hfrac{\text{eq}}{\ref{eq:m_f_v}}}{=} \int \dots \int f_{d_1}(\bm{x}) \prod_{h \in n(f_{d_1}) \setminus \{d_1\} } m_{h \rightarrow f_{d_1}}(h) \, d\bm{x}_{\setminus \{d_1\} }  \\
  & \overset{\hfrac{\text{fig}}{\ref{modelo_trueskill_2vs2}}}{\underset{\hfrac{\text{eq}}{\ref{eq:m_t_fd}}}{=}} \int \int \mathbb{I}(d_1 = t_a - t_b) N(t_a| \mu_a, \sigma_a^2)  N(t_b| \mu_b, \sigma_b^2)  dt_adt_b \\[0.25cm]
  & \overset{\hfrac{\text{eq}}{\ref{eq:integral_con_indicadora}}}{=} \int N(d_1 + t_b| \mu_a, \sigma_a^2)  N(t_b| \mu_a, \sigma_b^2)  dt_b \\
  & \overset{\hfrac{\text{eq}}{\ref{eq:simetria}}}{=} \int N(t_b| \mu_a - d_1 , \sigma_a^2)  N(t_b| \mu_b, \sigma_b^2)  dt_b \\
  & \overset{\hfrac{\text{eq}}{\ref{eq:multiplicacion_normales}}}{=} N( \mu_a - d_1 | \mu_b, \sigma_a^2 +\sigma_b^2  ) \\
  & \overset{\hfrac{\text{eq}}{\ref{eq:simetria}}}{=} N( d_1 | \mu_a - \mu_b, \sigma_a^2 +\sigma_b^2  )
 \end{split}
\end{equation}

General

\begin{equation} \label{eq:m_fd_d}
 m_{f_{d_1} \rightarrow d_1}(d_1) = N\Bigg(d_1 \ | \ \underbrace{\sum_{i \in A_a} \mu_i - \sum_{i \in A_b} \mu_i}_{\hfrac{\text{Diferencia}}{\text{esperada}} \, (\delta_1)} \ , \  \underbrace{\sum_{i \in A_a \cup A_b} \beta^2 + \sigma_i^2}_{\hfrac{\text{Varianza}}{\text{total}} \, (\vartheta_1) } \Bigg) = N(d_1 | \delta_1, \vartheta_1)
\end{equation}


\subsubsection{Mensajes ascendentes}

\paragraph{$\bm{m_{f_r \rightarrow d}(d)}:$}

\begin{equation}\label{eq:m_fr_d}
\begin{split}
m_{f_r \rightarrow d_1}(d_1) & \overset{\hfrac{\text{fig}}{\ref{modelo_trueskill_2vs2}}}{\underset{\hfrac{\text{eq}}{\ref{eq:m_f_v}}}{=}} \mathbb{I}(d_1 > 0)
\end{split}
\end{equation}


\paragraph{$\bm{m_{d \rightarrow f_d}(d)}:$}
\begin{equation}\label{eq:m_d_fd}
\begin{split}
m_{d_1 \rightarrow f_{d_1}}(d_1) & \overset{\hfrac{\text{eq}}{\ref{eq:m_v_f}}}{=} \prod_{g \in n(d_1) \setminus  \{f_{d_1} \}} m_{g \rightarrow d_1} (d_1) \\[0.3cm] 
 & \overset{\hfrac{\text{fig}}{\ref{modelo_trueskill_2vs2}}}{=} m_{f_r \rightarrow d_1}(d_1) \overset{\hfrac{\text{eq}}{\ref{eq:m_fr_d}}}{=} \mathbb{I}(d_1 > 0)
\end{split}
\end{equation}

\paragraph{$\bm{m_{f_{d_1} \rightarrow t_a}(t_a)}:$} (Caso ganador)
\begin{equation}\label{eq:m_fd_ta}
\begin{split}
m_{f_{d_1} \rightarrow t_a}(t_a) & \overset{\hfrac{\text{eq}}{\ref{eq:m_f_v}}}{=} \int \dots \int f_{d_1}(\bm{x}) \prod_{h \in n(f_{d_1}) \setminus \{t_a\} } m_{h \rightarrow f_{d_1}}(h) \, d\bm{x}_{\setminus \{t_a\} }  \\
&\overset{\hfrac{\text{fig}}{\ref{modelo_trueskill_2vs2}}}{\underset{\hfrac{\text{eq}}{\ref{eq:m_t_fd}}}{=}}  \int \dots \int \mathbb{I}(d_1 = t_a - t_b) \mathbb{I}(d_1 > 0) N(t_b | \mu_b , \sigma_b^2 )  \, d\bm{x}_{\setminus \{t_a\} } \\[0.1cm]
& \overset{\hfrac{\text{eq}}{\ref{eq:m_f_v}}}{=} \iint \mathbb{I}(d_1 = t_a - t_b) \mathbb{I}(d_1 > 0) N(t_b | \mu_b , \sigma_b^2 ) \, dd_1\,dt_b \\
& \overset{\hfrac{\text{eq}}{\ref{eq:integral_con_indicadora}}}{=} \int \mathbb{I}( t_a > t_b)  N(t_b | \mu_b , \sigma_b^2 ) \,dt_b  \\
& \overset{\hfrac{\text{fig}}{\ref{fig:m_fd_t}}}{=} \Phi (t_a| \mu_b, \sigma_b^2)  \overset{\hfrac{\mu_b}{\sigma_b}}{=}  \Phi \Big(t_a| \sum_{i \in A_b} \mu_i , \sum_{i \in A_b} \beta^2 + \sigma_i^2 \Big)
\end{split}
\end{equation}

\paragraph{$\bm{m_{f_{d_1} \rightarrow t_b}(t_b)}:$} (Caso perdedor)
\begin{equation}\label{eq:m_fd_tb}
\begin{split}
m_{f_{d_1} \rightarrow t_b}(t_b) &\overset{\hfrac{\text{eq}}{\ref{eq:m_f_v}}}{=} \int \dots \int f_{d_1}(\bm{x}) \prod_{h \in n(f_{d_1}) \setminus \{t_b\} } m_{h \rightarrow f_{d_1}}(h) \, d\bm{x}_{\setminus \{t_a\} }  \\
&\overset{\hfrac{\text{fig}}{\ref{modelo_trueskill_2vs2}}}{\underset{\hfrac{\text{eq}}{\ref{eq:m_t_fd}}}{=}}  \int \dots \int \mathbb{I}(d_1 = t_a - t_b) \mathbb{I}(d_1 > 0) N(t_a | \mu_a , \sigma_a^2 )  \, d\bm{x}_{\setminus \{t_b\} } \\[0.1cm]
&\overset{\hfrac{\text{eq}}{\ref{eq:m_f_v}}}{=} \iint \mathbb{I}(d_1 = t_a - t_b) \mathbb{I}(d_1 > 0)  N(t_a | \mu_a , \sigma_a^2 )  \, dd_1\,dt_a \\
&\overset{\hfrac{\text{eq}}{\ref{eq:integral_con_indicadora}}}{=} \int \mathbb{I}( t_a > t_b)   N(t_a | \mu_a , \sigma_a^2 ) \,dt_a \\
&\overset{\hfrac{\text{fig}}{\ref{fig:m_fd_t}}}{=} 1 - \Phi (t_b| \mu_a , \sigma_a^2 ) \overset{\hfrac{\mu_a}{\sigma_a}}{=} 1 - \Phi \Big(t_b| \sum_{i \in A_a} \mu_i , \sum_{i \in A_a} \beta^2 + \sigma_i^2 \Big)
\end{split}
\end{equation}

\begin{figure}[H]
\centering
  \begin{subfigure}[t]{0.48\textwidth}
  \includegraphics[width=\textwidth]{imagenes/m_d_ta.pdf}
  \caption{$m_{f_{d_1} \rightarrow t_a}(t_a)$: (Caso ganador)}
  \label{fig:m_fd_ta}
  \end{subfigure}
  \begin{subfigure}[t]{0.48\textwidth}
  \includegraphics[width=\textwidth]{imagenes/m_d_tb.pdf}
  \caption{$m_{f_{d_1} \rightarrow t_b}(t_b)$: (Caso perdedor)}
  \label{fig:m_fd_tb}
  \end{subfigure}
  \caption{Notar que en el caso ganador, $t_a$ es un valor fijo que entra como par\'ametro en la funci\'on $m_{f_{d_1} \rightarrow t_a}(t_a)$. El caso perdedor es an\'alogo.}
  \label{fig:m_fd_t}
\end{figure}

\paragraph{$\bm{m_{t_a \rightarrow f_{t_a}}(t_a)}:$} (Caso ganador)

\begin{equation}\label{eq:m_ta_ft}
\begin{split}
 m_{t_a \rightarrow f_{t_a}}(t_a) \overset{\hfrac{\text{eq}}{\ref{eq:m_v_f}}}{=} \prod_{g \in n(t_a) \setminus  \{f_{t_a} \}} m_{g \rightarrow t_a} (t_a)  \overset{\hfrac{\text{eq}}{\ref{eq:m_fd_ta}}}{=} \Phi(t_a|\mu_b,\sigma_b^2) \overset{\hfrac{\mu_b}{\sigma_b}}{=} \Phi \Big(t_a| \sum_{i \in A_b} \mu_i , \sum_{i \in A_b} \beta^2 + \sigma_i^2 \Big) 
\end{split}
\end{equation}

\paragraph{$\bm{m_{t_b \rightarrow f_{t_b}}(t_b)}:$} (Caso perdedor)
\begin{equation}\label{eq:m_tb_ft}
\begin{split}
 m_{t_b \rightarrow f_{t_b}}(t_b) \overset{\hfrac{\text{eq}}{\ref{eq:m_v_f}}}{=} \prod_{g \in n(t_b) \setminus  \{f_{t_b} \}} m_{g \rightarrow t_b} (t_b)  \overset{\hfrac{\text{eq}}{\ref{eq:m_fd_tb}}}{=} 1- \Phi(t_b|\mu_a,\sigma_a^2) \overset{\hfrac{\mu_a}{\sigma_a}}{=} 1 - \Phi \Big(t_b| \sum_{i \in A_a} \mu_i , \sum_{i \in A_a} \beta^2 + \sigma_i^2 \Big) 
\end{split}
\end{equation}

\paragraph{$\bm{m_{f_{t_a} \rightarrow p_1}(p_1)}:$} (Caso ganador)
\begin{equation}\label{eq:m_fta_p_inicial}
\begin{split}
m_{f_{t_a} \rightarrow p_1}(p_1)  &\overset{\hfrac{\text{eq}}{\ref{eq:m_f_v}}}{=} \int \dots \int f_{t_a}(\bm{x}) \prod_{h \in n(f_{t_a}) \setminus \{p_1\} } m_{h \rightarrow f_{t_a}}(h) \, d\bm{x}_{\setminus \{p_1\} }  \\
&\overset{\hfrac{\text{fig}}{\ref{modelo_trueskill_2vs2}}}{\underset{\hfrac{\text{eq}}{\ref{eq:m_ta_ft}}}{=}} \int \dots \int \mathbb{I}( t_a = p_1 + p_2) N(p_2| \mu_2, \beta^2 + \sigma_2^2 ) \, \Phi (t_a| \mu_b , \sigma_b^2 ) \, d\bm{x}_{\setminus \{p_1\} }\\[0.1cm]
&\overset{\hfrac{\text{eq}}{\ref{eq:m_f_v}}}{=} \iint \mathbb{I}( t_a = p_1 + p_2) \, N(p_2| \mu_2, \beta^2 + \sigma_2^2 ) \, \Phi (t_a| \mu_b , \sigma_b^2 ) \, dt_a dp_2 \\
&\overset{\hfrac{\text{eq}}{\ref{eq:integral_con_indicadora}}}{=} \int  \, N(p_2| \mu_2, \beta^2 + \sigma_2^2 ) \, \Phi (p_1 + p_2| \mu_b , \sigma_b^2 ) \, dp_2 \\
&\overset{\hfrac{\text{eq}}{\ref{eq:simetria}}}{=} \int  N(p_2| \mu_2, \beta^2 + \sigma_2^2 ) \, \Phi (p_1 | \mu_b - p_2 , \sigma_b^2) \, dp_2 \\
&= \kappa(p_1)
\end{split}
\end{equation}

La derivada de la funci\'on de distribución acumulada $\Phi(\,)$ es el valor de la densidad de la función de probabilidad $N(\,)$. Con esta idea en mente, tomamos la derivada de ambos lados de la igualdad:

\begin{equation}\label{eq:ta-p_derivada}
\begin{split}
\frac{\partial\kappa(x)}{\partial x} &= \frac{\partial}{\partial x} \int  N(y| \mu_y, \sigma_y^2 ) \,   \Phi (x | \mu_x -y , \sigma_x^2 ) \, dy \\
&= \int  N(y| \mu_y, \sigma_y^2 ) \, \frac{\partial}{\partial x} \,\Phi (x| \mu_x - y, \sigma_x^2 )  \, dy   \\
& \overset{\hfrac{\text{eq}}{\ref{eq:phi_norm}}}{=} \int  N(y| \mu_y, \sigma_y^2 ) \, N(x| \mu_x -y , \sigma_x^2)  \, dy  \\
& \overset{\hfrac{\text{eq}}{\ref{eq:simetria}}}{=} \int  N(y| \mu_y, \sigma_y^2 ) \, N(y| \mu_x  -x , \sigma_x^2)  \, dy  \\
&\overset{\hfrac{\text{eq}}{\ref{eq:multiplicacion_normales}}}{\underset{\hfrac{\text{eq}}{\ref{eq:simetria}}}{=}} N(x| \mu_x - \mu_y, \sigma_x^2 + \sigma_y^2) 
\end{split}
\end{equation}

Luego

\begin{equation}\label{eq:m_fta_p}
 m_{f_{t_a} \rightarrow p_1}(p_1) \overset{\hfrac{\text{eq}}{\ref{eq:m_fta_p_inicial}}}{\underset{\hfrac{\text{eq}}{\ref{eq:ta-p_derivada}}}{=}}  \Phi(p_1| \mu_b - \mu_2, \beta^2 + \sigma_2^2 + \sigma_b^2)  \overset{\hfrac{\mu_b}{\sigma_b}}{=}  \Phi\Big(p_1| \sum_{i \in A_b} \mu_i - \mu_2, \beta^2 + \sigma_2^2 + \sum_{i \in A_b} \beta^2 + \sigma_i^2 \Big) 
\end{equation}

\paragraph{$\bm{m_{f_{t_b} \rightarrow p_3}(p_3)}:$} (Caso perdedor)
\begin{equation}\label{eq:m_ftb_p}
\begin{split}
m_{f_{t_b} \rightarrow p_3}(p_3)&\overset{\hfrac{\text{eq}}{\ref{eq:m_f_v}}}{=} \int \dots \int f_{t_a}(\bm{x}) \prod_{h \in n(f_{t_a}) \setminus \{p_1\} } m_{h \rightarrow f_{t_a}}(h) \, d\bm{x}_{\setminus \{p_1\} }  \\
&\overset{\hfrac{\text{fig}}{\ref{modelo_trueskill_2vs2}}}{\underset{\hfrac{\text{eq}}{\ref{eq:m_tb_ft}}}{=}} \int \dots \int \mathbb{I}( t_b = p_3 + p_4) \, (1-\Phi (t_b| \mu_a , \sigma_a^2 )) \, N(p_4| \mu_4, \beta^2 + \sigma_4^2 ) \, d\bm{x}_{\setminus \{p_3\} }\\[0.1cm]
&\overset{\hfrac{\text{eq}}{\ref{eq:m_f_v}}}{=} \iint \mathbb{I}( t_b = p_3 + p_4) N(p_4| \mu_4, \beta^2 + \sigma_4^2 )  (1 - \Phi (t_b| \mu_a , \sigma_a^2) )\, dt_b dp_4 \\
&\overset{\hfrac{\text{eq}}{\ref{eq:integral_con_indicadora}}}{\underset{\hfrac{\text{eq}}{\ref{eq:simetria}}}{=}} \int N(p_4| \mu_4, \beta^2 + \sigma_4^2 )  (1 - \Phi (p_3 | \mu_a - p_4 , \sigma_a^2 ) ) \,  dp_4 \\
& =  \underbrace{\int N(p_4| \mu_4, \beta^2 + \sigma_4^2 )dp_4}_{1}  -  \underbrace{\int N(p_4| \mu_4, \beta^2 + \sigma_4^2 ) \Phi (p_3 | \mu_a - p_4 , \sigma_a^2 ) ) \, dp_4}_{\kappa(p_3)} \\
&\overset{\hfrac{\text{eq}}{\ref{eq:ta-p_derivada}}}{=} 1 - \Phi(p_3, \mu_a  - \mu_4, \beta^2 + \sigma_4^2 + \sigma_a^2)\\[0.1cm]
&\overset{\hfrac{\mu_a}{\sigma_a}}{=} 1 - \Phi\Big(p_3, \sum_{i \in A_a} \mu_i  - \mu_4, \beta^2 + \sigma_4^2 + \sum_{i \in A_a} \beta^2 + \sigma_i^2  \Big)
\end{split}
\end{equation}

\paragraph{$\bm{m_{p_1 \rightarrow f_{p_1}}(s_1)}:$} (Caso ganador)

\begin{equation}\label{eq:m_p1_fp}
\begin{split}
 m_{p_1 \rightarrow f_{p_1}}(p_1) \overset{\hfrac{\text{eq}}{\ref{eq:m_v_f}}}{=} \prod_{g \in n(p_1) \setminus  \{f_{p_1} \}} m_{g \rightarrow p_1} (p_1)  \overset{\hfrac{\text{eq}}{\ref{eq:m_fta_p}}}{=}  \Phi(p_1| \mu_b - \mu_2, \beta^2 + \sigma_2^2 + \sigma_b^2) 
\end{split}
\end{equation}


\paragraph{$\bm{m_{p_3 \rightarrow f_{p_3}}(s_1)}:$} (Caso perdedor)

\begin{equation}\label{eq:m_p3_fp}
\begin{split}
 m_{p_3 \rightarrow f_{p_3}}(p_3) \overset{\hfrac{\text{eq}}{\ref{eq:m_v_f}}}{=} \prod_{g \in n(p_3) \setminus  \{f_{p_3} \}} m_{g \rightarrow p_3} (p_3)  \overset{\hfrac{\text{eq}}{\ref{eq:m_ftb_p}}}{=}  1 - \Phi(p_3, \mu_a  - \mu_4, \beta^2 + \sigma_4^2 + \sigma_a^2) 
\end{split}
\end{equation}

\paragraph{$\bm{m_{f_{p_1} \rightarrow s_1}(s_1)}:$} (Caso ganador)
\begin{equation}\label{eq:m_fp_s1}
\begin{split}
m_{f_{p_1} \rightarrow s_1}(s_1) & \overset{\hfrac{\text{eq}}{\ref{eq:m_f_v}}}{=} \int \dots \int f_{p_1}(\bm{x}) \prod_{h \in n(f_{p_1}) \setminus \{s_1\} } m_{h \rightarrow f_{p_1}}(h) \, d\bm{x}_{\setminus \{s_1\} }  \\
&\overset{\hfrac{\text{fig}}{\ref{modelo_trueskill_2vs2}}}{\underset{\hfrac{\text{eq}}{\ref{eq:m_p1_fp}}}{=}} \int \dots \int N(p_1| s_1, \beta^2) \, \Phi(p_1| \mu_b - \mu_2, \beta^2 + \sigma_2^2 + \sigma_b^2 ) \, d\bm{x}_{\setminus \{s_1\} }
\\[0.1cm]
& \overset{\hfrac{\text{eq}}{\ref{eq:m_f_v}}}{\underset{\hfrac{\text{eq}}{\ref{eq:simetria}}}{=}} \int N(p_1| s_1, \beta^2) \, \Phi(\mu_2| \mu_b -  p_1, \beta^2 + \sigma_2^2 + \sigma_b^2) \, dp_1 \\[0.1cm]
&\overset{\hfrac{\text{eq}}{\ref{eq:ta-p_derivada}}}{=} \Phi(s_1| \mu_b - \mu_2, 2\beta^2 + \sigma_2^2 + \sigma_b^2)  
\end{split}
\end{equation}

General (N vs N)
\begin{equation}\label{eq:m_fp_s1_gral}
\begin{split}
m_{f_{p_1} \rightarrow s_1}(s_1) & \overset{\hfrac{\text{eq}}{\ref{eq:m_fp_s1}}}{=} \Phi(s_1| \mu_b - \mu_a + \mu_1, \sigma_b^2 +\sigma_a^2 - \sigma_1^2 )  \\
& \overset{\hfrac{\mu_b}{\sigma_b}}{=} \Phi\Big(s_1| \underbrace{\sum_{i \in A_b} \mu_i - \sum_{i \in A_a} \mu_i }_{-\hfrac{\text{Diferencia}}{\text{esperada}} \, -\delta_{ab} = \delta_{ba} } + \mu_1 , \underbrace{\sum_{i \in A_b \cup A_a} \beta^2 + \sigma_i^2}_{\hfrac{\text{Varianza}}{\text{total}} \, \vartheta^2} - \sigma_1^2   \Big) \\
& \overset{\hfrac{\delta}{\vartheta}}{=} \Phi(s_1|-\delta_{ab} + \mu_1,\vartheta^2-\sigma_1^2) \\
& \overset{\hfrac{\text{eq}}{\ref{eq:simetria}}}{=} 1- \Phi(0| \underbrace{\delta_{ab} - \mu_1 + s_1}_{\hfrac{\text{Diferencia esperada}}{\text{parametrizada}} \, \delta_1(s_1)},\underbrace{\vartheta^2-\sigma_1^2}_{\vartheta_1^2}) \\
& \overset{\hfrac{\delta_1}{\vartheta_1}}{=} 1- \Phi(0|\delta_1(s_1),\vartheta_1^2) \\
&\overset{\hfrac{\text{eq}}{\ref{eq:estandarizar}}}{=} 1- \Phi\Big(\frac{0-\delta_1(s_1)}{\vartheta_1}\Big)\\
&\overset{\hfrac{\text{eq}}{\ref{eq:simetria}}}{=} \Phi\Big(\frac{\delta_1(s_1)}{\vartheta_1}\Big)
\end{split}
\end{equation}

\textbf{Nota}: el mensaje $m_{f_{p_1} \rightarrow s_1}(s_1)$ computa la ``probabilidad de ganar parametrizada'', esto es la probabilidad de ganar si conociéramos la habilidad del jugador. Si conocemos la habilidad del jugador entonces hay que eliminar la varianza de la distribución de creencias de la varianza total (lo que hacemos en $\vartheta_1$) y hay que remplazar la media de la distribución de creencias por la verdadera habilidad en la diferencia esperada (lo que hacemos en $\delta_1(s_1)$).

\paragraph{$\bm{m_{f_{p_3} \rightarrow s_3}(s_3)}:$}

\begin{equation}\label{eq:m_fp_s3}
\begin{split}
m_{f_{p_3} \rightarrow s_3}(s_3) & \overset{\hfrac{\text{eq}}{\ref{eq:m_f_v}}}{=} \int \dots \int f_{p_3}(\bm{x}) \prod_{h \in n(f_{p_3}) \setminus \{s_3\} } m_{h \rightarrow f_{p_3}}(h) \, d\bm{x}_{\setminus \{s_3\} }  \\
&\overset{\hfrac{\text{fig}}{\ref{modelo_trueskill_2vs2}}}{\underset{\hfrac{\text{eq}}{\ref{eq:m_p3_fp}}}{=}} \int \dots \int N(p_3| s_3, \beta^2) (1 - \Phi(p_3, \mu_a  - \mu_4, \beta^2 + \sigma_4^2 + \sigma_a^2)) \, d\bm{x}_{\setminus \{s_3\} }\\
& \overset{\hfrac{\text{eq}}{\ref{eq:m_f_v}}}{\underset{\hfrac{\text{eq}}{\ref{eq:simetria}}}{=}} \int N(p_3| s_3, \beta^2) (1 - \Phi(\mu_4, \mu_a  - p_3, \beta^2 + \sigma_4^2 + \sigma_a^2)) \, dp_3 \\
&=\int N(p_3| s_3, \beta^2) \, dp_3 -  \int N(p_3| s_3, \beta^2)  \Phi(\mu_4, \mu_a  - p_3, \beta^2 + \sigma_4^2 + \sigma_a^2) \, dp_3 \\
&\overset{\hfrac{\text{eq}}{\ref{eq:ta-p_derivada}}}{=} 1 - \Phi\Big(s_3| \mu_a-  \mu_4, 2\beta^2 + \sigma_4^2 + \sigma_a^2 \Big)
\end{split}
\end{equation}

General (N vs N)

\begin{equation}
\begin{split}
m_{f_{p_3} \rightarrow s_3}(s_3) & \overset{\hfrac{\text{eq}}{\ref{eq:m_fp_s3}}}{=} 1 - \Phi\Big(s_3| \underbrace{\mu_a-\mu_b}_{-\delta_{ba}}+\mu_3, \underbrace{\sigma_a^2 + \sigma_b^2}_{\vartheta^2} - \sigma_3^2 \Big) \\
& \overset{\hfrac{\delta}{\vartheta}}{=} 1 - \Phi\Big(s_3| -\delta_{ba}+\mu_3, \vartheta^2- \sigma_3^2 \Big) \overset{\hfrac{\text{eq}}{\ref{eq:simetria}}}{=} \Phi\Big(0| \underbrace{\delta_{ba}-\mu_3+s_3}_{\delta_3(s_3)}, \underbrace{\vartheta^2- \sigma_3^2}_{\vartheta_3^2} \Big) \\
& \overset{\hfrac{\delta_3}{\vartheta_3}}{=} \Phi(0|\delta_3(s_3),\vartheta_3^2)  \overset{\hfrac{\text{eq}}{\ref{eq:estandarizar}}}{=}  \Phi\left(\frac{0-\delta_3(s_3)}{\vartheta_3}\right) \\
& \overset{\hfrac{\text{eq}}{\ref{eq:simetria}}}{=} \Phi\Big(\frac{-\delta_3(s_3)}{\vartheta_3}\Big)
\end{split}
\end{equation}

\textbf{Nota}: el mensaje $m_{f_{p_3} \rightarrow s_3}(s_3)$ computa la ``probabilidad de perder parametrizada'' (que es la misma que la probabilidad de ganar de su contrincante). Si conociéramos la habilidad del jugador entonces hay que eliminar la varianza de la distribución de creencias de la varianza total (lo que hacemos en $\vartheta_3$) y hay que remplazar la media de la distribución de creencias por la verdadera habilidad en la diferencia esperada (lo que hacemos en $\delta_3(s_3)$).

\subsubsection{Posterior ganador y perdedor}
\paragraph{Ganador}
\begin{equation}\label{eq:posterior_ganador}
 p(s_1|o,A) \overset{\hfrac{\text{eq}}{\ref{eq:marginal}}}{=} \prod_{h \in n(x_i)} m_{h \rightarrow x_i} \overset{\hfrac{\text{fig}}{\ref{modelo_trueskill_2vs2}}}{\underset{\hfrac{\text{eq}}{\ref{eq:m_fp_s1_gral}}}{=}}  N(s_1| \mu_1, \sigma_1^2)  \Phi\left(\frac{\delta_1(s_1)}{\vartheta_1}\right)
\end{equation}


\begin{figure}[H]
\centering
  \begin{subfigure}[t]{0.48\textwidth}
  \includegraphics[page=1,width=\textwidth]{imagenes/posterior_ganador.pdf}
  \caption{}
  \label{posterior_ganador_image}
  \end{subfigure}
  \begin{subfigure}[t]{0.48\textwidth}
  \includegraphics[page=2,width=\textwidth]{imagenes/posterior_ganador.pdf}
  \caption{}
  \label{posterior_ganador_distribution}
  \end{subfigure}
  \caption{Posterior ganador}
  \label{posterior_ganador}
\end{figure}

\paragraph{Perdedor}

\begin{equation}\label{eq:posterior_perdedor}
 p(s_1|o,A) \overset{\hfrac{\text{eq}}{\ref{eq:marginal}}}{=} \prod_{h \in n(x_i)} m_{h \rightarrow x_i} \overset{\hfrac{\text{fig}}{\ref{modelo_trueskill_2vs2}}}{\underset{\hfrac{\text{eq}}{\ref{eq:m_fp_s1_gral}}}{=}}  N(s_1| \mu_1, \sigma_1^2)  \Phi\left(\frac{-\delta_1(s_1)}{\vartheta_1}\right)
\end{equation}


\begin{figure}[H]
\centering
  \begin{subfigure}[t]{0.48\textwidth}
  \includegraphics[page=1,width=\textwidth]{imagenes/posterior_perdedor.pdf}
  \caption{}
  \label{posterior_perdedor_image}
  \end{subfigure}
  \begin{subfigure}[t]{0.48\textwidth}
  \includegraphics[page=2,width=\textwidth]{imagenes/posterior_perdedor.pdf}
  \caption{}
  \label{posterior_perdedor_distribution}
  \end{subfigure}
  \caption{Posterior perdedor}
  \label{posterior_perdedor}
\end{figure}

\section{Aproximaci\'on de la posterior}

La probabilidad de una diferencia cuando se conoce el resultado (Eq.~\ref{eq:p_d}) es una Normal truncada.

\begin{equation}\label{eq:p_d}
\begin{split}
 P(d_1) & \overset{\hfrac{\text{eq}}{\ref{eq:marginal}}}{=}   \prod_{h \in n(d_1)} m_{h \rightarrow d_1} \overset{\hfrac{\text{fig}}{\ref{modelo_trueskill_2vs2}}}{=} m_{f_{d_1} \rightarrow d_1}(d_1) \, m_{f_r \rightarrow d_1}(d_1) \overset{\hfrac{\text{eq}}{\ref{eq:m_fr_d}}}{\underset{\hfrac{\text{eq}}{\ref{eq:m_d_fd}}}{=}}  m_{f_{d_1} \rightarrow d_1}(d_1) \, m_{d_1 \rightarrow f_{d_1}}(d_1)  \\
 & = N(d_1|\delta_1,\vartheta_1) \mathbb{I}(d_1 > 0)
\end{split}
\end{equation}


Para tener una posterior normal lo que se hace es un buscar la normal que m\'as se aproxima a esta normal truncada

\vspace{0.3cm}

Se sabe que la Normal que mejor aproxima a una Normal truncada tiene como esperanza

\begin{equation}\label{eq:mean_aprox_double}
 E(X| a < X < b) = \mu + \sigma \frac{N(\frac{a-\mu}{\sigma}) - N(\frac{b-\mu}{\sigma}) }{\Phi(\frac{b-\mu}{\sigma}) - \Phi(\frac{a-\mu}{\sigma}) } = \mu + \sigma \frac{N(\alpha) - N(\beta) }{\Phi(\beta) - \Phi(\alpha) }
\end{equation}

done $\beta = \frac{b-\mu}{\sigma}$ y $\alpha = \frac{a-\mu}{\sigma}$.

Y la varianza 

\begin{equation}\label{eq:variance_aprox_double}
 V(X| a < X < b) = \sigma^2 \Bigg( 1 + \bigg(\frac{\alpha N(\alpha) - \beta N(\beta) }{\Phi(\beta) - \Phi(\alpha) }\bigg) - \bigg(\frac{N(\alpha) - N(\beta) }{\Phi(\beta) - \Phi(\alpha) }\bigg)^2 \Bigg)
\end{equation}

En el caso de que trabajemos con un \'unico truncamiento, estas funciones se pueden simplificar como sigue

\begin{equation}\label{eq:mean_aprox_}
 E(X|  X > a) = \mu + \sigma \frac{N(\alpha)}{1 - \Phi(\alpha) }
\end{equation}

\begin{equation}\label{eq:mean_aprox_}
 V(X|  X > a) = \sigma^2 \Bigg( 1 + \bigg(\frac{\alpha N(\alpha)}{1 - \Phi(\alpha) }\bigg) - \bigg(\frac{N(\alpha)}{1 - \Phi(\alpha) }\bigg)^2 \Bigg)
\end{equation}

Luego, la normal aproximada es 

\begin{equation}\label{p*_d}
 \widehat{P}(d_1) = N\Bigg(d1 \,  \bigg| \,  \mu + \sigma \frac{N(\alpha)}{1 - \Phi(\alpha) } \, , \,  \sigma^2 \bigg( 1 + \bigg(\frac{\alpha N(\alpha)}{1 - \Phi(\alpha) }\bigg) - \bigg(\frac{N(\alpha)}{1 - \Phi(\alpha) }\bigg)^2 \bigg) \, \Bigg)
\end{equation}

Y el mensaje aproximado es 

\begin{equation}\label{eq:m_d_fd_aprox}
 \widehat{m}_{d_1 \rightarrow f_{d_1}}(d_1)  = \frac{\widehat{P}(d_1)}{ m_{f_{d_1} \rightarrow d_1}(d_1)}
\end{equation}

Este mensaje aproximado es el resultado de la divisi\'on de dos Normales.
Seg\'un mis calculos, la divisi\'on de Normales no se simplifica en otra Normal.
Seg\'un Wolfram alpha, la divisi\'on de Normales se trasforma en una distribuci\'on de Cauchy. Es decir, 

\begin{equation}\label{eq:m_d_fd_aprox}
 \widehat{m}_{d_1 \rightarrow f_{d_1}}(d_1)  = \frac{\widehat{P}(d_1)}{ m_{f_{d_1} \rightarrow d_1}(d_1)} = Cauchy() \neq N()
\end{equation}


\vspace{0.3cm}

Sin embargo, en el paper de TrueSkill este mensaje con una distribuci\'on Normal.
Veamos que distribuci\'on Normal proponen.

\subsection{Descomponiendo la ecuaci\'on de actualizaci\'on}


\begin{equation}
 \widehat{m}_{d_1 \rightarrow f_{d_1}}(d_1) = N^*(\tau_d^\text{\tiny new},\pi_d^\text{\tiny new}) =  N(\frac{\tau_d^\text{\tiny new}}{\pi_d^\text{\tiny new}} \, , \,  \frac{1}{\pi_d^\text{\tiny new}})
\end{equation}






\end{document}
