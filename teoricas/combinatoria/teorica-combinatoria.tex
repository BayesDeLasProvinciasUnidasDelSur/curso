\documentclass[shownotes,aspectratio=169]{beamer}

\input{../../aux/tex/diapo_encabezado.tex}
\input{../../aux/tex/tikzlibrarybayesnet.code.tex}
 \mode<presentation>
 {
 %   \usetheme{Madrid}      % or try Darmstadt, Madrid, Warsaw, ...
 %   \usecolortheme{default} % or try albatross, beaver, crane, ...
 %   \usefonttheme{serif}  % or try serif, structurebold, ...
  \usetheme{Antibes}
  \setbeamertemplate{navigation symbols}{}
 }
 
\usepackage{todonotes}
\setbeameroption{show notes}

\newif\ifen
\newif\ifes
\newcommand{\en}[1]{\ifen#1\fi}
\newcommand{\es}[1]{\ifes#1\fi}
\estrue

%\title[Bayes del Sur]{}

\begin{document}

\color{black!85}
\large

 
%\setbeamercolor{background canvas}{bg=gray!15}

\begin{frame}[plain,noframenumbering]
 
 \begin{textblock}{90}(03,05)
 \centering \huge  \textcolor{black!40}{Creencias, datos y sorpresas}
\end{textblock}

 \begin{textblock}{47}(113,74)
\centering \Large  \textcolor{white!55}{Combinatoria}
\end{textblock}

 %\vspace{2cm}brown
%\maketitle
\Wider[2cm]{
\includegraphics[width=1\textwidth]{../../aux/static/peligro_predador}
}
\end{frame}

\begin{frame}[plain]
 
 \centering
 Resumir clase anterior
 
\end{frame}

\begin{frame}[plain]
 
 \centering
 Recordar que el likelihood no es otra cosa que contar caminos.
 
\end{frame}

\begin{frame}[plain]
 
 \centering
 Imagen de niña contanto contando
 
\end{frame}

\begin{frame}[plain]
\begin{textblock}{160}(0,4)
 \centering \LARGE 
 \es{Conjunto de pares ordenados}
 \end{textblock}
 \vspace{1cm}

 Dados dos conjuntos A y B, donde A tiene k elementos y B tiene m ele-
mentos, queremos determinar cuántos elementos tiene A $\times$ B.

\begin{align*}
\underbrace{m + m + \dots + m}_{k \text{ veces}} = km
\end{align*}
 
\end{frame}

\begin{frame}[plain]
\begin{textblock}{160}(0,4)
 \centering \LARGE 
 \es{Conjunto de pares ordenados}
 \end{textblock}
 \vspace{1cm}

Dados los conjuntos $A_1 , A_2 , \dots , A_n$ , donde $\#A_i = k_i$, ¿cuántos
elementos tiene $A_1 \times A_2 \times \dots A_n$?


\begin{align*}
k_1 k_2 \dots k_n
\end{align*}

\end{frame}

\begin{frame}[plain]
 \begin{textblock}{160}(0,4)
 \centering \LARGE 
 \es{Conjunto de funciones}
 \end{textblock}
 \vspace{1cm}
 
 Si $A$ es un conjunto de $n$ elementos y $B$ es un conjunto de $k$ elementos,
¿cuántas funciones de $A$ en $B$ se pueden definir?
 
\begin{align*}
 (f(a_1),f(a_2),\dots,f(a_n)) \in \underbrace{B \times B \times \dots \times B}_{n \text{ factores}}
\end{align*}

Luego

\begin{align*}
 k^n
\end{align*}


\end{frame}


\begin{frame}[plain]
 \begin{textblock}{160}(0,4)
 \centering \LARGE 
 \es{Bolitas numeradas en cajas numeradas}
 \end{textblock}
 \vspace{1cm}
 \centering
 
 ¿De cuántas maneras se pueden ubicar n bolitas numeradas en k cajas numeradas?
 
 \vspace{0.3cm}
 
 Notemos que esto es lo mismo que determinar cuántas funciones hay del tipo 
 
 \begin{align*}
  f(i) = \text{número de caja donde está ubicada la bolita } i
 \end{align*}
 
 Luego, hay $k^n$ formas de ubicar las bolitas.

 \end{frame}

 \begin{frame}[plain]
 \begin{textblock}{160}(0,4)
 \centering \LARGE 
 \es{Funciones inyectivas}
 \end{textblock}
 \vspace{1cm}
 \centering
 
 Si $A$ es un conjunto de $n$ elementos y $B$ es un conjunto de $m$ elementos, con $n \geq m$
¿cuántas funciones inyectivas de $A$ en $B$ se pueden definir?
 
\begin{align*}
 (f(a_1),f(a_2),\dots,f(a_n)) \in \underbrace{B \times B \times \dots \times B}_{n \text{ factores}} \text{ que tienen todas sus coordenadas distintas}
\end{align*}

Luego

\begin{align*}
 m(m-1)(m-2)\dots(m-(n-1)) = m(m-1)(m-2)\dots(m-n+1) = \frac{m!}{(m-n)!}
\end{align*}

 
 \end{frame}

 
 \begin{frame}[plain]
  
  ¿De cuántas maneras se pueden ubicar n bolitas numeradas en m cajas
numeradas de manera que haya a lo sumo una bolita por caja?


\begin{align*}
 m(m-1)(m-2)\dots(m-(n-1)) = m(m-1)(m-2)\dots(m-n+1) = \frac{m!}{(m-n)!}
\end{align*}

 \end{frame}


\begin{frame}[plain]
\centering
  \includegraphics[width=0.35\textwidth]{../../aux/static/pachacuteckoricancha.jpg}
\end{frame}



\end{document}



