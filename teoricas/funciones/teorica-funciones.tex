\documentclass[shownotes,aspectratio=169]{beamer}

\input{../../aux/tex/diapo_encabezado.tex}
\input{../../aux/tex/tikzlibrarybayesnet.code.tex}
 \mode<presentation>
 {
 %   \usetheme{Madrid}      % or try Darmstadt, Madrid, Warsaw, ...
 %   \usecolortheme{default} % or try albatross, beaver, crane, ...
 %   \usefonttheme{serif}  % or try serif, structurebold, ...
  \usetheme{Antibes}
  \setbeamertemplate{navigation symbols}{}
 }
 
\usepackage{todonotes}
\setbeameroption{show notes}

\newif\ifen
\newif\ifes
\newcommand{\en}[1]{\ifen#1\fi}
\newcommand{\es}[1]{\ifes#1\fi}
\estrue

%\title[Bayes del Sur]{}

\begin{document}

\color{black!85}
\large

 
%\setbeamercolor{background canvas}{bg=gray!15}

\begin{frame}[plain,noframenumbering]
 
 \begin{textblock}{90}(03,05)
 \centering \huge  \textcolor{black!40}{Creencias, datos y sorpresas}
\end{textblock}

 \begin{textblock}{47}(113,74)
\centering \Large  \textcolor{white!55}{Funciones}
\end{textblock}

 %\vspace{2cm}brown
%\maketitle
\Wider[2cm]{
\includegraphics[width=1\textwidth]{../../aux/static/peligro_predador}
}
\end{frame}

\begin{frame}[plain]
\begin{textblock}{160}(0,4)
 \centering \LARGE Intersubjetividad
\end{textblock}
\vspace{0.75cm}

 \begin{mdframed}[backgroundcolor=black!20]
 \centering
  Conoce de tal manera que puedas poner en
  
\textbf{correspondencia un\'ivoca} los fenómenos percibidos

por tu conciencia con algún esquema de operación

\textbf{que sea públicamente inteligible y reproducible}.
 \end{mdframed}

\end{frame}

\begin{frame}[plain]
\begin{textblock}{160}(0,4)
 \centering \LARGE Función matemática
\end{textblock}
\vspace{1.25cm}
 
\begin{quote}
 Operación $f$ que asigna a cada elemento de un conjunto (dominio) $X$
un único elemento de otro conjunto (imagen) $Y$.
\end{quote}

\begin{equation}
 f(x)=y
\end{equation}

\pause

 \begin{figure}[H]
    \centering
    \begin{subfigure}[t]{0.3\textwidth}
      \includegraphics[width=\textwidth]{../../aux/static/funcionMatematica.png}
            \caption*{$x \ \ \  \overset{f}{\longmapsto} \ \ \  y $}
    \end{subfigure}
\end{figure}
 
\end{frame}

\begin{frame}[plain]
\begin{textblock}{160}(0,4)
 \centering \LARGE Función matemática
\end{textblock}
\vspace{1.25cm}

AGREGAR ejemplos 

- $x + 2$

- $3x^2-1$

- $sin$

- $2^x$

- $1/x$

- $log x$

\end{frame}


\begin{frame}
 
 \begin{center}
  Pensar ejemplos de relaciones entre conjuntos 
  
  que sean y que no sean funciones
 \end{center}

 
\end{frame}

\begin{frame}[plain]
 \begin{textblock}{160}(0,4)
 \centering \LARGE
 \en{Data as emprirical functions}
 \es{Los datos como funciones empíricas}
\end{textblock}
\vspace{0.75cm}

\begin{textblock}{160}(0,20)
\begin{equation*}
 f(x) = y
\end{equation*}
\end{textblock}

\begin{textblock}{160}(43,33) 
\begin{itemize}
 \item[$x$] 
    \textbf{\en{Unit of analysis}\es{Unidad de análisis}} (UA)
 \item[$f$] 
   \en{\textbf{Variable} of the unit of analysis}
   \es{\textbf{Variable} de la unidad de análisis} (V)
 \item[$y$] 
   \en{\textbf{Value} of the variable}
   \es{\textbf{Resultado} o valor de la variable} (R)
\end{itemize}
\end{textblock}


\only<2>{
\begin{textblock}{160}(0,65) \centering
 \emph{Altura}(Gustavo) = $1.78$m
\end{textblock}
}

\only<3>{
\begin{textblock}{160}(0,65) \centering
 \emph{Ideología}(Partido Obrero) = Izquierda
\end{textblock}
}

\only<4>{
\begin{textblock}{160}(0,65) \centering
 \emph{Habilidad}(Maradona) $>$ \emph{Habilidad}(Messi)
\end{textblock}
}

\only<5>{
\begin{textblock}{140}(10,60)
\begin{framed} \centering
   \en{The meaning of data is implicit in their \textbf{operationalization}}
   \es{El significado preciso de la función depende de la \textbf{operacionalización}}
   \end{framed}
\end{textblock}
}
\end{frame}

\begin{frame}[plain]
 \begin{textblock}{160}(0,4)
 \centering \LARGE
 Los modelos causales como funciones teóricas
\end{textblock}
\vspace{0.75cm}

\begin{align*}
 q &= b_1p + d_1i + u_1 \\
 p &= b_2p + d_2w + u_2
\end{align*}

\vspace{0.5cm}

\pause
\centering
\tikz{        
    
    \node[latent] (q) {$Q$} ;
    \node[latent, right=of q] (p) {$P$} ;
    
    \node[latent, above= of q] (i) {$I$};
    \node[latent, left=of i,xshift=0.3cm] (u1) {$U_1$};
    
    \node[latent, above= of p] (w) {$W$};
    \node[latent, right=of w, xshift=-0.3cm] (u2) {$U_2$};
    
    \edge {u1} {q};
    \edge {u2} {p};
    \path[->] (q.east) edge[yshift=0.1cm] node[yshift=0.2cm]  {\small $b_2$} ([yshift=0cm]p.west);
    \path[->] (p.west) edge[yshift=-0.1cm] node[yshift=-0.2cm]  {\small $b_1$} ([yshift=0cm]q.east);
    \path[->] (i.south) edge node[yshift=0.1cm,xshift=0.2cm]  {\small $d_1$} ([yshift=0cm]q.north);
    \path[->] (w.south) edge node[yshift=0.1cm,xshift=-0.2cm]  {\small $d_2$} ([yshift=0cm]p.north);
    
}


\end{frame}


\begin{frame}[plain]
 Poblaciones

lorenz 

juego de la vida

\end{frame}


 
\begin{frame}[plain]
\centering
  \includegraphics[width=0.35\textwidth]{../../aux/static/pachacuteckoricancha.jpg}
\end{frame}






\end{document}



