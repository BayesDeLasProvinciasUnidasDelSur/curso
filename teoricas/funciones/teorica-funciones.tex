\documentclass[shownotes,aspectratio=169]{beamer}

\input{../../aux/tex/diapo_encabezado.tex}
% tikzlibrary.code.tex
%
% Copyright 2010-2011 by Laura Dietz
% Copyright 2012 by Jaakko Luttinen
%
% This file may be distributed and/or modified
%
% 1. under the LaTeX Project Public License and/or
% 2. under the GNU General Public License.
%
% See the files LICENSE_LPPL and LICENSE_GPL for more details.

% Load other libraries
\usetikzlibrary{shapes}
\usetikzlibrary{fit}
\usetikzlibrary{chains}
\usetikzlibrary{arrows}

% Latent node
\tikzstyle{latent} = [circle,fill=white,draw=black,inner sep=1pt,
minimum size=20pt, font=\fontsize{10}{10}\selectfont, node distance=1]
% Observed node
\tikzstyle{obs} = [latent,fill=gray!25]
% Invisible node
\tikzstyle{invisible} = [latent,minimum size=0pt,color=white, opacity=0, node distance=0]
% Constant node
\tikzstyle{const} = [rectangle, inner sep=0pt, node distance=0.1]
%state
\tikzstyle{estado} = [latent,minimum size=8pt,node distance=0.4]
%action
\tikzstyle{accion} =[latent,circle,minimum size=5pt,fill=black,node distance=0.4]


% Factor node
\tikzstyle{factor} = [rectangle, fill=black,minimum size=10pt, draw=black, inner
sep=0pt, node distance=1]
% Deterministic node
\tikzstyle{det} = [latent, rectangle]

% Plate node
\tikzstyle{plate} = [draw, rectangle, rounded corners, fit=#1]
% Invisible wrapper node
\tikzstyle{wrap} = [inner sep=0pt, fit=#1]
% Gate
\tikzstyle{gate} = [draw, rectangle, dashed, fit=#1]

% Caption node
\tikzstyle{caption} = [font=\footnotesize, node distance=0] %
\tikzstyle{plate caption} = [caption, node distance=0, inner sep=0pt,
below left=5pt and 0pt of #1.south east] %
\tikzstyle{factor caption} = [caption] %
\tikzstyle{every label} += [caption] %

\tikzset{>={triangle 45}}

%\pgfdeclarelayer{b}
%\pgfdeclarelayer{f}
%\pgfsetlayers{b,main,f}

% \factoredge [options] {inputs} {factors} {outputs}
\newcommand{\factoredge}[4][]{ %
  % Connect all nodes #2 to all nodes #4 via all factors #3.
  \foreach \f in {#3} { %
    \foreach \x in {#2} { %
      \path (\x) edge[-,#1] (\f) ; %
      %\draw[-,#1] (\x) edge[-] (\f) ; %
    } ;
    \foreach \y in {#4} { %
      \path (\f) edge[->,#1] (\y) ; %
      %\draw[->,#1] (\f) -- (\y) ; %
    } ;
  } ;
}

% \edge [options] {inputs} {outputs}
\newcommand{\edge}[3][]{ %
  % Connect all nodes #2 to all nodes #3.
  \foreach \x in {#2} { %
    \foreach \y in {#3} { %
      \path (\x) edge [->,#1] (\y) ;%
      %\draw[->,#1] (\x) -- (\y) ;%
    } ;
  } ;
}

% \factor [options] {name} {caption} {inputs} {outputs}
\newcommand{\factor}[5][]{ %
  % Draw the factor node. Use alias to allow empty names.
  \node[factor, label={[name=#2-caption]#3}, name=#2, #1,
  alias=#2-alias] {} ; %
  % Connect all inputs to outputs via this factor
  \factoredge {#4} {#2-alias} {#5} ; %
}

% \plate [options] {name} {fitlist} {caption}
\newcommand{\plate}[4][]{ %
  \node[wrap=#3] (#2-wrap) {}; %
  \node[plate caption=#2-wrap] (#2-caption) {#4}; %
  \node[plate=(#2-wrap)(#2-caption), #1] (#2) {}; %
}

% \gate [options] {name} {fitlist} {inputs}
\newcommand{\gate}[4][]{ %
  \node[gate=#3, name=#2, #1, alias=#2-alias] {}; %
  \foreach \x in {#4} { %
    \draw [-*,thick] (\x) -- (#2-alias); %
  } ;%
}

% \vgate {name} {fitlist-left} {caption-left} {fitlist-right}
% {caption-right} {inputs}
\newcommand{\vgate}[6]{ %
  % Wrap the left and right parts
  \node[wrap=#2] (#1-left) {}; %
  \node[wrap=#4] (#1-right) {}; %
  % Draw the gate
  \node[gate=(#1-left)(#1-right)] (#1) {}; %
  % Add captions
  \node[caption, below left=of #1.north ] (#1-left-caption)
  {#3}; %
  \node[caption, below right=of #1.north ] (#1-right-caption)
  {#5}; %
  % Draw middle separation
  \draw [-, dashed] (#1.north) -- (#1.south); %
  % Draw inputs
  \foreach \x in {#6} { %
    \draw [-*,thick] (\x) -- (#1); %
  } ;%
}

% \hgate {name} {fitlist-top} {caption-top} {fitlist-bottom}
% {caption-bottom} {inputs}
\newcommand{\hgate}[6]{ %
  % Wrap the left and right parts
  \node[wrap=#2] (#1-top) {}; %
  \node[wrap=#4] (#1-bottom) {}; %
  % Draw the gate
  \node[gate=(#1-top)(#1-bottom)] (#1) {}; %
  % Add captions
  \node[caption, above right=of #1.west ] (#1-top-caption)
  {#3}; %
  \node[caption, below right=of #1.west ] (#1-bottom-caption)
  {#5}; %
  % Draw middle separation
  \draw [-, dashed] (#1.west) -- (#1.east); %
  % Draw inputs
  \foreach \x in {#6} { %
    \draw [-*,thick] (\x) -- (#1); %
  } ;%
}


 \mode<presentation>
 {
 %   \usetheme{Madrid}      % or try Darmstadt, Madrid, Warsaw, ...
 %   \usecolortheme{default} % or try albatross, beaver, crane, ...
 %   \usefonttheme{serif}  % or try serif, structurebold, ...
  \usetheme{Antibes}
  \setbeamertemplate{navigation symbols}{}
 }
 
\usepackage{todonotes}
\setbeameroption{show notes}

\newif\ifen
\newif\ifes
\newcommand{\en}[1]{\ifen#1\fi}
\newcommand{\es}[1]{\ifes#1\fi}
\estrue

%\title[Bayes del Sur]{}

\begin{document}

\color{black!85}
\large

 
%\setbeamercolor{background canvas}{bg=gray!15}

\begin{frame}[plain,noframenumbering]
 
 \begin{textblock}{90}(03,05)
 \centering \huge  \textcolor{black!40}{Creencias, datos y sorpresas}
\end{textblock}

 \begin{textblock}{47}(113,74)
\centering \Large  \textcolor{white!55}{Funciones}
\end{textblock}

 %\vspace{2cm}brown
%\maketitle
\Wider[2cm]{
\includegraphics[width=1\textwidth]{../../aux/static/peligro_predador}
}
\end{frame}

\begin{frame}[plain]
\begin{textblock}{160}(0,4)
 \centering \LARGE Intersubjetividad
\end{textblock}
\vspace{0.75cm}

 \begin{mdframed}[backgroundcolor=black!20]
 \centering
  Conoce de tal manera que puedas poner en
  
\textbf{correspondencia un\'ivoca} los fenómenos percibidos

por tu conciencia con algún esquema de operación

\textbf{que sea públicamente inteligible y reproducible}.
 \end{mdframed}

\end{frame}

\begin{frame}[plain]
\begin{textblock}{160}(0,4)
 \centering \LARGE Función matemática
\end{textblock}
\vspace{1.25cm}
 
\begin{quote}
 Operación $f$ que asigna a cada elemento de un conjunto (dominio) $X$
un único elemento de otro conjunto (imagen) $Y$.
\end{quote}

\begin{equation}
 f(x)=y
\end{equation}

\pause

 \begin{figure}[H]
    \centering
    \begin{subfigure}[t]{0.3\textwidth}
      \includegraphics[width=\textwidth]{../../aux/static/funcionMatematica.png}
            \caption*{$x \ \ \  \overset{f}{\longmapsto} \ \ \  y $}
    \end{subfigure}
\end{figure}
 
\end{frame}

\begin{frame}[plain]
\begin{textblock}{160}(0,4)
 \centering \LARGE Función matemática
\end{textblock}
\vspace{1.25cm}

AGREGAR ejemplos 

- $x + 2$

- $3x^2-1$

- $sin$

- $2^x$

- $1/x$

- $log x$

\end{frame}


\begin{frame}
 
 \begin{center}
  Pensar ejemplos de relaciones entre conjuntos 
  
  que sean y que no sean funciones
 \end{center}

 
\end{frame}

\begin{frame}[plain]
 \begin{textblock}{160}(0,4)
 \centering \LARGE
 \en{Data as emprirical functions}
 \es{Los datos como funciones empíricas}
\end{textblock}
\vspace{0.75cm}

\begin{textblock}{160}(0,20)
\begin{equation*}
 f(x) = y
\end{equation*}
\end{textblock}

\begin{textblock}{160}(43,33) 
\begin{itemize}
 \item[$x$] 
    \textbf{\en{Unit of analysis}\es{Unidad de análisis}} (UA)
 \item[$f$] 
   \en{\textbf{Variable} of the unit of analysis}
   \es{\textbf{Variable} de la unidad de análisis} (V)
 \item[$y$] 
   \en{\textbf{Value} of the variable}
   \es{\textbf{Resultado} o valor de la variable} (R)
\end{itemize}
\end{textblock}


\only<2>{
\begin{textblock}{160}(0,65) \centering
 \emph{Altura}(Gustavo) = $1.78$m
\end{textblock}
}

\only<3>{
\begin{textblock}{160}(0,65) \centering
 \emph{Ideología}(Partido Obrero) = Izquierda
\end{textblock}
}

\only<4>{
\begin{textblock}{160}(0,65) \centering
 \emph{Habilidad}(Maradona) $>$ \emph{Habilidad}(Messi)
\end{textblock}
}

\only<5>{
\begin{textblock}{140}(10,60)
\begin{framed} \centering
   \en{The meaning of data is implicit in their \textbf{operationalization}}
   \es{El significado preciso de la función depende de la \textbf{operacionalización}}
   \end{framed}
\end{textblock}
}
\end{frame}

% \begin{frame}[plain]
%  \begin{textblock}{160}(0,4)
%  \centering \LARGE
%  Los modelos causales como funciones teóricas
% \end{textblock}
% \vspace{0.75cm}
% 
% \begin{align*}
%  q &= b_1p + d_1i + u_1 \\
%  p &= b_2p + d_2w + u_2
% \end{align*}
% 
% \vspace{0.5cm}
% 
% \pause
% \centering
% \tikz{        
%     
%     \node[latent] (q) {$Q$} ;
%     \node[latent, right=of q] (p) {$P$} ;
%     
%     \node[latent, above= of q] (i) {$I$};
%     \node[latent, left=of i,xshift=0.3cm] (u1) {$U_1$};
%     
%     \node[latent, above= of p] (w) {$W$};
%     \node[latent, right=of w, xshift=-0.3cm] (u2) {$U_2$};
%     
%     \edge {u1} {q};
%     \edge {u2} {p};
%     \path[->] (q.east) edge[yshift=0.1cm] node[yshift=0.2cm]  {\small $b_2$} ([yshift=0cm]p.west);
%     \path[->] (p.west) edge[yshift=-0.1cm] node[yshift=-0.2cm]  {\small $b_1$} ([yshift=0cm]q.east);
%     \path[->] (i.south) edge node[yshift=0.1cm,xshift=0.2cm]  {\small $d_1$} ([yshift=0cm]q.north);
%     \path[->] (w.south) edge node[yshift=0.1cm,xshift=-0.2cm]  {\small $d_2$} ([yshift=0cm]p.north);
%     
% }
% 
% 
% \end{frame}

\begin{frame}[plain]
 \begin{textblock}{160}(0,4)
 \centering \LARGE
 Modelos causales deterministas
\end{textblock}
\vspace{0.75cm}

\centering

\tikz{         
    \node[det, fill=black!15] (r) {$r$} ; 
    \node[const, left=of r, xshift=-2.35cm] (r_name) {\small \en{Result}\es{Ganar/perder}:}; 
    \node[const, right=of r] (dr) {\normalsize $ r = (d>0)$}; 

    \node[latent, above=of r, yshift=-0.45cm] (d) {$d$} ; %
    \node[const, right=of d] (dd) {\normalsize $ d=p_i-p_j$}; 
    \node[const, left=of d, xshift=-2.35cm] (s_name) {\small \en{Difference}\es{Diferencia}:};
    
    \node[latent, above=of d, xshift=-0.8cm, yshift=-0.45cm] (p1) {$p_i$} ; %
    \node[latent, above=of d, xshift=0.8cm, yshift=-0.45cm] (p2) {$p_j$} ; %
    \node[const, left=of p1, xshift=-1.55cm] (p_name) {\small \en{Performance}\es{Desempeño}:}; 

    \node[latent, above=of p1,yshift=0.7cm,fill=white] (s1) {$s_i$} ; %
    \node[latent, above=of p2,yshift=0.7cm,fill=white] (s2) {$s_j$} ; %
    
    \node[latent, above=of p1,xshift=-1cm, yshift=-0.45cm,fill=white] (u1) {$u_i$} ; 
    \node[latent, above=of p2,xshift=1cm, yshift=-0.45cm,fill=white] (u2) {$u_j$} ; 
    \node[const, left=of u1, xshift=-0.55cm] (u_name) {\small \en{Others factors}\es{Otros factores}:}; 
    
    \node[const, right=of p2] (dp2) {\normalsize $p = s + u$};

    \node[const, left=of s1, xshift=-1.55cm] (s_name) {\small \en{Skill}\es{Habilidad}:}; 
    
    \edge {d} {r};
    \edge {p1,p2} {d};
    \edge {s1} {p1};
    \edge {s2} {p2};
    \edge {u1} {p1};
    \edge {u2} {p2};
}


\end{frame}


\begin{frame}[plain]
 \begin{textblock}{160}(0,4)
 \centering \LARGE
 Modelos causales deterministas \\  \Large como funciones
\end{textblock}
\vspace{0.75cm}


\begin{textblock}{160}(0,28)
\begin{equation*}
 f(x) = y
\end{equation*}
\end{textblock}

\begin{textblock}{160}(55,45)
\centering
\begin{itemize}
 \item[$x$:] 
    {Condiciones iniciales} 
 \item[$f$:] 
    {Modelo causal determinista}
 \item[$y$:] 
    {Destino inevitable}
\end{itemize}
\end{textblock}



\end{frame}



\begin{frame}[plain]
 \begin{textblock}{160}(0,4)
 \centering \LARGE
 Modelos causales deterministas \\  \Large no lineales
\end{textblock}
\vspace{0.75cm}

\begin{align*}
 \text{Población}(t+1) = r \cdot \text{Población}(t)\cdot (1-\text{Población}(t))
\end{align*}

\vspace{0.5cm}

\pause

\centering

\tikz{         
    \node[latent] (n1) {$x_n$} ; 
    
    \node[latent, right=of n1, xshift=3cm] (n2) {$x_{n+1}$} ;
    
     \path[->] (n1) edge node[yshift=0.5cm] {$rx_n(1-x_n)$} (n2);

}
 
%lorenz 

%juego de la vida

\end{frame}

\begin{frame}[plain]
 \begin{textblock}{160}(0,4)
 \centering \LARGE
 Modelos causales deterministas \\  \Large no lineales
\end{textblock}
\vspace{0.75cm}

\only<1>{
\begin{textblock}{160}(0,22)
\centering
 \includegraphics[page={1},width=0.6\textwidth]{figuras/poblacion.pdf}
\end{textblock}
}

\only<2>{
\begin{textblock}{160}(0,22)
\centering
 \includegraphics[page={2},width=0.6\textwidth]{figuras/poblacion.pdf}
\end{textblock}
}

\only<3>{
\begin{textblock}{160}(0,22)
\centering
 \includegraphics[page={3},width=0.6\textwidth]{figuras/poblacion.pdf}
\end{textblock}
}

\only<4>{
\begin{textblock}{160}(0,22)
\centering
 \includegraphics[page={4},width=0.6\textwidth]{figuras/poblacion.pdf}
\end{textblock}
}

\only<5>{
\begin{textblock}{160}(0,22)
\centering
 \includegraphics[page={5},width=0.6\textwidth]{figuras/poblacion.pdf}
\end{textblock}
}

\only<6>{
\begin{textblock}{160}(0,22)
\centering
 \includegraphics[page={6},width=0.6\textwidth]{figuras/poblacion.pdf}
\end{textblock}
}

\only<7>{
\begin{textblock}{160}(0,22)
\centering
 \includegraphics[page={7},width=0.6\textwidth]{figuras/poblacion.pdf}
\end{textblock}
}

\only<8>{
\begin{textblock}{160}(0,22)
\centering
 \includegraphics[page={8},width=0.6\textwidth]{figuras/poblacion.pdf}
\end{textblock}
}


\only<9>{
\begin{textblock}{160}(0,22)
\centering
 \includegraphics[page={9},width=0.6\textwidth]{figuras/poblacion.pdf}
\end{textblock}
}


\only<10>{
\begin{textblock}{160}(0,22)
\centering
 \includegraphics[page={10},width=0.6\textwidth]{figuras/poblacion.pdf}
\end{textblock}
}

\only<11>{
\begin{textblock}{160}(0,22)
\centering
 \includegraphics[page={11},width=0.6\textwidth]{figuras/poblacion.pdf}
\end{textblock}
}

\end{frame}


\begin{frame}[plain]
 \begin{textblock}{160}(0,4)
 \centering \LARGE
 Modelos causales deterministas \\  \Large caos determinista
\end{textblock}
\vspace{0.75cm}

\only<1>{
\begin{textblock}{160}(0,22)
\centering
 \includegraphics[page={14},width=0.605\textwidth]{figuras/poblacion.pdf}
\end{textblock}
}

\only<2>{
\begin{textblock}{160}(0,22)
\centering
 \includegraphics[page={11},width=0.6\textwidth]{figuras/poblacion.pdf}
\end{textblock}
}

\only<3>{
\begin{textblock}{160}(0,22)
\centering
 \includegraphics[page={12},width=0.6\textwidth]{figuras/poblacion.pdf}
\end{textblock}
}

\only<4>{
\begin{textblock}{160}(0,22)
\centering
 \includegraphics[page={13},width=0.6\textwidth]{figuras/poblacion.pdf}
\end{textblock}
}

\end{frame}


\begin{frame}[plain]
 \begin{textblock}{160}(0,4)
 \centering \LARGE
 Modelos causales deterministas \\  \Large atractores
\end{textblock}
\vspace{0.75cm}

\begin{textblock}{150}(0,15)
\centering
 \includegraphics[width=0.8\textwidth]{figuras/camino_al_caos}
\end{textblock}

\only<2->{
\begin{textblock}{160}(35,38)
 Punto fijo
\end{textblock}
}

\only<3->{
\begin{textblock}{160}(70,38)
 Periódico
\end{textblock}
}

\only<4>{
\begin{textblock}{160}(130,38)
 Caótico
\end{textblock}
}

\end{frame}

\begin{frame}[plain]
 \begin{textblock}{160}(0,4)
 \centering \LARGE
 Efecto mariposa \\  \Large descubrimiento
\end{textblock}
\vspace{1cm}
% 
% At one point I wanted to examine a solution in greater detail,  so I stopped  the  computer  and  typed  in  the  twelve  numbers  from  a  row  that  the computer had printed earlier.  I started the computer again, and went out for a cup of coffee.  When I returned about an hour later, after the computer had generated  about  two  months  of  data,  I  found  that  the  new  solution  did  not agree  with  the  original  one.   At  first  I  suspected  trouble  with  the  computer, which occurred fairly often, but, when I compared the new solution step by step with the older one, I found that at first the solutions were the same, and thenthey would differ by one unit in the last decimal place, and then the differences would become larger and larger, doubling in magnitude in about four simulated days, until, after sixty days, the solutions were unrecognizably different.

\centering
  \includegraphics[width=0.7\textwidth]{figuras/lorenz.pdf}

\end{frame}

\begin{frame}[plain]
 \begin{textblock}{160}(0,4)
 \centering \LARGE
  \includegraphics[width=0.35\textwidth]{../../aux/static/game_of_life.png}
\end{textblock}


 \begin{textblock}{140}(10,26)
 \centering
Conway desarrolla el juego de la vida con el interés \\ de que su evolución sea ``impredecible''.  
 \end{textblock}

 \only<2->{
 \begin{textblock}{140}(10,38)
 Reglas:
 \begin{itemize}
 \item[$\bullet$] Sobreviven las celdas rodeadas por 2 o 3 celdas
 \item[$\bullet$] Mueren las celdas con 4 o más (sobrepoblación)
 \item[$\bullet$] Nacen las celdas rodeadas por exactamente 3
\end{itemize}
\end{textblock}
}

\end{frame}

\begin{frame}[plain]
 \begin{textblock}{150}(0,03)
 \centering
\tikz{     
    \onslide<1->{
    \node[const] (a) {\includegraphics[width=0.2\textwidth]{../../aux/static/game_of_life-forma_fijo.png}};
    \node[const, left=of a,xshift=-0.3cm] (na) {Fijo:};
    }
    
    \onslide<2->{
    \node[const, below=of a] (b) {\includegraphics[width=0.2\textwidth]{../../aux/static/game_of_life-forma_periodico.png}}; 
    \node[const, left=of b,xshift=-0.3cm] (nb) {Periódico:};
    }
    
    \onslide<3->{
    \node[const, below=of b] (c) {\includegraphics[width=0.2\textwidth]{../../aux/static/game_of_life-forma_caminante.png}};
    \node[const, left=of c,xshift=-0.3cm] (nc) {Periódico caminante:};
    }
    
    \onslide<4->{
    \node[const, below=of c] (d) {\includegraphics[width=0.2\textwidth]{../../aux/static/game_of_life-forma_periodico_creciente.png}}; 
    \node[const, left=of d,xshift=-0.3cm] (nd) {Periódico creciente:};
    }
    
    \onslide<5->{
    \node[const, below=of d, yshift=-0.4cm] (e) {\footnotesize \url{playgameoflife.com}};
    \node[const, left=of e,xshift=-0.3cm] (ne) {``Caótico'' finito:};
    }
    
    \onslide<6->{
    \node[const, below=of e, yshift=-0.3cm] (f) {Desconocido};
    \node[const, left=of f, xshift=-0.3cm] (nf) {``Caótico'' infinito: \ \ \ };
    }
}
\end{textblock} 
\end{frame}


\begin{frame}[plain]
\begin{textblock}{160}(0,4)
 \centering \LARGE
Turing completo \\ \Large Not
\end{textblock}
\vspace{1cm}
 
 
\includegraphics[width=0.48\textwidth]{../../aux/static/game_of_life-not_0.png}
\includegraphics[width=0.48\textwidth]{../../aux/static/game_of_life-not_1.png}
 
\end{frame}


\begin{frame}[plain]
\begin{textblock}{160}(0,4)
 \centering \LARGE
Turing completo \\ \Large And
\end{textblock}
\vspace{1cm}
 
\begin{textblock}{170}(0,34) 
\includegraphics[width=0.32\textwidth]{../../aux/static/game_of_life-and_01.png}
\includegraphics[width=0.32\textwidth]{../../aux/static/game_of_life-and_10.png}
\includegraphics[width=0.32\textwidth]{../../aux/static/game_of_life-and_11.png}
\end{textblock}

\end{frame}


\begin{frame}[plain]
\begin{textblock}{160}(0,4)
 \centering \LARGE
Turing completo \\ \Large Memoria
\end{textblock}
\vspace{1cm}
  
  \centering
\includegraphics[width=0.6\textwidth]{../../aux/static/game_of_life-latch.png}


\end{frame}




% 
% \begin{frame}[plain]
%  \begin{textblock}{160}(0,4)
%  \centering
%   \includegraphics[width=0.35\textwidth]{../../aux/static/game_of_life.png}
% \end{textblock}
% 
% Hay patrones estables, peródicos, periódicos crecientes, caoticos finitos, hay caóticos inf
% 
%  \begin{textblock}{160}(0,74)
%  \centering
%  El juego de la vida es turing completo  \\ 
% \url{https://www.alanzucconi.com/2020/10/13/conways-game-of-life/}
%  \end{textblock}
% }

%\end{frame}


\begin{frame}[plain]
\centering
  \includegraphics[width=0.35\textwidth]{../../aux/static/pachacuteckoricancha.jpg}
\end{frame}






\end{document}



