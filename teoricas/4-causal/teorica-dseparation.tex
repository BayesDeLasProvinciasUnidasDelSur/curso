\documentclass[shownotes,aspectratio=169]{beamer}

\input{../../auxiliar/tex/diapo_encabezado.tex}
% tikzlibrary.code.tex
%
% Copyright 2010-2011 by Laura Dietz
% Copyright 2012 by Jaakko Luttinen
%
% This file may be distributed and/or modified
%
% 1. under the LaTeX Project Public License and/or
% 2. under the GNU General Public License.
%
% See the files LICENSE_LPPL and LICENSE_GPL for more details.

% Load other libraries
\usetikzlibrary{shapes}
\usetikzlibrary{fit}
\usetikzlibrary{chains}
\usetikzlibrary{arrows}

% Latent node
\tikzstyle{latent} = [circle,fill=white,draw=black,inner sep=1pt,
minimum size=20pt, font=\fontsize{10}{10}\selectfont, node distance=1]
% Observed node
\tikzstyle{obs} = [latent,fill=gray!25]
% Invisible node
\tikzstyle{invisible} = [latent,minimum size=0pt,color=white, opacity=0, node distance=0]
% Constant node
\tikzstyle{const} = [rectangle, inner sep=0pt, node distance=0.1]
%state
\tikzstyle{estado} = [latent,minimum size=8pt,node distance=0.4]
%action
\tikzstyle{accion} =[latent,circle,minimum size=5pt,fill=black,node distance=0.4]


% Factor node
\tikzstyle{factor} = [rectangle, fill=black,minimum size=10pt, draw=black, inner
sep=0pt, node distance=1]
% Deterministic node
\tikzstyle{det} = [latent, rectangle]

% Plate node
\tikzstyle{plate} = [draw, rectangle, rounded corners, fit=#1]
% Invisible wrapper node
\tikzstyle{wrap} = [inner sep=0pt, fit=#1]
% Gate
\tikzstyle{gate} = [draw, rectangle, dashed, fit=#1]

% Caption node
\tikzstyle{caption} = [font=\footnotesize, node distance=0] %
\tikzstyle{plate caption} = [caption, node distance=0, inner sep=0pt,
below left=5pt and 0pt of #1.south east] %
\tikzstyle{factor caption} = [caption] %
\tikzstyle{every label} += [caption] %

\tikzset{>={triangle 45}}

%\pgfdeclarelayer{b}
%\pgfdeclarelayer{f}
%\pgfsetlayers{b,main,f}

% \factoredge [options] {inputs} {factors} {outputs}
\newcommand{\factoredge}[4][]{ %
  % Connect all nodes #2 to all nodes #4 via all factors #3.
  \foreach \f in {#3} { %
    \foreach \x in {#2} { %
      \path (\x) edge[-,#1] (\f) ; %
      %\draw[-,#1] (\x) edge[-] (\f) ; %
    } ;
    \foreach \y in {#4} { %
      \path (\f) edge[->,#1] (\y) ; %
      %\draw[->,#1] (\f) -- (\y) ; %
    } ;
  } ;
}

% \edge [options] {inputs} {outputs}
\newcommand{\edge}[3][]{ %
  % Connect all nodes #2 to all nodes #3.
  \foreach \x in {#2} { %
    \foreach \y in {#3} { %
      \path (\x) edge [->,#1] (\y) ;%
      %\draw[->,#1] (\x) -- (\y) ;%
    } ;
  } ;
}

% \factor [options] {name} {caption} {inputs} {outputs}
\newcommand{\factor}[5][]{ %
  % Draw the factor node. Use alias to allow empty names.
  \node[factor, label={[name=#2-caption]#3}, name=#2, #1,
  alias=#2-alias] {} ; %
  % Connect all inputs to outputs via this factor
  \factoredge {#4} {#2-alias} {#5} ; %
}

% \plate [options] {name} {fitlist} {caption}
\newcommand{\plate}[4][]{ %
  \node[wrap=#3] (#2-wrap) {}; %
  \node[plate caption=#2-wrap] (#2-caption) {#4}; %
  \node[plate=(#2-wrap)(#2-caption), #1] (#2) {}; %
}

% \gate [options] {name} {fitlist} {inputs}
\newcommand{\gate}[4][]{ %
  \node[gate=#3, name=#2, #1, alias=#2-alias] {}; %
  \foreach \x in {#4} { %
    \draw [-*,thick] (\x) -- (#2-alias); %
  } ;%
}

% \vgate {name} {fitlist-left} {caption-left} {fitlist-right}
% {caption-right} {inputs}
\newcommand{\vgate}[6]{ %
  % Wrap the left and right parts
  \node[wrap=#2] (#1-left) {}; %
  \node[wrap=#4] (#1-right) {}; %
  % Draw the gate
  \node[gate=(#1-left)(#1-right)] (#1) {}; %
  % Add captions
  \node[caption, below left=of #1.north ] (#1-left-caption)
  {#3}; %
  \node[caption, below right=of #1.north ] (#1-right-caption)
  {#5}; %
  % Draw middle separation
  \draw [-, dashed] (#1.north) -- (#1.south); %
  % Draw inputs
  \foreach \x in {#6} { %
    \draw [-*,thick] (\x) -- (#1); %
  } ;%
}

% \hgate {name} {fitlist-top} {caption-top} {fitlist-bottom}
% {caption-bottom} {inputs}
\newcommand{\hgate}[6]{ %
  % Wrap the left and right parts
  \node[wrap=#2] (#1-top) {}; %
  \node[wrap=#4] (#1-bottom) {}; %
  % Draw the gate
  \node[gate=(#1-top)(#1-bottom)] (#1) {}; %
  % Add captions
  \node[caption, above right=of #1.west ] (#1-top-caption)
  {#3}; %
  \node[caption, below right=of #1.west ] (#1-bottom-caption)
  {#5}; %
  % Draw middle separation
  \draw [-, dashed] (#1.west) -- (#1.east); %
  % Draw inputs
  \foreach \x in {#6} { %
    \draw [-*,thick] (\x) -- (#1); %
  } ;%
}


 \mode<presentation>
 {
 %   \usetheme{Madrid}      % or try Darmstadt, Madrid, Warsaw, ...
 %   \usecolortheme{default} % or try albatross, beaver, crane, ...
 %   \usefonttheme{serif}  % or try serif, structurebold, ...
  \usetheme{Antibes}
  \setbeamertemplate{navigation symbols}{}
 }
 
\usepackage{todonotes}
\setbeameroption{show notes}

\newif\ifen
\newif\ifes
\newcommand{\en}[1]{\ifen#1\fi}
\newcommand{\es}[1]{\ifes#1\fi}
\estrue

%\title[Bayes del Sur]{}

\begin{document}

\color{black!85}
\large
 
%\setbeamercolor{background canvas}{bg=gray!15}


\begin{frame}[plain,noframenumbering]
 
 \begin{textblock}{90}(00,05)
\begin{center}
 \huge  \textcolor{black!66}{Creencias adaptativas}
\end{center}
\end{textblock}

 %\vspace{2cm}brown
%\maketitle
\Wider[2cm]{
\includegraphics[width=1\textwidth]{../../auxiliar/static/peligro_predador}
}
\end{frame}

\begin{frame}[plain]
%  \begin{textblock}{160}(0,4)
%  \centering \Large
%  Judea Pearl
%  \end{textblock}
\centering
% \vspace{1cm}
 
\includegraphics[width=0.35\textwidth]{../../auxiliar/static/causality2-cover} 
 
  \normalsize
  \vspace{0.2cm}
  
\Wider[2cm]{\centering In this book, we shall express preference toward Laplace’s deterministic conception of causality}
 
\end{frame}


\begin{frame}[plain]
\begin{textblock}{160}(0,4)
 \centering \Large
 Modelos Causales
 \end{textblock}
 \centering
 \vspace{0.75cm}
 
 \tikz{
    \node[det] (l) {$l$} ; %
    \node[det, above=of l] (a) {$a$} ; %
    \node[det, above=of a,xshift=1.5cm] (t) {$t$} ; %
    \node[det, above=of a,xshift=-1.5cm] (e) {$e$} ; %
    \node[det, below=of t,xshift=1.5cm] (r) {$r$} ; %
    
    \edge {a} {l};
    \edge {t,e} {a};
    \edge {t} {r};
    
    \node[const, left= of e, xshift=-0.1cm] (cpd_e) {Entradera:};
    \node[const, right= of t, xshift=0.1cm] (cpd_t) {:Terremoto};
    \node[const, left= of a, xshift=-0.1cm] (cpd_a) {Alarma:};
    \node[const, right= of r, xshift=0.1cm] (cpd_r) {:Redes};
    \node[const, left= of l, xshift=-0.1cm] (cpd_l) {Llamada:};
    
    }


 
 \end{frame}


 \begin{frame}[plain]
 \begin{textblock}{160}(0,4)
 \centering \Large
 Causalidad
 \end{textblock}

 \centering \normalsize
 Causal relationships are expressed in the form of deterministic, functional equations, and probabilities are introduced through the assumption that certain variables in the equations are unobserved.

\end{frame} 
 
 \begin{frame}[plain]
\begin{textblock}{160}(0,4)
 \centering \Large
 Mecanismos Causales
 \end{textblock}
 \vspace{0.75cm}
 
 \begin{align*}
  \onslide<4->{e &= u_e \\ 
  t &= u_t \\}
  \onslide<1->{r &= f(t,u_r) = t \vee u_r \\}
  \onslide<2->{a &= f(e,t,u_a) = e \vee t \vee u_a \\}
  \onslide<3->{l &= f(a,u_l) = a \vee u_l}
 \end{align*} 
 \end{frame}

\begin{frame}[plain]
\begin{textblock}{160}(0,4)
 \centering \Large
 Probabilidad conjunta
 \end{textblock}
 
\begin{textblock}{160}(0,18)
 \centering

 \tikz{
    \node[det] (l) {$l$} ; %
    \node[det, above=of l] (a) {$a$} ; %
    \node[det, above=of a,xshift=1.5cm] (t) {$t$} ; %
    \node[det, above=of a,xshift=-1.5cm] (e) {$e$} ; %
    \node[det, below=of t,xshift=1.5cm] (r) {$r$} ; %
    
    \edge {a} {l};
    \edge {t,e} {a};
    \edge {t} {r};
    
    \node[const, left= of e, xshift=-0.1cm] (cpd_e) {
    \begin{tabular}{|c|c|}
        \hline
        $e^0$ & $e^1$ \\ \hline
        $0.999$ & $0.001$  \\ \hline
    \end{tabular}
    };
    \node[const, above= of cpd_e] (n_e) {Entradera:};
    
    \node[const, right= of t, xshift=0.1cm] (cpd_t) {
    \begin{tabular}{|c|c|}
        \hline
        $t^0$ & $t^1$ \\ \hline
        $0.992$ & $0.008$  \\ \hline
    \end{tabular}
    };
    \node[const, above= of cpd_t] (n_t) {Terremoto:};

    \node[const, left= of a, yshift=-0.7cm, xshift=-0.3cm] (cpd_a) {
    \begin{tabular}{|c|c|c|}
        \hline
        & $a^0$ & $a^1$ \\ \hline
       ($e^0, t^0$) & $0.99$ & $0.01$  \\ \hline
       ($e^1, t^0$) & $0.01$ & $0.99$  \\ \hline
       ($e^0, t^1$) & $0.01$ & $0.99$  \\ \hline
       ($e^1, t^1$) & $0.0001$ & $0.9999$  \\ \hline
    \end{tabular}
    };
    \node[const, above= of cpd_a] (n_a) {Alarma:};

    \node[const, right= of r, xshift=0.1cm] (cpd_r) {
    \begin{tabular}{|c|c|c|}
        \hline
        & \, $r^0$ \, & \, $r^1$ \,  \\ \hline
       $(t^0)$ & $0.99$ & $0.01$   \\ \hline
       $(t^1)$ & $0.01$ & $0.99$   \\ \hline
    \end{tabular}
    };
    \node[const, above= of cpd_r] (n_r) {Redes:};
 
    \node[const, right= of l, xshift=0.1cm] (cpd_l) {
    \begin{tabular}{|c|c|c|}
        \hline
        & \, $l^0$ \, & \, $l^1$ \,  \\ \hline
       $(t^0)$ & $0.99$ & $0.01$   \\ \hline
       $(t^1)$ & $0.01$ & $0.99$   \\ \hline
    \end{tabular}
    };
    \node[const, above= of cpd_l] (n_l) {Llamada:};    
 
 }
\end{textblock}
\end{frame}




\begin{frame}[plain]
\begin{textblock}{160}(0,4)
\centering \Large Flujos de inferencia
\end{textblock}

\only<1->{
\begin{textblock}{160}(0,18)
\centering
 \begin{tabular}{c c|c}
 & $\hfrac{\text{Intermedio}}{\text{no observable}}$ &   $\hfrac{\text{Intermedio}}{\text{observable}}$ \\
 & & \\
 $ e \rightarrow a \rightarrow l $    & \onslide<2->{$P(l) \overset{?}{=} P(l|e)$} & \onslide<3->{$P(l|a) \overset{?}{=} P(l|e,a)$} \\ 
 $ l \leftarrow a \leftarrow t $      &  \onslide<4->{$P(t) \overset{?}{=} P(t|l)$}  & \onslide<5->{$P(t|a) \overset{?}{=} P(t|a,l)$} \\ 
 $ a \leftarrow t \rightarrow r $     & \onslide<6->{$P(r) \overset{?}{=} P(r|a)$} & \onslide<7->{$P(r|t) \overset{?}{=} P(r|t,a)$} \\
 $ e \rightarrow a \leftarrow t $     & \onslide<8->{$P(t) \overset{?}{=} P(t|e)$} & \onslide<9->{$P(t|a) \overset{?}{=} P(t|e,a)$} \\
            $\overset{\downarrow}{l}$  &  & \onslide<10->{$P(t|l) \overset{?}{=} P(t|e,l)$}
 \end{tabular} 
 \end{textblock}
 }
 

\end{frame}


\begin{frame}[plain]
\begin{textblock}{160}(0,4)
 \centering \Large
 Grafo de factorización
\end{textblock}

\begin{textblock}{40}(5,12)
\centering
\begin{figure}[H]
\centering
  \scalebox{0.8}{
\tikz{ %
        
        
        \node[factor] (fa) {} ;
        \node[const,right=of fa] (pa) {$P(a|e,t)$} ;
        \node[det, below=of fa, yshift=0.1cm] (a) {$a$} ; %
        
        \node[factor, below=of a] (fl) {} ;
        \node[const,right=of fl] (pl) {$P(l|a)$} ;
        \node[det, below=of fl,yshift=0.1cm] (l) {$l$} ; %
        
        \node[det, above=of fa, xshift=-1.6cm,yshift=-0.1cm] (e) {$e$} ; %
        \node[factor, above=of e,yshift=-0.1cm] (fe) {} ;
        \node[const,right=of fe] (pe) {$P(e)$} ;
        
        \node[det, above=of fa, xshift=1.6cm,yshift=-0.1cm] (t) {$t$} ; %
        \node[factor, above=of t,yshift=-0.1cm] (ft) {} ;
        \node[const,right=of ft] (pt) {$P(t)$} ;
        \node[factor, below=of t, xshift=1.6cm,yshift=0.1cm] (fr) {} ;
        \node[det,below=of fr,yshift=0.1cm] (r) {$r$} ; %
        \node[const,right=of fr] (pr) {$P(r|t)$} ;
        
        \edge[-] {fa} {a,e,t};
        \edge[-] {fe} {e};
        \edge[-] {ft} {t};
        \edge[-] {fr} {r,t};
        \edge[-] {fl} {l,a};
        } 
}   
\end{figure}
\end{textblock}

\begin{textblock}{115}(45,30)
 
 \centering Nodos: \\ variables y funciones
 
 \vspace{1cm}
 
 \centering Ejes: \\ 
  ``la variable $v$ es argumento de la funci\'on $f$''
 
\end{textblock}
\end{frame}



\begin{frame}[plain]
\begin{textblock}{160}(0,4)
 \centering \Large
 Sum-product algorithm
\end{textblock}

\only<2->{
\begin{textblock}{80}(45,30)
\begin{description}
 \onslide<3->{\item[$v(n)$ :] Vecinos del nodo $n$} 
 \item[$m_{x \rightarrow f}(x)$ :] Mensaje de variable $x$ a factor $f$ 
 \item[$m_{f \rightarrow x}(x)$ :] Mensaje de factor $f$ a variable $x$
\end{description}
\end{textblock}
}


\only<1-3>{
\begin{textblock}{80}(45,16)
\centering
Algoritmo para calcular cualquier marginal \\ mediante pasaje de mensajes entre nodos
\end{textblock}
}


\only<4->{
\begin{textblock}{80}(45,16)
\begin{equation*}
P(x) = \prod_{f \in v(x)} m_{f \rightarrow x}(x)
\end{equation*} 
\end{textblock}
}

\only<5->{
\begin{textblock}{80}(45,56)
\begin{equation*}\label{eq:m_v_f}
m_{x \rightarrow f}(x) = \prod_{h \in \only<6>{\textcolor{red}}{v(x) \setminus \{f\}} } m_{h \rightarrow x}(x)
\end{equation*}
\end{textblock}
}

\only<7->{
\begin{textblock}{80}(45,68)
\begin{equation*}\label{eq:m_f_v}
 m_{f \rightarrow x}(x) = \onslide<9->{\sum_{\bm{h}} \Big(} \onslide<8->{f(\bm{h},x)} \prod_{h \in v(f) \setminus \{x\} } m_{h \rightarrow f}(h) \onslide<9->{\Big)} 
\end{equation*}
\end{textblock}
}

\only<1->{
\begin{textblock}{40}(0,20)
\centering
\begin{figure}[H]
\centering
  \scalebox{.8}{
\tikz{ %
        
        
        \node[factor] (fa) {} ;
        \node[det, below=of fa, minimum size=0.55cm, yshift=0.4cm] (a) {$a$} ; %
        
        \node[factor, below=of a, yshift=0.4cm] (fl) {} ;
        \node[det, below=of fl, minimum size=0.55cm,yshift=0.4cm] (l) {$l$} ; %
        
        \node[det, above=of fa, minimum size=0.55cm, xshift=-0.8cm, yshift=-0.4cm] (e) {$e$} ; %
        \node[factor, above=of e, yshift=-0.4cm] (fe) {} ;
        
        \node[det,minimum size=0.55cm, above=of fa, xshift=0.8cm, yshift=-0.4cm] (t) {$t$} ; %
        \node[factor, above=of t, yshift=-0.4cm] (ft) {} ;
        \node[factor, below=of t, xshift=0.8cm, yshift=0.4cm] (fr) {} ;
        \node[det,minimum size=0.55cm, below=of fr, yshift=0.4cm] (r) {$r$} ; %
        
        \edge[-] {fa} {a,e,t};
        \edge[-] {fe} {e};
        \edge[-] {ft} {t};
        \edge[-] {fr} {r,t};
        \edge[-] {fl} {l,a};
        } 
}   
\end{figure}
\end{textblock}
}

\end{frame}



\begin{frame}[plain]
\begin{textblock}{160}(0,4)
\centering \Large Flujos de inferencia
\end{textblock}

\only<1->{
\begin{textblock}{160}(0,18)
\centering
 \begin{tabular}{c c|c}
 & $\hfrac{\text{Intermedio}}{\text{no observable}}$ &   $\hfrac{\text{Intermedio}}{\text{observable}}$ \\
 & & \\
 $ e \rightarrow a \rightarrow l $    & $P(l) \only<1>{\overset{?}{=}}\only<2->{\bm{\neq}} P(l|e)$ & {$P(l|a) \overset{?}{=} P(l|e,a)$} \\ 
 $ l \leftarrow a \leftarrow t $      &  {$P(t) \overset{?}{=} P(t|l)$}  & {$P(t|a) \overset{?}{=} P(t|a,l)$} \\ 
 $ a \leftarrow t \rightarrow r $     & {$P(r) \overset{?}{=} P(r|a)$} & {$P(r|t) \overset{?}{=} P(r|t,a)$} \\
 $ e \rightarrow a \leftarrow t $     & {$P(t) \only<1>{\overset{?}{=}}\only<2->{\bm{=}} P(t|e)$} & {$P(t|a) \overset{?}{=} P(t|e,a)$} \\
            $\overset{\downarrow}{l}$  &  & {$P(t|l) \overset{?}{=} P(t|e,l)$}
 \end{tabular} 
 
 \vspace{0.7cm}
 
 \onslide<3>{¿Y el resto?}
 
 \end{textblock}
 }

\end{frame}


\begin{frame}[plain]
\begin{textblock}{160}(0,4)
\centering \Large Flujos de inferencia
\end{textblock}

\begin{textblock}{160}(0,18)
\centering
 \begin{tabular}{c c|c}
 & $\hfrac{\text{Intermedio}}{\text{no observable}}$ &   $\hfrac{\text{Intermedio}}{\text{observable}}$ \\
 & & \\
 $ e \rightarrow a \rightarrow l $    & $P(l) \neq P(l|e)$ & $P(l|a) \overset{\phantom{?}}{=} P(l|e,a)$ \\ 
 $ l \leftarrow a \leftarrow t $      &  $P(t) \neq P(t|l)$  & $P(t|a) \overset{\phantom{?}}{=} P(t|a,l)$ \\ 
 $ a \leftarrow t \rightarrow r $     & $P(r) \neq P(r|a)$ & $P(r|t) \overset{\phantom{?}}{=} P(r|t,a)$ \\
 $ e \rightarrow a \leftarrow t $     & $P(t) \overset{\phantom{?}}{=}  P(t|e)$ & $P(t|a) \neq P(t|e,a)$ \\
            $\overset{\downarrow}{l}$  & \phantom{$P(t) \overset{?}{=}  P(t|e)$} & $P(t|l) \only<1->{\neq} P(t|e,l)$
 \end{tabular} 
 \end{textblock}

 
 \only<2>{
 \begin{textblock}{140}(10,70)
  \centering \large
 Hay flujo de inferencia (``correlación'') entre los extremos de una cadena si:
 \begin{itemize}
  \item[$\bullet$] Todas las consecuencias comunes (o sus descendientes) son observables
  \item[$\bullet$] Ning\'una otra variable es observable
 \end{itemize}
 \end{textblock}
}
 
\end{frame}
% 
% 
% \begin{frame}[plain]
% \begin{textblock}{160}(0,4)
% \centering \Large Flujo de inferencia
% \end{textblock}
% \vspace{0.7cm}
% % 
% %  \only<2>{
% %  \centering
% % \includegraphics[width=0.66\textwidth]{../../auxiliar/static/preguntaseria.png}
% % }
% %  
%  
%  \only<1>{
%  \centering
%  Hay flujo de inferencia (``correlación'') entre los extremos de una cadena si:
%  \begin{itemize}
%   \item[$\bullet$] Todas las consecuencias comunes (o sus descendientes) son observables
%   \item[$\bullet$] Ning\'una otra variable es observable
%  \end{itemize}
% }
% 
% \end{frame}

\begin{frame}[plain]
 
 \huge \centering
 
 Intervención experimental
 
\end{frame}


\begin{frame}[plain]
\begin{textblock}{160}(0,4)
 \centering \Large
 Modelo Causal
 \end{textblock}
 \centering
 \vspace{0.75cm}
 
 \tikz{
    \node[det] (l) {$l$} ; %
    \node[det, above=of l] (a) {$a$} ; %
    \node[det, above=of a,xshift=1.5cm] (t) {$t$} ; %
    \node[det, above=of a,xshift=-1.5cm] (e) {$e$} ; %
    \node[det, below=of t,xshift=1.5cm] (r) {$r$} ; %
    \onslide<2>{\node[det, above=of t,xshift=-1.5cm] (c) {$c$} ;}
    
    
    \edge {a} {l};
    \edge {t,e} {a};
    \edge {t} {r};
    \onslide<2>{\edge {c} {e,t};}
    
    \node[const, left= of e, xshift=-0.1cm] (cpd_e) {Entradera:};
    \node[const, right= of t, xshift=0.1cm] (cpd_t) {:Terremoto};
    \onslide<2>{\node[const, right= of c, xshift=0.1cm] (cpd_c) {:Ciudad};}
    \node[const, left= of a, xshift=-0.1cm] (cpd_a) {Alarma:};
    \node[const, right= of r, xshift=0.1cm] (cpd_r) {:Redes};
    \node[const, left= of l, xshift=-0.1cm] (cpd_l) {Llamada:};
    
    }

 \end{frame}


 
\begin{frame}[plain]
\begin{textblock}{160}(0,4)
 \centering \Large
 Experimentos $Do(e=1)$
 \end{textblock}
 \centering
 \vspace{0.75cm}

 \tikz{
    \node[det] (l) {$l$} ; %
    \node[det, above=of l] (a) {$a$} ; %
    \node[det, above=of a,xshift=1.5cm] (t) {$t$} ; %
   \only<1>{\node[det, above=of a,xshift=-1.5cm] (e) {$e$} ;}
   \only<2>{\node[det, double, double distance=0.5mm, above=of a,xshift=-1.5cm] (e) {$e$} ;}
    \node[det, below=of t,xshift=1.5cm] (r) {$r$} ; %
    \node[det, above=of t,xshift=-1.5cm] (c) {$c$} ;
    
    
    \edge {a} {l};
    \edge {t,e} {a};
    \edge {t} {r};
    \only<1>{\edge {c} {e,t};}
    \only<2>{\edge {c} {t};}
    
     \node[const, left= of e, xshift=-0.1cm] (cpd_e) {Entradera:};
     \node[const, right= of t, xshift=0.1cm] (cpd_t) {:Terremoto};
     \node[const, right= of c, xshift=0.1cm] (cpd_c) {:Ciudad};
     \node[const, left= of a, xshift=-0.1cm] (cpd_a) {Alarma:};
     \node[const, right= of r, xshift=0.1cm] (cpd_r) {:Redes};
     \node[const, left= of l, xshift=-0.1cm] (cpd_l) {Llamada:};
    
    }

 
\end{frame}


\begin{frame}[plain]
 \begin{textblock}{160}(0,4)
 \centering \Large
 Estabilidad de los mecanismos causales
 \end{textblock}

 \only<1>{\centering
 Modelo causal
 \begin{align*}
  c &= u_c \\
  e &= f(c,u_e) \\ 
  t &= f(c,u_t) \\
  r &= f(t,u_r) \\
  a &= f(e,t,u_a) \\
  l &= f(a,u_l)
 \end{align*} 
 }
 
 \only<2>{\centering
 Experimento
 \begin{align*}
 c &= u_c \\
  e &= 1 \\ 
  t &= f(c,u_t) \\
  r &= f(t,u_r) \\
  a &= f(e,t,u_a) \\
  l &= f(a,u_l)
 \end{align*} 
 }
 
\end{frame}


\begin{frame}[plain]
 \begin{textblock}{160}(0,4)
 \centering \Large
Niveles (de detalle) de las preguntas científicas
 \end{textblock}
 \vspace{0.75cm}
 
 \large
 \begin{description}
  \item[Predicciones] ¿Llamarán dado que \emph{vimos} que entraron a la casa?\\[0.2cm] \pause
  \item[Efecto causal] ¿Llamarán si \emph{nos atrevemos} a entrar ahora a la casa?\\[0.2cm] \pause
  \item[Contrafáctico] ¿Hubieran llamado \emph{si} alguien hubiera entrado, dado que todavía no llamaron ni entraron?
 \end{description}
 
 \end{frame}

 
 
 \begin{frame}[plain]
 \begin{textblock}{160}(0,4)
 \centering \Large
 Modelos extendido \\
 \large Prior Ciudades
 \end{textblock}
 \vspace{0.75cm}
 
 \centering
 
 \begin{textblock}{160}(0,25)
  $P(c)$ \\[0.1cm]
    \begin{tabular}{|c|c|}
        \hline
        \,$c^0$\, & \,$c^1$\, \\ \hline
        \onslide<3>{$15/43$} & \onslide<3>{$28/43$}   \\ \hline
    \end{tabular}
  \end{textblock}

\begin{textblock}{160}(10,54) 
\onslide<2->{ 
 \begin{itemize}
  \item[$\bullet$] San Martín (San Juan) $15.000$
  \item[$\bullet$] San Martín (Buenos Aires) $28.000$
 \end{itemize}
}
\end{textblock}
\end{frame}
  
\begin{frame}[plain]
\begin{textblock}{160}(0,4)
 \centering \Large
 Modelos extendido \\
 \large Prior Entradera
 \end{textblock}
 \vspace{0.75cm}
 
 \centering
 
 
 \begin{textblock}{160}(0,25)
  $P(e|c)$ \\[0.1cm]
    \begin{tabular}{|c|c|c|}
        \hline
        & \, $e^0$ \, & \, $e^1$ \,  \\ \hline
       $(c^0)$ & \only<2->{$0.99$} & \only<2->{$0.01$}   \\ \hline
       $(c^1)$ & \only<3->{$0.95$} & \only<3->{$0.05$}   \\ \hline
    \end{tabular}
\end{textblock}
 
 \end{frame}
 
 \begin{frame}[plain]
\begin{textblock}{160}(0,4)
 \centering \Large
 Modelos extendido \\
 \large Prior Terremoto
 \end{textblock}
 \vspace{0.75cm}
 
 \centering

 \begin{textblock}{160}(0,25)
  $P(t|c)$ \\[0.1cm]
    \begin{tabular}{|c|c|c|}
        \hline
        & \, $t^0$ \, & \, $t^1$ \,  \\ \hline
       $(c^0)$ & \only<2->{$0.95$} & \only<2->{$0.05$}   \\ \hline
       $(c^1)$ & \only<3->{$0.999$} & \only<3->{$0.001$}   \\ \hline
    \end{tabular}
\end{textblock}

 
 \end{frame}



% \begin{frame}[plain]
% \begin{textblock}{160}(0,4)
%  \centering \Large
%  Grafo de factorización
% \end{textblock}
% 
% \begin{textblock}{40}(5,12)
% \centering
% \begin{figure}[H]
% \centering
%   \scalebox{0.8}{
% \tikz{ %
%         
%         
%         \node[factor] (fa) {} ;
%         %\node[const,left=of fa] (pa) {$P(a|e,t)$} ;
%         \node[det, fill=black!10, below=of fa, yshift=0.1cm] (a) {$a$} ; %
%         
%         \node[factor, below=of a] (fl) {} ;
%         %\node[const,right=of fl] (pl) {$P(l|a)$} ;
%         \node[det, below=of fl,yshift=0.1cm] (l) {$l$} ; %
%         
%         \node[det, above=of fa, xshift=-1.6cm,yshift=-0.1cm] (e) {$e$} ; %
%         \node[factor, above=of e,yshift=-0.1cm] (fe) {} ;
%         %\node[const,right=of fe] (pe) {$P(e)$} ;
%         
%         \node[det, above=of fa, xshift=1.6cm,yshift=-0.1cm] (t) {$t$} ; %
%         \node[factor, above=of t,yshift=-0.1cm] (ft) {} ;
%         %\node[const,right=of ft] (pt) {$P(t)$} ;
%         \node[factor, below=of t, xshift=1.6cm,yshift=0.1cm] (fr) {} ;
%         \node[det,below=of fr,yshift=0.1cm] (r) {$r$} ; %
%         %\node[const,left=of fr] (pr) {$P(r|t)$} ;
%         
%         
%         \edge[-] {fa} {a,e,t};
%         \edge[-] {fe} {e};
%         \edge[-] {ft} {t};
%         \edge[-] {fr} {r,t};
%         \edge[-] {fl} {l,a};
%         
%         \path[draw, ->, fill=black!50] (fe) edge[bend left,draw=black!50] node[right,color=black!75] {\only<1>{$P(e)$}} (e);
%         \path[draw, ->, fill=black!50] (ft) edge[bend left,draw=black!50] node[right,color=black!75] {$P(t)$} (t);
%         \path[draw, ->, fill=black!50] (fr) edge[bend left,draw=black!50] node[left,color=black!75] {\only<1->{$1$}} (t);
%         \path[draw, ->, fill=black!50] (fl) edge[bend left,draw=black!50] node[left,color=black!75] {$1$} (a);
%         \onslide<3->{\path[draw, ->, fill=black!50] (fa) edge[bend left,draw=black!50] node[left,color=black!75] {} (t);}
%         \onslide<2->{\path[draw, ->, fill=black!50] (fa) edge[bend left,draw=black!50] node[left,color=black!75] {} (e);}
%         \onslide<4->{\path[draw, ->, fill=black!50] (fr) edge[bend left,draw=black!50] node[right,color=black!75] {\only<5->{$P(r,a^*)$}} (r);}
%         \onslide<6->{\path[draw, ->, fill=black!50] (fa) edge[bend left,draw=black!50] node[right,color=black!75] {\only<7->{$P(a^*)$}} (a);}
%         \onslide<8->{\path[draw, ->, fill=black!50] (fl) edge[bend left,draw=black!50] node[right,color=black!75] {\only<9->{$P(l,a^*)$}} (l);}
%         }
% }   
% \end{figure}
% \end{textblock}
% 
% \begin{textblock}{100}(65,16)
%  $m_{f_e\rightarrow e}(e) = \onslide<1->{P(e)} $  \\
%  $m_{f_t\rightarrow t}(t) = \onslide<1->{P(t)}$  \\
%  $m_{f_r\rightarrow t}(t) = \onslide<1->{\sum_r \, P(r|t) = 1}$  \\
%  $m_{f_l\rightarrow a}\phantom{(a)} = \onslide<1->{\sum_l \, P(l|a^*) = 1} $  \\
%  $m_{f_a\rightarrow t}(t) = \onslide<3->{\sum_{e} \, P(e) P(a^*|e,t)}$  \\
%  $m_{f_a\rightarrow e}(e) = \onslide<2->{\sum_t P(t)P(a^*|e,t)}$  \\
%  $m_{f_r\rightarrow r}(r) = \onslide<4->{\sum_{et} \, P(e)P(t) P(a^*|e,t) P(r|t)} \onslide<5->{= P(r,a^*)}$ \\
%  $m_{f_a\rightarrow a}\phantom{(a)} = \onslide<6->{\sum_{et} \, P(e)P(t)P(a^*|e,t)}\onslide<7->{  = P(a^*)}$  \\
%  $m_{f_l\rightarrow l}(l) = \onslide<8->{P(a^*) P(l|a^*)} \onslide<9->{ = P(l,a^*)}$
%  \end{textblock}
% 
%  
% \only<10->{
% \begin{textblock}{80}(65,60)
% \begin{flalign*}
%   P(t,a^*) & = \sum_{e} \, P(e) P(t) P(a^*|e,t) \\ 
%   P(e,a^*) & = \sum_{t} \, P(e) P(t) P(a^*|e,t) \\
%   &&
%   \end{flalign*}
% \end{textblock}
% }
% 
%  
% \end{frame}
% 




\begin{frame}[plain]
\centering
  \includegraphics[width=0.35\textwidth]{../../auxiliar/static/pachacuteckoricancha.jpg}
\end{frame}


\end{document}



