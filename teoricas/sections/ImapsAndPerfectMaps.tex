
We have seen that a graph structure encodes a set of independencies.
What we want to talk about now is the question of how to take a distribution
that has a certain set of independencies that that it satisfies and encoded within a graph's structure.

\paragraph{Capturing independencies in $P$}

The set of all independiencies of $P$, 
\begin{equation}
 I(P) = \{(X \perp Y \mid Z) : P \models (X \perp Y \mid Z) \}
\end{equation}

We know that if $P$ factorizes over $G$ $\Rightarrow$ $G$ is an I-map for P

\begin{equation}
I(G) \subseteq  I(P)
\end{equation}

\begin{framed} \centering
 We want the graph that caputres us much of the indpendencies properties of $P$as posible
\end{framed}

\begin{definition}(Minimal I-map)
 I-map without rendundant edges ($P(Y|x^0) = P(Y|x^1)$)
\end{definition}

However, minimal I-mal may still not capture $I(P)$. For example, we show a minimal I-grap (right) for the ``student'' model (left) 
\begin{paracol}{2}
\begin{verbatim}
 D   I 
  \ /
   |
   v
   G
\end{verbatim}
 \switchcolumn
\begin{verbatim}
 D-->I 
 |   ^
 |  /
 v /
  G
\end{verbatim} 

\end{paracol}

Why this is a minimal I-map? Well
\begin{description}
 \item[Case remove $D->G$] we get $(D \perp G)$. NOT in our model 
 \item[Case remove $G->I$] we get $(G \perp I \mid D)$. NOT in our model
 \item[Case remove $D->I$] we get $(D \perp I \mid G)$. NOT in our model
\end{description}

There is a better minumum I-map, but this is a minimum I-map

\begin{definition}(Perfect Map)
$I(G) = I(P)$ 
\end{definition}

\paragraph{Converting BNs to MNs loses independcies}

\begin{itemize}
 \item Bn to MN: loses independencies in v-strucutre
 \item MN to BN: must add trinagulating edges to loops
\end{itemize}


Example: A scenario where there is a non perfect map.
This markov network (left) has an I-map (right), but not a perfect map-
\begin{paracol}{2}
\begin{verbatim}
     A
    / \
   /   \
  D     B
   \   /
    \ /
     C     
\end{verbatim}
\switchcolumn
\begin{verbatim}
     A
    / \
   v   v
   D-> B
   \   /
    v v
     C     
\end{verbatim}
\end{paracol}

Another example is the ``XOR model''.


\paragraph{Uniqueness of perfect Map}

\begin{itemize}
 \item $G_1: X \rightarrow Y $, $I(G_1) = \emptyset$
  \item $G_2: X \leftarrow Y $, $I(G_2) = \emptyset$
\end{itemize}
  Then, can represent exactly the same distributions

  \vspace{0.3cm}
  
  Which of the following graphs does not encode the same independencies as the others?
  
\begin{itemize}
 \item $G_a: X \rightarrow Y \rightarrow Z$,
  \item $G_b: X \leftarrow Y \rightarrow Z $,
\item $G_b: X \leftarrow Y \leftarrow  Z$,
\item $G_b: X \rightarrow Y \leftarrow Z $ (v-strucutre)
\end{itemize}

The v-structure holds $X \perp Z$ while the others $X \perp Z | Y $
Any of these not v-satuructre grpahs can represent each other.

\begin{definition}(I-equivalence)
 Two graphs $G_1$ and $G_2$ over $X_1, ... X_n$ are I-equivalent if $I(G_1)=I(G_2)$.
\end{definition}

\begin{framed}
 \centering
 There are certain propoerties of the grpah that are unidentifiables by lookin the indpendencies relations.
\end{framed}

\textbf{And most graph have many I-equivalent variants.
This is a problem to learn model from data.}




  
  





