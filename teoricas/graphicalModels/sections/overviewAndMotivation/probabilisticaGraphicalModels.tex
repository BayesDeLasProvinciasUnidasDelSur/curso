


\paragraph{Probabilistic Graphical Models}

\subparagraph{Model}

The model is a declarative representation of our understanding of the world. And the fact that it's declarative means that the representation stands on its own, which means that we can look into it and make sense of it aside from any algorithm that we might choose to apply on. 

\subparagraph{Probabilistic}

Probability theory is a framework that allows us to deal with uncertainty in ways that are principled and that bring to bear important and valuable tools.

So uncertainty comes in many forms and for many different reasons.
So, first it comes because we just have partial knowledge of the state of the world, 

\vspace{0.3cm}
Uncertainty comes because of 
\begin{itemize}
 \item Data: noisy observations. 
 \item Model: of modeling limitations, phenomena that are just not covered by our model
 \item And finally, some people would argue that the world is inherently stochastic
\end{itemize}

\subparagraph{Graphical}

Graphical representation gives us an intuitive and compact data structure for capturing high dimensional probability distributions.
It provides us at the same time, as we'll see later in the course,  suite of methods for efficient reasoning,
using general purpose algorithms that exploit the graphical structure.
And because of the way in which the graph structure encodes the parameterization of the probability distribution, we can represent these high-dimensional probability distribution efficiently using a very small number of parameters.
