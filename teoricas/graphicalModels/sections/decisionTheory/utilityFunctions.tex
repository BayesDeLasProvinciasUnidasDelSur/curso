
The Maximum Expected Utility function is not always what people prefers.
For example if we have a lottery that with probability $1$ gives you $3$ millions dolars, and a lottery taht with probability $0.8$ gives you $4$ millions dolars, then the MEU is greater in the second case, but people prefers the first one.

\vspace{0.3cm}

An other example is the St. Petersburg Paradox, in which the MEU is $\infty$ but experiments show that people payoff is $2$ dollars.

\vspace{0.3cm}

\begin{framed}
 \centering
So the payoff is not linear on the utility.
 \end{framed}

 So the utility could be modeled as a concave curve like the learning curves.
 This is the risk averse curve, how much you will pay for getting less risk.
 If the curve is linear, this is called ``risk neutral''.
 Convex curves is risk seaking. An example is behavior in Casinos.
 
 
 \vspace{0.3cm}
 
 The tipical utility curves a sigmoid function, centered at 0
 
 \paragraph{Multi-Attribute Utilities}
 
 All attributes affecting preferences must be integrated into one utility function. 
 
 
 
 
