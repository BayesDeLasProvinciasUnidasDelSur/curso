
\begin{framed} \centering
Which observation I need to made before take a decision 
\end{framed}

\paragraph{Value of information}

\begin{definition}[Value of Perfect information]
 $VPI(A|X)$ is the value of observing $X$ before choosing an action $A$
\end{definition}

If $D$ is our original influence diagram, we can define an other influence diagram where there is an edge between $X \rightarrow A$ $D_{X \rightarrow A}$

Then, we can define,

\begin{equation}\label{eq:vpi}
 VPI(A | X) = MEU(D_{X \leftarrow A}) - MEU(D)
\end{equation}

\begin{figure}[H]
 \centering
 \begin{subfigure}[c]{0.32\textwidth}
  \tikz{

 \node[latent, fill=black!0, yshift=1cm] (m) {M} ; %
 \node[latent, fill=black!0,right=of m,yshift=-0.5cm] (s) {S} ; %
 \node[latent, fill=red!20, yshift=-0.5cm, xshift=2cm] (f) {F};
 \node[latent, fill=green!40, below=of x,xshift=1cm] (u) {U} ;
 
 \edge {m,f} {u};
 \edge {m} {s};
 }  
 \caption{}
 \label{fig:market_survey_found_withOutObs}
 \end{subfigure}
 \begin{subfigure}[c]{0.32\textwidth}
  \tikz{

    \node[latent, fill=black!0, yshift=1cm] (m) {M} ; %
    \node[latent, fill=black!0,right=of m,yshift=-0.5cm] (s) {S} ; %
    \node[latent, fill=red!20, yshift=-0.5cm, xshift=2cm] (f) {F};
    \node[latent, fill=green!40, below=of x,xshift=1cm] (u) {U} ;
    \edge {m,f} {u};
    \edge {m} {s};
    \edge[red!40] {s} {f};
  }  
  \caption{}
 \label{fig:market_survey_found_withObs}
 \end{subfigure}
 \caption{}
 \label{fig:market_survey_found}
\end{figure}

\begin{theorem}(Value of information)
 \begin{itemize}
  \item $VPI(A|X) \geq 0$ 
  \item $VPI(A|X) = 0$ if and only if the optimal decision rule for $D$ is still optimal for $D_{X\rightarrow A}$
 \end{itemize}
\end{theorem}


\begin{framed}
\centering
 Information is useful $\leftrightarrow$ when it changes my decision. 
\end{framed}

\paragraph{Value of Information Example}

An entrepreneur is deciding agains funding a companies, and deciding to peek between two companies.
For each companie there is the state of the company, from $s^1:$ poor, to $s^3$ great.

\begin{figure}[H]
\centering
\tikz{

 \node[latent, fill=green!40] (u) {$U$} ;
 \node[const, right=of u, xshift=0.5cm] (Phi_u) {
     $ \Phi_U = 
      \begin{cases}
       1 & \text{ if company get founded} \\
       0 & \text{ otherwise}
      \end{cases}
    $
 }; 
 \node[latent, fill=red!40,yshift=1.33cm] (c) {$C$} ;
 \node[latent, fill=black!0,xshift=-1.5cm,yshift=1cm] (f1) {$F_1$} ; %
 \node[const, left=of f1, xshift=-0.3cm,yshift=-0.5cm] (phi_f1) {$\Phi_{F_1}$:
  \begin{tabular}{|l|l|l|} \hline
          & $f^0$ & $f^1$  \\ \hline
    $s^0$ & 0.9 & 0.1 \\ \hline
    $s^1$ & 0.6 & 0.4 \\ \hline
    $s^2$ & 0.1 & 0.9 \\ \hline
  \end{tabular}
 }; 
 \node[latent, fill=black!0,xshift=1.5cm,yshift=1cm] (f2) {$F_2$} ; %
 \node[latent, fill=black!0,xshift=-1cm,yshift=2.5cm] (s1) {$S_1$} ; %
 \node[const, left=of s1, xshift=-0.5cm] (Phi_s1) {$\Phi_{S_1}$:
    \begin{tabular}{|l|l|l|}\hline
    $s^0$ & $s^1$ & $s^2$ \\ \hline
    0.1 & 0.2 & 0.7  \\ \hline
    \end{tabular}
 }; 
 \node[latent, fill=black!0,xshift=1cm,yshift=2.5cm] (s2) {$S_2$} ; %
 \node[const, right=of s2, xshift=0.5cm] (Phi_s2) {$\Phi_{S_2}$:
    \begin{tabular}{|l|l|l|}\hline
    $s^0$ & $s^1$ & $s^2$ \\ \hline
    0.4 & 0.5 & 0.1  \\ \hline
    \end{tabular}
 }; 

  \edge {s1} {f1};
  \edge {s2} {f2};
  \edge {f1,f2,c} {u};
 }  
 \caption{}
 \label{fig:value_of_information_example_1}
\end{figure}

The chance of the founding of the company depends on the state of the companies.
Then, the two strategies without any information are,

\begin{equation*}
\begin{split}
 & EU[D(c_1)] = 0.1 \cdot 0.1 + 0.2 \cdot 0.4 + 0.7 \cdot 0.9 = 0.72 \\
 & EU[D(c_2)] = 0.4 \cdot 0.1 + 0.5 \cdot 0.4 + 0.1 \cdot 0.9 = 0.33
\end{split}
\end{equation*}

What happend if the agent makes an observation?
So imagen that they have acces to the real information about the company state.

\begin{figure}[H]
\centering
\tikz{

 \node[latent, fill=green!40] (u) {$U$} ;
 \node[const, right=of u, xshift=0.5cm] (Phi_u) {
     $ \Phi_U = 
      \begin{cases}
       1 & \text{ if company get founded} \\
       0 & \text{ otherwise}
      \end{cases}
    $
 }; 
 \node[latent, fill=red!40,yshift=1.33cm] (c) {$C$} ;
 \node[latent, fill=black!0,xshift=-1.5cm,yshift=1cm] (f1) {$F_1$} ; %
 \node[const, left=of f1, xshift=-0.3cm,yshift=-0.5cm] (phi_f1) {$\Phi_{F_1}$:
  \begin{tabular}{|l|l|l|} \hline
          & $f^0$ & $f^1$  \\ \hline
    $s^0$ & 0.9 & 0.1 \\ \hline
    $s^1$ & 0.6 & 0.4 \\ \hline
    $s^2$ & 0.1 & 0.9 \\ \hline
  \end{tabular}
 }; 
 \node[latent, fill=black!0,xshift=1.5cm,yshift=1cm] (f2) {$F_2$} ; %
 \node[latent, fill=black!0,xshift=-1cm,yshift=2.5cm] (s1) {$S_1$} ; %
 \node[const, left=of s1, xshift=-0.5cm] (Phi_s1) {$\Phi_{S_1}$:
    \begin{tabular}{|l|l|l|}\hline
    $s^0$ & $s^1$ & $s^2$ \\ \hline
    0.1 & 0.2 & 0.7  \\ \hline
    \end{tabular}
 }; 
 \node[latent, fill=black!0,xshift=1cm,yshift=2.5cm] (s2) {$S_2$} ; %
 \node[const, right=of s2, xshift=0.5cm] (Phi_s2) {$\Phi_{S_2}$:
    \begin{tabular}{|l|l|l|}\hline
    $s^0$ & $s^1$ & $s^2$ \\ \hline
    0.4 & 0.5 & 0.1  \\ \hline
    \end{tabular}
 }; 

  \edge {s1} {f1};
  \edge {s2} {f2};
  \edge {f1,f2,c} {u};
  \edge[red!60] {s2} {c};
 }  
 \caption{}
 \label{fig:value_of_information_example_2}
\end{figure}


If $S_2 = s^1$ and $C = c^2$, the expected utility is $0.1$.
If $S_2 = s^2$ and $C = c^2$, the expected utility is $0.4$. 
Both of this are lower than the $EU[D(c_1)] = 0.72$.
If $S_2 = s^3$ and $C = c^2$, the expected utility is $0.9$.
This is the only scenario that we prefer change the opinion.

 \begin{equation}
  \delta^*(C|S_2)
      \begin{cases}
       P(c^2) = 1 & \text{ if } S_2 = s^3 \\
       P(c^1) & \text{ otherwise }
      \end{cases}
 \end{equation}
 
 So the 
 \begin{equation}
  MEU(D_{S_2 \rightarrow C}) =  (1-P(S_2 = s^3)) \cdot EU[D(c^1)] + P(S_2 = s^3) \Phi_U(f^1,s^3) = 0.738
 \end{equation}
 
 If the probability of the state of company $1$ changes

 \begin{figure}[H]
\centering
\tikz{

 \node[latent, fill=green!40] (u) {$U$} ;
 \node[const, right=of u, xshift=0.5cm] (Phi_u) {
     $ \Phi_U = 
      \begin{cases}
       1 & \text{ if company get founded} \\
       0 & \text{ otherwise}
      \end{cases}
    $
 }; 
 \node[latent, fill=red!40,yshift=1.33cm] (c) {$C$} ;
 \node[latent, fill=black!0,xshift=-1.5cm,yshift=1cm] (f1) {$F_1$} ; %
 \node[const, left=of f1, xshift=-0.3cm,yshift=-0.5cm] (phi_f1) {$\Phi_{F_1}$:
  \begin{tabular}{|l|l|l|} \hline
          & $f^0$ & $f^1$  \\ \hline
    $s^0$ & 0.9 & 0.1 \\ \hline
    $s^1$ & 0.6 & 0.4 \\ \hline
    $s^2$ & 0.1 & 0.9 \\ \hline
  \end{tabular}
 }; 
 \node[latent, fill=black!0,xshift=1.5cm,yshift=1cm] (f2) {$F_2$} ; %
 \node[latent, fill=black!0,xshift=-1cm,yshift=2.5cm] (s1) {$S_1$} ; %
 \node[const, left=of s1, xshift=-0.5cm] (Phi_s1) {$\Phi_{S_1}$:
    \begin{tabular}{|l|l|l|}\hline
    $s^0$ & $s^1$ & $s^2$ \\ \hline
    \textbf{0.5} & \textbf{0.3} &  \textbf{0.2}  \\ \hline
    \end{tabular}
 }; 
 \node[latent, fill=black!0,xshift=1cm,yshift=2.5cm] (s2) {$S_2$} ; %
 \node[const, right=of s2, xshift=0.5cm] (Phi_s2) {$\Phi_{S_2}$:
    \begin{tabular}{|l|l|l|}\hline
    $s^0$ & $s^1$ & $s^2$ \\ \hline
    0.4 & 0.5 & 0.1  \\ \hline
    \end{tabular}
 }; 

  \edge {s1} {f1};
  \edge {s2} {f2};
  \edge {f1,f2,c} {u};
 }  
 \caption{}
 \label{fig:value_of_information_example_3}
\end{figure}

\begin{equation*}
\begin{split}
 & EU[D(c_1)] = \textbf{0.35} \\
 & EU[D(c_2)] = 0.33
\end{split}
\end{equation*}

If we add the observation of $S_2$ then we will change our decision either $S_2 = s^3, s^2$.

so 

 \begin{equation*}
  \delta^*(C|S_2)
      \begin{cases}
       P(c^2) = 1 & \text{ if } S_2 = s^3 , s^2 \\
       P(c^1) & \text{ otherwise }
      \end{cases}
 \end{equation*}
 
 So the 
 \begin{equation*}
  MEU(D_{S_2 \rightarrow C}) =  P(S_2 = s^1) \cdot EU[D(c^1)] + P(S_2 = s^2) \Phi_U(f^1,s^2) + P(S_2 = s^3) \Phi_U(f^1,s^3) = 0.43
 \end{equation*}
 
 
 \paragraph{Summary}
 
 \begin{itemize}
  \item influence diagrams (IDs) provide clear and coherent semantics for the value of making an observation: difference between values of two IDs
   \item Perfect information is valuable if and only if it induces a change in action in at least one context
 \end{itemize}

 
 
 

  
 
 









