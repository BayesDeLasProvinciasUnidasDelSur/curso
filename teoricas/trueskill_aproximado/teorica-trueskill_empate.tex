\documentclass[shownotes,aspectratio=169]{beamer}

\input{../../auxiliar/tex/diapo_encabezado.tex}
\input{../../auxiliar/tex/tikzlibrarybayesnet.code.tex}
 \mode<presentation>
 {
 %   \usetheme{Madrid}      % or try Darmstadt, Madrid, Warsaw, ...
 %   \usecolortheme{default} % or try albatross, beaver, crane, ...
 %   \usefonttheme{serif}  % or try serif, structurebold, ...
  \usetheme{Antibes}
  \setbeamertemplate{navigation symbols}{}
 }
 
\usepackage{todonotes}
\setbeameroption{show notes}

\newif\ifen
\newif\ifes
\newcommand{\en}[1]{\ifen#1\fi}
\newcommand{\es}[1]{\ifes#1\fi}
\estrue

%\title[Bayes del Sur]{}

\begin{document}

\color{black!85}
\large
 
%\setbeamercolor{background canvas}{bg=gray!15}


\begin{frame}[plain,noframenumbering]
 
 \begin{textblock}{90}(00,05)
\begin{center}
 \huge  \textcolor{black!66}{Creencias adaptativas}
\end{center}

\begin{textblock}{47}(113,72)
\centering \Large  \textcolor{white!55}{TrueSkill} \ \ \ \ \ \
\end{textblock}
% \begin{textblock}{47}(116,75)
% \centering \Large \ \ \ \ \ \textcolor{white!55}{aproximado}
% \end{textblock}

\end{textblock}

 %\vspace{2cm}brown
%\maketitle
\Wider[2cm]{
\includegraphics[width=1\textwidth]{../../auxiliar/static/peligro_predador}
}
\end{frame}



\begin{frame}[plain]
 \begin{textblock}{160}(0,4)
  \centering \Large Modelo de empate
 \end{textblock}
\vspace{1.25cm}
 
 \begin{equation*}
  p(r) = 
  \begin{cases}
   1 - \Phi(\varepsilon| \delta, \vartheta^2) & \text{ gana } \\
   \Phi(\varepsilon| \delta, \vartheta^2) - \Phi(-\varepsilon| \delta, \vartheta^2) & \text{ empata } \\
   \Phi(-\varepsilon| \delta, \vartheta^2) & \text{ pierde }
  \end{cases}
 \end{equation*}

 \vspace{0.5cm}

 \centering
\includegraphics[page=1,width=0.49\textwidth]{figures/draw}

 
\end{frame}

\begin{frame}[plain]
 \centering \Large \vspace{1cm}
 
 >Qu\'e valor de margen $\varepsilon$ deberiamos usar?

\end{frame}

\begin{frame}[plain]
 \begin{textblock}{160}(0,4)
  \centering \Large Propuesta original \\ \normalsize (Herbrich et al. 2006)
 \end{textblock}
\centering 

\only<2-3>{
\begin{textblock}{160}(0,18)
\begin{align*}
 p(r=\text{empate}) = \Phi(\varepsilon| \delta, \vartheta^2) - \Phi(-\varepsilon| \delta, \vartheta^2) 
\end{align*}
\end{textblock}
}

\only<4->{
\begin{textblock}{160}(0,18)
\begin{align*}
 p(r=\text{empate}) = \Phi(\varepsilon| 0 , N\beta^2) - \Phi(-\varepsilon| 0, N\beta^2) 
\end{align*}
\end{textblock}
}


\only<1-2>{
\begin{textblock}{160}(0,40)
\centering \Large 
Usar la frecuencia empírica de empates
\end{textblock}
}

\only<3>{
\begin{textblock}{160}(0,40)
\centering
La frecuencia de empates depende \\ de la diferencia de habilidad $\delta$, \\ \textbf{que justamente no conocemos!}
\end{textblock}
}

\only<4>{
\begin{textblock}{160}(0,40)
\centering
Suponer que la frecuencia surge \\ de jugadores que sabemos\\ tienen misma habilidad \\
\end{textblock}
}


\only<5>{
 \begin{textblock}{160}(0,40)
 \centering
\includegraphics[page=2,width=0.49\textwidth]{figures/draw}
\end{textblock}
}


\end{frame}


\begin{frame}[plain]
 \begin{textblock}{160}(0,4)
  \centering \Large Modelo de empate
 \end{textblock}
\vspace{1.25cm}
 
 \begin{equation*}
  p(r) = 
  \begin{cases}
   1 - \Phi(\varepsilon| \delta, \vartheta^2) & \text{ gana } \\
   \Phi(\varepsilon| \delta, \vartheta^2) - \Phi(-\varepsilon| \delta, \vartheta^2) & \text{ empata } \\
   \Phi(-\varepsilon| \delta, \vartheta^2) & \text{ pierde }
  \end{cases}
 \end{equation*}

 \vspace{0.5cm}

 \centering
\includegraphics[page=1,width=0.49\textwidth]{figures/draw}

 
\end{frame}


\begin{frame}[plain]
 \centering \Large \vspace{1cm}
 
 >Qu\'e valor de margen $\varepsilon$ deberiamos usar?

\end{frame}

\begin{frame}[plain]
 \begin{textblock}{160}(0,4)
  \centering \Large Propuesta original \\ \normalsize (Herbrich et al. 2006)
 \end{textblock}
\centering 

\only<2-3>{
\begin{textblock}{160}(0,18)
\begin{align*}
 p(r=\text{empate}) = \Phi(\varepsilon| \delta, \vartheta^2) - \Phi(-\varepsilon| \delta, \vartheta^2) 
\end{align*}
\end{textblock}
}

\only<4->{
\begin{textblock}{160}(0,18)
\begin{align*}
 p(r=\text{empate}) = \Phi(\varepsilon| 0 , N\beta^2) - \Phi(-\varepsilon| 0, N\beta^2) 
\end{align*}
\end{textblock}
}


\only<1-2>{
\begin{textblock}{160}(0,40)
\centering \Large 
Usar la frecuencia empírica de empates
\end{textblock}
}

\only<3>{
\begin{textblock}{160}(0,40)
\centering
La frecuencia de empates depende \\ de la diferencia de habilidad $\delta$, \\ \textbf{que justamente no conocemos!}
\end{textblock}
}

\only<4>{
\begin{textblock}{160}(0,40)
\centering
Suponer que la frecuencia surge \\ de jugadores que sabemos\\ tienen misma habilidad \\
\end{textblock}
}


\only<5>{
 \begin{textblock}{160}(0,40)
 \centering
\includegraphics[page=2,width=0.49\textwidth]{figures/draw}
\end{textblock}
}


\end{frame}



\end{document}



