\documentclass[shownotes,aspectratio=169]{beamer}

\input{../../aux/tex/diapo_encabezado.tex}
\input{../../aux/tex/tikzlibrarybayesnet.code.tex}
 \mode<presentation>
 {
 %   \usetheme{Madrid}      % or try Darmstadt, Madrid, Warsaw, ...
 %   \usecolortheme{default} % or try albatross, beaver, crane, ...
 %   \usefonttheme{serif}  % or try serif, structurebold, ...
  \usetheme{Antibes}
  \setbeamertemplate{navigation symbols}{}
 }
 
\usepackage{todonotes}
\setbeameroption{show notes}

\newif\ifen
\newif\ifes
\newcommand{\en}[1]{\ifen#1\fi}
\newcommand{\es}[1]{\ifes#1\fi}
\estrue

%\title[Bayes del Sur]{}

\begin{document}

\color{black!85}
\large


 \begin{textblock}{90}(00,05)
\begin{center}
 \huge  \textcolor{black!66}{Creencias adaptativas}
\end{center}
\end{textblock}

 %\vspace{2cm}brown
%\maketitle
\Wider[2cm]{
\includegraphics[width=1\textwidth]{../../aux/static/peligro_predador}
}
\end{frame}


\begin{frame}[plain]
\begin{textblock}{160}(0,4)
\centering \Large Flujos de inferencia
\end{textblock}

\only<2->{
\begin{textblock}{160}(0,18)
\centering
 \begin{tabular}{c c|c}
 & $\hfrac{\text{Intermedio}}{\text{no observable}}$ &   $\hfrac{\text{Intermedio}}{\text{observable}}$ \\
 & & \\
 $ p \rightarrow r \rightarrow a $    & \onslide<3->{$P(a) \overset{?}{=} P(a|p)$} & \onslide<4->{$P(a|r) \overset{?}{=} P(a|p,r)$} \\ 
 $a \leftarrow r \leftarrow p $      &  \onslide<5->{$P(p) \overset{?}{=} P(p|a)$}  & \onslide<6->{$P(p|r) \overset{?}{=} P(p|a,r)$} \\ 
 $ r \leftarrow p \rightarrow c $     & \onslide<7->{$P(c) \overset{?}{=} P(c|r)$} & \onslide<8->{$P(c|p) \overset{?}{=} P(c|r,p)$} \\
 $ d \rightarrow r \leftarrow p $     & \onslide<9->{$P(p) \overset{?}{=} P(p|d)$} & \onslide<10->{$P(p|r) \overset{?}{=} P(p|d,r)$} \\
                                      & \onslide<11->{$P(p) \overset{?}{=} P(p|d)$} & \onslide<11->{$P(p|a) \overset{?}{=} P(p|d,a)$}
 \end{tabular} 
 \end{textblock}
 }
 
 
\only<13->{
\footnotesize
\begin{textblock}{140}(10,60)

\begin{framed}

 Un flujo de inferencia permenece abierto si:
 \begin{itemize}
  \item[$\bullet$] Toda consecuencia com\'un (o alguno de sus descendientes) es observable
  \item[$\bullet$] Ning\'una otra variable es observable
 \end{itemize}
 \end{framed}

\end{textblock}

}

\only<14>{
\begin{textblock}{160}(0,92)
\tiny \centering
\href{http://93.174.95.29/_ads/6B4DC58C0028F2BBEAA2CC9204F01845}{Pearl, J. 2018. The Book of Why: The New Science of Cause and Effect}
\end{textblock}
}


\end{frame}


 
\begin{frame}[plain]
\centering
  \includegraphics[width=0.35\textwidth]{../../aux/static/pachacuteckoricancha.jpg}
\end{frame}




\end{document}



