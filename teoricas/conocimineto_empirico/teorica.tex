\documentclass[shownotes,aspectratio=169]{beamer}

\input{../../aux/tex/diapo_encabezado.tex}
% tikzlibrary.code.tex
%
% Copyright 2010-2011 by Laura Dietz
% Copyright 2012 by Jaakko Luttinen
%
% This file may be distributed and/or modified
%
% 1. under the LaTeX Project Public License and/or
% 2. under the GNU General Public License.
%
% See the files LICENSE_LPPL and LICENSE_GPL for more details.

% Load other libraries
\usetikzlibrary{shapes}
\usetikzlibrary{fit}
\usetikzlibrary{chains}
\usetikzlibrary{arrows}

% Latent node
\tikzstyle{latent} = [circle,fill=white,draw=black,inner sep=1pt,
minimum size=20pt, font=\fontsize{10}{10}\selectfont, node distance=1]
% Observed node
\tikzstyle{obs} = [latent,fill=gray!25]
% Invisible node
\tikzstyle{invisible} = [latent,minimum size=0pt,color=white, opacity=0, node distance=0]
% Constant node
\tikzstyle{const} = [rectangle, inner sep=0pt, node distance=0.1]
%state
\tikzstyle{estado} = [latent,minimum size=8pt,node distance=0.4]
%action
\tikzstyle{accion} =[latent,circle,minimum size=5pt,fill=black,node distance=0.4]


% Factor node
\tikzstyle{factor} = [rectangle, fill=black,minimum size=10pt, draw=black, inner
sep=0pt, node distance=1]
% Deterministic node
\tikzstyle{det} = [latent, rectangle]

% Plate node
\tikzstyle{plate} = [draw, rectangle, rounded corners, fit=#1]
% Invisible wrapper node
\tikzstyle{wrap} = [inner sep=0pt, fit=#1]
% Gate
\tikzstyle{gate} = [draw, rectangle, dashed, fit=#1]

% Caption node
\tikzstyle{caption} = [font=\footnotesize, node distance=0] %
\tikzstyle{plate caption} = [caption, node distance=0, inner sep=0pt,
below left=5pt and 0pt of #1.south east] %
\tikzstyle{factor caption} = [caption] %
\tikzstyle{every label} += [caption] %

\tikzset{>={triangle 45}}

%\pgfdeclarelayer{b}
%\pgfdeclarelayer{f}
%\pgfsetlayers{b,main,f}

% \factoredge [options] {inputs} {factors} {outputs}
\newcommand{\factoredge}[4][]{ %
  % Connect all nodes #2 to all nodes #4 via all factors #3.
  \foreach \f in {#3} { %
    \foreach \x in {#2} { %
      \path (\x) edge[-,#1] (\f) ; %
      %\draw[-,#1] (\x) edge[-] (\f) ; %
    } ;
    \foreach \y in {#4} { %
      \path (\f) edge[->,#1] (\y) ; %
      %\draw[->,#1] (\f) -- (\y) ; %
    } ;
  } ;
}

% \edge [options] {inputs} {outputs}
\newcommand{\edge}[3][]{ %
  % Connect all nodes #2 to all nodes #3.
  \foreach \x in {#2} { %
    \foreach \y in {#3} { %
      \path (\x) edge [->,#1] (\y) ;%
      %\draw[->,#1] (\x) -- (\y) ;%
    } ;
  } ;
}

% \factor [options] {name} {caption} {inputs} {outputs}
\newcommand{\factor}[5][]{ %
  % Draw the factor node. Use alias to allow empty names.
  \node[factor, label={[name=#2-caption]#3}, name=#2, #1,
  alias=#2-alias] {} ; %
  % Connect all inputs to outputs via this factor
  \factoredge {#4} {#2-alias} {#5} ; %
}

% \plate [options] {name} {fitlist} {caption}
\newcommand{\plate}[4][]{ %
  \node[wrap=#3] (#2-wrap) {}; %
  \node[plate caption=#2-wrap] (#2-caption) {#4}; %
  \node[plate=(#2-wrap)(#2-caption), #1] (#2) {}; %
}

% \gate [options] {name} {fitlist} {inputs}
\newcommand{\gate}[4][]{ %
  \node[gate=#3, name=#2, #1, alias=#2-alias] {}; %
  \foreach \x in {#4} { %
    \draw [-*,thick] (\x) -- (#2-alias); %
  } ;%
}

% \vgate {name} {fitlist-left} {caption-left} {fitlist-right}
% {caption-right} {inputs}
\newcommand{\vgate}[6]{ %
  % Wrap the left and right parts
  \node[wrap=#2] (#1-left) {}; %
  \node[wrap=#4] (#1-right) {}; %
  % Draw the gate
  \node[gate=(#1-left)(#1-right)] (#1) {}; %
  % Add captions
  \node[caption, below left=of #1.north ] (#1-left-caption)
  {#3}; %
  \node[caption, below right=of #1.north ] (#1-right-caption)
  {#5}; %
  % Draw middle separation
  \draw [-, dashed] (#1.north) -- (#1.south); %
  % Draw inputs
  \foreach \x in {#6} { %
    \draw [-*,thick] (\x) -- (#1); %
  } ;%
}

% \hgate {name} {fitlist-top} {caption-top} {fitlist-bottom}
% {caption-bottom} {inputs}
\newcommand{\hgate}[6]{ %
  % Wrap the left and right parts
  \node[wrap=#2] (#1-top) {}; %
  \node[wrap=#4] (#1-bottom) {}; %
  % Draw the gate
  \node[gate=(#1-top)(#1-bottom)] (#1) {}; %
  % Add captions
  \node[caption, above right=of #1.west ] (#1-top-caption)
  {#3}; %
  \node[caption, below right=of #1.west ] (#1-bottom-caption)
  {#5}; %
  % Draw middle separation
  \draw [-, dashed] (#1.west) -- (#1.east); %
  % Draw inputs
  \foreach \x in {#6} { %
    \draw [-*,thick] (\x) -- (#1); %
  } ;%
}


 \mode<presentation>
 {
 %   \usetheme{Madrid}      % or try Darmstadt, Madrid, Warsaw, ...
 %   \usecolortheme{default} % or try albatross, beaver, crane, ...
 %   \usefonttheme{serif}  % or try serif, structurebold, ...
  \usetheme{Antibes}
  \setbeamertemplate{navigation symbols}{}
 }
 
\usepackage{todonotes}
\setbeameroption{show notes}

\newif\ifen
\newif\ifes
\newcommand{\en}[1]{\ifen#1\fi}
\newcommand{\es}[1]{\ifes#1\fi}
\estrue

%\title[Bayes del Sur]{}

\begin{document}

\color{black!85}
\large

 
%\setbeamercolor{background canvas}{bg=gray!15}

\begin{frame}[plain,noframenumbering]
 
 \begin{textblock}{90}(00,05)
\begin{center}
 \huge  \textcolor{black!66}{Creencias, datos y sorpresas}
\end{center}
\end{textblock}

 %\vspace{2cm}brown
%\maketitle
\Wider[2cm]{
\includegraphics[width=1\textwidth]{../../aux/static/peligro_predador}
}
\end{frame}

% 
% \begin{frame}[plain]
% \begin{textblock}{160}(0,4)
%  \centering
%  \LARGE \textcolor{black!85}{\en{Today}\es{Hoy}}
% \end{textblock}
% 
% \begin{itemize}
%  \item[$\bullet$] Human dispersal
%  \item[$\bullet$] Biomass (dentro de los vertebrados terrestres)
%  \item[$\bullet$] Empatia
%  \item[$\bullet$] Evolución cultural
%  \item[$\bullet$] Ciencia como intersubjectividad (muto entendimiento)
%  \item[$\bullet$] Base empírica 
%  \item[$\bullet$] Matriz de datos 
%  \item[$\bullet$] Ciencia empírica 
%  \item[$\bullet$] Niveles de conocimiento
%  \item[$\bullet$] Incertidumbre
%  \item[$\bullet$] Creencias honestas
%  \item[$\bullet$] Modelos causales
%  \item[$\bullet$] La lógica de la ciencia empírica
%  \item[$\bullet$] Selección de modelo
% \end{itemize}
% 
% \end{frame}

\begin{frame}[plain]
\begin{textblock}{160}(0,4)
 \centering \LARGE
 Humanos
\end{textblock}
\vspace{1.2cm}
\Wider[1cm]{
\includegraphics[width=1\textwidth]{../../aux/static/mapamundi2.jpg}
}
\end{frame}

\begin{frame}[plain]
\begin{textblock}{160}(0,4)
 \centering \LARGE
 Ocupamos todos los nichos ecológicos
\end{textblock}
\Wider[2cm]{
\includegraphics[width=1\textwidth]{../../aux/static/inuit_igloo_low}
}
\end{frame}



\begin{frame}[plain]
\begin{textblock}{160}(0,4)
 \centering \LARGE
 \en{The cognitive hypotesis}
 \es{La hipótesis cognitiva}
\end{textblock}
\centering \vspace{1cm}
 \includegraphics[width=0.5\textwidth]{../../aux/static/cerebros}  
\end{frame}



\begin{frame}[plain]

 \begin{textblock}{160}(5,30)
\includegraphics[width=0.33\textwidth]{../../aux/static/evolucionLitica.jpg} 
\end{textblock}
\begin{textblock}{160}(0,40)
 \centering \LARGE
 \en{Cultural \\ evolution}
 \es{Evolución \\ cultural}
\end{textblock}
\begin{textblock}{160}(105,1)
\includegraphics[width=0.285\textwidth]{../../aux/static/debian_tree.png}
\end{textblock}
\end{frame}


\begin{frame}[plain]
\begin{textblock}{160}(0,4)
 \centering \LARGE 
 La hipótesis cultural
 \end{textblock}
\vspace{1cm}

\includegraphics[width=1\textwidth]{../../aux/static/evolucionCultural.jpg}

\end{frame}
% 
% \begin{frame}[plain]
% \begin{textblock}{160}(0,4)
%  \centering \LARGE 
%  \en{Evolution of empathy}
%  \es{La evolución de la empatía}
% \end{textblock}
% \vspace{1cm}
%  
% %Wired To mutual understanding
% %Cableados para la comprensión mutua
%  \centering
% \includegraphics[width=0.7\textwidth]{../../aux/static/empatia}
% 
% % Cooperative breeding permited the evolution of extended life span, prolonged childhoods
% 
% % \centering
% % \en{Cooperative breeding permited the evolution of empathy}
% % \es{La crianza cooperative permitió la evolución de la empatía}
%  
% %Humans are often eager to understand others, to be understood, and to cooperate. 
% %Los seres humanos suelen estar ansiosos por comprender a los demás, por ser comprendidos y por cooperar. 
% 
% 
% \end{frame}

 
\begin{frame}[plain]
\centering
  \includegraphics[width=0.35\textwidth]{../../aux/static/pachacuteckoricancha.jpg}
\end{frame}






\end{document}



