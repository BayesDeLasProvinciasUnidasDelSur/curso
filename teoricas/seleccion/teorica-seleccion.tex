\documentclass[shownotes,aspectratio=169]{beamer}

\input{../../auxiliar/tex/diapo_encabezado.tex}
\input{../../auxiliar/tex/tikzlibrarybayesnet.code.tex}
 \mode<presentation>
 {
 %   \usetheme{Madrid}      % or try Darmstadt, Madrid, Warsaw, ...
 %   \usecolortheme{default} % or try albatross, beaver, crane, ...
 %   \usefonttheme{serif}  % or try serif, structurebold, ...
  \usetheme{Antibes}
  \setbeamertemplate{navigation symbols}{}
 }
 
\usepackage{todonotes}
\setbeameroption{show notes}

\newif\ifen
\newif\ifes
\newcommand{\en}[1]{\ifen#1\fi}
\newcommand{\es}[1]{\ifes#1\fi}
\estrue

%\title[Bayes del Sur]{}

\begin{document}

\color{black!85}
\large

\begin{frame}[plain,noframenumbering]
 
 \begin{textblock}{90}(00,05)
\begin{center}
 \huge  \textcolor{black!66}{Creencias, datos y sorpresas}
\end{center}
\end{textblock}

 %\vspace{2cm}brown
%\maketitle
\Wider[2cm]{
\includegraphics[width=1\textwidth]{../../auxiliar/static/peligro_predador}
}
\end{frame}


\begin{frame}[plain]

\vspace{0.75cm}

 \begin{center}
  \Large >Cuál es la forma \'optima de actualizar creencias 
  
dadas restricciones: modelos y datos?
 \end{center}

\begin{textblock}{160}(0,86)
\tiny \centering
\href{https://doi.org/10.1016/S0888-613X(03)00051-3}{Teorema de Cox}
\end{textblock}

\end{frame}
 
 

\begin{frame}[plain]
\begin{textblock}{160}(0,4)
 \centering \LARGE 
 \en{Hidden variables and honest reasoning}
 \es{Creencias y razonamiento \textbf{honesto}}
 \end{textblock}
 \vspace{1cm}
 
\begin{center}
  Regla 2 \\
\LARGE
\textbf{Contextualizar las creencias \\ en partes iguales}
 \end{center}

 \begin{align*}
 P(r|s_2) = \frac{P(r, s_2)}{P(s_2)}
 \end{align*} 

 
\end{frame}



\begin{frame}[plain]
\begin{textblock}{160}(0,4)
\centering \LARGE  Las reglas de la probabilidad
\end{textblock}

\vspace{1cm}



\begin{equation*}
  \text{Marginal}_{i} = \sum_j \text{Conjunta}_{ij}  \ \ \ \ \ \ \ \ \ \ \ \  \text{Condicional}_{j|i} = \frac{\text{Conjunta}_{ij}}{\text{Marginal}_{i}}
\end{equation*}

\pause
\vspace{1cm}


\begin{columns}[t]
\begin{column}{0.5\textwidth}
 \centering \textbf{Regla de la suma}
 
 
\begin{equation*}
 P(r) = \sum_j P(r,s_j)
\end{equation*}
 
 \justifying \footnotesize
  Cualquier distribución marginal puede ser obtenida integrando la distribución conjunta

 \end{column}
 \begin{column}{0.5\textwidth}
\centering  \textbf{Regla del producto}

\begin{equation*}
 P(r,s) = P(s|r) P(r)
\end{equation*}

 \justifying \footnotesize
Cualquier distribución conjunta puede ser expresada como el producto de distribuciones condicionales uni-dimensionles.

\end{column}
\end{columns}

\end{frame}

\begin{frame}[plain]
\begin{textblock}{160}(0,4)
\centering \LARGE 
\st{Teorema de Bayes} \\
\Large El corolario de Laplace
\end{textblock}

\only<1-4>{
\begin{textblock}{160}(0,34)
 \begin{align*}
  \phantom{P(r_i)} P(r_i|s_2) & = \frac{P(r_i, s_2)}{P(s_2)} \onslide<4>{= \frac{P(s_2|r_i)P(r_i)}{P(s_2)}}
  \only<2>{\\ P(s_2 | r_i) &= \frac{P(r_i, s_2)}{P(r_i)}}
  \only<3-4>{\\[0.27cm]P(r_i) P(s_2 | r_i)  &= P(r_i, s_2)}
 \end{align*}
\end{textblock}
}

\only<5>{
\begin{textblock}{160}(0,34)
 \begin{align*}
  \phantom{P(r_i)} P(r_i|s_2) & = \frac{P(s_2|r_i)P(r_i)}{P(s_2)} \phantom{= \frac{P(r_i, s_2)}{P(s_2)} \ }
  \end{align*}
\end{textblock}
}



\only<6>{
\begin{textblock}{160}(0,43)
\begin{equation*}
P(\text{Hip\'otesis }|\text{ Dato}) = \frac{P(\text{Dato }|\text{ Hip\'otesis}) P(\text{Hip\'otesis})}{P(\text{Dato})}
\end{equation*}
\end{textblock}
}


\only<7>{
\begin{textblock}{160}(0,37.75)
\begin{equation*}
\underbrace{P(\text{Hip\'otesis }|\text{ Datos})}_{\text{\scriptsize Posteriori}} = \frac{\overbrace{P(\text{Datos }|\text{ Hip\'otesis})}^{\text{\scriptsize Verosimilitud}} \overbrace{P(\text{Hip\'otesis})}^{\text{\scriptsize Priori}} }{\underbrace{P(\text{Datos})}_{\text{\scriptsize Evidencia}}}
\end{equation*}
\end{textblock}
}

\vspace{0.2cm}

\only<8->{  
%\vspace{0.3cm}
\Wider[2cm]{
\begin{textblock}{160}(0,34.25) 
\begin{equation*}
\underbrace{P(\text{Hip\'otesis }|\text{ Datos, Modelo})}_{\text{\scriptsize Posteriori}} = \frac{\overbrace{P(\text{Datos }|\text{ Hip\'otesis, Modelo})}^{\text{\scriptsize Verosimilitud}} \overbrace{P(\text{Hip\'otesis }|\text{ Modelo})}^{\text{\scriptsize Priori}} }{\underbrace{P(\text{Datos }|\text{ Modelo})}_{\text{\scriptsize Evidencia}}}
\end{equation*}

\end{textblock}

}
}

\end{frame}

\begin{frame}[plain]

\vspace{0.75cm}

 \begin{center}
 \Large ¿Y nuestras creencias respecto de modelos alternativos?  
 \end{center}

\end{frame}


\begin{frame}[plain]
\begin{textblock}{160}(0,4)
 \centering \Large Modelos \\
\end{textblock}

\begin{textblock}{160}(16,24) 
\begin{equation*}
 P(\text{Modelo}|\text{Datos}) = \frac{\only<1>{P(\text{Datos}|\text{Modelo})\,}\only<2->{\overbrace{P(\text{Datos}|\text{Modelo})}^{\text{Evidencia}}}P(\text{Modelo})}{ P(\text{Datos})} \phantom{\frac{\overbrace{P(\text{Datos}|\text{Modelo})}^{\text{Evidencia}}}{ P(\text{Datos})}}
\end{equation*}
\end{textblock}

\only<3>{
\begin{textblock}{160}(0,50) 
\begin{equation*}
 P(\text{Datos}) = \sum_i P(\text{Datos}|\text{Modelo})P(\text{Modelo})
\end{equation*}
\end{textblock}
}


\end{frame}

\begin{frame}[plain]
\begin{textblock}{160}(0,4)
 \centering \Large Evidencia \\
\end{textblock}
\only<1>{
\begin{textblock}{160}(0,30) 
\begin{align*}
P(\text{Datos}|\text{Modelo}) & = \sum_{i}^H P(\text{Datos}|\text{Hip\'otesis}_i,\text{Modelo}) P(\text{Hip\'otesis}_i|\text{Modelo}) 
\end{align*}
\end{textblock}
}

\only<2->{
\begin{textblock}{160}(0,30) 
\begin{align*}
P(\text{Datos}|\text{Modelo}) & = \sum_{i}^H P(\text{Datos}|\text{Hip\'otesis}_i,\text{Modelo}) P(\text{Hip\'otesis}_i|\text{Modelo}) \\
& = P(\text{D}_1|\text{M}) P(\text{D}_2|\text{D}_1, \text{M}) \dots P(\text{D}_N|\text{D}_{N-1} \dots \text{D}_1, \text{M})  
\end{align*}
\end{textblock}
}


% \only<7>{
% \begin{textblock}{128}(0,92)
% \centering \tiny Sobre comparaci\'on de modelos ver: \href{http://xyala.cap.ed.ac.uk/teaching/tutorials/phylogenetics/Bayesian_Workshop/PDFs/Kass\%20and\%20Raftery\%201995.pdf}{Kass \& Raftery. Bayes factors. 1995.}
% \end{textblock}
% }
\end{frame}

\begin{frame}[plain]
\begin{textblock}{160}(0,4)
\centering \Large Modelos lineales
\end{textblock}
 \vspace{1.25cm}
 
\begin{equation*}
y(\bm{x},\bm{w}) = \sum_{i=0}^{M-1} w_i \phi_i(\bm{x}) = \bm{w}^T \bm{\phi}(\bm{x})
\end{equation*}

\vspace{0.5cm}
\pause
% 
% \begin{equation*}
%  t = y(\vm{x},\vm{w}) + \epsilon  \ \ \ \ \  \text{con} \ \ \  \epsilon \sim \N(0,\beta^{-1})
% \end{equation*}
 
 \begin{equation*}
p(t \, | \, \bm{x}, \bm{w}, \beta) = \N(t \, | \, y(\bm{x},\bm{w}) , \beta^{-1})
\end{equation*}
\vspace{0.025cm}
\pause
 
\begin{equation*}
P(\bm{t} | \bm{x}, \bm{w}, \beta) = \prod_{i=1}^n \N(t_i | \bm{w}^T \bm{\phi}(\bm{x}_i) , \beta^{-1}) = \N(\bm{t}|\bm{w}^T \bm{\Phi}, \beta^{-1} \vm{I})
\end{equation*}
\vspace{0.05cm}
\pause

\begin{equation*}
 \bm{\Phi} =
  \begin{pmatrix}
    \phi_0(\bm{x}_1) & \phi_1(\bm{x}_1) & \dots & \phi_{M-1}(\bm{x}_1)\\
    \vdots & \vdots & \ddots & \vdots \\
    \phi_0(\bm{x}_N) & \phi_1(\bm{x}_N) & \dots & \phi_{M-1}(\bm{x}_N)
  \end{pmatrix}
  = 
  \begin{pmatrix}
   \bm{\phi}(\vm{x}_1)^T \\
   \vdots \\
   \bm{\phi}(\vm{x}_N)^T \\
  \end{pmatrix}
\end{equation*}

 
\end{frame}



\begin{frame}[plain]
\begin{textblock}{160}(0,4)
 \centering \Large Solución fecuentista
\end{textblock}
\vspace{1.25cm}


\begin{equation*}
 \underset{\bm{w}}{\text{ max }} P(\bm{t} | \bm{x}, \bm{w}, \beta) = \underset{\bm{w}}{\text{ min }} \sum_{i=1}^{n}  (t_i - \bm{w}^T\bm{\phi}(\bm{x}_i))^2 
\end{equation*}

\end{frame}


\begin{frame}[plain]
\begin{textblock}{160}(0,4)
 \Large \centering Soluci\'on completa
\end{textblock}
 \vspace{1cm}

\begin{equation*}
 p(\vm{w}) = N(\vm{w}| \vm{0}, \alpha^{-1} \vm{I})
\end{equation*}
 
\begin{figure}[H]
    \centering
    \tikz{

    \node[latent, fill=black!100, minimum size=2pt] (x) {} ; %
    \node[const, right=of x] (c_x) {$\vm{X}$};
    \node[latent, fill=black!20, yshift=-1.5cm] (t) {$\bm{t}$} ; %
    \node[latent, fill=black!100, yshift=-1.5cm , xshift=-2cm,minimum size=2pt] (beta)
    {} ; %
    \node[const, above=of beta] (c_beta) {$\beta$};
    \node[latent, fill=black!0, yshift=-1.5cm, xshift=2cm] (w) {$\vm{w}$};
    \node[latent, fill=black!100, xshift=2cm, minimum size=2pt] (alpha) {} ; %
    \node[const, right=of alpha] (c_alpha) {$\alpha$};
    
    \edge {x,beta,w} {t};
    \edge {alpha} {w};
    
    \node[invisible, fill=black!0, minimum size=0pt, xshift=-0.52cm] (data_inv) {} ; %
      
    }  
\end{figure}
\end{frame}


\begin{frame}[plain]
\begin{textblock}{160}(0,4)
 \Large \centering Soluci\'on completa
\end{textblock}
 \vspace{0.75cm}



 La distribuci\'on \textbf{posteriori} sobre $\vm{w}$ tiene como media

\begin{equation}
 \vm{m}_N = \beta  \vm{S}_N\vm{\Phi}^T \vm{t}
\end{equation}

y como covarianzas

\begin{equation}
 \vm{S}_N^{-1} = \alpha \vm{I} + \beta \vm{\Phi}^T\vm{\Phi}
\end{equation}

\pause


Y la \textbf{evidencia} del modelo es

\begin{equation}
 P(t) = \N(\bm{t}| \vm{0}, \beta^{-1} \vm{I} + \alpha^{-1}\bm{\Phi}\bm{\Phi}^T  )
\end{equation}

\vspace{0.75cm}

\small
\Wider[-8cm]{
\begin{mdframed}[backgroundcolor=black!15]\centering
 Cap\'itulo 2 de Bishop
\end{mdframed}
}

\end{frame}

% 
% \begin{frame}
% \begin{textblock}{128}(0,8)
% \begin{center}
%  \normalsize Propiedades importantes para la vida
% \end{center}
% \end{textblock}
% \vspace{0.75cm}
% 
% \tiny
% 
% Dadas las distribuciones
% \begin{align*}
%     P(\vm{x}) &=  \N(\vm{x}|\bm{\mu},\bm{\Lambda}^{-1}) \\
%    P(\vm{y}|\vm{x}) &=  \N(\vm{y}|\vm{A}\vm{x}+\vm{b},\vm{L}^{-1})    
% \end{align*}
% 
% 
%  Luego,  
% \begin{align*}
%     P(\vm{y}) &=  \N(\vm{y}|\vm{A}\bm{\mu}+\vm{b},\, \vm{L}^{-1} + \vm{A}\vm{\Lambda}^{-1}\vm{A}^T) \\
%    P(\vm{x}|\vm{y}) &=  \N(\vm{x}|\vm{\Sigma}[\vm{A}^T \vm{L}(\vm{y}-\vm{b})+ \bm{\Lambda\mu}],\,  \vm{\Sigma})\label{eq:post}
% \end{align*}
% 
% donde, 
% \begin{equation*}
%  \vm{\Sigma} = (\vm{\Lambda} + \vm{A}^T \vm{LA} )^{-1}
% \end{equation*}
% 
% \vspace{0.3cm}
% 
% \small
% \begin{mdframed}[backgroundcolor=black!15]\centering
%  Leer cap\'itulo 2 de Bishop y/o anexo de la pr\'actica
% \end{mdframed}
% 
% \end{frame}


\begin{frame}[plain]

\Wider[-3cm]{
 \begin{figure}
\begin{subfigure}[t]{0.32\textwidth} 
\caption*{Verosimilitud} 
\end{subfigure}
\begin{subfigure}[t]{0.32\textwidth}
\caption*{Priori/Posteriori} 
\includegraphics[width=\textwidth]{figures/pdf/linearRegression_posterior_0.pdf} 
\end{subfigure}
\begin{subfigure}[t]{0.32\textwidth}
\caption*{Data space} 
\includegraphics[width=\textwidth]{figures/pdf/linearRegression_dataSpace_0.pdf} 
\end{subfigure}


\begin{subfigure}[c]{0.32\textwidth}
\onslide<2->{\includegraphics[width=\textwidth]{figures/pdf/linearRegression_likelihood_1.pdf}}
\end{subfigure}
\begin{subfigure}[c]{0.32\textwidth}
\onslide<3->{\includegraphics[width=\textwidth]{figures/pdf/linearRegression_posterior_1.pdf}}
\end{subfigure}
\begin{subfigure}[c]{0.32\textwidth}
\onslide<4->{\includegraphics[width=\textwidth]{figures/pdf/linearRegression_dataSpace_1.pdf}}
\end{subfigure}

\begin{subfigure}[c]{0.32\textwidth}
\onslide<5->{\includegraphics[width=\textwidth]{figures/pdf/linearRegression_likelihood_2.pdf}}
\end{subfigure}
\begin{subfigure}[c]{0.32\textwidth}
\onslide<6->{\includegraphics[width=\textwidth]{figures/pdf/linearRegression_posterior_2.pdf}}
\end{subfigure}
\begin{subfigure}[c]{0.32\textwidth}
\onslide<7->{\includegraphics[width=\textwidth]{figures/pdf/linearRegression_dataSpace_2.pdf}}
\end{subfigure}

\end{figure}
}
\end{frame}


\begin{frame}[plain]
\begin{textblock}{160}(0,4)
\centering  \Large Regresi\'on lineal Bayesiana
\end{textblock}


\begin{textblock}{80}(0,20)
 \centering
 \onslide<1->{Funci\'on objetivo}
\end{textblock}

\begin{textblock}{80}(80,20)
 \centering
 \onslide<2>{Modelos polinomiales}
\end{textblock}

\begin{textblock}{160}(0,26)
     \centering 
       \onslide<1->{\includegraphics[width=0.45\textwidth]{figures/pdf/model_selection_true_and_sample} }
       \onslide<2>{\includegraphics[width=0.45\textwidth]{figures/pdf/model_selection_MAP} }
\end{textblock}

\end{frame}

\begin{frame}[plain]
\begin{textblock}{160}(0,4)
 \centering \Large Evidencia vs Verosimilitud
\end{textblock}


\begin{textblock}{80}(0,22)
 \centering
Evidencia conjunta
\end{textblock}

\begin{textblock}{80}(80,22)
 \centering
 M\'axima verosimilitud
\end{textblock}


\begin{textblock}{160}(0,24)
     \centering 
       \begin{figure}[H]     
     \centering 
     \begin{subfigure}[b]{0.47\textwidth}
       \includegraphics[width=1\textwidth]{figures/pdf/model_selection_evidence}
     \end{subfigure}
     \begin{subfigure}[b]{0.49\textwidth}
       \includegraphics[width=1\textwidth]{figures/pdf/model_selection_maxLikelihood}
     \end{subfigure}
\end{figure}
\end{textblock}


\end{frame}



% \begin{frame}
%  \begin{textblock}{108}(10,08)
%  \begin{center}
%   \large Evidencia
%  \end{center}
% \end{textblock}
% 
% \begin{textblock}{108}(10,22)
% \begin{align*}
%  P(\text{C}|\text{D},\text{M}) = \frac{P(\text{D}|\text{C},\text{M})P(\text{C}|\text{M})}{\underbrace{P(\text{D}|\text{M})}_{\text{Evidencia}}}
% \end{align*}
% \end{textblock}
% 
% 
% 
% \only<2>{
% \begin{textblock}{108}(10,44)
% \begin{align*}
%  P(\text{D}|\text{M}) = \sum_C P(\text{D}|\text{C},\text{M})\, P(\text{C}|\text{M})
% \end{align*}
% \end{textblock}
% 
% }
% 
% 
% \only<3>{
% \begin{textblock}{108}(10,44)
% \begin{align*}
%  P(\text{D}|\text{M}) = \sum_C \underbrace{P(\text{D}|\text{C},\text{M})}_{\hfrac{\text{\tiny Predicci\'on}}{\text{\tiny de D dado C}}}\underbrace{P(\text{C}|\text{M})}_{\hfrac{\text{\tiny Creencia de}}{\text{\tiny C a Priori}}} %\text{La predicci\'on de los datos observados a partir de las creencias a priori}
% \end{align*}
% \end{textblock}
% 
% }
% 
% \only<4->{
% \begin{textblock}{108}(10,44)
% \begin{align*}
%  P(\text{D}|\text{M}) = \underbrace{ \sum_C \underbrace{P(\text{D}|\text{C},\text{M})}_{\hfrac{\text{\tiny Predicci\'on}}{\text{\tiny de D dado C}}}\underbrace{P(\text{C}|\text{M})}_{\hfrac{\text{\tiny Creencia de}}{\text{\tiny C a Priori}}}}_{\hfrac{\text{\tiny Predicci\'on de la datos observados}}{\text{\tiny pesando todas las creencias a priori}}}
% \end{align*}
% \end{textblock}
% }
% 
% 
% 
% 
% \end{frame}



\begin{frame}[plain]
\begin{textblock}{160}(0,4)
\centering \Large  Evidencia
\end{textblock}


 \begin{textblock}{120}(20,14)
  \centering
  \includegraphics[width=0.9\textwidth]{figures/pdf/evidencia_de_modelos_alternativos} 
 \end{textblock} 
 
 
 \begin{textblock}{100}(30,83)
  \centering
 \ \ \ \  Balance natural entre complejidad y predicci\'on
  \end{textblock}
\end{frame}


\begin{frame}[plain]

\centering
  \includegraphics[width=0.55\textwidth]{../../auxiliar/static/pachacuteckoricancha.jpg}
\end{frame}





\end{document}



