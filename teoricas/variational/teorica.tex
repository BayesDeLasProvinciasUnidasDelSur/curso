\documentclass[shownotes]{beamer}
\input{../../aux/tex/diapo_encabezado.tex}
% tikzlibrary.code.tex
%
% Copyright 2010-2011 by Laura Dietz
% Copyright 2012 by Jaakko Luttinen
%
% This file may be distributed and/or modified
%
% 1. under the LaTeX Project Public License and/or
% 2. under the GNU General Public License.
%
% See the files LICENSE_LPPL and LICENSE_GPL for more details.

% Load other libraries
\usetikzlibrary{shapes}
\usetikzlibrary{fit}
\usetikzlibrary{chains}
\usetikzlibrary{arrows}

% Latent node
\tikzstyle{latent} = [circle,fill=white,draw=black,inner sep=1pt,
minimum size=20pt, font=\fontsize{10}{10}\selectfont, node distance=1]
% Observed node
\tikzstyle{obs} = [latent,fill=gray!25]
% Invisible node
\tikzstyle{invisible} = [latent,minimum size=0pt,color=white, opacity=0, node distance=0]
% Constant node
\tikzstyle{const} = [rectangle, inner sep=0pt, node distance=0.1]
%state
\tikzstyle{estado} = [latent,minimum size=8pt,node distance=0.4]
%action
\tikzstyle{accion} =[latent,circle,minimum size=5pt,fill=black,node distance=0.4]


% Factor node
\tikzstyle{factor} = [rectangle, fill=black,minimum size=10pt, draw=black, inner
sep=0pt, node distance=1]
% Deterministic node
\tikzstyle{det} = [latent, rectangle]

% Plate node
\tikzstyle{plate} = [draw, rectangle, rounded corners, fit=#1]
% Invisible wrapper node
\tikzstyle{wrap} = [inner sep=0pt, fit=#1]
% Gate
\tikzstyle{gate} = [draw, rectangle, dashed, fit=#1]

% Caption node
\tikzstyle{caption} = [font=\footnotesize, node distance=0] %
\tikzstyle{plate caption} = [caption, node distance=0, inner sep=0pt,
below left=5pt and 0pt of #1.south east] %
\tikzstyle{factor caption} = [caption] %
\tikzstyle{every label} += [caption] %

\tikzset{>={triangle 45}}

%\pgfdeclarelayer{b}
%\pgfdeclarelayer{f}
%\pgfsetlayers{b,main,f}

% \factoredge [options] {inputs} {factors} {outputs}
\newcommand{\factoredge}[4][]{ %
  % Connect all nodes #2 to all nodes #4 via all factors #3.
  \foreach \f in {#3} { %
    \foreach \x in {#2} { %
      \path (\x) edge[-,#1] (\f) ; %
      %\draw[-,#1] (\x) edge[-] (\f) ; %
    } ;
    \foreach \y in {#4} { %
      \path (\f) edge[->,#1] (\y) ; %
      %\draw[->,#1] (\f) -- (\y) ; %
    } ;
  } ;
}

% \edge [options] {inputs} {outputs}
\newcommand{\edge}[3][]{ %
  % Connect all nodes #2 to all nodes #3.
  \foreach \x in {#2} { %
    \foreach \y in {#3} { %
      \path (\x) edge [->,#1] (\y) ;%
      %\draw[->,#1] (\x) -- (\y) ;%
    } ;
  } ;
}

% \factor [options] {name} {caption} {inputs} {outputs}
\newcommand{\factor}[5][]{ %
  % Draw the factor node. Use alias to allow empty names.
  \node[factor, label={[name=#2-caption]#3}, name=#2, #1,
  alias=#2-alias] {} ; %
  % Connect all inputs to outputs via this factor
  \factoredge {#4} {#2-alias} {#5} ; %
}

% \plate [options] {name} {fitlist} {caption}
\newcommand{\plate}[4][]{ %
  \node[wrap=#3] (#2-wrap) {}; %
  \node[plate caption=#2-wrap] (#2-caption) {#4}; %
  \node[plate=(#2-wrap)(#2-caption), #1] (#2) {}; %
}

% \gate [options] {name} {fitlist} {inputs}
\newcommand{\gate}[4][]{ %
  \node[gate=#3, name=#2, #1, alias=#2-alias] {}; %
  \foreach \x in {#4} { %
    \draw [-*,thick] (\x) -- (#2-alias); %
  } ;%
}

% \vgate {name} {fitlist-left} {caption-left} {fitlist-right}
% {caption-right} {inputs}
\newcommand{\vgate}[6]{ %
  % Wrap the left and right parts
  \node[wrap=#2] (#1-left) {}; %
  \node[wrap=#4] (#1-right) {}; %
  % Draw the gate
  \node[gate=(#1-left)(#1-right)] (#1) {}; %
  % Add captions
  \node[caption, below left=of #1.north ] (#1-left-caption)
  {#3}; %
  \node[caption, below right=of #1.north ] (#1-right-caption)
  {#5}; %
  % Draw middle separation
  \draw [-, dashed] (#1.north) -- (#1.south); %
  % Draw inputs
  \foreach \x in {#6} { %
    \draw [-*,thick] (\x) -- (#1); %
  } ;%
}

% \hgate {name} {fitlist-top} {caption-top} {fitlist-bottom}
% {caption-bottom} {inputs}
\newcommand{\hgate}[6]{ %
  % Wrap the left and right parts
  \node[wrap=#2] (#1-top) {}; %
  \node[wrap=#4] (#1-bottom) {}; %
  % Draw the gate
  \node[gate=(#1-top)(#1-bottom)] (#1) {}; %
  % Add captions
  \node[caption, above right=of #1.west ] (#1-top-caption)
  {#3}; %
  \node[caption, below right=of #1.west ] (#1-bottom-caption)
  {#5}; %
  % Draw middle separation
  \draw [-, dashed] (#1.west) -- (#1.east); %
  % Draw inputs
  \foreach \x in {#6} { %
    \draw [-*,thick] (\x) -- (#1); %
  } ;%
}



\mode<presentation>
{
%   \usetheme{Madrid}      % or try Darmstadt, Madrid, Warsaw, ...
%   \usecolortheme{default} % or try albatross, beaver, crane, ...
%   \usefonttheme{default}  % or try serif, structurebold, ...
 \usetheme{Antibes}
 \usecolortheme[rgb={0.6,0.75,0}]{structure}%divido los RGB por 252
 \setbeamercolor{block title}{fg=white,bg=azuluca}
 \xdefinecolor{azuluca}{rgb}{0.02, 0.2, 0.18}
 \definecolor{greenblue}{rgb}{0.1, 0.55, 0.5}

 \setbeamercolor{palette quaternary}{fg=white,bg=azuluca}
 \setbeamertemplate{caption}[numbered]
 \setbeamercolor{item projected}{bg=black}
 \setbeamertemplate{enumerate items}[default]
 \setbeamertemplate{navigation symbols}{}
 %\setbeamercovered{transparent}
 \setbeamercolor{block title}{fg=black}
 \setbeamercolor{local structure}{fg=black}

}

\usepackage{todonotes}
\setbeameroption{show notes}

\title[Inferencia variacional]{Inferencia variacional}

\institute[Bayes del sur]{\includegraphics[width=0.85\textwidth]{../../aux/static/peligro_predador}}

\date{\today}

\begin{document}

\small

\begin{frame}[noframenumbering]
 
 %\vspace{2cm}
\maketitle
 \end{frame}
 
\section{Problema}
 
\begin{frame}
\begin{textblock}{128}(0,8)
\begin{center}
 \large Enfoque Bayesiano
\end{center}
\end{textblock}
\vspace{1cm}
Dado un modelo,
\begin{equation*}
 p(x, \theta) = p(x | \theta) p(\theta)
\end{equation*}

\vspace{0.25cm}
\pause
\begin{center}
 Queremos calcular
\end{center}

$\bullet$ La posterior: 
\begin{equation*}
 p(\theta|x) = \frac{p(x | \theta) p(\theta)}{p(x)}
\end{equation*}

% $\bullet$ La evidencia: 
% \begin{equation*}
%  p(x) = \int p(x | \theta) p(\theta) d\theta
% \end{equation*}
% 
% $\bullet$ La predicci\'on: 
% \begin{equation*}
%  p(\tilde{x}|x) = \int p(\tilde{x} | \theta, x) p(\theta, x) d\theta
% \end{equation*}
% 



\end{frame}

\subsection{Aproximaciones}

\begin{frame}
\begin{textblock}{128}(0,8)
\begin{center}
 \large Aproximaciones
\end{center}
\end{textblock}


\begin{columns}[t]
\begin{column}{0.5\textwidth}
 \centering \textbf{Inferencia variacional}
 
\begin{itemize}
  \item[$\circ$] Sesgado
  \item[$\circ$] R\'apido y escalable
 \end{itemize}

 
 \end{column}
 \begin{column}{0.5\textwidth}
\centering  \textbf{Cadenas de Markov}

\begin{itemize}
  \item[$\circ$] No sesgado
  \item[$\circ$] Lento y no escalable
 \end{itemize}

\end{column}
\end{columns}


\end{frame}
 
\section{Inferencia variacional}

\begin{frame}
\begin{textblock}{128}(0,8)
\begin{center}
 \large Inferencia variacional
\end{center}
\end{textblock}

Encontrar la mejor aproximaci\'on a la posterior

\begin{equation*}
 p(\theta|x) \approx q(\theta)
\end{equation*}

\vspace{0.3cm}
Que minimice una distancia $D$
\begin{equation*}
 q(\theta) = \underset{q \in \mathcal{Q}}{\text{arg min}} \ \ D(q(\theta) \ || \ p(\theta|x))
\end{equation*}
\end{frame}
 
\begin{frame}
\begin{textblock}{128}(0,8)
\begin{center}
 \large Solcuiones
\end{center}
\end{textblock}

$\bullet$ \textbf{Variational Bayes}

\begin{equation*}
 q(\theta) = \underset{q \in \mathcal{Q}}{\text{arg min}} \ \ \text{KL}(q(\theta) \ || \ p(\theta|x))
\end{equation*}

\pause
\vspace{0.5cm}

$\bullet$ \textbf{Expecatation Propagation}

\begin{equation*}
 q(\theta) = \underset{q \in \mathcal{Q}}{\text{arg min}} \ \ \text{KL}(p(\theta|x) \ || \  q(\theta))
\end{equation*}

\end{frame}

\subsection{Kullback-Leibler}

\begin{frame}
 \begin{textblock}{128}(0,8)
\begin{center}
 \large Kullback-Leibler
\end{center}
\end{textblock}
\vspace{0.5cm}

\begin{columns}[t]
\begin{column}{0.5\textwidth}
\centering
$\text{KL}(q(\theta) \ || \ p(\theta|x))$

\begin{equation*}
  \int q(\theta) \log \frac{q(\theta)}{p(\theta|x)} d\theta
\end{equation*}

 \end{column}
 \begin{column}{0.5\textwidth}
\centering
 $\text{KL}( p(\theta|x) \ || \ q(\theta))$

 \begin{equation*}
\int p(\theta|x) \log \frac{p(\theta|x)}{q(\theta)} d\theta
\end{equation*}

 
\end{column}
\end{columns}

 \vspace{0.5cm}
\todo[inline]{IMAGEN}

\end{frame}


\section{Variationl Bayes}

\begin{frame}
 \begin{textblock}{128}(0,8)
\begin{center}
 \large Variationl Bayes
\end{center}
\end{textblock}

\begin{equation*}
 \begin{split}
 \log p(x) &= \log p(x) \int q(\theta) d\theta =  \int q(\theta) \log p(x) d\theta \\
 & = \int q(\theta) \log \frac{p(x,\theta)}{p(\theta|x)} d\theta \\
 & = \int q(\theta) \log \frac{p(x,\theta)}{p(\theta|x)} \frac{q(\theta)}{q(\theta)} d\theta \\
 & = \int q(\theta) \log \frac{p(x,\theta)}{q(\theta)} d\theta + \int q(\theta) \log \frac{q(\theta)}{p(\theta|x)} d\theta 
 \end{split}
\end{equation*}


\end{frame}

\begin{frame}

\begin{equation*}
 \log p(x)  = \underbrace{\int q(\theta) \log \frac{p(x,\theta)}{q(\theta)} d\theta}_{\mathcal{L}(q(\theta))} + \underbrace{\int q(\theta) \log \frac{q(\theta)}{p(\theta|x)} d\theta}_{\text{KL}(q(\theta) \ || \ p(\theta|x))}  
\end{equation*}


\pause

\begin{itemize}
 \item[$\bullet$] Evidence Lower Bound $\log p(x) \geq \mathcal{L}(q(\theta))$
 \item[$\bullet$] $ \underset{q \in \mathcal{Q}}{\text{arg max}} \ \mathcal{L}(q(\theta)) \Leftrightarrow\underset{q \in \mathcal{Q}}{\text{arg min}} \ \text{KL}(q(\theta) \ || \ p(\theta|x))$ 
\end{itemize}


\end{frame}

\subsection{Evidence Lower Bound}

\begin{frame}
 \begin{textblock}{128}(0,8)
\begin{center}
 \large Evidence Lower Bound
\end{center}
\end{textblock}

\begin{equation*}
 \begin{split}
 \mathcal{L}(q(\theta)) &= \int q(\theta) \log \frac{p(x,\theta)}{q(\theta)} d\theta \\
 &= \int q(\theta) \log \frac{p(x|\theta) p(\theta)}{q(\theta)} d\theta \\
 & = \int q(\theta) \log p(x|\theta) d\theta + \int q(\theta) \log  \frac{ p(\theta)}{q(\theta)} d\theta \\
 & = \int q(\theta) \log p(x|\theta) d\theta - \int q(\theta) \log  \frac{ q(\theta)}{p(\theta)} d\theta \\
 & = \underbrace{{q(\theta)}[\log p(x|\theta)]}_{\text{m\'axima verosimilitud}} - \underbrace{\text{KL}(q(\theta) \ || \ p(\theta))}_{\text{prior}}
 \end{split}
\end{equation*}
 
\end{frame}


\begin{frame}
 \begin{textblock}{128}(0,8)
\begin{center}
 \large Mean field approximation
\end{center}
\end{textblock}
\vspace{0.5cm}

Restricci\'on de independencia sobre $q$ para que se pueda factorizar,
\begin{equation*}
 q(\theta) = \prod_{i=1}^m q_i(\theta_i)
\end{equation*}

\textbf{Block coordinate assent}: En cada paso fijamos todos los factores $q_i(\theta_i)$ salvo uno, y optimizamos con respeto al restante $q_j(\theta_j)$

\begin{equation*}
 \underset{q_j \in \mathcal{Q}}{\text{arg max}} \ \mathcal{L}(q_j(\theta_j)\prod_{i\neq j}^m q_i(\theta_i))
\end{equation*}

\end{frame}

\begin{frame}
 
 
 \todo[inline]{SEGUIR ACA}
 
 \Wider[3cm]{
\begin{equation*}
 \begin{split}
 \mathcal{L}(q(\theta)) &= \int q(\theta) \log \frac{p(x,\theta)}{q(\theta)} d\theta \\
 & = \int q(\theta) \log p(x,\theta)d\theta - \int q(\theta) \log q(\theta) d\theta \\
 & = \int q_j(\theta_j) \big(\int \log p(x,\theta)\prod_{i\neq j}^m q_i(\theta_i) d\theta_i\big)d\theta_j - \int q_j(\theta_j) \log q_j(\theta_j)  d\theta_j + \text{cont} \\
  & = E_j [E_{i\neq j} \log p(x,\theta) ] - E_j [ \log q_j(\theta_j) ] + \text{cont} \\
\end{split}
\end{equation*}
 }
 
\end{frame}


\end{document}



