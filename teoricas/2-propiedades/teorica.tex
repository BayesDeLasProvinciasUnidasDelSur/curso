\documentclass[shownotes,aspectratio=169]{beamer}

\input{../../auxiliar/tex/diapo_encabezado.tex}
% tikzlibrary.code.tex
%
% Copyright 2010-2011 by Laura Dietz
% Copyright 2012 by Jaakko Luttinen
%
% This file may be distributed and/or modified
%
% 1. under the LaTeX Project Public License and/or
% 2. under the GNU General Public License.
%
% See the files LICENSE_LPPL and LICENSE_GPL for more details.

% Load other libraries
\usetikzlibrary{shapes}
\usetikzlibrary{fit}
\usetikzlibrary{chains}
\usetikzlibrary{arrows}

% Latent node
\tikzstyle{latent} = [circle,fill=white,draw=black,inner sep=1pt,
minimum size=20pt, font=\fontsize{10}{10}\selectfont, node distance=1]
% Observed node
\tikzstyle{obs} = [latent,fill=gray!25]
% Invisible node
\tikzstyle{invisible} = [latent,minimum size=0pt,color=white, opacity=0, node distance=0]
% Constant node
\tikzstyle{const} = [rectangle, inner sep=0pt, node distance=0.1]
%state
\tikzstyle{estado} = [latent,minimum size=8pt,node distance=0.4]
%action
\tikzstyle{accion} =[latent,circle,minimum size=5pt,fill=black,node distance=0.4]


% Factor node
\tikzstyle{factor} = [rectangle, fill=black,minimum size=10pt, draw=black, inner
sep=0pt, node distance=1]
% Deterministic node
\tikzstyle{det} = [latent, rectangle]

% Plate node
\tikzstyle{plate} = [draw, rectangle, rounded corners, fit=#1]
% Invisible wrapper node
\tikzstyle{wrap} = [inner sep=0pt, fit=#1]
% Gate
\tikzstyle{gate} = [draw, rectangle, dashed, fit=#1]

% Caption node
\tikzstyle{caption} = [font=\footnotesize, node distance=0] %
\tikzstyle{plate caption} = [caption, node distance=0, inner sep=0pt,
below left=5pt and 0pt of #1.south east] %
\tikzstyle{factor caption} = [caption] %
\tikzstyle{every label} += [caption] %

\tikzset{>={triangle 45}}

%\pgfdeclarelayer{b}
%\pgfdeclarelayer{f}
%\pgfsetlayers{b,main,f}

% \factoredge [options] {inputs} {factors} {outputs}
\newcommand{\factoredge}[4][]{ %
  % Connect all nodes #2 to all nodes #4 via all factors #3.
  \foreach \f in {#3} { %
    \foreach \x in {#2} { %
      \path (\x) edge[-,#1] (\f) ; %
      %\draw[-,#1] (\x) edge[-] (\f) ; %
    } ;
    \foreach \y in {#4} { %
      \path (\f) edge[->,#1] (\y) ; %
      %\draw[->,#1] (\f) -- (\y) ; %
    } ;
  } ;
}

% \edge [options] {inputs} {outputs}
\newcommand{\edge}[3][]{ %
  % Connect all nodes #2 to all nodes #3.
  \foreach \x in {#2} { %
    \foreach \y in {#3} { %
      \path (\x) edge [->,#1] (\y) ;%
      %\draw[->,#1] (\x) -- (\y) ;%
    } ;
  } ;
}

% \factor [options] {name} {caption} {inputs} {outputs}
\newcommand{\factor}[5][]{ %
  % Draw the factor node. Use alias to allow empty names.
  \node[factor, label={[name=#2-caption]#3}, name=#2, #1,
  alias=#2-alias] {} ; %
  % Connect all inputs to outputs via this factor
  \factoredge {#4} {#2-alias} {#5} ; %
}

% \plate [options] {name} {fitlist} {caption}
\newcommand{\plate}[4][]{ %
  \node[wrap=#3] (#2-wrap) {}; %
  \node[plate caption=#2-wrap] (#2-caption) {#4}; %
  \node[plate=(#2-wrap)(#2-caption), #1] (#2) {}; %
}

% \gate [options] {name} {fitlist} {inputs}
\newcommand{\gate}[4][]{ %
  \node[gate=#3, name=#2, #1, alias=#2-alias] {}; %
  \foreach \x in {#4} { %
    \draw [-*,thick] (\x) -- (#2-alias); %
  } ;%
}

% \vgate {name} {fitlist-left} {caption-left} {fitlist-right}
% {caption-right} {inputs}
\newcommand{\vgate}[6]{ %
  % Wrap the left and right parts
  \node[wrap=#2] (#1-left) {}; %
  \node[wrap=#4] (#1-right) {}; %
  % Draw the gate
  \node[gate=(#1-left)(#1-right)] (#1) {}; %
  % Add captions
  \node[caption, below left=of #1.north ] (#1-left-caption)
  {#3}; %
  \node[caption, below right=of #1.north ] (#1-right-caption)
  {#5}; %
  % Draw middle separation
  \draw [-, dashed] (#1.north) -- (#1.south); %
  % Draw inputs
  \foreach \x in {#6} { %
    \draw [-*,thick] (\x) -- (#1); %
  } ;%
}

% \hgate {name} {fitlist-top} {caption-top} {fitlist-bottom}
% {caption-bottom} {inputs}
\newcommand{\hgate}[6]{ %
  % Wrap the left and right parts
  \node[wrap=#2] (#1-top) {}; %
  \node[wrap=#4] (#1-bottom) {}; %
  % Draw the gate
  \node[gate=(#1-top)(#1-bottom)] (#1) {}; %
  % Add captions
  \node[caption, above right=of #1.west ] (#1-top-caption)
  {#3}; %
  \node[caption, below right=of #1.west ] (#1-bottom-caption)
  {#5}; %
  % Draw middle separation
  \draw [-, dashed] (#1.west) -- (#1.east); %
  % Draw inputs
  \foreach \x in {#6} { %
    \draw [-*,thick] (\x) -- (#1); %
  } ;%
}


 \mode<presentation>
 {
 %   \usetheme{Madrid}      % or try Darmstadt, Madrid, Warsaw, ...
 %   \usecolortheme{default} % or try albatross, beaver, crane, ...
 %   \usefonttheme{serif}  % or try serif, structurebold, ...
  \usetheme{Antibes}
  \setbeamertemplate{navigation symbols}{}
 }
 
\estrue




%\title[Bayes del Sur]{}

\begin{document}

\color{black!85}
\large

 
%\setbeamercolor{background canvas}{bg=gray!15}


\begin{frame}[plain,noframenumbering]


\begin{textblock}{160}(0,0)
\includegraphics[width=1\textwidth]{../../auxiliar/static/deforestacion}
\end{textblock}

\begin{textblock}{80}(18,9)
\textcolor{black!15}{\fontsize{44}{55}\selectfont Verdades}
\end{textblock}

\begin{textblock}{47}(85,70)
\centering \textcolor{black!15}{{\fontsize{52}{65}\selectfont Empíricas}}
\end{textblock}

\begin{textblock}{80}(100,28)
\LARGE  \textcolor{black!15}{\rotatebox[origin=tr]{-3}{\scalebox{9}{\scalebox{1}[-1]{$p$}}}}
\end{textblock}

\begin{textblock}{80}(66,43)
\LARGE  \textcolor{black!15}{\scalebox{6}{$=$}}
\end{textblock}

\begin{textblock}{80}(36,29)
\LARGE  \textcolor{black!15}{\scalebox{9}{$p$}}
\end{textblock}

\vspace{2cm}
\maketitle


%
% \begin{textblock}{160}(01,81)
% \footnotesize \textcolor{black!5}{Congreso Bayesiano Plurinacional 2023} \\}
% \end{textblock}

\end{frame}



\begin{frame}[plain,noframenumbering]

% \begin{textblock}{160}(0,0)
% \includegraphics[width=1.18\textwidth]{../../aux/static/fotosintesis}
% \end{textblock}
\begin{textblock}{160}(0,-15)
\includegraphics[width=1\textwidth]{../../auxiliar/static/tsimane}
\end{textblock}


% VERSION 2
\begin{textblock}{160}(6,36)
\LARGE \rotatebox[origin=tr]{18}{\textcolor{black!95}{\fontsize{22}{0}\selectfont \textbf{La función}}}
\end{textblock}
\begin{textblock}{160}(41,32)
\LARGE \rotatebox[origin=tr]{23}{\textcolor{black!95}{\fontsize{22}{0}\selectfont \textbf{de}}}
\end{textblock}
\begin{textblock}{160}(50.5,23)
\LARGE \rotatebox[origin=tr]{28}{\textcolor{black!95}{\fontsize{22}{0}\selectfont \textbf{costo}}}
\end{textblock}
\begin{textblock}{160}(68,5.3)
\LARGE \rotatebox[origin=tr]{26}{\textcolor{black!95}{\fontsize{22}{0}\selectfont \textbf{epistémico}}}
\end{textblock}
\begin{textblock}{160}(104,5.5)
\LARGE \rotatebox[origin=tr]{8}{\textcolor{black!95}{\fontsize{22}{0}\selectfont \textbf{-}}}
\end{textblock}
\begin{textblock}{160}(110,3)
\LARGE \rotatebox[origin=tr]{-14}{\textcolor{black!95}{\fontsize{22}{0}\selectfont \textbf{evolutiva}}}
\end{textblock}


\begin{textblock}{55}[0,0](120,22)
\begin{turn}{-57}
\parbox{7cm}{\sloppy\setlength\parfillskip{0pt}
\textcolor{black!0}{\ \ \ \ \ Capítulo 2} \\
\small\textcolor{black!5}{\hspace{-0.15cm} Ventajas a favor de la:} \\
\small\textcolor{black!5}{\hspace{-1.45cm} Diversificación (propiedad epistémica)}\\
\small\textcolor{black!5}{\hspace{-1.7cm} Cooperación (propiedad evolutiva mayor)}\\
\small\textcolor{black!5}{ \hspace{-1.75cm}Especialización (propiedad meta-epistémica)} \\
\small\textcolor{black!5}{\hspace{-2cm} Coexistencia (propiedad ecológica).\\ }}
\end{turn}
\end{textblock}


\end{frame}



\begin{frame}[plain]
\begin{textblock}{160}(0,4)
 \centering \LARGE Vida \\
 \Large Transiciones evolutivas mayores
\end{textblock}
\vspace{1.7cm} \centering




\scalebox{1.5}{
\tikz{            
    \node[accion] (i1) {} ;
    \node[accion, yshift=0.6cm, xshift=0.4cm] (i2) {} ;
    \node[accion, yshift=0.6cm, xshift=-0.4cm] (i3) {} ;
    \node[const, yshift=0.3cm, xshift=0.4cm] (i) {};
    
    \node[const, yshift=-0.8cm] (ni) {$\hfrac{\text{Individuos}}{\text{solitarios}}$};
    
    \node[const, yshift=1.2cm, xshift=1.5cm] (m1) {$\hfrac{\text{Formación}}{\text{de grupos}}$};
    
    \node[const, right=of i, xshift=2cm] (c) {};
    \node[accion, below=of c, yshift=0.35cm, xshift=0.4cm] (c1) {} ;
    \node[accion, above=of c, yshift=-0.35cm, xshift=0.6cm] (c2) {} ;
    \node[accion, above=of c, yshift=-0.35cm, xshift=0.2cm] (c3) {} ;
    \node[const, right=of c, xshift=0.6cm] (cc) {};
    
    \node[const, right=of ni, xshift=1.3cm] (nc) {$\hfrac{\text{Grupos}}{\text{cooperativos}}$};

    \node[const, right=of m1, xshift=1.2cm] (m2) {$\hfrac{\text{Transición}}{\text{mayor}}$};
    
    \node[const, right=of cc, xshift=2cm] (t) {};
    \node[accion, below=of t, yshift=0.35cm, xshift=0.4cm] (t1) {} ;
    \node[accion, above=of t, yshift=-0.35cm, xshift=0.6cm] (t2) {} ;
    \node[accion, above=of t, yshift=-0.35cm, xshift=0.2cm] (t3) {} ;
    
    \node[const, right=of nc, xshift=1.1cm] (nt) {$\hfrac{\text{Unidad de}}{\text{nivel superior}}$};

    \edge {i} {c};
    \edge {cc} {t};
    
    \plate {transition} {(t1)(t2)(t3)} {}; %
    }
}


\end{frame}


\begin{frame}[plain]
\begin{textblock}{160}(0,4)
 \centering \LARGE Cooperación \\
 \Large Células que viven en células
\end{textblock}
\vspace{1.3cm} \centering

\includegraphics[width=1\textwidth]{../../auxiliar/static/cloroplastos}

\end{frame}

\begin{frame}[plain]
\begin{textblock}{160}(0,4)
 \centering \LARGE Cooperación \\
 \Large Organismos multicelulares
\end{textblock}
\vspace{1.3cm} \centering

\includegraphics[width=1\textwidth]{../../auxiliar/static/fotosintesis}

\end{frame}

\begin{frame}[plain]
\begin{textblock}{160}(0,4)
 \centering \LARGE Cooperación \\
 \Large Sistemas sociales
\end{textblock}
\vspace{1.3cm} \centering

\includegraphics[width=0.75\textwidth]{../../auxiliar/static/hormigas}

\end{frame}

\begin{frame}[plain]
\begin{textblock}{160}(0,4)
 \centering \LARGE Cooperación \\
 \Large Comunidad ecológica (biocenosis)
\end{textblock}
\vspace{1.3cm} \centering

\includegraphics[width=0.70\textwidth]{../../auxiliar/static/tsimane}

\end{frame}


\begin{frame}[plain]
\begin{textblock}{160}(0,4)
 \centering \LARGE Complejidad actual de la vida \\
 \Large Distribución de biomasa 
\end{textblock}
\vspace{2cm} \centering
\includegraphics[width=1\textwidth]{../../auxiliar/static/biomass.jpg}

\vspace{0.3cm}

\footnotesize Bar-On et al 2018
\end{frame}

\begin{frame}[plain]
\begin{textblock}{160}(0,4)
 \centering \LARGE Complejidad actual de la vida
\end{textblock}
\vspace{1.5cm} \centering \Large 

¿Cómo se explica esta tendencia a favor de la \\ diversificación, cooperación, especialización y coexistencia?

\end{frame}


\begin{frame}[plain]
\begin{textblock}{160}(0,4)
 \centering \LARGE
Crecimiento de los linajes
\end{textblock}
\vspace{1cm}

\begin{equation*} 
\omega(T) = \prod_t^T f(\Aa(t)) \onslide<2>{\approx r^T }
\end{equation*}

\vspace{0.3cm}

\begin{equation*}
f(\Aa) =
\begin{cases}
 1.5 & \Aa = \text{ Cara } \\
 0.6 & \Aa = \text{ Seca }
\end{cases}
\end{equation*}

\pause \centering \vspace{1cm}

\onslide<2>{
¿Cuál es la tasa de crecimiento $r$?
}

\end{frame}

\begin{frame}[plain]
\begin{textblock}{160}(0,4)
 \centering \LARGE
Poblaciones de tamaño infinito
\end{textblock}
\vspace{1cm}

\only<1>{
\begin{textblock}{160}(0,22)
\begin{equation*}
\langle \omega(t) \rangle = \sum_{\omega(t)} \omega(t) \cdot  P(\omega(t))
\end{equation*}
\end{textblock}
}

\only<2->{
\begin{textblock}{160}(0,12)
\begin{equation*}
\begin{split}
\langle \omega(1) \rangle & = 1.5 \cdot \frac{1}{2} + 0.6 \cdot  \frac{1}{2} = 1.05 \\ 
\onslide<3->{\langle \omega(2) \rangle &=  1.5^2 \cdot \frac{1}{4} + 2 (0.6 \cdot 1.5 \cdot \frac{1}{4} ) + 0.6^2 \cdot \frac{1}{4}= 1.05^2 }
\end{split}
\end{equation*}
\end{textblock}
}

\only<4>{
\begin{textblock}{140}(10,36)
\begin{figure}[H]
    \centering
    \begin{subfigure}[b]{0.5\linewidth}
    \includegraphics[width=\linewidth]{figures/pdf/ergodicity_expectedValue.pdf}
    \end{subfigure}
\end{figure}
\end{textblock}
}

\end{frame}


\begin{frame}[plain]
\begin{textblock}{160}(0,4)
 \centering \LARGE
Trayectorias individuales en el tiempo
\end{textblock}
\vspace{1cm}

\begin{textblock}{140}(10,10)
\begin{figure}[H]
    \centering
    \begin{subfigure}[b]{0.49\linewidth}
    \includegraphics[width=\linewidth]{figures/pdf/ergodicity_individual_trayectories_y.pdf}
    \end{subfigure}
\end{figure}
\end{textblock}


\only<2->{
\begin{textblock}{140}(10,58)
\begin{equation*} 
\begin{split}
\omega(T) & = \prod^T_{t=1} f(\Aa(t)) = f(\text{\en{head}\es{cara}})^{n_1} f(\text{\en{tail}\es{sello}})^{n_2}  \approx r^T \\
\onslide<3->{\lim_{T \rightarrow \infty} \omega(T)^{1/T} & =  r}  \onslide<4->{= 1.5^{1/2} \cdot 0.6^{1/2}} \onslide<5>{ \approx 0.95 } 
\end{split}
\end{equation*}
\end{textblock}
}

\end{frame}


\begin{frame}[plain]
\begin{textblock}{160}(0,4)
 \centering \LARGE
Cooperación
\end{textblock}
\vspace{1cm}


\begin{figure}[H]
\centering
\scalebox{0.75}{
\tikz{

    \node[latent, minimum size=2cm ] (x1_0) {$\omega_1(t)$} ;
    \node[latent, below=of x1_0, minimum size=2cm ] (x2_0) {$\omega_2(t)$} ;

    \node[latent, right=of x1_0, minimum size=3cm ] (x1_0g) {$ \omega_1(t)\cdot f(\text{Cara})$} ;
    \node[latent, right=of x2_0, minimum size=1.8cm, xshift=0.6cm , align=left] (x2_0g) {$\omega_2(t)\cdot$\\$f(\text{Seca})$} ;
    
    \node[latent, right=of x1_0g, minimum size=3.8cm, yshift=-1.33cm, align=right] (x_0) {$\omega_1(t)\cdot f(\text{Cara})$\\$+\omega_2(t)\cdot f(\text{Seca})$ } ;
    
    \node[const, above=of x_0] (nx_0) {$\overbrace{\text{Pool}\hspace{2.5cm}\text{Share}}^{\text{\normalsize Coopera\en{tion}\es{ci\'on}}}$} ;
    
    \node[latent, right=of x1_0g, minimum size=2.5cm,  xshift=4.5cm] (x1_1) {$\omega_1(t+1)$ } ;
    \node[latent, below=of x1_1, minimum size=2.5cm, yshift=0.7cm] (x2_1) {$\omega_2(t+1)$ } ;
    
    \edge {x1_0} {x1_0g};
    \edge {x2_0} {x2_0g};
    \edge {x1_0g,x2_0g} {x_0};
    \edge {x_0} {x1_1,x2_1};
    
}
}
\end{figure}
\end{frame}

\begin{frame}[plain]
\begin{textblock}{160}(0,4)
 \centering \LARGE
 Cooperación
\end{textblock}
\vspace{1.3cm}

\centering

\begin{textblock}{140}(05,14
 )
\only<1-3>{\includegraphics[width=0.5\linewidth]{figures/pdf/ergodicity_desertion0.pdf}}\only<4>{\includegraphics[width=0.5\linewidth]{figures/pdf/ergodicity_desertion1.pdf}}\only<5->{\includegraphics[width=0.5\linewidth]{figures/pdf/ergodicity_desertion.pdf}}
\end{textblock}


\only<2->{
\begin{textblock}{160}(0,57)
\begin{equation*} 
\begin{split}
\omega(t+1) & \ \, = \frac{1}{N} \big(\overbrace{\omega(t) \, f(\text{Cara}) \, n_c + \omega(t) \, f(\text{Seca}) \, n_s }^{\text{Fondo común}}  \big) \\
& \onslide<3->{\overset{\hfrac{\lim }{N\rightarrow \infty}}{=}  \omega(t) \, (\underbrace{f(\text{Cara}) \, p  + f(\text{Seca}) \, (1-p) }_{\text{Tasa de crecimiento}})   \\}
\end{split}
\end{equation*}
\end{textblock}
}

\end{frame}

\begin{frame}[plain]
\begin{textblock}{160}(0,4)
 \centering \LARGE
 Cooperación
\end{textblock}
\vspace{1.3cm} \centering

 \begin{tabular}{|l|c|c|c|c|c|}
     \hline
         & {\small \, $\omega(0)$ } & {\small \  $f(\cdot) \ $}  & {\small \, $\omega(1)$ } & {\small \  $f(\cdot) \ $}  & {\small \,  $\omega(2)$ }  \\ \hline \hline
        A no-coop& $1$ & $1.5$ &  $1.5$ & $0.6$ & $\bm{0.9}$ \\ \hline
        B no-coop & $1$ & $0.6$ & $0.6$ & $1.5$ & $\bm{0.9}$ \\ \hline\hline
        A coop & $1$ & $1.5$ & $1.05$ & $0.6$ & $\bm{1.1}$ \\ \hline
        B coop & $1$ & $0.6$ & $1.05$ & $1.5$ & $\bm{1.1}$\\ \hline
    \end{tabular}
    
    \pause
    
    \vspace{1cm}
    
    \Large
    
    La reducción de fluctuaciones produce un \\  aumento en las tasas de crecimiento

\end{frame}


\begin{frame}[plain]

\begin{textblock}{160}(0,-16)
\includegraphics[width=1\textwidth]{../../auxiliar/static/madre-chimpance.jpg}
\end{textblock}

\begin{textblock}{80}(80,4)
 \centering \LARGE Crianza cooperativa \\
 \Large Coevolución genético-cultural
\end{textblock}
\vspace{1cm}

\end{frame}


\begin{frame}[plain]

\begin{textblock}{170}(-6,9)
\centering
\includegraphics[width=0.95\textwidth]{figures/agricultura.pdf} \ \ \ \ \
\end{textblock}

\begin{textblock}{160}(0,4)
 \centering \LARGE La transición cultural
\end{textblock}
\vspace{0.3cm}


\end{frame}

\begin{frame}[plain]
\begin{textblock}{191}(-13,0)
 \centering
 \includegraphics[width=1\textwidth]{../../auxiliar/static/terrazas_arroz_c}
\end{textblock}

 \begin{textblock}{160}(0,4)
  \LARGE \centering \textcolor{black!5}{Tecnologías de reciprocidad ecológica}\\ 
 \end{textblock} 

 \begin{textblock}{70}(88,60)
  \Large \textcolor{black!5}{Domesticación}\\
 \end{textblock} 


\end{frame}
% % 
% \begin{frame}[plain]
%  \begin{textblock}{160}(0,4)
%   \LARGE \centering \textcolor{black!85}{China}
%  \end{textblock} 
% 
% 
% 
% \begin{textblock}{80}(0,12)
%   \Large \centering \textcolor{black!85}{}
% \end{textblock} 
% \begin{textblock}{160}(0,60)
%   \centering
% \includegraphics[width=1\textwidth]{../../auxiliar/static/chineseRiverShips.jpg}
%   \end{textblock} 
% 
% \begin{textblock}{70}(10,10) \footnotesize
%  $\bullet$ Seda ($\sim -1500$) \\
%  $\bullet$ Puentes flotantes ($\sim -1100$) \\
%  $\bullet$ Altos hornos de fundición ($\sim -750/-450$) \\
%  $\bullet$ Molino de agua ($\sim -500$) \\
%  $\bullet$ Canales artificiales navegación ($\sim -500$) \\
%  $\bullet$ Arado de hierro ($\sim -500$) \\
%  $\bullet$ Cámara de fotos ($\sim -450$) \\
%  $\bullet$ Helicóptero de juegete ($\sim -400$) \\
%  $\bullet$ Tinta ($\sim -250$) \\
%  $\bullet$ Porcelena ($\sim -200$) \\
%  %$\bullet$ Higrómetros ($\sim -200$) \\
%  $\bullet$ Burocracia por concurso ($\sim 0$) \\
%  %$\bullet$ Sismógrafo ($\sim 100$ ) \\
%  $\bullet$ Refinamiento de petróleo ($\sim 100$ ) \\
%  $\bullet$ Brújula ($\sim 100$ ) 
%  \end{textblock} 
% 
% 
%  
%  \begin{textblock}{70}(90,10) \footnotesize
%  $\bullet$ Fútbol ($\sim 200$) \\
%  $\bullet$ Control biológico de pestes ($\sim 300$) \\
%  $\bullet$ Pózos de petróleo ($\sim 350$ ) \\
%  $\bullet$ Fósforos ($\sim 550$) \\
%  $\bullet$ Papel higiénico ($\sim 600$) \\
%  $\bullet$ Imprenta ($\sim 650$ ) \\
%  $\bullet$ Amalmaga dental ($\sim 650$) \\
%  $\bullet$ Papel moneda ($\sim 700$ ) \\
%  $\bullet$ Relojería de escape ($\sim 700$) \\
%  $\bullet$ Espejos ($\sim 800$) \\
%  $\bullet$ Vacunas ($\sim 950$) \\
%  $\bullet$ Pólvora ($\sim 1000$) \\
%  $\bullet$ Cepillo de dientes ($\sim 1450$)
%  \end{textblock} 
%  
%   
% \end{frame}


\begin{frame}[plain]
\begin{textblock}{95}(0,22) \centering
\includegraphics[width=0.95\textwidth]{../../auxiliar/static/polynesia.png}
\end{textblock} 

\begin{textblock}{60}(95,08.5) \centering
\includegraphics[width=0.95\textwidth]{../../auxiliar/static/tonga_barco.jpg}
\end{textblock} 

\begin{textblock}{160}(0,4)
\centering \LARGE \textcolor{black!85}{Agricultura $\mapsto$ Población $\mapsto$ Centros de innovación }
\end{textblock}

\end{frame}



\begin{frame}[plain]
\begin{textblock}{160}(0,4)
  \LARGE \centering \textcolor{black!85}{Tecnologías de reciprocidad social} \\
  \Large La obligación universal de dar y recibir
 \end{textblock} % 
\vspace{1.1cm}

\centering
 \includegraphics[width=0.593\textwidth]{../../auxiliar/static/bali-offerings.jpg}
 \includegraphics[width=0.397\textwidth]{../../auxiliar/static/pachamama.jpg}
 
\end{frame}


\begin{frame}[plain]
\begin{textblock}{160}(0,4)
\LARGE \centering \textcolor{black!85}{Principio de reciprocidad} \\
\Large El problema que da inicio a la teoría de la probabilidad
\end{textblock} 
\vspace{2cm} \centering

Pascal-Fermat (1654)

\vspace{0.3cm}

Tiramos dos veces la moneda: \\ \justify 
$\bullet$ Rojas hace un favor cuando sale Seca en la primera y en la segunda \\
$\bullet$ Negras hace un favor en caso contrario. \\[0.2cm]

\centering

\tikz{
\node[latent, draw=white, yshift=0.7cm, minimum size=0.1cm] (b0) {};
\node[latent,below=of b0,yshift=0.7cm, xshift=-1cm] (r1) {$S$};
\node[latent,below=of b0,yshift=0.7cm, xshift=1cm] (r2) {$C$};

\node[latent, below=of r1, draw=white, yshift=0.8cm, minimum size=0.1cm] (bc11) {};
\node[accion, below=of r2, yshift=0cm] (bc12) {};
\node[latent,below=of bc11,yshift=0.8cm, xshift=-0.5cm] (r1d2) {$S$};
\node[latent,below=of bc11,yshift=0.8cm, xshift=0.5cm] (r1d3) {$C$};

\node[accion,below=of r1d2,yshift=0cm, color=red] (br1d2) {};
\node[accion,below=of r1d3,yshift=0cm] (br1d3) {};
\edge[-] {b0} {r1,r2};
\edge[-] {r1} {bc11};
\edge[-] {r2} {bc12};
\edge[-] {bc11} {r1d2,r1d3};
\edge[-] {r1d2} {br1d2};
\edge[-] {r1d3} {br1d3};
}



\vspace{0.3cm}

\onslide<2>{
\Wider[2cm]{
\centering
\Large ¿Cuál es el valor justo de la reciprocidad en contexto de incertidumbre?
}
}

\end{frame}

\begin{frame}[plain]
\begin{textblock}{160}(0,4)
 \centering \LARGE Apuestas
\end{textblock}
\vspace{1cm}

Las casas de apuestas ofrecen pagos $Q_c$ y $Q_s$ por Cara y Seca

\vspace{1cm} \pause

$\bullet$ ¿Cuál es la apuesta óptima, $b_c$ y $b_s$? \\ \pause
$\bullet$ ¿Cuáles pagos $Q_c$ y $Q_s$ convienen?

\end{frame}



\begin{frame}[plain]
\begin{textblock}{160}(0,4)
 \centering \LARGE Apuesta individual \\
 \Large $b = b_c = 1 - b_s$
\end{textblock}
\vspace{1.5cm} 

\begin{equation*}
\omega(T) = (\underbrace{(\omega(0) \, \overbrace{b \,  Q_c}^{\text{Cara}})}_{\omega(1)} \, \overbrace{(1-b) \, Q_s}^{\text{Seca}}) \dots \onslide<2->{= \omega(0) \,  (b \,  Q_c)^{n_c}  \,  ((1-b) \, Q_s)^{n_s}}
\end{equation*}
\onslide<3->{donde $n_c + n_s = T$ son las veces que salió Cara y Seca en $T$ intentos.}

\onslide<4->{
\begin{equation*}
\begin{split}
 \omega(T) = \omega(0) \,  r^T &= \omega(0) \,  (b \,  Q_c)^{n_c}  \,  ((1-b) \, Q_s)^{n_s} \\
\onslide<5->{r &= \underbrace{(b \,  Q_c)^{n_c/T}  \,  ((1-b) \, Q_s)^{n_s/T}}_{\text{Tasa de crecimiento}} }
\end{split}
\end{equation*}
}

\onslide<6->{Con frecuencia típica $p = \lim_{T \rightarrow \infty} n_c/T$}
\end{frame}


\begin{frame}[plain]
\begin{textblock}{160}(0,4)
 \centering \LARGE Apuesta individual  \\
 %\only<5->{\Large Maximizar $r$ }
 \end{textblock}
\vspace{1.2cm}  \centering

\only<1>{
\begin{textblock}{160}(0,16)
 \begin{equation*}
 r = (b \,  Q_c)^{p}  \,  ((1-b) \, Q_s)^{1-p} 
\end {equation*}
\end{textblock}
}

\only<2>{
\begin{textblock}{160}(0,16)
 \begin{equation*}
\frac{r_b}{r_d} = \frac{(b \  Q_c)^{p}  \,  ((1-b) \, Q_s \, )^{1-p} }{(d \  Q_c)^{p}  \,  ((1-d) \, Q_s \, )^{1-p}  }
\end {equation*}
\end{textblock}
}

\only<3-4>{
\begin{textblock}{160}(0,16)
 \begin{equation*}
\frac{r_b}{r_d} = \frac{(b \, \cancel{Q_c})^{p}  \,  ((1-b) \, \bcancel{Q_s})^{1-p} }{(d \, \cancel{Q_c})^{p}  \,  ((1-d) \, \bcancel{Q_s})^{1-p}  }
\end {equation*}
\end{textblock}
}

\only<5->{
\begin{textblock}{160}(0,16)
 \begin{equation*}
 \begin{split}
 r \propto \ &b^{\,p}  \,  (1-b)^{1-p} \\
 \onslide<7->{\underset{b}{\text{arg max}} \ \  &b^{\,p}  \,  (1-b)^{1-p} = p}
 \end{split}
\end {equation*}
\end{textblock}
}


\only<4-5>{
\begin{textblock}{160}(0,46) \centering
\Large La apuesta no depende de los pagos que ofrece la casa! \\
\large (No cualquier función de costo ad-hoc tiene esta propiedad)

 \end{textblock}
}


\only<6>{
\begin{textblock}{160}(0,26) \centering
\includegraphics[width=0.6\textwidth, page=6]{figures/tasa-temporal2.pdf} 
\end{textblock}
}

\only<7->{
\begin{textblock}{160}(0,46) \centering
\Large Las apuestas multiplicativas garantizan la inferencia! \\
\large (No cualquier función de costo ad-hoc tiene este propiedad)
\\[0.4cm]

\Large

\onslide<8->{La teoría de la probabilidad es un juego de apuestas multiplicativo! \\
\large (Máxima verosimilitud no lo es, salvo en tiempo infinito)
}
\end{textblock}
}


\only<9>{
\begin{textblock}{140}(10,73) \centering
\normalsize
\begin{equation*}
P(\text{Hipótesis} = h|\text{Datos} = \{d_1, \dots, d_n\}, M) \propto P(h|M) \ P(d_1|h,M) \, P(d_2|h, d_1,M) \dots
\end{equation*}
\end{textblock}
}


\only<10>{
\begin{textblock}{140}(10,68.75) \centering
\normalsize
\begin{equation*}
P(\text{Hipótesis} = h|\text{Datos} = \{d_1, \dots, d_n\}, M) \propto \overbrace{P(h|M)}^{\text{prior}} \, \underbrace{P(d_1|h,M) \, P(d_2|h, d_1,M)}_{}  \dots
\end{equation*}
\end{textblock}
}


\only<10>{
\begin{textblock}{80}(80,85.5)
\scriptsize Predicciones a piori, \textbf{las apuestas de las hipótesis}
\end{textblock}
}



% 
% \vspace{0.3cm} 
% \onslide<4->{Dados los pagos recíprocos $Q_c = 1/p$ y $Q_s = 1 / (1-p)$}
% \begin{equation*}
% \begin{split}
% \onslide<5->{r & =  (b / p)^{p}  \,  ((1-b) / (1-p))^{1-p}  \onslide<6->{\overset{b=p}{=} 1} }
% \end{split}
% \end{equation*}

\end{frame}



%
% \begin{frame}[plain]
% \begin{textblock}{160}(0,4)
% \LARGE \centering Teoría de la probabilidad \\
% \Large Breve resumen
% \end{textblock}
% \vspace{1cm} \centering \Large
%
%
%  \only<1>{
%   \begin{textblock}{80}(-10,20) \centering
%  \scalebox{0.8}{
% \tikz{ %
%          \node[factor, minimum size=1cm] (p1) {} ;
%          \node[factor, minimum size=1cm, xshift=1.5cm] (p2) {} ;
%          \node[factor, minimum size=1cm, xshift=3cm] (p3) {} ;
%
%
%          \node[const, above=of p1, yshift=0.1cm] (np1) {\Large $?$};
%          \node[const, above=of p2, yshift=0.1cm] (np2) {\Large $?$};
%          \node[const, above=of p3, yshift=0.1cm] (np3) {\Large $?$};
%          }
% }
% \end{textblock}
% }
%
% \only<2>{
%   \begin{textblock}{80}(-10,20) \centering
%  \scalebox{0.8}{
% \tikz{ %
%          \node[factor, minimum size=1cm] (p1) {} ;
%          \node[factor, minimum size=1cm, xshift=1.5cm] (p2) {} ;
%          \node[factor, minimum size=1cm, xshift=3cm] (p3) {} ;
%
%
%          \node[const, above=of p1, yshift=0.12cm] (np1) {\Large $0$};
%          \node[const, above=of p2, yshift=0.12cm] (np2) {\Large $1$};
%          \node[const, above=of p3, yshift=0.12cm] (np3) {\Large $0$};
%          }
% }
% \end{textblock}
% %
% }
%
% \only<2->{
% \begin{textblock}{100}(55,24) \centering
% \large Principio de integridad (Regla de la suma) \\
% \normalsize
% Las distribuciónes de creencia posibles son la que suman 1
% \end{textblock}
% }
%
% \only<3>{
%   \begin{textblock}{80}(-10,20) \centering
%  \scalebox{0.8}{
% \tikz{ %
%          \node[factor, minimum size=1cm] (p1) {} ;
%          \node[factor, minimum size=1cm, xshift=1.5cm] (p2) {} ;
%          \node[factor, minimum size=1cm, xshift=3cm] (p3) {} ;
%
%
%          \node[const, above=of p1, yshift=-0.05cm] (np1) {\Large $1/5$};
%          \node[const, above=of p2, yshift=-0.05cm] (np2) {\Large $3/5$};
%          \node[const, above=of p3, yshift=-0.05cm] (np3) {\Large $1/5$};
%          }
% }
% \end{textblock}
% %
% }
%
% \only<4>{
%   \begin{textblock}{80}(-10,20) \centering
%  \scalebox{0.8}{
% \tikz{ %
%          \node[factor, minimum size=1cm] (p1) {} ;
%          \node[factor, minimum size=1cm, xshift=1.5cm] (p2) {} ;
%          \node[factor, minimum size=1cm, xshift=3cm] (p3) {} ;
%
%
%          \node[const, above=of p1, yshift=-0.05cm] (np1) {\Large $1/3$};
%          \node[const, above=of p2, yshift=-0.05cm] (np2) {\Large $1/3$};
%          \node[const, above=of p3, yshift=-0.05cm] (np3) {\Large $1/3$};
%          }
% }
% \end{textblock}
% %
% }
%
% \only<4->{
% \begin{textblock}{100}(55,42) \centering
% \large Principio de indiferencia (Máxima incertidumbre) \\
% \normalsize
% Primer acuerdo intersubjetivo en contextos de incertidumbre
% \end{textblock}
% }
%
%
% \only<5-7>{
%  \begin{textblock}{80}(-10,20) \centering
%  \scalebox{0.8}{
% \tikz{ %
%          \node[factor, minimum size=1cm] (p1) {\includegraphics[width=0.05\textwidth]{../../auxiliar/static/cerradura.png}} ;
%          \node[det, minimum size=1cm, xshift=1.5cm] (p2) {\includegraphics[width=0.06\textwidth]{../../auxiliar/static/dedo.png}} ;
%          \node[factor, minimum size=1cm, xshift=3cm] (p3) {} ;
%
%
%          \node[const, above=of p1, yshift=0.1cm] (np1) {\Large $?$};
%          \node[const, above=of p2, yshift=0.1cm] (np2) {\Large $0$};
%          \node[const, above=of p3, yshift=0.1cm] (np3) {\Large $?$};
%          }
% }
% \end{textblock}
% }
%
%
% \only<5->{
% \begin{textblock}{80}(-10,35) \centering
% Datos \\[0.2cm]
%
% \only<
% 7->{
%  Modelo causal \\[0.3cm]
% \scalebox{0.6}{
% \tikz{
%
%     \node[latent] (d) {\includegraphics[width=0.10\textwidth]{../../auxiliar/static/dedo.png}} ;
%     \node[const,left=of d] (nd) {\Large $s$} ;
%
%     \node[latent, above=of d, xshift=-1.5cm] (r) {\includegraphics[width=0.12\textwidth]{../../auxiliar/static/regalo.png}} ;
%     \node[const,left=of r] (nr) {\Large $r$} ;
%
%
%     \node[latent, fill=black!30, above=of d, xshift=1.5cm] (c) {\includegraphics[width=0.12\textwidth]{../../auxiliar/static/cerradura.png}} ;
%     \node[const,left=of c] (nc) {\Large $c$} ;
%
%     \edge {r,c} {d};
% }
% }
% }
% \end{textblock}
% }
%
%
% \only<5>{
% \begin{textblock}{160}(0,64) \centering \large
% ¿Cómo podemos dar continuidad a los acuerdos intersubjetivos?
% \end{textblock}
% }
%
% \only<6->{
% \begin{textblock}{110}(50,60) \centering
% \large Principio de coherencia (Regla del producto)\\
% \normalsize
% Creencia previa que sigue siendo compatible con datos y modelo \\[-0.3cm]
% \only<8>{
% \begin{equation*}
% P(r_i|s_2, M) \propto \underbrace{P(s_2|r_i,M)}_{\text{predicción}} \, \underbrace{P(r_i|M)}_{\text{prior}}
% \end{equation*}
% }
% \end{textblock}
% }
%
% \only<9>{
% \begin{textblock}{140}(10,73) \centering
% \normalsize
% \begin{equation*}
% P(\text{Hipótesis} = h|\text{Datos} = \{d_1, \dots, d_n\}, M) \propto P(h|M) \ P(d_1|h,M) \, P(d_2|h, d_1,M) \dots
% \end{equation*}
% \end{textblock}
% }
%
%
% \only<10>{
% \begin{textblock}{140}(10,68.75) \centering
% \normalsize
% \begin{equation*}
% P(\text{Hipótesis} = h|\text{Datos} = \{d_1, \dots, d_n\}, M) \propto \overbrace{P(h|M)}^{\text{prior}} \, \underbrace{P(d_1|h,M) \, P(d_2|h, d_1,M)}_{}  \dots
% \end{equation*}
% \end{textblock}
% }
%
%
% \only<10>{
% \begin{textblock}{80}(80,85.5)
% \scriptsize Predicciones a piori, \textbf{las apuestas de las hipótesis}
% \end{textblock}
% }
%
%
%
% \only<8->{
%  \begin{textblock}{80}(-10,20) \centering
%  \scalebox{0.8}{
% \tikz{ %
%          \node[factor, minimum size=1cm] (p1) {\includegraphics[width=0.05\textwidth]{../../auxiliar/static/cerradura.png}} ;
%          \node[det, minimum size=1cm, xshift=1.5cm] (p2) {\includegraphics[width=0.06\textwidth]{../../auxiliar/static/dedo.png}} ;
%          \node[factor, minimum size=1cm, xshift=3cm] (p3) {} ;
%
%
%          \node[const, above=of p1, yshift=-0.05cm] (np1) {\Large $1/3$};
%          \node[const, above=of p2, yshift=0.1cm] (np2) {\Large $0$};
%          \node[const, above=of p3, yshift=-0.05cm] (np3) {\Large $2/3$};
%          }
% }
% \end{textblock}
% }
%
%
% \end{frame}


\begin{frame}[plain]
\begin{textblock}{160}(0,4)
 \centering \LARGE Apuesta cooperativa
\end{textblock}
\vspace{1.25cm} 

\begin{equation*}
\onslide<2->{\text{fondo común}_t = (\omega_t \, b \, Q_c ) \, n_c + (\omega_t \, (1-b) \, Q_s ) \, n_s }
\end{equation*} \\[-0.3cm]
\onslide<3->{donde $n_c + n_s = N$ son las veces que salió Cara y Seca en el grupo de tamaño $N$}
\begin{equation*}
\begin{split}
\onslide<4->{\omega_{t+1} & =  \frac{1}{N} \, \text{fondo común}_t \\ }
\onslide<5->{& = \omega_t  \left( b \, Q_c  \, \frac{n_c}{N} +  (1-b) \, Q_s \, \frac{n_s}{N} \right) } 
\end{split}
\end{equation*} \\[0.3cm]

\onslide<6->{Cuando el grupo es de tamaño infinito, la cuota siempre tiene el mismo valor}
\begin{equation*}
\begin{split}
\onslide<6->{\lim_{N \rightarrow \infty} \omega_{t+1} & = \omega_t  \underbrace{\left( b \, Q_c  \, p +  (1-b) \, Q_s \, (1-p)\right)}_{\text{Promedio de pagos}} } 
\end{split}
\end{equation*}
\end{frame}


\begin{frame}[plain]
\begin{textblock}{160}(0,4)
 \centering \LARGE Apuesta cooperativa\\
 \Large Con pagos de la teoría de la probabilidad $Q_c = Q_s = 1$
\end{textblock}
\vspace{1.75cm} 

\begin{textblock}{160}(0,19)
\begin{equation*}
\omega_{t+1} = \omega_t \, \left( b \, \frac{n_c}{N} +  (1-b) \, \frac{n_s}{N} \right)
\end{equation*}
\end{textblock}


\only<2>{
\begin{textblock}{160}(0,30) \centering
\includegraphics[width=0.55\textwidth, page=3]{figures/tasa-temporal2.pdf} 
\end{textblock}
}

\only<3>{
\begin{textblock}{160}(0,30) \centering
\includegraphics[width=0.55\textwidth, page=4]{figures/tasa-temporal2.pdf} 
\end{textblock}
}

\only<4>{
\begin{textblock}{160}(0,30) \centering
\includegraphics[width=0.55\textwidth, page=5]{figures/tasa-temporal2.pdf} 
\end{textblock}
}

\only<5->{
\begin{textblock}{160}(0,38) \centering
\Large Cooperativamente existe una ventaja a favor de la especialización! \\
\large Apostar (casi) todos los recursos a la opción más frecuente
\begin{equation*}
\underset{b}{\text{arg max}} \lim_{N \rightarrow \infty} \omega_t = 1
\end{equation*}
\end{textblock}
}


\only<6>{
\begin{textblock}{160}(0,70) \centering
\Large ¿Rompimos la propiedad epistémica? 
\end{textblock}
}


\only<7->{
\begin{textblock}{160}(0,70) \centering
\Large Si, pero tenemos una propiedad meta-epistémica \\
\large La ciencia es cooperativa porque así aumenta su tasa de crecimiento \\ 
\only<8->{Y el crecimiento es mayor cuando arriesgamos todo por una (la mejor) opción!}
\end{textblock}
}

\end{frame}

\begin{frame}[plain]
\begin{textblock}{160}(0,4)
 \centering \LARGE La apuesta científica \\
 \Large ¿Opciones seguras o descubrimientos?
 \end{textblock}
\vspace{1.75cm} 

\only<2->{
\begin{textblock}{130}(15,22)
$\bullet$ $p > 0.5$: probabilidad de la opción segura \\[0.1cm]
$\bullet$ $1 \leq Q_c \leq 1/p$: pago de la opción segura \\[0.1cm]
$\bullet$ $Q_s > 1/(1-p)$: pago del descubrimiento
\end{textblock}
}

\only<3->{
\begin{textblock}{80}(85,24) 
\begin{equation*}
\underbrace{\left( b \, Q_c  \, p +  (1-b) \, Q_s \, (1-p)\right)}_{\text{Tasa de crecimiento}}  
\end{equation*}
\end{textblock}
}

\only<4>{
\begin{textblock}{130}(15,40) \centering
\includegraphics[width=0.57\textwidth, page=7]{figures/tasa-temporal2}
\end{textblock}
}

\only<5>{
\begin{textblock}{160}(0,52) \centering
\Large Política científica: \\[0.4cm]

\Large Conviene poner todos los recursos en los descubrimientos \\
\large A pesar de que la mayoría fracase individualmente
\end{textblock}
}

\end{frame}

% 
% \begin{frame}[plain]
% \begin{textblock}{160}(0,4)
%  \centering \LARGE Mi apuesta científica \\
%  \Large Observación: involución cultural por pérdida de diversidad cultural
%  \end{textblock}
% \vspace{1.75cm} \centering
% 
% \includegraphics[width=0.65\textwidth]{../../auxiliar/static/output/cisma.png}
% 
% \end{frame}


\begin{frame}[plain]
\begin{textblock}{160}(0,4)
 \centering \LARGE Mi apuesta científica \\
 \Large Causa: pérdida global de diversidad cultural (1850 - )
 \end{textblock}
\vspace{1.25cm} \centering

\includegraphics[width=1\textwidth]{../../auxiliar/static/genocidio_patagonia.jpg}

% \begin{textblock}{160}(0,70)
%   \centering \textcolor{black!15}{\textbf{Patagonia $\sim$ 1880}}
%  \end{textblock} 

\end{frame}


\begin{frame}[plain]
\begin{textblock}{160}(0,4)
 \centering \LARGE Mi apuesta científica \\
 \Large Consecuencia: pérdida global de biodiversidad 
 \end{textblock}
\vspace{1.25cm} \centering


\begin{textblock}{155}(02.5,20)
 \centering
\includegraphics[width=0.49\textwidth]{../../auxiliar/static/chaco1984.jpg}
\includegraphics[width=0.49\textwidth]{../../auxiliar/static/chaco2022.jpg}
\end{textblock}

\end{frame}
% 
% \begin{frame}[plain]
% \begin{textblock}{160}(0,4)
%  \centering \LARGE Mi apuesta científica \\
%  \Large Consecuencia: pérdida global de biodiversidad 
%  \end{textblock}
% \vspace{1.25cm} \centering
% 
% 
% \begin{textblock}{155}(02.5,30)
%  \centering
% \includegraphics[width=0.49\textwidth]{../../auxiliar/static/brasil1986.jpg}
% \includegraphics[width=0.49\textwidth]{../../auxiliar/static/brasil2020.jpg}
% \end{textblock}
% 
% \end{frame}
% 



\begin{frame}[plain]
\begin{textblock}{160}(0,4)
 \centering \LARGE Mi apuesta científica \\
 \Large Pregunta: ¿es posible la coexistencia?
 \end{textblock}
\vspace{1.25cm} \centering


\begin{textblock}{160}(0,24)
\large Casa de apuestas \ vs \  Población apostadora cooperativa
\end{textblock}

\only<2->{
\begin{textblock}{160}(0,35) 
\centering

Pascal-Fermat (1654)

\Large 
¿Cuál es el valor justo de la reciprocidad en contexto de incertidumbre? \\[0.6cm]

\large
\only<3->{
El inverso de la probabilidad 

$Q_c = 1/p$ y $Q_s = 1/(1-p)$
}
\onslide<4->{
\begin{equation*} 
 r = b \, \cancel{Q_c  \, p} +  (1-b) \, \bcancel{Q_s \, (1-p)} = 1  
\end{equation*}
}
\Large
\onslide<5->{Los pagos recíprocos garantizan la coexistencia!}

\large
\onslide<6->{La ventaja de la cooperación es que nunca perdemos}

\end{textblock}
}

\end{frame}

%
% \begin{frame}[plain]
% \begin{textblock}{160}(0,4)
%  \centering \LARGE Laboratorio Pacha Pampas \\
%  \Large Tecnologías de reciprocidad
%  \end{textblock}
% \vspace{0.7cm} \centering
%
%
% \Large
%
% El conocimiento empírico lo tienen las plantas
%
% El saber las comunidades indígenas campesinas
%
% La ciencia moderna tiene que hacer silencio y escuchar
%
%
%
%
%
% \begin{textblock}{160}(0,66) \centering
% \textcolor{black!85}{\rotatebox[origin=tr]{-0}{\scalebox{4.5}{\bet}}}
% \end{textblock}
% \begin{textblock}{160}(0,66) \centering
% \textcolor{black!85}{\rotatebox[origin=tr]{3}{\scalebox{4.5}{$p$}}}
% \end{textblock}
% \begin{textblock}{160}(03,70) \huge \centering
% Pacha \hspace{1cm} Pampas
% \end{textblock}
%
%
%
% \end{frame}



\begin{frame}[plain,noframenumbering]

\begin{textblock}{96}(0,6.5)\centering
{\transparent{0.9}\includegraphics[width=0.8\textwidth]{../../auxiliar/static/inti.png}}
\end{textblock}

\begin{textblock}{160}(96,5.5)
\includegraphics[width=0.35\textwidth]{../../auxiliar/static/pachacuteckoricancha}
\end{textblock}

\end{frame}






\end{document}



