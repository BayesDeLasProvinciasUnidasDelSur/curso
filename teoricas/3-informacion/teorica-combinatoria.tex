\documentclass[shownotes,aspectratio=169]{beamer}

\input{../../auxiliar/tex/diapo_encabezado.tex}
% tikzlibrary.code.tex
%
% Copyright 2010-2011 by Laura Dietz
% Copyright 2012 by Jaakko Luttinen
%
% This file may be distributed and/or modified
%
% 1. under the LaTeX Project Public License and/or
% 2. under the GNU General Public License.
%
% See the files LICENSE_LPPL and LICENSE_GPL for more details.

% Load other libraries
\usetikzlibrary{shapes}
\usetikzlibrary{fit}
\usetikzlibrary{chains}
\usetikzlibrary{arrows}

% Latent node
\tikzstyle{latent} = [circle,fill=white,draw=black,inner sep=1pt,
minimum size=20pt, font=\fontsize{10}{10}\selectfont, node distance=1]
% Observed node
\tikzstyle{obs} = [latent,fill=gray!25]
% Invisible node
\tikzstyle{invisible} = [latent,minimum size=0pt,color=white, opacity=0, node distance=0]
% Constant node
\tikzstyle{const} = [rectangle, inner sep=0pt, node distance=0.1]
%state
\tikzstyle{estado} = [latent,minimum size=8pt,node distance=0.4]
%action
\tikzstyle{accion} =[latent,circle,minimum size=5pt,fill=black,node distance=0.4]


% Factor node
\tikzstyle{factor} = [rectangle, fill=black,minimum size=10pt, draw=black, inner
sep=0pt, node distance=1]
% Deterministic node
\tikzstyle{det} = [latent, rectangle]

% Plate node
\tikzstyle{plate} = [draw, rectangle, rounded corners, fit=#1]
% Invisible wrapper node
\tikzstyle{wrap} = [inner sep=0pt, fit=#1]
% Gate
\tikzstyle{gate} = [draw, rectangle, dashed, fit=#1]

% Caption node
\tikzstyle{caption} = [font=\footnotesize, node distance=0] %
\tikzstyle{plate caption} = [caption, node distance=0, inner sep=0pt,
below left=5pt and 0pt of #1.south east] %
\tikzstyle{factor caption} = [caption] %
\tikzstyle{every label} += [caption] %

\tikzset{>={triangle 45}}

%\pgfdeclarelayer{b}
%\pgfdeclarelayer{f}
%\pgfsetlayers{b,main,f}

% \factoredge [options] {inputs} {factors} {outputs}
\newcommand{\factoredge}[4][]{ %
  % Connect all nodes #2 to all nodes #4 via all factors #3.
  \foreach \f in {#3} { %
    \foreach \x in {#2} { %
      \path (\x) edge[-,#1] (\f) ; %
      %\draw[-,#1] (\x) edge[-] (\f) ; %
    } ;
    \foreach \y in {#4} { %
      \path (\f) edge[->,#1] (\y) ; %
      %\draw[->,#1] (\f) -- (\y) ; %
    } ;
  } ;
}

% \edge [options] {inputs} {outputs}
\newcommand{\edge}[3][]{ %
  % Connect all nodes #2 to all nodes #3.
  \foreach \x in {#2} { %
    \foreach \y in {#3} { %
      \path (\x) edge [->,#1] (\y) ;%
      %\draw[->,#1] (\x) -- (\y) ;%
    } ;
  } ;
}

% \factor [options] {name} {caption} {inputs} {outputs}
\newcommand{\factor}[5][]{ %
  % Draw the factor node. Use alias to allow empty names.
  \node[factor, label={[name=#2-caption]#3}, name=#2, #1,
  alias=#2-alias] {} ; %
  % Connect all inputs to outputs via this factor
  \factoredge {#4} {#2-alias} {#5} ; %
}

% \plate [options] {name} {fitlist} {caption}
\newcommand{\plate}[4][]{ %
  \node[wrap=#3] (#2-wrap) {}; %
  \node[plate caption=#2-wrap] (#2-caption) {#4}; %
  \node[plate=(#2-wrap)(#2-caption), #1] (#2) {}; %
}

% \gate [options] {name} {fitlist} {inputs}
\newcommand{\gate}[4][]{ %
  \node[gate=#3, name=#2, #1, alias=#2-alias] {}; %
  \foreach \x in {#4} { %
    \draw [-*,thick] (\x) -- (#2-alias); %
  } ;%
}

% \vgate {name} {fitlist-left} {caption-left} {fitlist-right}
% {caption-right} {inputs}
\newcommand{\vgate}[6]{ %
  % Wrap the left and right parts
  \node[wrap=#2] (#1-left) {}; %
  \node[wrap=#4] (#1-right) {}; %
  % Draw the gate
  \node[gate=(#1-left)(#1-right)] (#1) {}; %
  % Add captions
  \node[caption, below left=of #1.north ] (#1-left-caption)
  {#3}; %
  \node[caption, below right=of #1.north ] (#1-right-caption)
  {#5}; %
  % Draw middle separation
  \draw [-, dashed] (#1.north) -- (#1.south); %
  % Draw inputs
  \foreach \x in {#6} { %
    \draw [-*,thick] (\x) -- (#1); %
  } ;%
}

% \hgate {name} {fitlist-top} {caption-top} {fitlist-bottom}
% {caption-bottom} {inputs}
\newcommand{\hgate}[6]{ %
  % Wrap the left and right parts
  \node[wrap=#2] (#1-top) {}; %
  \node[wrap=#4] (#1-bottom) {}; %
  % Draw the gate
  \node[gate=(#1-top)(#1-bottom)] (#1) {}; %
  % Add captions
  \node[caption, above right=of #1.west ] (#1-top-caption)
  {#3}; %
  \node[caption, below right=of #1.west ] (#1-bottom-caption)
  {#5}; %
  % Draw middle separation
  \draw [-, dashed] (#1.west) -- (#1.east); %
  % Draw inputs
  \foreach \x in {#6} { %
    \draw [-*,thick] (\x) -- (#1); %
  } ;%
}


 \mode<presentation>
 {
 %   \usetheme{Madrid}      % or try Darmstadt, Madrid, Warsaw, ...
 %   \usecolortheme{default} % or try albatross, beaver, crane, ...
 %   \usefonttheme{serif}  % or try serif, structurebold, ...
  \usetheme{Antibes}
  \setbeamertemplate{navigation symbols}{}
 }
 
\usepackage{todonotes}
\setbeameroption{show notes}

\newif\ifen
\newif\ifes
\newcommand{\en}[1]{\ifen#1\fi}
\newcommand{\es}[1]{\ifes#1\fi}
\estrue

%\title[Bayes del Sur]{}

\begin{document}

\color{black!85}
\large

 
%\setbeamercolor{background canvas}{bg=gray!15}

\begin{frame}[plain,noframenumbering]
 
 \begin{textblock}{90}(03,05)
 \centering \huge  \textcolor{black!40}{Creencias, datos y sorpresas}
\end{textblock}

 \begin{textblock}{47}(113,74)
\centering \Large  \textcolor{white!55}{Universos paralelos}
\end{textblock}

 %\vspace{2cm}brown
%\maketitle
\Wider[2cm]{
\includegraphics[width=1\textwidth]{../../auxiliar/static/peligro_predador}
}
\end{frame}

\begin{frame}[plain]
 \begin{textblock}{160}(0,4)
 \centering \LARGE
Determinismo
\end{textblock}
\vspace{0.75cm}

\begin{center} 
El principio de razón suficiente \\

\Large
\textbf{Nada puede ocurrir sin una causa que la produzca}
\end{center} 
 
\end{frame}


\begin{frame}[plain]
\begin{textblock}{160}(0,4)
 \centering \LARGE 
 \es{Incertidumbre \textbf{honesta}}
 \end{textblock}
\vspace{1cm}

 \begin{center}
 Principio de razón insuficiente \\
   \Large
\textbf{Dividir las creencias en partes iguales \\ (por los caminos del modelo causal)}
 \end{center}
\end{frame}


\begin{frame}[plain]
 \begin{textblock}{160}(0,4)
 \centering \LARGE \es{Universos paralelos: Monty Hall}
 \end{textblock}
 \vspace{-1cm}

 
 \only<1-4>{
 \begin{textblock}{80}(0,22) \centering
\scalebox{1}{
 \tikz{        
    \node[latent] (d) {\includegraphics[width=0.1\textwidth]{../../auxiliar/static/dedo.png}} ;
    \node[const,above=of d] (nd) {\Large $s$} ;
    \node[latent, above=of d, xshift=-1.5cm] (r) {\includegraphics[width=0.12\textwidth]{../../auxiliar/static/regalo.png}} ;
    \node[const,above=of r] (nr) {\Large $r$} ;
    \node[latent, fill=black!30, above=of d, xshift=1.5cm] (c) {\includegraphics[width=0.12\textwidth]{../../auxiliar/static/cerradura.png}} ;
    \node[const,above=of c] (nc) {\Large $c=1$} ;
    \edge {r,c} {d};
}
}
\end{textblock}
 }
 
\only<2->{
 \begin{textblock}{80}(70,11) \centering
\scalebox{1.2}{
\tikz{
\onslide<2->{
\node[latent, draw=white, yshift=0.8cm] (b0) {$1$};
\node[latent,below=of b0,yshift=0.8cm, xshift=-2cm] (r1) {$r_1$};
\node[latent,below=of b0,yshift=0.8cm] (r2) {$r_2$};
\node[latent,below=of b0,yshift=0.8cm, xshift=2cm] (r3) {$r_3$};

\node[latent, below=of r1, draw=white, yshift=0.8cm] (br1) {$\frac{1}{3}$};
\node[latent, below=of r2, draw=white, yshift=0.8cm] (br2) {$\frac{1}{3}$};
\node[latent, below=of r3, draw=white, yshift=0.8cm] (br3) {$\frac{1}{3}$};
}\onslide<3->{
\node[latent,below=of br1,yshift=0.8cm] (c11) {$c_1$};
\node[latent,below=of br2,yshift=0.8cm] (c12) {$c_1$};
\node[latent,below=of br3,yshift=0.8cm] (c13) {$c_1$};

\node[latent, below=of c11, draw=white, yshift=0.8cm] (bc11) {$\frac{1}{3}$};
\node[latent, below=of c12, draw=white, yshift=0.8cm] (bc12) {$\frac{1}{3}$};
\node[latent, below=of c13, draw=white, yshift=0.8cm] (bc13) {$\frac{1}{3}$};
}\onslide<4->{
\node[latent,below=of bc11,yshift=0.8cm, xshift=-0.7cm] (r1d2) {$s_2$};
\node[latent,below=of bc11,yshift=0.8cm, xshift=0.7cm] (r1d3) {$s_3$};
\node[latent,below=of bc12,yshift=0.8cm] (r2d3) {$s_3$};
\node[latent,below=of bc13,yshift=0.8cm] (r3d2) {$s_2$};

\node[latent,below=of r1d2,yshift=0.8cm,draw=white] (br1d2) {$\frac{1}{3}\frac{1}{2}$};
\node[latent,below=of r1d3,yshift=0.8cm, draw=white] (br1d3) {$\frac{1}{3}\frac{1}{2}$};
\node[latent,below=of r2d3,yshift=0.8cm,draw=white] (br2d3) {$\frac{1}{3}$};
\node[latent,below=of r3d2,yshift=0.8cm,draw=white] (br3d2) {$\frac{1}{3}$};
}\onslide<2->{
\edge[-] {b0} {r1,r2,r3};
\edge[-] {r1} {br1};
\edge[-] {r2} {br2};
\edge[-] {r3} {br3};
}\onslide<3->{
\edge[-] {br1} {c11};
\edge[-] {br2} {c12};
\edge[-] {br3} {c13};
\edge[-] {c11} {bc11};
\edge[-] {c12} {bc12};
\edge[-] {c13} {bc13};
}\onslide<4->{
\edge[-] {bc11} {r1d2,r1d3};
\edge[-] {bc12} {r2d3};
\edge[-] {bc13} {r3d2};
\edge[-] {r1d2} {br1d2};
\edge[-] {r1d3} {br1d3};
\edge[-] {r2d3} {br2d3};
\edge[-] {r3d2} {br3d2};
}
}
}
\end{textblock}
}


  \only<5-18>{
 \begin{textblock}{80}(0,22)
  \centering
  $P(r,s)$ \\ \vspace{0.3cm}
 \begin{tabular}{c|c|c|c||c} \setlength\tabcolsep{0.4cm} 
        & \, $r_1$ \, &  \, $r_2$ \, & \, $r_3$ \, & \\ \hline 
  { $s_2$}  & \onslide<6->{$1/6$} & \onslide<8->{$0$} & \onslide<10->{$1/3$} & \onslide<13->{$1/2$} \\ \hline
       {$s_3$} & \onslide<7->{$1/6$} & \onslide<9->{$1/3$} & \onslide<11->{$0$} & \onslide<14->{$1/2$} \\ \hline
              & \onslide<15->{$1/3$} &  \onslide<16->{$1/3$} & \onslide<16->{$1/3$}  & \onslide<17->{$1$} \\ 
\end{tabular}
\end{textblock}
}

\only<19>{
 \begin{textblock}{80}(0,22)
  \centering
  $P(r,s_2)$ \\ \vspace{0.3cm}
 \begin{tabular}{c|c|c|c||c} \setlength\tabcolsep{0.4cm} 
        & \, $r_1$ \, &  \, $r_2$ \, & \, $r_3$ \, & \\ \hline 
        { $s_2$}  & \onslide<6->{$1/6$} & \onslide<8->{$0$} & \onslide<10->{$1/3$} & \onslide<13->{$1/2$} \\ \hline
\end{tabular}
\end{textblock}
}


\only<20->{
 \begin{textblock}{80}(0,22)
  \centering
  $P(r|s_2)$ \\ \vspace{0.3cm}
 \begin{tabular}{c|c|c|c||c} \setlength\tabcolsep{0.4cm} 
        & \, $r_1$ \, &  \, $r_2$ \, & \, $r_3$ \, & \phantom{$1/2$}\\ \hline 
  { $s_2$}  & \onslide<6->{$1/3$} & \onslide<8->{$0$} & \onslide<10->{$2/3$} & \onslide<13->{$1$} \\ \hline
\end{tabular}
\end{textblock}
}


\only<12-16>{
\begin{textblock}{80}(0,58)
 \centering 
\begin{center}
 Regla de la suma
 \end{center} 
 
 $P(s_i) = \sum_{j} P(r_j,s_i)$ 
 \\
 
\end{textblock}
}
 
\only<18-20>{
\begin{textblock}{80}(0,58)
 \centering 
\begin{center}
 Regla del producto
 \end{center} 
 \begin{equation*}
P(r_i|s_2) = \frac{P(r_i,s_2)}{P(s_2)} 
 \end{equation*}
 
\end{textblock}
}
 

\only<21>{
\begin{textblock}{80}(0,53)
\centering

\tikz{
         \node[factor, minimum size=1cm] (p1) {\includegraphics[width=0.05\textwidth]{../../auxiliar/static/cerradura.png}} ;
         \node[det, minimum size=1cm, xshift=1.5cm] (p2) {\includegraphics[width=0.06\textwidth]{../../auxiliar/static/dedo.png}} ;
         \node[factor, minimum size=1cm, xshift=3cm] (p3) {} ;

         \node[const, above=of p1, yshift=.15cm] (fp1) {$1/3$};
         \node[const, above=of p2, yshift=.15cm] (fp2) {$0$};
         \node[const, above=of p3, yshift=.15cm] (fp3) {$2/3$};
         \node[const, below=of p2, yshift=-.10cm, xshift=0.3cm] (dedo) {};
         
        } 
 \end{textblock}
}
 
 
\end{frame}



\begin{frame}[plain]
\begin{textblock}{160}(0,4)
 \centering \LARGE 
 \en{Likelihood}
 \es{Verosimilutd: predicción del dato dada la hipótesis}
 \end{textblock}
 \vspace{-1cm}

 
\begin{textblock}{80}(70,11) \centering
\scalebox{1.2}{
\tikz{
\only<3->{\phantom}{\node[latent, draw=white, yshift=0.8cm] (b0) {$1$};}
\only<8->{\phantom}{\node[latent,below=of b0,yshift=0.8cm, xshift=-2cm] (r1) {$r_1$};}
\only<3-7,10-11>{\phantom}{\node[latent,below=of b0,yshift=0.8cm] (r2) {$r_2$};}
\only<3-9>{\phantom}{\node[latent,below=of b0,yshift=0.8cm, xshift=2cm] (r3) {$r_3$};}

\only<8->{\phantom}{\node[latent, below=of r1, draw=white, yshift=0.8cm] (br1) {$\frac{1}{3}$};}
\only<3-7,10-11>{\phantom}{\node[latent, below=of r2, draw=white, yshift=0.8cm] (br2) {$\frac{1}{3}$};}
\only<3-9>{\phantom}{\node[latent, below=of r3, draw=white, yshift=0.8cm] (br3) {$\frac{1}{3}$};}
\only<8->{\phantom}{\node[latent,below=of br1,yshift=0.8cm] (c11) {$c_1$};}
\only<3-7,10-11>{\phantom}{\node[latent,below=of br2,yshift=0.8cm] (c12) {$c_1$};}
\only<3-9>{\phantom}{\node[latent,below=of br3,yshift=0.8cm] (c13) {$c_1$};}

\only<8->{\phantom}{\node[latent, below=of c11, draw=white, yshift=0.8cm] (bc11) {$\frac{1}{3}$};}
\only<3-7,10-11>{\phantom}{\node[latent, below=of c12, draw=white, yshift=0.8cm] (bc12) {$\frac{1}{3}$};}
\only<3-9>{\phantom}{\node[latent, below=of c13, draw=white, yshift=0.8cm] (bc13) {$\frac{1}{3}$};}
\only<8->{\phantom}{\node[latent,below=of bc11,yshift=0.8cm, xshift=-0.7cm] (r1d2) {$s_2$};}
\only<8->{\phantom}{\node[latent,below=of bc11,yshift=0.8cm, xshift=0.7cm] (r1d3) {$s_3$};}
\only<3-7,10-11>{\phantom}{\node[latent,below=of bc12,yshift=0.8cm] (r2d3) {$s_3$};}
\only<3-9>{\phantom}{\node[latent,below=of bc13,yshift=0.8cm] (r3d2) {$s_2$};}

\only<8->{\phantom}{\node[latent,below=of r1d2,yshift=0.8cm,draw=white] (br1d2) {$\frac{1}{3}\frac{1}{2}$};}
\only<8->{\phantom}{\node[latent,below=of r1d3,yshift=0.8cm, draw=white] (br1d3) {$\frac{1}{3}\frac{1}{2}$};}
\only<3-7,10-11>{\phantom}{\node[latent,below=of r2d3,yshift=0.8cm,draw=white] (br2d3) {$\frac{1}{3}$};}
\only<3-9>{\phantom}{\node[latent,below=of r3d2,yshift=0.8cm,draw=white] (br3d2) {$\frac{1}{3}$};}

\only<3->{\phantom}{\edge[-] {b0} {r1};}
\only<3->{\phantom}{\edge[-] {b0} {r2};}
\only<3->{\phantom}{\edge[-] {b0} {r3};}
\only<8->{\phantom}{\edge[-] {r1} {br1};}
\only<3-7,10-11>{\phantom}{\edge[-] {r2} {br2};}
\only<3-9>{\phantom}{\edge[-] {r3} {br3};}
\only<8->{\phantom}{\edge[-] {br1} {c11};}
\only<3-7,10-11>{\phantom}{\edge[-] {br2} {c12};}
\only<3-9>{\phantom}{\edge[-] {br3} {c13};}
\only<8->{\phantom}{\edge[-] {c11} {bc11};}
\only<3-7,10-11>{\phantom}{\edge[-] {c12} {bc12};}
\only<3-9>{\phantom}{\edge[-] {c13} {bc13};}
\only<8->{\phantom}{\edge[-] {bc11} {r1d2,r1d3};}
\only<3-7,10-11>{\phantom}{\edge[-] {bc12} {r2d3};}
\only<3-9>{\phantom}{\edge[-] {bc13} {r3d2};}
\only<8->{\phantom}{\edge[-] {r1d2} {br1d2};}
\only<8->{\phantom}{\edge[-] {r1d3} {br1d3};}
\only<3-7,10-11>{\phantom}{\edge[-] {r2d3} {br2d3};}
\only<3-9>{\phantom}{\edge[-] {r3d2} {br3d2};}
}
}
\end{textblock}


\only<1->{
 \begin{textblock}{80}(0,16)
  \centering
  $P(s_2|r_i)$ \\ \vspace{0.1cm} 
  \onslide<2->{
  \begin{tabular}{c|c|c|c} \setlength\tabcolsep{0.4cm} 
          & \, \only<3-7>{\bm}{$r_1$} \, &  \, \only<8-9>{\bm}{$r_2$} \, & \, \only<10-11>{\bm}{$r_3$} \, \\ \hline 
   $s_2$ & \onslide<7->{$1/2$} & \onslide<9->{$0$} & \onslide<11->{$1$}  \\ \hline
\end{tabular} 
}
\end{textblock}
}

\only<4->{
\begin{textblock}{80}(0,40)
 \begin{equation*}
  P(s|r_{\only<4-7>{1}\only<8-9>{2}\only<10-11>{3}}) = \frac{P(r_{\only<4-7>{1}\only<8-9>{2}\only<10-11>{3}}, s)}{P(r_{\only<4-7>{1}\only<8-9>{2}\only<10-11>{3}})}
 \end{equation*}
\end{textblock}
}


 \only<5>{
 \begin{textblock}{80}(0,57)
  \centering
  $P(r,s)$\\ \vspace{0.1cm}
 \begin{tabular}{c|c|c|c||c} \setlength\tabcolsep{0.4cm} 
          & \, $r_1$ \, &  \, $r_2$ \, & \, $r_3$ \, & \\ \hline 
   $s_2$ & $1/6$ & $0$ & $1/3$     & $1/2$ \\ \hline
   $s_3$ & $1/6$ & $1/3$ & $0$     & $1/2$ \\ \hline \hline
         & $1/3$ &  $1/3$ & $1/3$  & $1$ \\ 
\end{tabular} 
\end{textblock}
}

\only<6->{
\begin{textblock}{80}(0,57)
  \centering
  \only<6>{$P(r_1, s)$}\only<7>{$P(s|r_1)$}\only<8>{$P(r_2, s)$}\only<9>{$P(s|r_2)$}\only<10>{$P(r_3, s)$}\only<11>{$P(s|r_3)$} \\ \vspace{0.1cm}
 \begin{tabular}{c|c|c|c||c} \setlength\tabcolsep{0.4cm} 
          & \, $r_1$ \, &  \, $r_2$ \, & \, $r_3$ \, &  \phantom{$1/2$} \\ \hline 
   \only<7->{\bm}{$s_2$} & \only<6>{$1/6$}\only<7>{\bm{$1/2$}} & \only<8>{$0$}\only<9>{\bm{$0$}} & \only<10>{$1/3$}\only<11>{\bm{$1$}} &  \\ \hline
   $s_3$ & \only<6>{$1/6$}\only<7>{$1/2$} & \only<8>{$1/3$}\only<9>{$1$} & \only<10-11>{$0$}  &  \\ \hline \hline
         & \only<6>{$1/3$}\only<7>{$1$} & \only<8>{$1/3$}\only<9>{$1$} & \only<10>{$1/3$}\only<11>{$1$}  &  \\ 
\end{tabular} 
\end{textblock}
}

\end{frame}

\begin{frame}[plain]
\begin{textblock}{160}(0,4)
 \centering \LARGE 
 \en{Likelihood}
 \es{Verosimilitud: los caminos del modelo causal}
 \end{textblock}

 \begin{textblock}{160}(0,18)
  \centering
  $P(s_2|r_i)$ \\ \vspace{0.1cm} 
  \begin{tabular}{c|c|c|c} \setlength\tabcolsep{0.4cm} 
          & \, $r_1$ \, &  \, $r_2$ \, & \, $r_3$ \, \\ \hline 
   $s_2$ & $1/2$ & $0$ & $1$  \\ \hline
\end{tabular}
\end{textblock}

\only<2>{
 \begin{textblock}{160}(0,49)
 \begin{align*}
  P(s_2|r_i, M) = \frac{\text{\textbf{Caminos que generan} $\bm{s_2}$ dada la hipótesis $r_i$ y el modelo}}{\text{\textbf{Caminos totales} dada la hipótesis $r_i$ y el modelo }}
 \end{align*}
\end{textblock}
}
 
\end{frame}




\begin{frame}[plain]
 
 \centering
 Imagen de niña contanto contando
 
\end{frame}

\begin{frame}[plain]
\begin{textblock}{160}(0,4)
 \centering \LARGE 
 \es{Conjunto de pares ordenados}
 \end{textblock}
 \vspace{1cm}

 Dados dos conjuntos A y B, donde A tiene k elementos y B tiene m ele-
mentos, queremos determinar cuántos elementos tiene A $\times$ B.

\begin{align*}
\underbrace{m + m + \dots + m}_{k \text{ veces}} = km
\end{align*}
 
\end{frame}

\begin{frame}[plain]
\begin{textblock}{160}(0,4)
 \centering \LARGE 
 \es{Conjunto de pares ordenados}
 \end{textblock}
 \vspace{1cm}

Dados los conjuntos $A_1 , A_2 , \dots , A_n$ , donde $\#A_i = k_i$, ¿cuántos
elementos tiene $A_1 \times A_2 \times \dots A_n$?


\begin{align*}
k_1 k_2 \dots k_n
\end{align*}

\end{frame}

\begin{frame}[plain]
 \begin{textblock}{160}(0,4)
 \centering \LARGE 
 \es{Conjunto de funciones}
 \end{textblock}
 \vspace{1cm}
 
 Si $A$ es un conjunto de $n$ elementos y $B$ es un conjunto de $k$ elementos,
¿cuántas funciones de $A$ en $B$ se pueden definir?
 
\begin{align*}
 (f(a_1),f(a_2),\dots,f(a_n)) \in \underbrace{B \times B \times \dots \times B}_{n \text{ factores}}
\end{align*}

Luego

\begin{align*}
 k^n
\end{align*}


\end{frame}


\begin{frame}[plain]
 \begin{textblock}{160}(0,4)
 \centering \LARGE 
 \es{Bolitas numeradas en cajas numeradas}
 \end{textblock}
 \vspace{1cm}
 \centering
 
 ¿De cuántas maneras se pueden ubicar n bolitas numeradas en k cajas numeradas?
 
 \vspace{0.3cm}
 
 Notemos que esto es lo mismo que determinar cuántas funciones hay del tipo 
 
 \begin{align*}
  f(i) = \text{número de caja donde está ubicada la bolita } i
 \end{align*}
 
 Luego, hay $k^n$ formas de ubicar las bolitas.

 \end{frame}

 \begin{frame}[plain]
 \begin{textblock}{160}(0,4)
 \centering \LARGE 
 \es{Funciones inyectivas}
 \end{textblock}
 \vspace{1cm}
 \centering
 
 Si $A$ es un conjunto de $n$ elementos y $B$ es un conjunto de $m$ elementos, con $n \geq m$
¿cuántas funciones inyectivas de $A$ en $B$ se pueden definir?
 
\begin{align*}
 (f(a_1),f(a_2),\dots,f(a_n)) \in \underbrace{B \times B \times \dots \times B}_{n \text{ factores}} \text{ que tienen todas sus coordenadas distintas}
\end{align*}

Luego

\begin{align*}
 m(m-1)(m-2)\dots(m-(n-1)) = m(m-1)(m-2)\dots(m-n+1) = \frac{m!}{(m-n)!}
\end{align*}

 
 \end{frame}

 
 \begin{frame}[plain]
  
  ¿De cuántas maneras se pueden ubicar n bolitas numeradas en m cajas
numeradas de manera que haya a lo sumo una bolita por caja?


\begin{align*}
 m(m-1)(m-2)\dots(m-(n-1)) = m(m-1)(m-2)\dots(m-n+1) = \frac{m!}{(m-n)!}
\end{align*}

 \end{frame}


\begin{frame}[plain]
\centering
  \includegraphics[width=0.35\textwidth]{../../auxiliar/static/pachacuteckoricancha.jpg}
\end{frame}



\end{document}



