\documentclass[shownotes,aspectratio=169]{beamer}

\input{../../auxiliar/tex/diapo_encabezado.tex}
\input{../../auxiliar/tex/tikzlibrarybayesnet.code.tex}
 \mode<presentation>
 {
 %   \usetheme{Madrid}      % or try Darmstadt, Madrid, Warsaw, ...
 %   \usecolortheme{default} % or try albatross, beaver, crane, ...
 %   \usefonttheme{serif}  % or try serif, structurebold, ...
  \usetheme{Antibes}
  \setbeamertemplate{navigation symbols}{}
 }
\estrue
\usepackage{todonotes}
\setbeameroption{show notes}

\usepackage{listings}
\lstset{
  aboveskip=3mm,
  belowskip=3mm,
  showstringspaces=true,
  columns=flexible,
  basicstyle={\ttfamily},
  breaklines=true,
  breakatwhitespace=true,
  tabsize=4,
  showlines=true
}


\begin{document}

\color{black!85}
\large

\begin{frame}[plain,noframenumbering]
\begin{textblock}{80}(54,14)
 \huge \textcolor{black!50}{Sorpresa}
\end{textblock}

\begin{textblock}{47}(113,73.5)
\centering \LARGE  \textcolor{black!5}{Supervivencia}
\end{textblock}

\begin{textblock}{80}(100,27)
\LARGE  \textcolor{black!10}{Creencia}
\end{textblock}

\begin{textblock}{80}(44,61)
\LARGE  \textcolor{black!15}{Dato}
\end{textblock}

\begin{textblock}{160}(01,87)
\scriptsize \textcolor{black!5}{Bayes de la Provincias Unidas del Sur, 2022.}
\end{textblock}

\begin{textblock}{160}(01,01)
\normalsize \textcolor{black!60}{Práctica 0:\ Coexistencia}
\end{textblock}


 %\vspace{2cm}brown
%\maketitle
\Wider[2cm]{
\includegraphics[width=1\textwidth]{../../auxiliar/static/peligro_predador}
}
\end{frame}



\begin{frame}[plain,noframenumbering]
\begin{textblock}{160}(00,10)
\centering
\huge Práctica 0 \\[0.4cm] 
\LARGE Teoría de la probabilidad: \\ tecnología de reciprocidad
\end{textblock}

\begin{textblock}{160}(00,48) \centering

Seminario abierto.

Bayes de la Provincias Unidas del Sur.


\vspace{1.5cm}

Primer cuatrimestre 2022

\vspace{.3cm}

Buenos Aires
\end{textblock}



\end{frame}

\begin{frame}[plain]
\begin{textblock}{160}(00,04)
\centering
\LARGE Principio de reciprocidad \\
\large Cartas entre Pascal y Fermat, 1648
\end{textblock}
\vspace{1.5cm}



Tiramos dos veces una moneda: \\
$\bullet$ Si sale seca (S) en la primera y en la segunda, te hago un favor (Color rojo). \\
$\bullet$ Caso contrario (C), me hacés un favor (Color negro). \\
\begin{center}
\tikz{
\node[latent, draw=white, yshift=0.7cm, minimum size=0.1cm] (b0) {};
\node[latent,below=of b0,yshift=0.7cm, xshift=-1cm] (r1) {$S$};
\node[latent,below=of b0,yshift=0.7cm, xshift=1cm] (r2) {$C$};

\node[latent, below=of r1, draw=white, yshift=0.8cm, minimum size=0.1cm] (bc11) {};
\node[accion, below=of r2, draw=white, yshift=0cm] (bc12) {};
\node[latent,below=of bc11,yshift=0.8cm, xshift=-0.5cm] (r1d2) {$S$};
\node[latent,below=of bc11,yshift=0.8cm, xshift=0.5cm] (r1d3) {$C$};

\node[accion,below=of r1d2,yshift=0cm, color=red] (br1d2) {};
\node[accion,below=of r1d3,yshift=0cm] (br1d3) {};
\edge[-] {b0} {r1,r2};
\edge[-] {r1} {bc11};
\edge[-] {r2} {bc12};
\edge[-] {bc11} {r1d2,r1d3};
\edge[-] {r1d2} {br1d2};
\edge[-] {r1d3} {br1d3};
}
\end{center}


\centering
¿Cuál es el valor justo de la reciprocidad en este contexto de incertidumbre?

\end{frame}


\begin{frame}[plain]
\begin{textblock}{160}(0,4)
 \centering \LARGE Principio de coexistencia
\end{textblock}
\vspace{1cm}

\normalsize

Las casas de apuestas ofrecen pagos $q_c$ y $q_s$ por Cara y Seca (multiplican lo apostado).
Supongamos que se conoce la probabilidad $p$ de Cara.

\vspace{0.3cm} \pause

$\bullet$ ¿Cuál es la apuesta óptima, las proporciones $b_c$ y $b_s$ de recursos que conviene apostar a Cara y Seca? \\ \pause
$\bullet$ ¿Existe un conjunto de pagos $q_c$ y $q_s$ que garantice la coexistencia en el tiempo entre la casa de apuestas y una población de apostadoras, o una población cooperadora siempre puede romper cualquier pago que ofrezca la casa de apuestas a través de la especialización?

\end{frame}



\begin{frame}[plain]
\begin{textblock}{160}(0,4)
 \centering \LARGE Apostar a las dos opciones \\
 \Large $b = b_c = 1 - b_s$
\end{textblock}
\vspace{1.5cm} 

Como apostadores queremos maximizar nuestros recursos $\omega_T$,
\begin{equation*}
\omega_T = (\underbrace{(\omega_0 \, \overbrace{b_c \,  q_c}^{\text{Cara}})}_{\omega_1} \, \overbrace{b_s \, q_s}^{\text{Seca}}) \dots \onslide<2->{= \omega_0 \,  (b_c \,  q_c)^{n_c}  \,  (b_s \, q_s)^{n_s}}
\end{equation*}
\onslide<3->{donde $n_c + n_s = T$ son las veces que salió Cara y Seca en $T$ intentos.}

\onslide<4->{
\begin{equation*}
\begin{split}
\omega_0 \,  r^T &= \lim_{T \rightarrow \infty} \omega_0 \,  (b_c \,  q_c)^{n_c}  \,  (b_s \, q_s)^{n_s} \\
\onslide<5->{r &= \lim_{T \rightarrow \infty} (b_c \,  q_c)^{n_c/T}  \,  (b_s \, q_s)^{n_s/T} } \onslide<6->{=  (b_c \,  q_c)^{p}  \,  (b_s \, q_s)^{1-p}} 
\end{split}
\end{equation*}
}

\onslide<7->{Queremos maximizar la tasa de crecimiento $r$}
\end{frame}

\begin{frame}[plain]
\begin{textblock}{160}(0,4)
 \centering \LARGE Apostar a una sola opción \\
 \Large $l = b_c > 0$ y $b_s=0$ o al revés
\end{textblock}
\vspace{1.5cm} 

Como apostadores queremos maximizar nuestros recursos $\omega_T$,
\begin{equation*}
\begin{split}
\onslide<2->{\omega_T = \overbrace{(\omega_0 \, (1-l)) + (\omega_0 \, l \, q_c)}^{\omega_1} \, (1-l) \dots }
\onslide<3->{&=\omega_0  ( \overbrace{(1-l) + (l \, q_c)}^{\text{Cara}}) \, \overbrace{(1-l)}^{\text{Seca}} \dots } \\
\onslide<4->{&=\omega_0 \, (1 +  l \, (q_c - 1))^{n_c} \,  (1-l)^{n_s}}
\end{split}
\end{equation*}

\onslide<5->{donde $n_c + n_s = T$ son las veces que salió Cara y Seca en $T$ intentos.}

 \onslide<6->{
\begin{equation*}
\begin{split}
\omega_0 \,  r^T &= \lim_{T \rightarrow \infty} \omega_0 \,  (1 +  l \, (q_c - 1))^{n_c} \,  (1-l)^{n_s} \\
\onslide<7->{r &= (1 +  l \, (q_c - 1))^{p} \,  (1-l)^{1-p}}
\end{split}
\end{equation*}
}

\onslide<8->{Queremos maximizar la tasa de crecimiento $r$}
 
\end{frame}

\begin{frame}[plain]
\begin{textblock}{160}(0,4)
 \centering \LARGE Apostar a las dos opciones \\
 \Large Maximizar $r$
 \end{textblock}
\vspace{1.75cm} 

 \begin{equation*}
r =   (b \,  q_c)^{p}  \,  ((1-b) \, q_s)^{1-p} 
\end {equation*}

\vspace{0.3cm} 
\onslide<2->{Dados los pagos recíprocos $q_c = 1/p$ y $q_s = 1 / (1-p)$}
\begin{equation*}
\begin{split}
\onslide<3->{r & =  (b / p)^{p}  \,  ((1-b) / (1-p))^{1-p}  \onslide<8->{\overset{b=p}{=} 1}  \\ }
\onslide<4->{\log r &= p \, \log \frac{b}{p} + (1-p) \log  \frac{1-b}{1-p} } \\
\end{split}
\end{equation*}

\onslide<5->{Podemos maximizar $r$ encontrnado su punto crı́tico en b (es cóncava).}
\begin{equation*}
\begin{split}
\onslide<6->{0 &= \frac{d}{db} p \, \log \frac{b}{p} + (1-p) \log  \frac{1-b}{1-p} \\}
\onslide<7->{b &= p}
\end{split}
\end{equation*}

\end{frame}


\begin{frame}[plain]
\begin{textblock}{160}(0,4)
 \centering \LARGE Apostar a las dos opciones \\
 \Large Maximizar $r$ con pagos sesgados
\end{textblock}
\vspace{1.5cm} 

Sea $q_c = 1/q$ y $q_s = 1 / (1-q)$ con $q \in [0,1]$ \\[0.3cm]

\onslide<2->{Podemos maximizar $r$ encontrnado su punto crı́tico en b (es cóncava).}
\begin{equation*}
\begin{split}
\onslide<3->{0 &= \frac{d}{db} p \, \log \frac{b}{q} + (1-p) \log  \frac{1-b}{1-q} \\}
\onslide<4->{b &= p}
\end{split}
\end{equation*}

\Large

\vspace{0.4cm}

\centering
\onslide<5->{La apuesta óptima siempre se obtiene dividiendo \\ los recursos en la misma proproción que la probabilidad $p$}


\end{frame}


\begin{frame}[plain]
\begin{textblock}{160}(0,4)
 \centering \LARGE Apostar a una opción \\
 \Large Tarea
 \end{textblock}
\vspace{1.75cm} 

$\bullet$ Maximizar $r$ \\
$\bullet$ Mostrar que es equivalente a apostar en las dos opciones \\

\end{frame}


\begin{frame}[plain]
\begin{textblock}{160}(0,4)
 \centering \LARGE Apostar a las dos opciones (Cooperativo)
\end{textblock}
\vspace{1.75cm} 

Como cooperadores queremos maximizar nuestra cuota del fondo común,
\begin{equation*}
\onslide<2->{\text{fondo común}_t = (\omega_t \, b \, q_c ) \, n_c + (\omega_t \, (1-b) \, q_s ) \, n_s }
\end{equation*}
\onslide<3->{donde $n_c + n_s = N$ son las veces que salió Cara y Seca en el grupo de tamaño $N$}
\begin{equation*}
\begin{split}
\onslide<4->{\text{couta} = \omega_{t+1} & = \frac{1}{N} \left( (\omega_t \, b \, q_c ) \, n_c  + (\omega_t \, (1-b) \, q_s ) \, n_s }\\
\onslide<5->{& = \omega_t  \left( b \, q_c  \, \frac{n_c}{N} +  (1-b) \, q_s \, \frac{n_s}{N} \right) } 
\end{split}
\end{equation*}

\onslide<6->{Cuando el grupo es de tamaño infinito, la cuota siempre tiene el mismo valor}
\begin{equation*}
\begin{split}
\onslide<7->{ \omega_{t+1} & = \lim_{N \rightarrow \infty } \omega_t  \left( b \, q_c  \, \frac{n_c}{N} +  (1-b) \, q_s \, \frac{n_s}{N} \right) } \onslide<8->{= \omega_t  \underbrace{\left( b \, q_c  \, p +  (1-b) \, q_s \, (1-p)\right)}_{\text{Promedio de pagos}} } 
\end{split}
\end{equation*}
\end{frame}


\begin{frame}[plain]
\begin{textblock}{160}(0,4)
 \centering \LARGE Apostar a las dos opciones (Cooperativo)\\
 \Large Maximizar $r$
\end{textblock}
\vspace{1.75cm} 

\begin{equation*}
r = \left( b \, q_c  \, p +  (1-b) \, q_s \, (1-p)\right)
\end{equation*}

\vspace{0.3cm} 
\onslide<2->{Dados los pagos recíprocos $q_c = 1/p$ y $q_s = 1 / (1-p)$}

\begin{equation*}
\begin{split}
\onslide<3->{r &= \left( b \, \frac{1}{p}  \, p +  (1-b) \, \frac{1}{1-p} \, (1-p)\right)}  \\ \onslide<4->{&= b + (1 - b) = 1}
\end{split}
\end{equation*}


\Large

\vspace{0.4cm}

\centering
\onslide<5->{Si el pago es recíproco \\ cooperativamentre nunca perdemos.}


\end{frame}

\begin{frame}[plain]
\begin{textblock}{160}(0,4)
 \centering \LARGE Apostar a las dos opciones (Cooperativo)\\
 \Large Tarea
\end{textblock}
\vspace{1.75cm} 

$\bullet$ Maximizar $r$ con pagos sesgados en grupo de tamaño 2 \\
$\bullet$ Maximizar $r$ con pagos sesgados en grupo de tamaño 3 \\
$\bullet$ Maximizar $r$ con pagos sesgados en grupo de tamaño infinito.

\vspace{1cm}

Los pagos sesgados son $q_c = 1/q$ y $q_s = 1 / (1-q)$ con $q \neq p$ \\[0.3cm]


\end{frame}

\begin{frame}[plain]
\centering
  \includegraphics[width=0.35\textwidth]{../../auxiliar/static/pachacuteckoricancha.jpg}
\end{frame}






\end{document}



