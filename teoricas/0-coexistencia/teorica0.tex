\documentclass[shownotes,aspectratio=169]{beamer}

\input{../../auxiliar/tex/diapo_encabezado.tex}
% tikzlibrary.code.tex
%
% Copyright 2010-2011 by Laura Dietz
% Copyright 2012 by Jaakko Luttinen
%
% This file may be distributed and/or modified
%
% 1. under the LaTeX Project Public License and/or
% 2. under the GNU General Public License.
%
% See the files LICENSE_LPPL and LICENSE_GPL for more details.

% Load other libraries
\usetikzlibrary{shapes}
\usetikzlibrary{fit}
\usetikzlibrary{chains}
\usetikzlibrary{arrows}

% Latent node
\tikzstyle{latent} = [circle,fill=white,draw=black,inner sep=1pt,
minimum size=20pt, font=\fontsize{10}{10}\selectfont, node distance=1]
% Observed node
\tikzstyle{obs} = [latent,fill=gray!25]
% Invisible node
\tikzstyle{invisible} = [latent,minimum size=0pt,color=white, opacity=0, node distance=0]
% Constant node
\tikzstyle{const} = [rectangle, inner sep=0pt, node distance=0.1]
%state
\tikzstyle{estado} = [latent,minimum size=8pt,node distance=0.4]
%action
\tikzstyle{accion} =[latent,circle,minimum size=5pt,fill=black,node distance=0.4]


% Factor node
\tikzstyle{factor} = [rectangle, fill=black,minimum size=10pt, draw=black, inner
sep=0pt, node distance=1]
% Deterministic node
\tikzstyle{det} = [latent, rectangle]

% Plate node
\tikzstyle{plate} = [draw, rectangle, rounded corners, fit=#1]
% Invisible wrapper node
\tikzstyle{wrap} = [inner sep=0pt, fit=#1]
% Gate
\tikzstyle{gate} = [draw, rectangle, dashed, fit=#1]

% Caption node
\tikzstyle{caption} = [font=\footnotesize, node distance=0] %
\tikzstyle{plate caption} = [caption, node distance=0, inner sep=0pt,
below left=5pt and 0pt of #1.south east] %
\tikzstyle{factor caption} = [caption] %
\tikzstyle{every label} += [caption] %

\tikzset{>={triangle 45}}

%\pgfdeclarelayer{b}
%\pgfdeclarelayer{f}
%\pgfsetlayers{b,main,f}

% \factoredge [options] {inputs} {factors} {outputs}
\newcommand{\factoredge}[4][]{ %
  % Connect all nodes #2 to all nodes #4 via all factors #3.
  \foreach \f in {#3} { %
    \foreach \x in {#2} { %
      \path (\x) edge[-,#1] (\f) ; %
      %\draw[-,#1] (\x) edge[-] (\f) ; %
    } ;
    \foreach \y in {#4} { %
      \path (\f) edge[->,#1] (\y) ; %
      %\draw[->,#1] (\f) -- (\y) ; %
    } ;
  } ;
}

% \edge [options] {inputs} {outputs}
\newcommand{\edge}[3][]{ %
  % Connect all nodes #2 to all nodes #3.
  \foreach \x in {#2} { %
    \foreach \y in {#3} { %
      \path (\x) edge [->,#1] (\y) ;%
      %\draw[->,#1] (\x) -- (\y) ;%
    } ;
  } ;
}

% \factor [options] {name} {caption} {inputs} {outputs}
\newcommand{\factor}[5][]{ %
  % Draw the factor node. Use alias to allow empty names.
  \node[factor, label={[name=#2-caption]#3}, name=#2, #1,
  alias=#2-alias] {} ; %
  % Connect all inputs to outputs via this factor
  \factoredge {#4} {#2-alias} {#5} ; %
}

% \plate [options] {name} {fitlist} {caption}
\newcommand{\plate}[4][]{ %
  \node[wrap=#3] (#2-wrap) {}; %
  \node[plate caption=#2-wrap] (#2-caption) {#4}; %
  \node[plate=(#2-wrap)(#2-caption), #1] (#2) {}; %
}

% \gate [options] {name} {fitlist} {inputs}
\newcommand{\gate}[4][]{ %
  \node[gate=#3, name=#2, #1, alias=#2-alias] {}; %
  \foreach \x in {#4} { %
    \draw [-*,thick] (\x) -- (#2-alias); %
  } ;%
}

% \vgate {name} {fitlist-left} {caption-left} {fitlist-right}
% {caption-right} {inputs}
\newcommand{\vgate}[6]{ %
  % Wrap the left and right parts
  \node[wrap=#2] (#1-left) {}; %
  \node[wrap=#4] (#1-right) {}; %
  % Draw the gate
  \node[gate=(#1-left)(#1-right)] (#1) {}; %
  % Add captions
  \node[caption, below left=of #1.north ] (#1-left-caption)
  {#3}; %
  \node[caption, below right=of #1.north ] (#1-right-caption)
  {#5}; %
  % Draw middle separation
  \draw [-, dashed] (#1.north) -- (#1.south); %
  % Draw inputs
  \foreach \x in {#6} { %
    \draw [-*,thick] (\x) -- (#1); %
  } ;%
}

% \hgate {name} {fitlist-top} {caption-top} {fitlist-bottom}
% {caption-bottom} {inputs}
\newcommand{\hgate}[6]{ %
  % Wrap the left and right parts
  \node[wrap=#2] (#1-top) {}; %
  \node[wrap=#4] (#1-bottom) {}; %
  % Draw the gate
  \node[gate=(#1-top)(#1-bottom)] (#1) {}; %
  % Add captions
  \node[caption, above right=of #1.west ] (#1-top-caption)
  {#3}; %
  \node[caption, below right=of #1.west ] (#1-bottom-caption)
  {#5}; %
  % Draw middle separation
  \draw [-, dashed] (#1.west) -- (#1.east); %
  % Draw inputs
  \foreach \x in {#6} { %
    \draw [-*,thick] (\x) -- (#1); %
  } ;%
}


 \mode<presentation>
 {
 %   \usetheme{Madrid}      % or try Darmstadt, Madrid, Warsaw, ...
 %   \usecolortheme{default} % or try albatross, beaver, crane, ...
 %   \usefonttheme{serif}  % or try serif, structurebold, ...
  \usetheme{Antibes}
  \setbeamertemplate{navigation symbols}{}
 }
 
\usepackage{todonotes}
\setbeameroption{show notes}

\newif\ifen
\newif\ifes
\newcommand{\en}[1]{\ifen#1\fi}
\newcommand{\es}[1]{\ifes#1\fi}
\estrue



\newcommand{\E}{\en{S}\es{E}}
\newcommand{\A}{\en{E}\es{A}}
\newcommand{\Ee}{\en{s}\es{e}}
\newcommand{\Aa}{\en{e}\es{a}}


%\title[Bayes del Sur]{}

\begin{document}

\color{black!85}
\large

 
%\setbeamercolor{background canvas}{bg=gray!15}

\begin{frame}[plain,noframenumbering]
\begin{textblock}{80}(34,14)
 \huge \textcolor{black!50}{Sorpresa}
\end{textblock}

\begin{textblock}{47}(113,73.5)
\centering \LARGE  \textcolor{black!5}{Supervivencia}
\end{textblock}

\begin{textblock}{80}(100,27)
\LARGE  \textcolor{black!10}{Creencia}
\end{textblock}

\begin{textblock}{80}(44,61)
\LARGE  \textcolor{black!15}{Dato}
\end{textblock}

\begin{textblock}{160}(01,87)
\scriptsize \textcolor{black!5}{Bayes de la Provincias Unidas del Sur, 2022.}
\end{textblock}

\begin{textblock}{160}(01,01)
\normalsize \textcolor{black!60}{0.\ Coexistencia}
\end{textblock}

 %\vspace{2cm}brown
%\maketitle
\Wider[2cm]{
\includegraphics[width=1\textwidth]{../../auxiliar/static/peligro_predador}
}
\end{frame}

\begin{frame}[plain,noframenumbering]
\begin{textblock}{160}(00,12)
\centering
\huge Teoría de la probabilidad: \\ 0.\ principio de coexistencia
\end{textblock}

\begin{textblock}{160}(00,42) \centering

Seminario abierto.

Bayes de la Provincias Unidas del Sur.


\vspace{1.5cm}

23 Marzo 2022

\vspace{.3cm}

Buenos Aires
\end{textblock}



\end{frame}

% 
% \begin{frame}[plain]
% \begin{textblock}{160}(0,4)
%  \centering
%  \LARGE \textcolor{black!85}{\en{Today}\es{Hoy}}
% \end{textblock}
% 
% \begin{itemize}
%  \item[$\bullet$] Human dispersal
%  \item[$\bullet$] Biomass (dentro de los vertebrados terrestres)
%  \item[$\bullet$] Empatia
%  \item[$\bullet$] Evolución cultural
%  \item[$\bullet$] Ciencia como intersubjectividad (muto entendimiento)
%  \item[$\bullet$] Base empírica 
%  \item[$\bullet$] Matriz de datos 
%  \item[$\bullet$] Ciencia empírica 
%  \item[$\bullet$] Niveles de conocimiento
%  \item[$\bullet$] Incertidumbre
%  \item[$\bullet$] Creencias honestas
%  \item[$\bullet$] Modelos causales
%  \item[$\bullet$] La lógica de la ciencia empírica
%  \item[$\bullet$] Selección de modelo
% \end{itemize}
% 
% \end{frame}


\begin{frame}[plain]
\begin{textblock}{160}(0,4)
 \centering \LARGE
Vida
\end{textblock}
\vspace{1cm}
\includegraphics[width=1\textwidth]{../../auxiliar/static/biomass.jpg}
\end{frame}

\begin{frame}[plain]
\begin{textblock}{160}(0,4)
 \centering \LARGE
Crecimiento de los linajes
\end{textblock}
\vspace{1cm}

\begin{equation*} 
\omega(T) = \prod_t^T f(\Aa(t)) \approx r^T 
\end{equation*}

\vspace{0.3cm}

\begin{equation*}
f(\Aa) =
\begin{cases}
 1.5 & \Aa = \text{ \en{Head}\es{Cara} } \\
 0.6 & \Aa = \text{ \en{Tail}\es{Sello} }
\end{cases}
\end{equation*}

\pause \centering \vspace{1cm} 

¿Cuál es la tasa de crecimiento $r$?

\end{frame}

\begin{frame}[plain]
\begin{textblock}{160}(0,4)
 \centering \LARGE
Poblaciones de tamaño infinito
\end{textblock}
\vspace{1cm}

\only<1>{
\begin{textblock}{160}(0,22)
\begin{equation*}
\langle \omega \rangle_t = \sum_{\omega} \omega \cdot  P(\omega)
\end{equation*}
\end{textblock}
}

\only<2->{
\begin{textblock}{160}(0,12)
\begin{equation*}
\begin{split}
\langle \omega \rangle_1 & = 1.5 \cdot \frac{1}{2} + 0.6 \cdot  \frac{1}{2} = 1.05 \\ 
\onslide<3->{\langle \omega \rangle_2 &=  1.5^2 \cdot \frac{1}{4} + 2 (0.6 \cdot 1.5 \cdot \frac{1}{4} ) + 0.6^2 \cdot \frac{1}{4}= 1.05^2 }
\end{split}
\end{equation*}
\end{textblock}
}

\only<4>{
\begin{textblock}{140}(10,36)
\begin{figure}[H]
    \centering
    \begin{subfigure}[b]{0.5\linewidth}
    \includegraphics[width=\linewidth]{figures/pdf/ergodicity_expectedValue.pdf}
    \end{subfigure}
\end{figure}
\end{textblock}
}

\end{frame}


\begin{frame}[plain]
\begin{textblock}{160}(0,4)
 \centering \LARGE
Trayectorias individuales en el tiempo
\end{textblock}
\vspace{1cm}

\begin{textblock}{140}(10,10)
\begin{figure}[H]
    \centering
    \begin{subfigure}[b]{0.49\linewidth}
    \includegraphics[width=\linewidth]{figures/pdf/ergodicity_individual_trayectories_y.pdf}
    \end{subfigure}
\end{figure}
\end{textblock}


\only<2->{
\begin{textblock}{140}(10,58)
\begin{equation*} 
\begin{split}
\omega(T) & = \prod^T_{t=1} f(\Aa(t)) = f(\text{\en{head}\es{cara}})^{n_1} f(\text{\en{tail}\es{sello}})^{n_2}  \approx r^T \\
\onslide<3->{\left( \lim_{T \rightarrow \infty} \omega_e(T) \right)^{1/T} & =  r}  \onslide<4->{= 1.5^{1/2} \cdot 0.6^{1/2}} \onslide<5>{ \approx 0.95 } 
\end{split}
\end{equation*}
\end{textblock}
}

\end{frame}


\begin{frame}[plain]
\begin{textblock}{160}(0,4)
 \centering \LARGE
Cooperación
\end{textblock}
\vspace{1cm}


\begin{figure}[H]
\centering
\scalebox{0.75}{
\tikz{

    \node[latent, minimum size=2cm ] (x1_0) {$\omega_1(t)$} ;
    \node[latent, below=of x1_0, minimum size=2cm ] (x2_0) {$\omega_2(t)$} ;

    \node[latent, right=of x1_0, minimum size=3cm ] (x1_0g) {$ \omega_1(t)\cdot f(\Aa_1(t))$} ;
    \node[latent, right=of x2_0, minimum size=1.8cm, xshift=0.6cm , align=left] (x2_0g) {$\omega_2(t)\cdot$\\$f(\Aa_2(t))$} ;
    
    \node[latent, right=of x1_0g, minimum size=3.8cm, yshift=-1.33cm, align=right] (x_0) {$\omega_1(t)\cdot f(\Aa_1(t))$\\$+\omega_2(t)\cdot f(\Aa_2(t))$ } ;
    
    \node[const, above=of x_0] (nx_0) {$\overbrace{\text{Pool}\hspace{2.5cm}\text{Share}}^{\text{\normalsize Coopera\en{tion}\es{ci\'on}}}$} ;
    
    \node[latent, right=of x1_0g, minimum size=2.5cm,  xshift=4.5cm] (x1_1) {$\omega_1(t+1)$ } ;
    \node[latent, below=of x1_1, minimum size=2.5cm, yshift=0.7cm] (x2_1) {$\omega_2(t+1)$ } ;
    
    \edge {x1_0} {x1_0g};
    \edge {x2_0} {x2_0g};
    \edge {x1_0g,x2_0g} {x_0};
    \edge {x_0} {x1_1,x2_1};
    
}
}
\end{figure}
\end{frame}

\begin{frame}[plain]
\begin{textblock}{160}(0,4)
 \centering \LARGE
 Cooperación
\end{textblock}
\vspace{1.3cm}

\centering


\only<1>{\includegraphics[width=0.6\linewidth]{figures/pdf/ergodicity_desertion0.pdf}}\only<2>{\includegraphics[width=0.6\linewidth]{figures/pdf/ergodicity_desertion1.pdf}}\only<3->{\includegraphics[width=0.6\linewidth]{figures/pdf/ergodicity_desertion.pdf}}
\onslide<4>{
\begin{equation*}\label{eq:posterior_multinivel}
\begin{split}
\underbrace{P(\texttt{coop}|\vec{\Aa}^{\,1:T})}_{\text{\large \en{Multilevel selection}\es{Selección multinivel}}} = \sum_{g} \ \ \underbrace{P(\texttt{coop}|\vec{\Aa}^{\,1:T}, g)}_{\text{\large \en{Individual selection}\es{Selección de individuos}}} \ \ \ \cdot \underbrace{P(g|\vec{\Aa}^{\,1:T})}_{\text{\large \en{Group selection}\es{Selección de grupo}}}
\end{split}
\end{equation*}
}

\end{frame}

\begin{frame}[plain]
\begin{textblock}{160}(0,4)
 \centering \LARGE
 Especialización
\end{textblock}
\vspace{1cm}

\begin{textblock}{150}(05,18)
\begin{equation*}
f(\Ee,\Aa) \propto \Ee^\Aa(1-\Ee)^\Aa \text{  \ \ \  con \ \ \  } \Ee \in [0,1]
\end{equation*}
\end{textblock}


\begin{textblock}{150}(05,30)
\begin{figure}[H]
    \centering
    \begin{subfigure}[b]{0.49\linewidth}
    \only<2>{\includegraphics[width=1\linewidth]{figures/pdf/tasa-temporal-0-a.pdf}}\only<3>{\includegraphics[width=1\linewidth]{figures/pdf/tasa-temporal-0-b1.pdf}}\only<4>{\includegraphics[width=1\linewidth]{figures/pdf/tasa-temporal-0-b.pdf}}
\only<5->{\includegraphics[width=1\linewidth]{figures/pdf/tasa-temporal-0.pdf}}
    \end{subfigure}
    \ 
    \begin{subfigure}[b]{0.49\linewidth}
    \only<1-5>{\phantom{\includegraphics[width=1\linewidth]{figures/pdf/tasa-temporal-2-a.pdf}}}\only<6>{\includegraphics[width=1\linewidth]{figures/pdf/tasa-temporal-2-a.pdf}}\only<7>{\includegraphics[width=1\linewidth]{figures/pdf/tasa-temporal-2-b.pdf}}\only<8>{\includegraphics[width=1\linewidth]{figures/pdf/tasa-temporal-2.pdf}}
    \end{subfigure}
    \label{fig:tasa-temporal-2}
\end{figure}
\end{textblock}
\end{frame}

\begin{frame}[plain]
\begin{textblock}{160}(0,4)
 \centering \LARGE
Crianza cooperativa
\end{textblock}
\vspace{1cm}

\includegraphics[width=1\textwidth]{../../auxiliar/static/evolucionCultural.jpg}

\end{frame}


\begin{frame}[plain]
\begin{textblock}{160}(0,4)
 \centering \LARGE
La transición cultural
\end{textblock}
\vspace{0.3cm}

\centering
\includegraphics[width=0.95\textwidth]{figures/agricultura.pdf} \ \ \ \ \ 

\end{frame}


\begin{frame}[plain]
 \begin{textblock}{160}(0,4)
  \LARGE \centering \textcolor{black!85}{Agricultura $\mapsto$ Población $\mapsto$ Centros de innovación }
 \end{textblock} 

\vspace{1cm}
\begin{textblock}{150}(5,16)
 \centering
 \includegraphics[width=0.95\textwidth]{../../auxiliar/static/terrazas_arroz_c}
\end{textblock}

\end{frame}

\begin{frame}[plain]
 \begin{textblock}{160}(0,4)
  \LARGE \centering \textcolor{black!85}{China}
 \end{textblock} 



\begin{textblock}{80}(0,12)
  \Large \centering \textcolor{black!85}{}
\end{textblock} 
\begin{textblock}{160}(0,60)
  \centering
\includegraphics[width=1\textwidth]{../../auxiliar/static/chineseRiverShips.jpg}  
  \end{textblock} 

\begin{textblock}{70}(10,10) \footnotesize
 $\bullet$ Seda ($\sim -1500$) \\
 $\bullet$ Puentes flotantes ($\sim -1100$) \\
 $\bullet$ Altos hornos de fundición ($\sim -750/-450$) \\
 $\bullet$ Molino de agua ($\sim -500$) \\
 $\bullet$ Canales artificiales navegación ($\sim -500$) \\
 $\bullet$ Arado de hierro ($\sim -500$) \\
 $\bullet$ Cámara de fotos ($\sim -450$) \\
 $\bullet$ Helicóptero de juegete ($\sim -400$) \\
 $\bullet$ Tinta ($\sim -250$) \\
 $\bullet$ Porcelena ($\sim -200$) \\
 %$\bullet$ Higrómetros ($\sim -200$) \\
 $\bullet$ Burocracia por concurso ($\sim 0$) \\
 %$\bullet$ Sismógrafo ($\sim 100$ ) \\
 $\bullet$ Refinamiento de petróleo ($\sim 100$ ) \\
 $\bullet$ Brújula ($\sim 100$ ) 
 \end{textblock} 


 
 \begin{textblock}{70}(90,10) \footnotesize
 $\bullet$ Fútbol ($\sim 200$) \\
 $\bullet$ Control biológico de pestes ($\sim 300$) \\
 $\bullet$ Pózos de petróleo ($\sim 350$ ) \\
 $\bullet$ Fósforos ($\sim 550$) \\
 $\bullet$ Papel higiénico ($\sim 600$) \\
 $\bullet$ Imprenta ($\sim 650$ ) \\
 $\bullet$ Amalmaga dental ($\sim 650$) \\
 $\bullet$ Papel moneda ($\sim 700$ ) \\
 $\bullet$ Relojería de escape ($\sim 700$) \\
 $\bullet$ Espejos ($\sim 800$) \\
 $\bullet$ Vacunas ($\sim 950$) \\
 $\bullet$ Pólvora ($\sim 1000$) \\
 $\bullet$ Cepillo de dientes ($\sim 1450$)
 \end{textblock} 
 
  
\end{frame}


\begin{frame}[plain]
\begin{textblock}{95}(0,18) \centering
\includegraphics[width=0.95\textwidth]{../../auxiliar/static/polynesia.png}  
\end{textblock} 

\begin{textblock}{60}(95,04.5) \centering
\includegraphics[width=0.95\textwidth]{../../auxiliar/static/tonga_barco.jpg}  
\end{textblock} 

\begin{textblock}{160}(0,4)
\centering \LARGE  \textcolor{black!85}{Polinesia - América}
\end{textblock}

\end{frame}

\begin{frame}[plain]
\begin{textblock}{160}(0,4)
\centering \LARGE  \textcolor{black!85}{Involución cultural por aislamiento} \\
\Large Tasmania
\end{textblock}

\begin{textblock}{160}(0,18)
  \centering
\includegraphics[width=0.6\textwidth]{../../auxiliar/static/output/tasmania.png}  
  \end{textblock} 

\end{frame}


\begin{frame}[plain]
\begin{textblock}{160}(0,4)
\centering \LARGE  \textcolor{black!85}{Involución cultural por aislamiento} \\
\Large Europa occidental

\end{textblock}

\only<1>{
\begin{textblock}{160}(0,22) \centering
\includegraphics[width=0.55\textwidth]{../../auxiliar/static/output/cisma.png}  
  \end{textblock} 
}

\end{frame}


\begin{frame}[plain]
\begin{textblock}{160}(0,4)
\centering \LARGE  \textcolor{black!85}{Europa occidental} \\ 
\Large Criterio de autoridad como fundamento del ``saber auténtico''
\end{textblock}
\vspace{2cm}

\centering

\includegraphics[width=0.225\linewidth]{../../auxiliar/static/digesto1553.jpg}
\hspace{0.8cm}
\includegraphics[width=0.267\linewidth]{../../auxiliar/static/malleus.jpeg}

\hspace{0.2cm} Libris terribilis ($\sim 1000$) \hspace{0.2cm} Malleus maleficarum ($\sim 1480$)

\end{frame}


\begin{frame}[plain]
\begin{textblock}{160}(0,4)
\centering \LARGE  \textcolor{black!85}{Migración y Pandemia} \\ 
\end{textblock}
\vspace{2cm}

\centering

\only<1>{
\begin{textblock}{160}(0,14)
Universalis Cosmographia, 1507.

\includegraphics[width=0.74\linewidth]{../../auxiliar/static/mapaWaldseemuller.jpg}
\end{textblock}
}

\only<2>{
\begin{textblock}{160}(0,14)
Uso de la tierra 

\vspace{0.4cm}

\includegraphics[width=0.65\linewidth]{../../auxiliar/static/americaLandUse.png}

\hspace{1cm} 1500 \hspace{4cm} 1600
%Doi $10.1177/0959683610386983$
\end{textblock}
}
\end{frame}

\begin{frame}[plain]
\begin{textblock}{160}(0,4)
  \LARGE \centering \textcolor{black!85}{La plata, moneda oficial China} \\
  \Large Potosí 1546
  
 \end{textblock} 
\vspace{1.5cm}

\centering


\includegraphics[width=0.9\textwidth]{../../auxiliar/static/plata-potosi.jpg}  
\end{frame}

\begin{frame}[plain]
\begin{textblock}{160}(0,4)
  \LARGE \centering Ruptura del cerco medieval \\
  \Large Batalla de Lepanto 1571  
\end{textblock} 
\vspace{1.5cm}
\centering
\includegraphics[width=0.66\textwidth]{../../auxiliar/static/lepanto.png}  
\end{frame}


\begin{frame}[plain]
\begin{textblock}{160}(0,4)
\centering \LARGE Criterio de universalidad colonial-moderno \\
%\Large La revolución científica (1550 - )
\end{textblock}
\vspace{1.5cm}
\centering

Obra de tal modo que la máxima de tu voluntad pueda valer\\
siempre como principio de una legislación universal (Kant)

\vspace{1.6cm}

\onslide<2>{
Universalidad limitada a los hombres blancos:

\vspace{0.2cm}

Es justo que los varones virtuosos y humanos dominen sobre todos \\
los que no tienen estas cualidades (Ginés de Sepúlveda)
}

\end{frame}



\begin{frame}[plain]
\begin{textblock}{160}(0,4)
  \LARGE \centering \textcolor{black!85}{La guerra contra el narcotráfico}
 \end{textblock} 
\vspace{1cm}

\centering

\begin{textblock}{45}(05,18)
\includegraphics[width=1\textwidth]{../../auxiliar/static/opium-war.jpg}  
\end{textblock} 

\begin{textblock}{105}(50,21)
\includegraphics[width=0.95\textwidth]{figures/china.pdf}
\end{textblock} 

\end{frame}

\begin{frame}[plain]
\begin{textblock}{160}(0,4)
  \LARGE \centering \textcolor{black!85}{La era eurocéntrica (1850 - )}
 \end{textblock} 

\vspace{1cm}
 \centering
\includegraphics[width=0.8\textwidth]{../../auxiliar/static/mapaMercator.jpg}
\end{frame} 


\begin{frame}[plain]
\begin{textblock}{160}(0,4)
  \LARGE \centering \textcolor{black!85}{Era de los genocidios y pérdida cultural}
 \end{textblock} 

\vspace{1cm}
 \centering
\includegraphics[width=1\textwidth]{../../auxiliar/static/genocidio_patagonia.jpg}

\begin{textblock}{160}(0,70)
  \centering \textcolor{black!15}{\textbf{Patagonia $\sim$ 1880}}
 \end{textblock} 


\end{frame} 


\begin{frame}[plain]
\begin{textblock}{160}(0,4)
  \LARGE \centering \textcolor{black!85}{La crisis ecológica actual}
 \end{textblock} 

\vspace{1cm}
 \centering
\includegraphics[width=1\textwidth]{../../auxiliar/static/deforestation-brazil.jpg}
\end{frame} 


\begin{frame}[plain]
\begin{textblock}{160}(0,4)
  \LARGE \centering \textcolor{black!85}{Tecnologías de reciprocidad} \\
  \Large Evolución de instituciones que reducen las fluctuaciones ecológicas
 \end{textblock} % 
\vspace{1.1cm}

\centering
 \includegraphics[width=0.593\textwidth]{../../auxiliar/static/bali-offerings.jpg} 
 \includegraphics[width=0.397\textwidth]{../../auxiliar/static/pachamama.jpg} 
 
\end{frame}


\begin{frame}[plain]
\begin{textblock}{160}(0,4)
\LARGE \centering \textcolor{black!85}{Principio de reciprocidad} \\
\Large El problema que da inicio a la teoría de la probabilidad
\end{textblock} 
\vspace{2cm} \centering

Pascal-Fermat (1650)

\vspace{0.3cm}

Tiramos dos veces la moneda: \\ \justify 
$\bullet$ Si sale seca en la primera y en la segunda, vos me das a mi. \\
$\bullet$ Caso contrario, yo te doy a vos. \\

\centering

\tikz{
\node[latent, draw=white, yshift=0.7cm, minimum size=0.1cm] (b0) {};
\node[latent,below=of b0,yshift=0.7cm, xshift=-1cm] (r1) {$S$};
\node[latent,below=of b0,yshift=0.7cm, xshift=1cm] (r2) {$C$};

\node[latent, below=of r1, draw=white, yshift=0.8cm, minimum size=0.1cm] (bc11) {};
\node[accion, below=of r2, draw=white, yshift=0cm] (bc12) {};
\node[latent,below=of bc11,yshift=0.8cm, xshift=-0.5cm] (r1d2) {$S$};
\node[latent,below=of bc11,yshift=0.8cm, xshift=0.5cm] (r1d3) {$C$};

\node[accion,below=of r1d2,yshift=0cm, color=red] (br1d2) {};
\node[accion,below=of r1d3,yshift=0cm] (br1d3) {};
\edge[-] {b0} {r1,r2};
\edge[-] {r1} {bc11};
\edge[-] {r2} {bc12};
\edge[-] {bc11} {r1d2,r1d3};
\edge[-] {r1d2} {br1d2};
\edge[-] {r1d3} {br1d3};
}



\vspace{0.3cm}

\onslide<2>{
\Wider[2cm]{
\centering
\Large ¿Cuál es el valor justo de la reciprocidad en contexto de incertidumbre?
}
}

\end{frame}



\begin{frame}[plain,noframenumbering]
\begin{textblock}{160}(0,4)
\LARGE \centering Clase que viene
\end{textblock} 
\vspace{1cm} \centering


\vspace{0.3cm}

\Large Los principios de la inferencia Bayesiana \\ \justify \large
$\bullet$ Principio de razón suficiente \\
$\bullet$ Principio de indiferencia \\
$\bullet$ Principio de reciprocidad \\
$\bullet$ Principio de coherencia \\
$\bullet$ Axiomas de Kologorov \\
$\bullet$ Las reglas de la probabilidad \\
$\bullet$ Evaluación observacional de modelos causales alternativos \\
\end{frame}
 
 
\begin{frame}[plain]
\centering
  \includegraphics[width=0.35\textwidth]{../../auxiliar/static/pachacuteckoricancha.jpg}
\end{frame}






\end{document}



