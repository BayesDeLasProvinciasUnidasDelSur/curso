\documentclass[shownotes,aspectratio=169]{beamer}

\input{../../aux/tex/diapo_encabezado.tex}
\input{../../aux/tex/tikzlibrarybayesnet.code.tex}
 \mode<presentation>
 {
 %   \usetheme{Madrid}      % or try Darmstadt, Madrid, Warsaw, ...
 %   \usecolortheme{default} % or try albatross, beaver, crane, ...
 %   \usefonttheme{serif}  % or try serif, structurebold, ...
  \usetheme{Antibes}
  \setbeamertemplate{navigation symbols}{}
 }
 
\usepackage{todonotes}
\setbeameroption{show notes}

\newif\ifen
\newif\ifes
\newcommand{\en}[1]{\ifen#1\fi}
\newcommand{\es}[1]{\ifes#1\fi}
\estrue

%\title[Bayes del Sur]{}

\begin{document}

\color{black!85}
\large

 
%\setbeamercolor{background canvas}{bg=gray!15}

\begin{frame}[plain,noframenumbering]
 
 \begin{textblock}{90}(03,05)
 \centering \huge  \textcolor{black!40}{Creencias, datos y sorpresas}
\end{textblock}

 \begin{textblock}{47}(113,74)
\centering \Large  \textcolor{white!55}{Ergodicidad}
\end{textblock}

 %\vspace{2cm}brown
%\maketitle
\Wider[2cm]{
\includegraphics[width=1\textwidth]{../../aux/static/peligro_predador}
}
\end{frame}




\begin{frame}[plain]
 \begin{textblock}{160}(0,4)
  \centering \LARGE La moneda
 \end{textblock}

\vspace{1cm}

\includegraphics[width=1\textwidth]{../../aux/static/plata-potosi}
 
\end{frame}


\begin{frame}[plain]
 The Unfinished game: Pascal, Fermat and the letters

 Antes de que la teoría de la probabilidad hubiera sido desarrollada, de que la palabra probabilidad estuviera en el volcabulario, 

Supongamos que dos jugadores hacen apuestas iguales sobre quién ganará una serie de tres lanzamientos de una moneda justa.
Cada uno tira su moneda.
El que obtiene más caras gana.
Antes de terminar se ven obligados a escapar. 
Las apuestas son juegos clandestinos.
¿Cuál es la forma justa de repartir las apuestas?
Esta es la pregunta que buscan resolver Pascal y Fermat.
 
\end{frame}

\begin{frame}[plain]
 \begin{textblock}{160}(0,4)
  \centering \LARGE La moneda
 \end{textblock}

\vspace{1cm}

%\includegraphics[width=1\textwidth]{../../aux/static/triangleYanghui.gif}

\end{frame}


\begin{frame}[plain]
 \begin{textblock}{160}(0,4)
  \centering \Large Contar los caminos de la moneda
 \end{textblock}
 
 Combinatoria
 
\end{frame}


\begin{frame}[plain]
 \begin{textblock}{160}(0,4)
  \centering \Large Binomial
 \end{textblock}
 
 
\end{frame}




\begin{frame}[plain]
 \begin{textblock}{160}(0,4)
  \centering \Large Los orígenes de la economía
 \end{textblock}
 \vspace{1cm}
 
 \begin{align*}
 \Delta \,r = 
      \begin{cases*}
       (+0.5)\cdot r & \ \ \text{Cara} \\
       (-0.4)\cdot r & \ \ \text{Seca}
    \end{cases*}
 \end{align*}

 \vspace{0.5cm}
 
 \centering
 
 >Apostar o no apostar?
 
\end{frame}

\begin{frame}[plain]
 \begin{textblock}{160}(0,4)
  \centering \Large The rule of thumb
 \end{textblock}
 
 \begin{align*}
 \text{Decisión} = 
 \begin{cases*}
  \text{Tomar puesta} & E[\Delta r] > 0 \\ 
  \text{Rechazar apuesta} & E[\Delta r] < 0 \\
 \end{cases*}
\end{align*}

\begin{align*}
  E[\Delta r] = 0.5 \cdot 1/2 + (-0.4) \cdot 1/2
\end{align*}

\end{frame}


\begin{frame}[plain]
 \begin{textblock}{160}(0,4)
  \centering \Large 
 \end{textblock}

 Phil. Trans. R. Soc. A(2011)369, 4913–4931doi:10.1098/rsta.2011.0065
 
 The time resolution of the St Petersburg paradox
 
 \end{frame}

\begin{frame}[plain]
\centering
  \includegraphics[width=0.35\textwidth]{../../aux/static/pachacuteckoricancha.jpg}
\end{frame}






\end{document}



