\documentclass[shownotes,aspectratio=169]{beamer}
\input{../../aux/tex/diapo_encabezado.tex}
\input{../../aux/tex/tikzlibrarybayesnet.code.tex}
 \mode<presentation>
 {
 %   \usetheme{Madrid}      % or try Darmstadt, Madrid, Warsaw, ...
 %   \usecolortheme{default} % or try albatross, beaver, crane, ...
 %   \usefonttheme{serif}  % or try serif, structurebold, ...
  \usetheme{Antibes}
  \setbeamertemplate{navigation symbols}{}
 }
 
\usepackage{todonotes}
\setbeameroption{show notes}

\newif\ifen
\newif\ifes
\newcommand{\en}[1]{\ifen#1\fi}
\newcommand{\es}[1]{\ifes#1\fi}
\estrue

%\title[Bayes del Sur]{}

\begin{document}

 \color{black!85}

%\setbeamercolor{background canvas}{bg=gray!15}

\begin{frame}[plain,noframenumbering]
 
 \begin{textblock}{90}(00,05)
\begin{center}
 \huge  \textcolor{black!66}{Creencias, datos y sorpresas}
\end{center}
\end{textblock}

 %\vspace{2cm}brown
%\maketitle
\Wider[2cm]{
\includegraphics[width=1\textwidth]{../../aux/static/peligro_predador}
}
\end{frame}


\begin{frame}[plain]
\begin{textblock}{160}(0,4)
 \centering
 \LARGE \textcolor{black!85}{\en{Science}\es{Ciencia}}
\end{textblock}
\vspace{0.75cm}
\Large

 \begin{center}
 \en{The goal of science are \\ \textbf{intersubjective agreements}}
 \es{El objetivo de la ciencia son \\ \textbf{acuerdos intersubjectivos}}
 \end{center}

\vspace{0.5cm}
 
 \begin{center}
\en{Disagreements in social sciences \\ are mostly semantic}
\es{Los mayoría de los desacuerdos en las \\ ciencias sociales son semánticos}
 \end{center}

\end{frame}


\begin{frame}[plain]
 \begin{textblock}{160}(0,4)
 \centering \LARGE
 \en{Data as functions}
 \es{Los datos como funciones}
\end{textblock}
\vspace{0.75cm}

\begin{textblock}{160}(0,20)
\begin{equation*}
 f(x) = y
\end{equation*}
\end{textblock}

\begin{textblock}{160}(54,33)
\begin{itemize}
 \item[$x$] 
    \textbf{\en{Unit of analysis}\es{Unidad de análisis}}
 \item[$f$] 
   \en{\textbf{Variable} of the unit of analysis}
   \es{\textbf{Variable} de la unidad de análisis}
 \item[$y$] 
   \en{\textbf{Value} of the variable}
   \es{\textbf{Valor} de la variable}
\end{itemize}
\end{textblock}

\only<2>{
\begin{textblock}{130}(15,60)
\begin{mdframed}
 $\bullet$ 
   \en{It is a function because each $f(x)$ is associated with a unique $y$}
   \es{Es una función porque cada $f(x)$ está asociado a un \'unico $y$}
   \\[0.1cm]
 $\bullet$ 
   \en{To understand each other, it is necessary to make this functions explicit}
   \es{Para entendernos es necesario hacer explícita esa función}
 \end{mdframed}
\end{textblock}
}
\end{frame}

\begin{frame}[plain]
 \begin{textblock}{160}(0,4)
 \centering \LARGE 
 \en{Empirical basis and theoretical data}
 \es{Base empírica y datos teóricos}
\end{textblock}
\vspace{1.5cm}

\centering
\includegraphics[width=0.45\textwidth]{figures/baseEmpirica.pdf}

\vspace{0.25cm}

\flushleft
\en{The set of hypotheses we assume define an empirical basis}
\es{El conjunto de hipótesis que dejamos fuera de toda duda definen la base empírica}
\\[0.2cm]

\todo[inline]{Presentar por pasos}

\Wider[-2cm]{
\begin{itemize}
 \en{\item[\textbf{FEB}:] Filosofical Empirical Basis (Empty set)}
 \es{\item[\textbf{BEF}:] Base empírica filosófica (Conjunto vacío)}
 %
 \en{\item[\textbf{EEB}:] Epistemological Empirical Basis (Common sence)}
 \es{\item[\textbf{BEE}:] Base empírica epistemológica (Sentido común)}
 %
 \en{\item[\textbf{MEB}:] Methodological Empirical Basis (Assumed estimates)}
 \es{\item[\textbf{BEM}:] Base empírica metodológica (Estimaciones fuera de toda duda)}
 %
 \en{\item[\textbf{TD}:] Theoretical Data (Estimators under development)}
 \es{\item[\textbf{DT}:] Datos teóricos (Estimadores en desarrollo)}
\end{itemize}
}

\end{frame}

\begin{frame}[plain]
\begin{textblock}{160}(0,4)
\centering  \LARGE 
 \en{How to measure skills?}
 \es{¿Cómo medir habilidades?}
 \end{textblock}
\vspace{1cm}

\centering
 
 \includegraphics[width=0.35\textwidth]{../../aux/static/elo}
  
 
\end{frame}



\begin{frame}[plain]
 \begin{textblock}{160}(0,4)
\centering  \LARGE 
\en{Hidden variable, data and causal model}
\es{Variable oculta, datos y modelo causal}
 \end{textblock}
\vspace{1cm}

\centering
 
\tikz{         
    \node[det, fill=black!15] (r) {$r_{ij}$} ; 
    \node[const, left=of r, xshift=-1.35cm] (r_name) {\small \en{Result}\es{Resultado}:}; 
    \node[const, right=of r] (dr) {\normalsize $ r_{ij} = (d_{ij}>0)$}; 

    \node[latent, above=of r, yshift=-0.45cm] (d) {$d_{ij}$} ; %
    \node[const, right=of d] (dd) {\normalsize $ d_{ij}=p_i-p_j$}; 
    \node[const, left=of d, xshift=-1.35cm] (d_name) {\small \en{Difference}\es{Diferencia}:};
    
    \node[latent, above=of d, xshift=-0.8cm, yshift=-0.45cm] (p1) {$p_i$} ; %
    \node[latent, above=of d, xshift=0.8cm, yshift=-0.45cm] (p2) {$p_j$} ; %
    \node[const, left=of p1, xshift=-0.55cm] (p_name) {\small \en{Performance}\es{Rendimiento}:}; 

    \node[accion, above=of p1,yshift=0.3cm,fill=white] (s1) {} ; %
    \node[const, right=of s1] (ds1) {$s_i$};
    \node[accion, above=of p2,yshift=0.3cm,fill=white] (s2) {} ; %
    \node[const, right=of s2] (ds2) {$s_j$};
    
    \node[const, right=of p2] (dp2) {\normalsize $p \sim \N(s,\beta^2)$};

    \node[const, left=of s1, xshift=-.85cm] (s_name) {\small \en{Skill}\es{Habilidad}:}; 
    
    \edge {d} {r};
    \edge {p1,p2} {d};
    \edge {s1} {p1};
    \edge {s2} {p2};
}

 
\end{frame}

\begin{frame}[plain]
\only<1-3>{
 \begin{textblock}{160}(0,4)
 \centering \LARGE 
 \en{Hidden variables and beliefs}
 \es{Variables ocultas y creencias}
 \end{textblock}
}

\only<4->{
\begin{textblock}{160}(0,4)
 \centering \LARGE 
 \en{Hidden variables and \textbf{honest} beliefs}
 \es{Variables ocultas y creencias \textbf{honestas}}
 \end{textblock}

}

\vspace{1cm}


\only<1>{
 \begin{textblock}{160}(0,30)\centering
 \includegraphics[width=0.55\textwidth]{figures/montyHall_0}     
 \end{textblock}
}

%reason for one outcome to occur more often than any other, then the events are assigned equal probabilities. This is called the principle of insufficient reason, or principle of indifference, and goes back to Laplace.

\only<2>{
 \begin{textblock}{160}(0,30)\centering
 \includegraphics[width=0.55\textwidth]{figures/montyHall_1}     
 \end{textblock}
}

\only<3-4>{
 \begin{textblock}{160}(0,30)\centering
 \includegraphics[width=0.55\textwidth]{figures/montyHall_2}     
 \end{textblock}
}

\only<5>{
 \begin{textblock}{160}(0,30)\centering
 \includegraphics[width=0.55\textwidth]{figures/montyHall_4}     
 \end{textblock}
}
\only<6>{
 \begin{textblock}{160}(0,30)\centering
 \includegraphics[width=0.55\textwidth]{figures/montyHall_5}     
 \end{textblock}
}
 
\end{frame}



\begin{frame}[plain]
\begin{textblock}{160}(0,4)
 \centering \LARGE 
 \en{Honesty maximizes entropy}
 \es{La honestidad maximiza la entropía}
 \end{textblock}

\en{Entropy (or expected information)}
\es{Entropía (o informaci\'on esperada)}
\begin{equation*}
 H(X) = \sum_{x\in X} P(x) \  \cdot \underbrace{(-\log P(x))}_{\hfrac{\text{\scriptsize \en{Information extractable}\es{Información extraíble}}}{\text{\scriptsize \en{from the surprise}\es{de la sorpresa}}}}
\end{equation*}

\pause

\vspace{0.3cm}

\Wider[-1.5cm]{
\begin{mdframed}[backgroundcolor=black!15]
\centering
 \en{Maximum expected information $\Leftrightarrow$ Maximum uncertainty}
 \es{M\'axima información esperada $\Leftrightarrow$ Máxima incertidumbre}
 \end{mdframed}


}
 
\end{frame}


\begin{frame}[plain]
\begin{textblock}{160}(0,4)
 \centering \LARGE 
 \en{Principle of maximum uncertainty: counting paths}
 \es{Principio de máxima incertidumbre: contar caminos}
 \end{textblock}
\vspace{1.25cm} 

 \begin{equation*}
  \text{\en{Belief}\es{Creencia}}(x) = \frac{\text{\en{Number of paths on which $x$ becomes true}\es{Cantidad de caminos en los que $x$ se hace verdadera}}}{\text{\en{Total number of paths}\es{Cantidad de caminos totales}}}
 \end{equation*}
 
\end{frame}

\begin{frame}[plain]
\begin{textblock}{160}(0,4)
 \centering \LARGE 
 \en{Example: Monty Hall}
 \es{Ejemplo: Monty Hall}
 \end{textblock}
 
 
 
 
 
\end{frame}

 
\begin{frame}[plain]
\centering
  \includegraphics[width=0.35\textwidth]{../../aux/static/pachacuteckoricancha.jpg}
\end{frame}






\end{document}



