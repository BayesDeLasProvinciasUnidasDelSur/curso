\documentclass[shownotes,aspectratio=169]{beamer}

\input{../../aux/tex/diapo_encabezado.tex}
% tikzlibrary.code.tex
%
% Copyright 2010-2011 by Laura Dietz
% Copyright 2012 by Jaakko Luttinen
%
% This file may be distributed and/or modified
%
% 1. under the LaTeX Project Public License and/or
% 2. under the GNU General Public License.
%
% See the files LICENSE_LPPL and LICENSE_GPL for more details.

% Load other libraries
\usetikzlibrary{shapes}
\usetikzlibrary{fit}
\usetikzlibrary{chains}
\usetikzlibrary{arrows}

% Latent node
\tikzstyle{latent} = [circle,fill=white,draw=black,inner sep=1pt,
minimum size=20pt, font=\fontsize{10}{10}\selectfont, node distance=1]
% Observed node
\tikzstyle{obs} = [latent,fill=gray!25]
% Invisible node
\tikzstyle{invisible} = [latent,minimum size=0pt,color=white, opacity=0, node distance=0]
% Constant node
\tikzstyle{const} = [rectangle, inner sep=0pt, node distance=0.1]
%state
\tikzstyle{estado} = [latent,minimum size=8pt,node distance=0.4]
%action
\tikzstyle{accion} =[latent,circle,minimum size=5pt,fill=black,node distance=0.4]


% Factor node
\tikzstyle{factor} = [rectangle, fill=black,minimum size=10pt, draw=black, inner
sep=0pt, node distance=1]
% Deterministic node
\tikzstyle{det} = [latent, rectangle]

% Plate node
\tikzstyle{plate} = [draw, rectangle, rounded corners, fit=#1]
% Invisible wrapper node
\tikzstyle{wrap} = [inner sep=0pt, fit=#1]
% Gate
\tikzstyle{gate} = [draw, rectangle, dashed, fit=#1]

% Caption node
\tikzstyle{caption} = [font=\footnotesize, node distance=0] %
\tikzstyle{plate caption} = [caption, node distance=0, inner sep=0pt,
below left=5pt and 0pt of #1.south east] %
\tikzstyle{factor caption} = [caption] %
\tikzstyle{every label} += [caption] %

\tikzset{>={triangle 45}}

%\pgfdeclarelayer{b}
%\pgfdeclarelayer{f}
%\pgfsetlayers{b,main,f}

% \factoredge [options] {inputs} {factors} {outputs}
\newcommand{\factoredge}[4][]{ %
  % Connect all nodes #2 to all nodes #4 via all factors #3.
  \foreach \f in {#3} { %
    \foreach \x in {#2} { %
      \path (\x) edge[-,#1] (\f) ; %
      %\draw[-,#1] (\x) edge[-] (\f) ; %
    } ;
    \foreach \y in {#4} { %
      \path (\f) edge[->,#1] (\y) ; %
      %\draw[->,#1] (\f) -- (\y) ; %
    } ;
  } ;
}

% \edge [options] {inputs} {outputs}
\newcommand{\edge}[3][]{ %
  % Connect all nodes #2 to all nodes #3.
  \foreach \x in {#2} { %
    \foreach \y in {#3} { %
      \path (\x) edge [->,#1] (\y) ;%
      %\draw[->,#1] (\x) -- (\y) ;%
    } ;
  } ;
}

% \factor [options] {name} {caption} {inputs} {outputs}
\newcommand{\factor}[5][]{ %
  % Draw the factor node. Use alias to allow empty names.
  \node[factor, label={[name=#2-caption]#3}, name=#2, #1,
  alias=#2-alias] {} ; %
  % Connect all inputs to outputs via this factor
  \factoredge {#4} {#2-alias} {#5} ; %
}

% \plate [options] {name} {fitlist} {caption}
\newcommand{\plate}[4][]{ %
  \node[wrap=#3] (#2-wrap) {}; %
  \node[plate caption=#2-wrap] (#2-caption) {#4}; %
  \node[plate=(#2-wrap)(#2-caption), #1] (#2) {}; %
}

% \gate [options] {name} {fitlist} {inputs}
\newcommand{\gate}[4][]{ %
  \node[gate=#3, name=#2, #1, alias=#2-alias] {}; %
  \foreach \x in {#4} { %
    \draw [-*,thick] (\x) -- (#2-alias); %
  } ;%
}

% \vgate {name} {fitlist-left} {caption-left} {fitlist-right}
% {caption-right} {inputs}
\newcommand{\vgate}[6]{ %
  % Wrap the left and right parts
  \node[wrap=#2] (#1-left) {}; %
  \node[wrap=#4] (#1-right) {}; %
  % Draw the gate
  \node[gate=(#1-left)(#1-right)] (#1) {}; %
  % Add captions
  \node[caption, below left=of #1.north ] (#1-left-caption)
  {#3}; %
  \node[caption, below right=of #1.north ] (#1-right-caption)
  {#5}; %
  % Draw middle separation
  \draw [-, dashed] (#1.north) -- (#1.south); %
  % Draw inputs
  \foreach \x in {#6} { %
    \draw [-*,thick] (\x) -- (#1); %
  } ;%
}

% \hgate {name} {fitlist-top} {caption-top} {fitlist-bottom}
% {caption-bottom} {inputs}
\newcommand{\hgate}[6]{ %
  % Wrap the left and right parts
  \node[wrap=#2] (#1-top) {}; %
  \node[wrap=#4] (#1-bottom) {}; %
  % Draw the gate
  \node[gate=(#1-top)(#1-bottom)] (#1) {}; %
  % Add captions
  \node[caption, above right=of #1.west ] (#1-top-caption)
  {#3}; %
  \node[caption, below right=of #1.west ] (#1-bottom-caption)
  {#5}; %
  % Draw middle separation
  \draw [-, dashed] (#1.west) -- (#1.east); %
  % Draw inputs
  \foreach \x in {#6} { %
    \draw [-*,thick] (\x) -- (#1); %
  } ;%
}


 \mode<presentation>
 {
 %   \usetheme{Madrid}      % or try Darmstadt, Madrid, Warsaw, ...
 %   \usecolortheme{default} % or try albatross, beaver, crane, ...
 %   \usefonttheme{serif}  % or try serif, structurebold, ...
  \usetheme{Antibes}
  \setbeamertemplate{navigation symbols}{}
 }
 
\usepackage{todonotes}
\setbeameroption{show notes}

\newif\ifen
\newif\ifes
\newcommand{\en}[1]{\ifen#1\fi}
\newcommand{\es}[1]{\ifes#1\fi}
\estrue

%\title[Bayes del Sur]{}

\begin{document}

\color{black!85}
\large

 
%\setbeamercolor{background canvas}{bg=gray!15}

\begin{frame}[plain,noframenumbering]
 
 \begin{textblock}{90}(00,05)
\begin{center}
 \huge  \textcolor{black!66}{Creencias, datos y sorpresas}
\end{center}
\end{textblock}

 %\vspace{2cm}brown
%\maketitle
\Wider[2cm]{
\includegraphics[width=1\textwidth]{../../aux/static/peligro_predador}
}
\end{frame}


\begin{frame}[plain]
\begin{textblock}{160}(0,4)
 \centering
 \LARGE \textcolor{black!85}{\en{Today}\es{Hoy}}
\end{textblock}

\begin{itemize}
 \item[$\bullet$] Intersubjectividad
 \item[$\bullet$] Base empírica 
 \item[$\bullet$] Matriz de datos 
 \item[$\bullet$] Ciencia empírica 
 \item[$\bullet$] Niveles de conocimiento
 \item[$\bullet$] Incertidumbre
 \item[$\bullet$] Creencias honestas
 \item[$\bullet$] Modelos causales
 \item[$\bullet$] La lógica de la ciencia empírica
 \item[$\bullet$] Selección de modelo
\end{itemize}


\end{frame}

\begin{frame}[plain]
\begin{textblock}{160}(0,4)
 \centering
 \LARGE \textcolor{black!85}{\en{Science as intersubjectivity}\es{Ciencia como intersubjectividad}}
\end{textblock}
\vspace{0.75cm}

 
\begin{center}
conoce de tal manera que puedas \emph{poner en}

\emph{correspondencia unívoca los fenómenos percibidos por}

\emph{tu conciencia mediante algún \textbf{esquema de operación}}

\emph{\textbf{que sea públicamente inteligible y reproducible}}

\vspace{0.2cm}
{\small \hspace{5cm} Juan Samaja. Metodología y Epistemología.}
\end{center}


%  \begin{center}
% \en{The ambiguity of natural language \\ prevents building agreements}
% \es{La ambiguedad del lenguaje natural \\ impide construir acuerdos}
%  \end{center}

\end{frame}

\begin{frame}[plain] 
 \begin{textblock}{160}(0,4)
 \centering \LARGE
 \en{Empirical Basis}
 \es{Base empírica}
\end{textblock}
\vspace{1cm}


\centering 
\en{We call \emph{empirical basis} the portion of the world that a certain community momentarily leaves out of all doubt}
\es{Llamamos \emph{base empírica} a la porción del mundo que \\ cierta comunidad deja momentaneamente fuera de toda duda}

\vspace{0.1cm}
{\hspace{4cm} \small Gregorio Klimovsky. Desventuras del conocimiento científico.}

\vspace{1.3cm}

%Para aceptar la entidad de nuestros objetos más cotidianos, el árbol, la mesa, el libro son suficiente los supuestos del sentido común. No hay mayor controversia al respecto. 
\Wider[-2cm]{\normalsize
\begin{itemize}
 % Sin embargo ciertos círculos filosóficos los ponen en duda y pretenden justificar incluso la existencia misma de la realidad externa.
 \onslide<2->{
 \en{\item[\textbf{FEB}:] Filosofical Empirical Basis (Empty set or primary perceptions)}
 \es{\item[\textbf{BEF}:] Base empírica filosófica (Conjunto vacío o percepciones primarias)}
 }
 %Cuando relajamos la exigencia y aceptamos los objetos que nos son cotidianamente evidentes, obtenemos una base empírica significativamente más amplia que Klimovsky llama base empírica epistemológica, pues los epistemólogos admitirían lo que cualquier persona tomaría por cierto en su vida cotidiana y pondrían en duda al conjunto de la teorías científicas.
 \onslide<3->{
 \en{\item[\textbf{EEB}:] Epistemological Empirical Basis (Common sence)}
 \es{\item[\textbf{BEE}:] Base empírica epistemológica (Sentido común)}
 }
 % Los científicos por su parte trabajan constantemente con objetos que obtienen mediante teorías o técnicas especiales. los llama base empírica metodológica (BEM), en alusión a la metodología de trabajo normal de la actividad científica.
 \onslide<4->{
 \en{\item[\textbf{MEB}:] Methodological Empirical Basis (Constructed data)}
 \es{\item[\textbf{BEM}:] Base empírica metodológica (Datos construidos)}
 }
\end{itemize}
}

\end{frame}

\begin{frame}[plain]
 \begin{textblock}{160}(0,4)
 \centering \LARGE 
 \en{Levels of empírical basis}%\en{Empirical basis and theoretical data}
 \es{Niveles de base empírica}%\es{Base empírica y datos teóricos}
\end{textblock}
\vspace{1.2cm}

\centering

La base empírica puede ampliarse o reducirse 

aumentando o disminuyendo el conjunto de supuestos 

que una comunidad esta dispuesta a no poner en duda

\vspace{0.5cm}

\en{\includegraphics[page=1,width=0.45\textwidth]{figures/baseEmpirica.pdf}}
\es{\includegraphics[page=2,width=0.45\textwidth]{figures/baseEmpirica.pdf}}

\vspace{0.5cm}


\Wider[-2cm]{\normalsize
\begin{itemize}
 \onslide<2->{
 \es{\item[\textbf{BEE}:] Precios en los negocios}
 }
 \onslide<3->{
 \es{\item[\textbf{BEM}$_1$:] Inflación}
 }
 \onslide<4->{
 \es{\item[\textbf{BEM}$_2$:] Producto Bruto Interno}
 }
 \onslide<5->{
 \es{\item[\textbf{DT}:] Bienestar del pueblo}
 }
\end{itemize}
}



\end{frame}


\begin{frame}[plain]
 \begin{textblock}{160}(0,4)
 \centering \LARGE
 \en{Data as functions}
 \es{Los datos como funciones}
\end{textblock}
\vspace{0.75cm}

\begin{textblock}{160}(0,20)
\begin{equation*}
 f(x) = y
\end{equation*}
\end{textblock}

\begin{textblock}{160}(54,33)
\begin{itemize}
 \item[$x$] 
    \textbf{\en{Unit of analysis}\es{Unidad de análisis}}
 \item[$f$] 
   \en{\textbf{Variable} of the unit of analysis}
   \es{\textbf{Variable} de la unidad de análisis}
 \item[$y$] 
   \en{\textbf{Value} of the variable}
   \es{\textbf{Valor} de la variable}
\end{itemize}
\end{textblock}


\only<2>{
\begin{textblock}{160}(0,65) \centering
 \emph{Altura}(Gustavo) = $1.78$m
\end{textblock}
}


\only<3>{
\begin{textblock}{160}(0,65) \centering
 \emph{Inflación}(Argentina) = $2.3\%$
\end{textblock}
}

\only<4>{
\begin{textblock}{160}(0,65) \centering
 \emph{Ideología}(Partido Obrero) = Izquierda
\end{textblock}
}

\only<5>{
\begin{textblock}{160}(0,65) \centering
 \emph{Habilidad}(Maradona) $>$ \emph{Habilidad}(Messi)
\end{textblock}
}

\only<6>{
\begin{textblock}{70}(45,60)
\begin{framed} \centering
   \en{\textbf{The data is a function!}}
   \es{\textbf{Los datos son funciones!}}
 \end{framed}
\end{textblock}
}
\end{frame}

\begin{frame}[plain]
 \begin{textblock}{160}(0,4)
 \centering \LARGE
 \en{Esquema indicador}
 \es{Esquema indicador}
\end{textblock}
\vspace{0.75cm}

\begin{framed} \centering
   \en{The meaning of data is implicit in their \textbf{operationalization}}
   \es{El significado preciso del dato está implícito en su \textbf{operacionalización}}
 \end{framed}

\end{frame}



\begin{frame}[plain]
\begin{textblock}{160}(0,4)
\centering  \LARGE 
 \en{How to measure skills?}
 \es{¿Cómo medir habilidades?}
 \end{textblock}
\vspace{1cm}

\centering
 
 \includegraphics[width=0.35\textwidth]{../../aux/static/elo}
  
 
\end{frame}

\begin{frame}[plain]
\only<1-3>{
 \begin{textblock}{160}(0,4)
 \centering \LARGE 
 \en{Hidden variables and beliefs}
 \es{Variables ocultas y creencias}
 \end{textblock}
}

\only<4->{
\begin{textblock}{160}(0,4)
 \centering \LARGE 
 \en{Hidden variables and \textbf{honest} beliefs}
 \es{Variables ocultas y creencias \textbf{honestas}}
 \end{textblock}

}

\vspace{1cm}


\only<1>{
 \begin{textblock}{160}(0,30)\centering
 \includegraphics[width=0.55\textwidth]{figures/montyHall_0}     
 \end{textblock}
}

%reason for one outcome to occur more often than any other, then the events are assigned equal probabilities. This is called the principle of insufficient reason, or principle of indifference, and goes back to Laplace.

\only<2>{
 \begin{textblock}{160}(0,30)\centering
 \includegraphics[width=0.55\textwidth]{figures/montyHall_1}     
 \end{textblock}
}

\only<3-4>{
 \begin{textblock}{160}(0,30)\centering
 \includegraphics[width=0.55\textwidth]{figures/montyHall_2}     
 \end{textblock}
}

\only<5>{
 \begin{textblock}{160}(0,30)\centering
 \includegraphics[width=0.55\textwidth]{figures/montyHall_4}     
 \end{textblock}
}
\only<6>{
 \begin{textblock}{160}(0,30)\centering
 \includegraphics[width=0.55\textwidth]{figures/montyHall_5}     
 \end{textblock}
}

\end{frame}



\begin{frame}[plain]
\begin{textblock}{160}(0,4)
 \centering \LARGE 
 \en{Honesty maximizes entropy}
 \es{La honestidad maximiza la entropía}
 \end{textblock}

\en{Entropy (or expected information)}
\es{Entropía (o informaci\'on esperada)}
\begin{equation*}
 H(X) = \sum_{x\in X} P(x) \  \cdot \underbrace{(-\log P(x))}_{\hfrac{\text{\scriptsize \en{Information extractable}\es{Información extraíble}}}{\text{\scriptsize \en{from the surprise}\es{de la sorpresa}}}}
\end{equation*}

\pause

\vspace{0.3cm}

\Wider[-1.5cm]{
\begin{mdframed}[backgroundcolor=black!15]
\centering
 \en{Maximum expected information $\Leftrightarrow$ Maximum uncertainty}
 \es{M\'axima información esperada $\Leftrightarrow$ Máxima incertidumbre}
 \end{mdframed}


}
 
\end{frame}


\begin{frame}[plain]
\begin{textblock}{160}(0,4)
 \centering \LARGE 
 \en{Principle of maximum uncertainty: counting paths}
 \es{Principio de máxima incertidumbre: contar caminos}
 \end{textblock}
\vspace{1.25cm} 

 \begin{equation*}
  \text{\en{Belief}\es{Creencia}}(x) = \frac{\text{\en{Number of paths on which $x$ becomes true}\es{Cantidad de caminos en los que $x$ se hace verdadera}}}{\text{\en{Total number of paths}\es{Cantidad de caminos totales}}}
 \end{equation*}
 
\end{frame}

\begin{frame}[plain]
\begin{textblock}{160}(0,4)
 \centering \LARGE 
 \en{Example: Monty Hall}
 \es{Ejemplo: Monty Hall}
 \end{textblock}
 \vspace{-1cm}

 \only<1-2>{
 \begin{textblock}{160}(0,30)\centering
 \includegraphics[width=0.55\textwidth]{figures/montyHall_7}     
 \end{textblock}
 
}

\onslide<2>{\includegraphics[width=0.4\textwidth]{figures/montyHall_model.pdf}}
 

\only<3->{
\begin{columns}[c]
\begin{column}{0.4\textwidth}
 \centering 
 
 \includegraphics[width=0.8\textwidth]{figures/montyHall_model.pdf}
\end{column}
 \begin{column}{0.6\textwidth}

 \centering
 
 Caminos:
 \vspace{0.2cm}
 
 \begin{tabular}{c|c|c|c|c}
        
        & $d_1$ & $d_2$ & $d_3$ & \\ \hline 
        $r_1$ & $0$ & $1$ & $1$ & $2$\\ \hline
        $r_2$ & $0$ & $0$ & $1$ & $1$ \\ \hline
        $r_3$ & $0$ & $1$ & $0$ & $1$\\ \hline
              & $0$ & $2$ & $2$ & $\bm{4}$ \\ 
\end{tabular}

 \end{column}
\end{columns}
}


\only<3->{\begin{textblock}{130}(15,60) 
 Joint belief: $  B(H = h_i, D= d_j) = \frac{n_{ij}}{N}$
\end{textblock}}

\only<3->{\begin{textblock}{130}(15,68.7) 
Marginal belief: $ B(H = h_i) =   \frac{r_i}{N}$\end{textblock}}

\only<4>{\begin{textblock}{130}(15,76)
Conditional belief: $ B(D= d_j| H=h_i) = \frac{n_{ij}}{r_i} \color{white} \frac{B(H = h_i , D = d_j)}{B(H = h_i)}  $
\end{textblock}}

\only<5>{\begin{textblock}{130}(15,76)
Conditional belief: $ B(D= d_j| H=h_i) = \frac{n_{ij}}{r_i} \frac{N}{N} \color{white} \frac{B(H = h_i , D = d_j)}{B(H = h_i)}  $
\end{textblock}}

\only<6>{\begin{textblock}{130}(15,76)
Conditional belief: $ B(D= d_j| H=h_i) = \frac{n_{ij}}{N} \frac{N}{r_i} = \frac{B(H = h_i , D = d_j)}{B(H = h_i)}  $
\end{textblock}}


\end{frame}



\begin{frame}[plain]
 \begin{textblock}{160}(0,4)
\centering  \LARGE 
\en{Hidden variable, data and causal model}
\es{Variable oculta, datos y modelo causal}
 \end{textblock}
\vspace{1cm}

\centering
 
\tikz{         
    \node[det, fill=black!15] (r) {$r_{ij}$} ; 
    \node[const, left=of r, xshift=-1.35cm] (r_name) {\small \en{Result}\es{Resultado}:}; 
    \node[const, right=of r] (dr) {\normalsize $ r_{ij} = (d_{ij}>0)$}; 

    \node[latent, above=of r, yshift=-0.45cm] (d) {$d_{ij}$} ; %
    \node[const, right=of d] (dd) {\normalsize $ d_{ij}=p_i-p_j$}; 
    \node[const, left=of d, xshift=-1.35cm] (d_name) {\small \en{Difference}\es{Diferencia}:};
    
    \node[latent, above=of d, xshift=-0.8cm, yshift=-0.45cm] (p1) {$p_i$} ; %
    \node[latent, above=of d, xshift=0.8cm, yshift=-0.45cm] (p2) {$p_j$} ; %
    \node[const, left=of p1, xshift=-0.55cm] (p_name) {\small \en{Performance}\es{Rendimiento}:}; 

    \node[accion, above=of p1,yshift=0.3cm,fill=white] (s1) {} ; %
    \node[const, right=of s1] (ds1) {$s_i$};
    \node[accion, above=of p2,yshift=0.3cm,fill=white] (s2) {} ; %
    \node[const, right=of s2] (ds2) {$s_j$};
    
    \node[const, right=of p2] (dp2) {\normalsize $p \sim \N(s,\beta^2)$};

    \node[const, left=of s1, xshift=-.85cm] (s_name) {\small \en{Skill}\es{Habilidad}:}; 
    
    \edge {d} {r};
    \edge {p1,p2} {d};
    \edge {s1} {p1};
    \edge {s2} {p2};
}

 
\end{frame}


 
\begin{frame}[plain]
\centering
  \includegraphics[width=0.35\textwidth]{../../aux/static/pachacuteckoricancha.jpg}
\end{frame}






\end{document}



