\documentclass[shownotes,aspectratio=169]{beamer}

\input{../../aux/tex/diapo_encabezado.tex}
% tikzlibrary.code.tex
%
% Copyright 2010-2011 by Laura Dietz
% Copyright 2012 by Jaakko Luttinen
%
% This file may be distributed and/or modified
%
% 1. under the LaTeX Project Public License and/or
% 2. under the GNU General Public License.
%
% See the files LICENSE_LPPL and LICENSE_GPL for more details.

% Load other libraries
\usetikzlibrary{shapes}
\usetikzlibrary{fit}
\usetikzlibrary{chains}
\usetikzlibrary{arrows}

% Latent node
\tikzstyle{latent} = [circle,fill=white,draw=black,inner sep=1pt,
minimum size=20pt, font=\fontsize{10}{10}\selectfont, node distance=1]
% Observed node
\tikzstyle{obs} = [latent,fill=gray!25]
% Invisible node
\tikzstyle{invisible} = [latent,minimum size=0pt,color=white, opacity=0, node distance=0]
% Constant node
\tikzstyle{const} = [rectangle, inner sep=0pt, node distance=0.1]
%state
\tikzstyle{estado} = [latent,minimum size=8pt,node distance=0.4]
%action
\tikzstyle{accion} =[latent,circle,minimum size=5pt,fill=black,node distance=0.4]


% Factor node
\tikzstyle{factor} = [rectangle, fill=black,minimum size=10pt, draw=black, inner
sep=0pt, node distance=1]
% Deterministic node
\tikzstyle{det} = [latent, rectangle]

% Plate node
\tikzstyle{plate} = [draw, rectangle, rounded corners, fit=#1]
% Invisible wrapper node
\tikzstyle{wrap} = [inner sep=0pt, fit=#1]
% Gate
\tikzstyle{gate} = [draw, rectangle, dashed, fit=#1]

% Caption node
\tikzstyle{caption} = [font=\footnotesize, node distance=0] %
\tikzstyle{plate caption} = [caption, node distance=0, inner sep=0pt,
below left=5pt and 0pt of #1.south east] %
\tikzstyle{factor caption} = [caption] %
\tikzstyle{every label} += [caption] %

\tikzset{>={triangle 45}}

%\pgfdeclarelayer{b}
%\pgfdeclarelayer{f}
%\pgfsetlayers{b,main,f}

% \factoredge [options] {inputs} {factors} {outputs}
\newcommand{\factoredge}[4][]{ %
  % Connect all nodes #2 to all nodes #4 via all factors #3.
  \foreach \f in {#3} { %
    \foreach \x in {#2} { %
      \path (\x) edge[-,#1] (\f) ; %
      %\draw[-,#1] (\x) edge[-] (\f) ; %
    } ;
    \foreach \y in {#4} { %
      \path (\f) edge[->,#1] (\y) ; %
      %\draw[->,#1] (\f) -- (\y) ; %
    } ;
  } ;
}

% \edge [options] {inputs} {outputs}
\newcommand{\edge}[3][]{ %
  % Connect all nodes #2 to all nodes #3.
  \foreach \x in {#2} { %
    \foreach \y in {#3} { %
      \path (\x) edge [->,#1] (\y) ;%
      %\draw[->,#1] (\x) -- (\y) ;%
    } ;
  } ;
}

% \factor [options] {name} {caption} {inputs} {outputs}
\newcommand{\factor}[5][]{ %
  % Draw the factor node. Use alias to allow empty names.
  \node[factor, label={[name=#2-caption]#3}, name=#2, #1,
  alias=#2-alias] {} ; %
  % Connect all inputs to outputs via this factor
  \factoredge {#4} {#2-alias} {#5} ; %
}

% \plate [options] {name} {fitlist} {caption}
\newcommand{\plate}[4][]{ %
  \node[wrap=#3] (#2-wrap) {}; %
  \node[plate caption=#2-wrap] (#2-caption) {#4}; %
  \node[plate=(#2-wrap)(#2-caption), #1] (#2) {}; %
}

% \gate [options] {name} {fitlist} {inputs}
\newcommand{\gate}[4][]{ %
  \node[gate=#3, name=#2, #1, alias=#2-alias] {}; %
  \foreach \x in {#4} { %
    \draw [-*,thick] (\x) -- (#2-alias); %
  } ;%
}

% \vgate {name} {fitlist-left} {caption-left} {fitlist-right}
% {caption-right} {inputs}
\newcommand{\vgate}[6]{ %
  % Wrap the left and right parts
  \node[wrap=#2] (#1-left) {}; %
  \node[wrap=#4] (#1-right) {}; %
  % Draw the gate
  \node[gate=(#1-left)(#1-right)] (#1) {}; %
  % Add captions
  \node[caption, below left=of #1.north ] (#1-left-caption)
  {#3}; %
  \node[caption, below right=of #1.north ] (#1-right-caption)
  {#5}; %
  % Draw middle separation
  \draw [-, dashed] (#1.north) -- (#1.south); %
  % Draw inputs
  \foreach \x in {#6} { %
    \draw [-*,thick] (\x) -- (#1); %
  } ;%
}

% \hgate {name} {fitlist-top} {caption-top} {fitlist-bottom}
% {caption-bottom} {inputs}
\newcommand{\hgate}[6]{ %
  % Wrap the left and right parts
  \node[wrap=#2] (#1-top) {}; %
  \node[wrap=#4] (#1-bottom) {}; %
  % Draw the gate
  \node[gate=(#1-top)(#1-bottom)] (#1) {}; %
  % Add captions
  \node[caption, above right=of #1.west ] (#1-top-caption)
  {#3}; %
  \node[caption, below right=of #1.west ] (#1-bottom-caption)
  {#5}; %
  % Draw middle separation
  \draw [-, dashed] (#1.west) -- (#1.east); %
  % Draw inputs
  \foreach \x in {#6} { %
    \draw [-*,thick] (\x) -- (#1); %
  } ;%
}


 \mode<presentation>
 {
 %   \usetheme{Madrid}      % or try Darmstadt, Madrid, Warsaw, ...
 %   \usecolortheme{default} % or try albatross, beaver, crane, ...
 %   \usefonttheme{serif}  % or try serif, structurebold, ...
  \usetheme{Antibes}
  \setbeamertemplate{navigation symbols}{}
 }
 
\usepackage{todonotes}
\setbeameroption{show notes}

\newif\ifen
\newif\ifes
\newcommand{\en}[1]{\ifen#1\fi}
\newcommand{\es}[1]{\ifes#1\fi}
\estrue

%\title[Bayes del Sur]{}

\begin{document}

\color{black!85}
\large

 
%\setbeamercolor{background canvas}{bg=gray!15}

\begin{frame}[plain,noframenumbering]
 
 \begin{textblock}{90}(00,05)
\begin{center}
 \huge  \textcolor{black!66}{Creencias, datos y sorpresas}
\end{center}
\end{textblock}

 %\vspace{2cm}brown
%\maketitle
\Wider[2cm]{
\includegraphics[width=1\textwidth]{../../aux/static/peligro_predador}
}
\end{frame}


\begin{frame}[plain]
\only<1-2>{
 \begin{textblock}{160}(0,4)
 \centering \LARGE 
 \en{Hidden variables and beliefs}
 \es{Hipótesis y creencias}
 \end{textblock}
}

\only<3->{
\begin{textblock}{160}(0,4)
 \centering \LARGE 
 \en{Hidden variables and \textbf{honest} beliefs}
 \es{Hipótesis y creencias \textbf{honestas}}
 \end{textblock}

}

\vspace{1cm}


\only<1>{
 \begin{textblock}{160}(0,30)\centering
 \includegraphics[width=0.55\textwidth]{figures/montyHall_0}     
 \end{textblock}
}

%reason for one outcome to occur more often than any other, then the events are assigned equal probabilities. This is called the principle of insufficient reason, or principle of indifference, and goes back to Laplace.

\only<2-3>{
 \begin{textblock}{160}(0,30)\centering
 \includegraphics[width=0.55\textwidth]{figures/montyHall_1}     
 \end{textblock}
}


\only<4>{
 \begin{textblock}{160}(0,30)\centering
 \includegraphics[width=0.55\textwidth]{figures/montyHall_2}     
 \end{textblock}
}

\end{frame}

\begin{frame}[plain]
\begin{textblock}{160}(0,4)
\centering \LARGE Fundamento de la validez del conocimiento \\ \onslide<2->{\Large El reconocimiento mutuo (o imperativo categórico)}
\end{textblock}
\vspace{0.75cm}

\only<2>{
 \begin{center} \Large
   \textbf{Obra de modo tal que tu praxis pueda \\ también valer como legislación universal}
 \end{center} 
 }
 
\end{frame}


\begin{frame}[plain]
\begin{textblock}{160}(0,4)
\centering \LARGE La honestidad intelectual \\ \Large principio intercultural de acuerdos intersubjetivos
\end{textblock}
\vspace{0.75cm}

\only<2>{
 \begin{center} \Large
   \textbf{Afirmar sólo lo que se sabe: \\ maximizar incertidumbre dada la información disponible}
 \end{center} 
}

\end{frame}
 

\begin{frame}[plain]
\begin{textblock}{160}(0,4)
 \centering \LARGE 
 \en{Hidden variables and \textbf{honest} beliefs}
 \es{Hipótesis y creencias \textbf{honestas}}
 \end{textblock}
\vspace{1cm}

 \begin{center}
 Principio de honestidad \\
   \LARGE
\textbf{Dividir las creencias \\ en partes iguales}
 \end{center}
\end{frame}


\begin{frame}[plain]
\begin{textblock}{160}(0,4)
 \centering \LARGE 
 \en{Hidden variables and \textbf{honest} beliefs}
 \es{Hipótesis y creencias \textbf{honestas}}
 \end{textblock}
\vspace{1cm}

 
\begin{textblock}{160}(20,22)
\onslide<2->{Modelo causal} \\ \vspace{0.3cm}
 \tikz{            
    \node[latent,] (r) {\includegraphics[width=0.06\textwidth]{../../aux/static/regalo.png}} ;
    \node[const,left=of r] (nr) {\Large $r$} ;    
    
    \onslide<2->{
    \node[latent, below=of r] (d) {\includegraphics[width=0.05\textwidth]{../../aux/static/dedo.png}} ;
    \node[const, left=of d] (nd) {\Large $s$} ;

    \edge {r} {d};
    }
}
\end{textblock}

\only<1-2>{
\begin{textblock}{160}(65,33)
\scalebox{1.5}{
\tikz{
    \node[factor, minimum size=1cm] (p1) {} ;
    \node[factor, minimum size=1cm, xshift=1.5cm] (p2) {} ;
    \node[factor, minimum size=1cm, xshift=3cm] (p3) {} ;

    \node[const, above=of p1, yshift=.15cm] (fp1) {$1/3$};
    \node[const, above=of p2, yshift=.15cm] (fp2) {$1/3$};
    \node[const, above=of p3, yshift=.15cm] (fp3) {$1/3$};
    \node[const, below=of p2, yshift=-.10cm, xshift=0.3cm] (dedo) {};
    
    \node[invisible, xshift=4.75cm] (s-dist) {};
    \node[invisible, yshift=-1cm] (s-dist) {};
    \node[invisible, yshift=1.2cm] (s-dist) {};
    } 
}
\end{textblock}
}

\only<3>{
\begin{textblock}{160}(65,33)
\scalebox{1.5}{
\tikz{ %
        
         \node[factor, minimum size=1cm] (p1) {} ;
         \node[det, minimum size=1cm, xshift=1.5cm] (p2) {\includegraphics[width=0.03\textwidth]{../../aux/static/dedo.png}} ;
         \node[factor, minimum size=1cm, xshift=3cm] (p3) {} ;
%         
%         
         \node[const, above=of p1, yshift=.15cm] (fp1) {$?$};
         \node[const, above=of p2, yshift=.15cm] (fp2) {$0$};
         \node[const, above=of p3, yshift=.15cm] (fp3) {$?$};
         \node[const, below=of p2, yshift=-.10cm, xshift=0.3cm] (dedo) {};
         
%         \node[const, above=of p2, xshift=.8cm, yshift=.15cm] (fp3) {$66\%$};
%         
         \node[invisible, xshift=4.75cm] (s-dist) {};
         \node[invisible, yshift=-1cm] (s-dist) {};
         \node[invisible, yshift=1.2cm] (s-dist) {};
%         
%         \plate[color=red] {no} {(p1)} {}; %
%         \plate {si} {(p2)(p3)} {}; %
        
        } 
}
\end{textblock}
}

\end{frame}

\begin{frame}[plain]
\begin{textblock}{160}(0,4)
 \centering \LARGE 
 \en{Hidden variables and \textbf{honest} beliefs}
 \es{Hipótesis y creencias \textbf{honestas}}
 \end{textblock}
\vspace{1cm}

 \begin{center}
 Principio de honestidad \\
   \LARGE
\textbf{Dividir las creencias en partes iguales \\ (por los caminos del modelo causal)}
 \end{center}

\end{frame}

\begin{frame}[plain]
\begin{textblock}{160}(0,4)
 \centering \LARGE 
 \en{Hidden variables and \textbf{honest} beliefs}
 \es{Hipótesis y creencias \textbf{honestas}}
 \end{textblock}
\vspace{1cm}

\only<1-4>{
\begin{textblock}{160}(20,22)
Modelo causal \\ \vspace{0.3cm}
 \tikz{            
    \node[latent,] (r) {\includegraphics[width=0.06\textwidth]{../../aux/static/regalo.png}} ;
    \node[const,left=of r] (nr) {\Large $r$} ;    
    
    \node[latent, below=of r] (d) {\includegraphics[width=0.05\textwidth]{../../aux/static/dedo.png}} ;
    \node[const, left=of d] (nd) {\Large $s$} ;

    \edge {r} {d};
}
\end{textblock}
}


\only<2->{ 
\begin{textblock}{80}(60,18) \centering
\scalebox{1.1}{
\tikz{
\onslide<2->{ 
\node[latent, draw=white, yshift=0.6cm] (b0) {$ 1$};

\node[latent,below=of b0,yshift=0.6cm, xshift=-3cm] (r1) {$r_1$};
\node[latent,below=of b0,yshift=0.6cm] (r2) {$r_2$};
\node[latent,below=of b0,yshift=0.6cm, xshift=3cm] (r3) {$r_3$};

\node[latent, below=of r1, draw=white, yshift=0.6cm] (br1) {$\frac{1}{3}$};
\node[latent, below=of r2, draw=white, yshift=0.6cm] (br2) {$\frac{1}{3}$};
\node[latent, below=of r3, draw=white, yshift=0.6cm] (br3) {$\frac{1}{3}$};
}
\onslide<3->{
\node[latent,below=of br1,yshift=0.6cm, xshift=-0.7cm] (r1d2) {$s_2$};
\node[latent,below=of br1,yshift=0.6cm, xshift=0.7cm] (r1d3) {$s_3$};

\node[latent,below=of r1d2,yshift=0.6cm,draw=white] (br1d2) {$\frac{1}{3}\frac{1}{2}$};
\node[latent,below=of r1d3,yshift=0.6cm, draw=white] (br1d3) {$\frac{1}{3}\frac{1}{2}$};
}
\onslide<4->{
\node[latent,below=of br2,yshift=0.6cm, xshift=-0.7cm] (r2d1) {$s_1$};
\node[latent,below=of br2,yshift=0.6cm, xshift=0.7cm] (r2d3) {$s_3$};
\node[latent,below=of br3,yshift=0.6cm, xshift=-0.7cm] (r3d1) {$s_1$};
\node[latent,below=of br3,yshift=0.6cm, xshift=0.7cm] (r3d2) {$s_2$};

\node[latent,below=of r2d1,yshift=0.6cm, draw=white] (br2d1) {$\frac{1}{3}\frac{1}{2}$};
\node[latent,below=of r2d3,yshift=0.6cm,draw=white] (br2d3) {$\frac{1}{3}\frac{1}{2}$};
\node[latent,below=of r3d1,yshift=0.6cm, draw=white] (br3d1) {$\frac{1}{3}\frac{1}{2}$};
\node[latent,below=of r3d2,yshift=0.6cm,draw=white] (br3d2) {$\frac{1}{3}\frac{1}{2}$};
}
\onslide<2->{
\edge[-] {b0} {r1,r2,r3};
\edge[-] {r1} {br1};
\edge[-] {r2} {br2};
\edge[-] {r3} {br3};
}
\onslide<3->{
\edge[-] {br1} {r1d2,r1d3};
\edge[-] {r1d2} {br1d2};
\edge[-] {r1d3} {br1d3};
}
\onslide<4->{
\edge[-] {br2} {r2d1, r2d3};
\edge[-] {br3} {r3d1,r3d2};
\edge[-] {r2d1} {br2d1};
\edge[-] {r2d3} {br2d3};
\edge[-] {r3d1} {br3d1};
\edge[-] {r3d2} {br3d2};
}
}
}
\end{textblock}
}
 

\only<5->{
 \begin{textblock}{65}(0,22)
  \centering
  Creencia$(r,s)$ \\ \vspace{0.3cm}
 \begin{tabular}{c|c|c|c||c} \setlength\tabcolsep{0.4cm} 
        & \, $r_1$ \, &  \, $r_2$ \, & \, $r_3$ \, & \\ \hline 
  $s_1$  & \onslide<6->{$0$} & \onslide<7->{$1/6$} & \onslide<7->{$1/6$} & \\ \hline
  $s_2$  & \onslide<8->{$1/6$} & \onslide<8->{$0$} & \onslide<8->{$1/6$} &  \\ \hline
       $s_3$ & \onslide<9->{$1/6$} & \onslide<9->{$1/6$} & \onslide<9->{$0$} &  \\ \hline \hline
              & & &  & \\ 
\end{tabular}
\end{textblock}
}

\end{frame}

\begin{frame}[plain]
\only<1>{
\begin{textblock}{160}(0,4)
 \centering \LARGE 
 \en{Hidden variables and \textbf{honest} beliefs}
 \es{Hipótesis y creencias \textbf{honestas}}
 \end{textblock}
}

\only<2->{
 \begin{textblock}{160}(0,4)
 \centering \LARGE 
 \en{Hidden variables and honest reasoning}
 \es{Creencias y razonamiento \textbf{honesto}}
 \end{textblock}
 }
 
\vspace{1cm}

 \begin{textblock}{160}(0,22)
  \centering
  Creencia$(r,s)$ \\ \vspace{0.3cm}
 \begin{tabular}{c|c|c|c||c} \setlength\tabcolsep{0.4cm} 
        & \, $r_1$ \, &  \, $r_2$ \, & \, $r_3$ \, & \\ \hline 
  $s_1$ & $0$ & $1/6$ & $1/6$ & \onslide<6->{$1/3$} \\ \hline
  $s_2$ & $1/6$ & $0$ & $1/6$ & \onslide<6->{$1/3$} \\ \hline
  $s_3$ & $1/6$ & $1/6$ & $0$ & \onslide<6->{$1/3$} \\ \hline \hline
        & \onslide<4->{$1/3$} & \onslide<4->{$1/3$} & \onslide<4->{$1/3$} & \onslide<7>{$1$} \\ 
\end{tabular}

\vspace{0.3cm}

\onslide<3->{
\begin{align*}
 \text{Creencia}(r) = \ ?
\end{align*}
}
\vspace{-1cm}
\onslide<5->{
\begin{align*}
 \text{Creencia}(s) = \ ?
\end{align*}
}
\end{textblock}

\end{frame}


\begin{frame}[plain]
\begin{textblock}{160}(0,4)
 \centering \LARGE 
 \en{Hidden variables and honest reasoning}
 \es{Creencias y razonamiento \textbf{honesto}}
 \end{textblock}
 
\begin{textblock}{160}(0,25)\centering
 \begin{center}
 Regla 1 \\
   \LARGE
\textbf{Integrar las creencias \\ en partes iguales}
 \end{center}
 
 \begin{equation*}
  \text{Creencia}(r_i) = \sum_j \text{Creencia}(r_i, s_j) 
 \end{equation*}
 
\end{textblock}

 
\end{frame}


\begin{frame}[plain]
 \begin{textblock}{160}(0,4)
 \centering \LARGE 
 \en{Hidden variables and honest reasoning}
 \es{Creencias y razonamiento \textbf{honesto}}
 \end{textblock}
 
\vspace{1cm}
\only<1-2>{
 \begin{textblock}{160}(0,22)
  \centering
  Creencia$(r,s)$ \\ \vspace{0.3cm}
 \begin{tabular}{c|c|c|c||c} \setlength\tabcolsep{0.4cm} 
        & \, $r_1$ \, &  \, $r_2$ \, & \, $r_3$ \, & \phantom{\bm{$1/3$}} \\ \hline 
  $s_1$ & $0$ & $1/6$ & $1/6$ & $1/3$ \\ \hline
  $s_2$ & $1/6$ & $0$ & $1/6$ & $1/3$ \\ \hline
  $s_3$ & $1/6$ & $1/6$ & $0$ & $1/3$ \\ \hline \hline
        & $1/3$ & $1/3$ & $1/3$ & $1$ \\ 
\end{tabular}
\end{textblock}
}

\only<2-3>{
\begin{textblock}{160}(0,60)
\begin{align*}
 \text{Creencia}(r|s_2) = \ ?
\end{align*}
\end{textblock}
}

\only<3->{
 \begin{textblock}{160}(0,22)
  \centering
  \only<3-6>{Creencia$(r,s_2)$}\only<7>{Creencia$(r|s_2)$} \\ \vspace{0.3cm}
 \begin{tabular}{c|c|c|c||c} \setlength\tabcolsep{0.4cm} 
        & \, $r_1$ \, &  \, $r_2$ \, & \, $r_3$ \, &  \phantom{\bm{$1/3$}} \\ \hline 
  &  &  &  & \\ \hline
  $s_2$ & \only<3-5>{$1/6$}\only<6>{$\frac{1}{6}/\frac{1}{3}$}\only<7>{$1/2$} & $0$ & \only<3-5>{$1/6$}\only<6>{$\frac{1}{6}/\frac{1}{3}$}\only<7>{$1/2$} & \only<3-4>{$1/3$}\only<5>{\bm{$1/3$}}\only<6>{$\frac{1}{3}/\frac{1}{3}$}\only<7>{$1$} \\ \hline
  &  &  & &  \\ 
\end{tabular}
\end{textblock}
}

\only<4->{
\begin{textblock}{160}(7,57)
\centering
\scalebox{1.2}{
\tikz{ %
        
         \node[factor, minimum size=1cm] (p1) {} ;
         \node[det, minimum size=1cm, xshift=1.5cm] (p2) {\includegraphics[width=0.03\textwidth]{../../aux/static/dedo.png}} ;
         \node[factor, minimum size=1cm, xshift=3cm] (p3) {} ;

         \node[const, above=of p1, yshift=.15cm] (fp1) {$1/2$};
         \node[const, above=of p2, yshift=.15cm] (fp2) {$0$};
         \node[const, above=of p3, yshift=.15cm] (fp3) {$1/2$};
         \node[const, below=of p2, yshift=-.10cm, xshift=0.3cm] (dedo) {};
         
         \node[invisible, xshift=4.75cm] (s-dist) {};
         \node[invisible, yshift=-1cm] (s-dist) {};
         \node[invisible, yshift=1.2cm] (s-dist) {};
        
        } 
}
\end{textblock}
}

\end{frame}


\begin{frame}[plain]
\begin{textblock}{160}(0,4)
 \centering \LARGE 
 \en{Hidden variables and honest reasoning}
 \es{Creencias y razonamiento \textbf{honesto}}
 \end{textblock}
 \vspace{1cm}
 
\begin{center}
  Regla 2 \\
\LARGE
\textbf{Contextualizar las creencias \\ en partes iguales}
 \end{center}

 \begin{align*}
 P(r|s_2) = \frac{P(r, s_2)}{P(s_2)}
 \end{align*} 

 
\end{frame}



\begin{frame}[plain]
\begin{textblock}{160}(0,4)
\centering \LARGE  Las reglas de la probabilidad
\end{textblock}


\vspace{0.75cm}



\begin{equation*}
  \text{Marginal}_{i} = \sum_j \text{Conjunta}_{ij}  \ \ \ \ \ \ \ \ \ \ \ \  \text{Condicional}_{j|i} = \frac{\text{Conjunta}_{ij}}{\text{Marginal}_{i}}
\end{equation*}

\pause
\vspace{0.75cm}


\begin{columns}[t]
\begin{column}{0.5\textwidth}
 \centering \textbf{Regla de la suma}
 
 
\begin{equation*}
 P(r) = \sum_j P(r,s_j)
\end{equation*}
 
 \justifying \footnotesize
  Cualquier distribución marginal puede ser obtenida integrando la distribución conjunta

 \end{column}
 \begin{column}{0.5\textwidth}
\centering  \textbf{Regla del producto}

\begin{equation*}
 P(r,s) = P(s|r) P(r)
\end{equation*}

 \justifying \footnotesize
Cualquier distribución conjunta puede ser expresada como el producto de distribuciones condicionales uni-dimensionles.

\end{column}
\end{columns}

\end{frame}

 \begin{frame}[plain]
 \begin{textblock}{160}(0,4)
  \centering \LARGE Las reglas de la probabilidad
 \end{textblock}
 
 
\vspace{1.2cm}
 
  \textbf{Teorema de Cox} \\ 
  
  \vspace{0.3cm}
  
  \'Unicas reglas de razonamiento que grantizan:
\begin{itemize}
 \item[$\bullet$] Representar creencias con valores reales 
 \item[$\bullet$] Actualizar creencias en la direcci\'on de la evidencia
 \item[$\bullet$] Consistencia
 \end{itemize}
 
% In 1946, thanks to Cox, it was now a theorem that any set of rules for conducting inference, in which we represent degrees of plausibility by real numbers, is necessarily either equivalent to the Laplace-Jefreys rules, or inconsistent
 
\only<2>{
 \begin{textblock}{160}(0,85)
  \scriptsize \centering  \href{https://doi.org/10.1119/1.1990764}{Cox, R (1946). Probability, frequency and reasonable expectation.}
 \end{textblock}
 }
%  
 \end{frame}

 
\begin{frame}[plain]
\begin{textblock}{160}(0,4)
\centering \LARGE La teoría de la probabilidad \\ \Large como formalización de la honestidad intelectual
\end{textblock}
\vspace{0.75cm}

\only<2>{
 \begin{center} \Large
 Prové las (única) reglas que garantizan maximizar incertidumbre  \\ dada las evidencias empíricas y formales (datos y modelos)
 \end{center} 
}

\end{frame}
 

\begin{frame}[plain]
\begin{textblock}{160}(0,4)
 \centering \LARGE 
 \en{Model: Monty Hall}
 \es{Modelo: Monty Hall}
 \end{textblock}
 \vspace{-1cm}

 \only<1>{
 \begin{textblock}{80}(0,22)
 \centering
 \includegraphics[width=0.8\textwidth]{figures/montyHall_model_0.pdf}
 \end{textblock}
 }
 
 \only<2-3>{
 \begin{textblock}{80}(0,22)
 \centering
 \includegraphics[width=0.8\textwidth]{figures/montyHall_model_1.pdf}
 \end{textblock}
 }

\only<4->{
 \begin{textblock}{80}(0,22)
 \centering
 \includegraphics[width=0.8\textwidth]{figures/montyHall_model_2.pdf}
 \end{textblock}
 }
 

\only<1-2>{ 
 \begin{textblock}{80}(70,30)
 \centering
\includegraphics[width=1\textwidth]{figures/montyHall_2}
 \end{textblock}
}

\only<3-4>{ 
 \begin{textblock}{80}(70,30)
 \centering
\includegraphics[width=1\textwidth]{figures/montyHall_6}
 \end{textblock}
}

\only<5>{ 
 \begin{textblock}{80}(70,30)
 \centering
\includegraphics[width=1\textwidth]{figures/montyHall_7}
 \end{textblock}
}

\end{frame}


\begin{frame}[plain]
\begin{textblock}{160}(0,4)
 \centering \LARGE 
 \en{Model: Monty Hall}
 \es{Modelo: Monty Hall}
 \end{textblock}
 \vspace{-1cm}

 \only<1-4>{
 \begin{textblock}{80}(0,22)
 \centering
 \includegraphics[width=0.8\textwidth]{figures/montyHall_model.pdf}
 \end{textblock}
 }

  \only<5-18>{
 \begin{textblock}{80}(0,22)
  \centering
  $P(r,s)$ \\ \vspace{0.3cm}
 \begin{tabular}{c|c|c|c||c} \setlength\tabcolsep{0.4cm} 
        & \, $r_1$ \, &  \, $r_2$ \, & \, $r_3$ \, & \\ \hline 
  { $s_2$}  & \onslide<6->{$1/6$} & \onslide<8->{$0$} & \onslide<10->{$1/3$} & \onslide<13->{$1/2$} \\ \hline
       {$s_3$} & \onslide<7->{$1/6$} & \onslide<9->{$1/3$} & \onslide<11->{$0$} & \onslide<14->{$1/2$} \\ \hline
              & \onslide<15->{$1/3$} &  \onslide<16->{$1/3$} & \onslide<16->{$1/3$}  & \onslide<17->{$1$} \\ 
\end{tabular}
\end{textblock}
}

\only<19>{
 \begin{textblock}{80}(0,22)
  \centering
  $P(r,s_2)$ \\ \vspace{0.3cm}
 \begin{tabular}{c|c|c|c||c} \setlength\tabcolsep{0.4cm} 
        & \, $r_1$ \, &  \, $r_2$ \, & \, $r_3$ \, & \\ \hline 
        { $s_2$}  & \onslide<6->{$1/6$} & \onslide<8->{$0$} & \onslide<10->{$1/3$} & \onslide<13->{$1/2$} \\ \hline
\end{tabular}
\end{textblock}
}


\only<20->{
 \begin{textblock}{80}(0,22)
  \centering
  $P(r|s_2)$ \\ \vspace{0.3cm}
 \begin{tabular}{c|c|c|c||c} \setlength\tabcolsep{0.4cm} 
        & \, $r_1$ \, &  \, $r_2$ \, & \, $r_3$ \, & \phantom{$1/2$}\\ \hline 
  { $s_2$}  & \onslide<6->{$1/3$} & \onslide<8->{$0$} & \onslide<10->{$2/3$} & \onslide<13->{$1$} \\ \hline
\end{tabular}
\end{textblock}
}


\only<12-16>{
\begin{textblock}{80}(0,58)
 \centering 
\begin{center}
 Regla de la suma
 \end{center} 
 
 $P(s_i) = \sum_{j} P(r_j,s_i)$ 
 \\
 
\end{textblock}
}
 
\only<18-20>{
\begin{textblock}{80}(0,58)
 \centering 
\begin{center}
 Regla del producto
 \end{center} 
 \begin{equation*}
P(r_i|s_2) = \frac{P(r_i,s_2)}{P(s_2)} 
 \end{equation*}
 
\end{textblock}
}
 

\only<21>{
\begin{textblock}{80}(0,53)
\centering
\includegraphics[width=0.8\textwidth]{figures/montyHall_8}     
 \end{textblock}
}
 
\onslide<2->{ 
\begin{textblock}{80}(70,11) \centering
\scalebox{1.2}{
\tikz{
\onslide<2->{
\node[latent, draw=white, yshift=0.8cm] (b0) {$1$};
\node[latent,below=of b0,yshift=0.8cm, xshift=-2cm] (r1) {$r_1$};
\node[latent,below=of b0,yshift=0.8cm] (r2) {$r_2$};
\node[latent,below=of b0,yshift=0.8cm, xshift=2cm] (r3) {$r_3$};

\node[latent, below=of r1, draw=white, yshift=0.8cm] (br1) {$\frac{1}{3}$};
\node[latent, below=of r2, draw=white, yshift=0.8cm] (br2) {$\frac{1}{3}$};
\node[latent, below=of r3, draw=white, yshift=0.8cm] (br3) {$\frac{1}{3}$};
}
\onslide<3->{
\node[latent,below=of br1,yshift=0.8cm] (c11) {$c_1$};
\node[latent,below=of br2,yshift=0.8cm] (c12) {$c_1$};
\node[latent,below=of br3,yshift=0.8cm] (c13) {$c_1$};

\node[latent, below=of c11, draw=white, yshift=0.8cm] (bc11) {$\frac{1}{3}$};
\node[latent, below=of c12, draw=white, yshift=0.8cm] (bc12) {$\frac{1}{3}$};
\node[latent, below=of c13, draw=white, yshift=0.8cm] (bc13) {$\frac{1}{3}$};
}
\onslide<4->{
\node[latent,below=of bc11,yshift=0.8cm, xshift=-0.7cm] (r1d2) {$s_2$};
\node[latent,below=of bc11,yshift=0.8cm, xshift=0.7cm] (r1d3) {$s_3$};
\node[latent,below=of bc12,yshift=0.8cm] (r2d3) {$s_3$};
\node[latent,below=of bc13,yshift=0.8cm] (r3d2) {$s_2$};

\node[latent,below=of r1d2,yshift=0.8cm,draw=white] (br1d2) {$\frac{1}{3}\frac{1}{2}$};
\node[latent,below=of r1d3,yshift=0.8cm, draw=white] (br1d3) {$\frac{1}{3}\frac{1}{2}$};
\node[latent,below=of r2d3,yshift=0.8cm,draw=white] (br2d3) {$\frac{1}{3}$};
\node[latent,below=of r3d2,yshift=0.8cm,draw=white] (br3d2) {$\frac{1}{3}$};
}
\onslide<2->{
\edge[-] {b0} {r1,r2,r3};
\edge[-] {r1} {br1};
\edge[-] {r2} {br2};
\edge[-] {r3} {br3};
}
\onslide<3->{
\edge[-] {br1} {c11};
\edge[-] {br2} {c12};
\edge[-] {br3} {c13};
\edge[-] {c11} {bc11};
\edge[-] {c12} {bc12};
\edge[-] {c13} {bc13};
}
\onslide<4->{
\edge[-] {bc11} {r1d2,r1d3};
\edge[-] {bc12} {r2d3};
\edge[-] {bc13} {r3d2};
\edge[-] {r1d2} {br1d2};
\edge[-] {r1d3} {br1d3};
\edge[-] {r2d3} {br2d3};
\edge[-] {r3d2} {br3d2};
}
}
}
\end{textblock}
}
 
\end{frame}

\begin{frame}[plain]
\begin{textblock}{160}(0,4)
\centering \LARGE 
\st{Teorema de Bayes} \\
\Large El corolario de Laplace
\end{textblock}

\only<1-4>{
\begin{textblock}{160}(0,34)
 \begin{align*}
  \phantom{P(r_i)} P(r_i|s_2) & = \frac{P(r_i, s_2)}{P(s_2)} \onslide<4>{= \frac{P(s_2|r_i)P(r_i)}{P(s_2)}}
  \only<2>{\\ P(s_2 | r_i) &= \frac{P(r_i, s_2)}{P(r_i)}}
  \only<3-4>{\\[0.27cm]P(r_i) P(s_2 | r_i)  &= P(r_i, s_2)}
 \end{align*}
\end{textblock}
}

\only<5>{
\begin{textblock}{160}(0,34)
 \begin{align*}
  \phantom{P(r_i)} P(r_i|s_2) & = \frac{P(s_2|r_i)P(r_i)}{P(s_2)} \phantom{= \frac{P(r_i, s_2)}{P(s_2)} \ }
  \end{align*}
\end{textblock}
}



\only<6>{
\begin{textblock}{160}(0,43)
\begin{equation*}
P(\text{Hip\'otesis }|\text{ Dato}) = \frac{P(\text{Dato }|\text{ Hip\'otesis}) P(\text{Hip\'otesis})}{P(\text{Dato})}
\end{equation*}
\end{textblock}
}


\only<7>{
\begin{textblock}{160}(0,37.75)
\begin{equation*}
\underbrace{P(\text{Hip\'otesis }|\text{ Datos})}_{\text{\scriptsize Posteriori}} = \frac{\overbrace{P(\text{Datos }|\text{ Hip\'otesis})}^{\text{\scriptsize Verosimilitud}} \overbrace{P(\text{Hip\'otesis})}^{\text{\scriptsize Priori}} }{\underbrace{P(\text{Datos})}_{\text{\scriptsize Evidencia}}}
\end{equation*}
\end{textblock}
}

\vspace{0.2cm}

\only<8->{  
%\vspace{0.3cm}
\Wider[2cm]{
\begin{textblock}{160}(0,34.25) 
\begin{equation*}
\underbrace{P(\text{Hip\'otesis }|\text{ Datos, Modelo})}_{\text{\scriptsize Posteriori}} = \frac{\overbrace{P(\text{Datos }|\text{ Hip\'otesis, Modelo})}^{\text{\scriptsize Verosimilitud}} \overbrace{P(\text{Hip\'otesis }|\text{ Modelo})}^{\text{\scriptsize Priori}} }{\underbrace{P(\text{Datos }|\text{ Modelo})}_{\text{\scriptsize Evidencia}}}
\end{equation*}

\end{textblock}

}
}
% 
% \only<8->{  
% \begin{textblock}{100}(30,65)
% \begin{mdframed}[backgroundcolor=black!30]
% \centering \vspace{0.05cm}
% El \textbf{modelo} es lo que permite relacionar 
% 
% los \textbf{datos} con nuestras \textbf{hipótesis}! 
% \vspace{0.1cm}
% \end{mdframed}
% \end{textblock}
% }





\end{frame}


\begin{frame}[plain]
\begin{textblock}{160}(0,4)
\centering \LARGE El teorema de Bayes \\ \Large como generalización de la hipótesis indicadora
\end{textblock}
\vspace{0.75cm}

\only<2>{
 \begin{center} \Large
 Principio de validez universal para determinar las creencias \\ honestas dadas las evidencias empíricas y formales
 \end{center} 
}
  
\end{frame}

 



\begin{frame}[plain]
\begin{textblock}{160}(0,4)
 \centering \LARGE 
 \en{Likelihood}
 \es{Verosimilutd: predicción del dato dada la hipótesis}
 \end{textblock}
 \vspace{-1cm}

 
\begin{textblock}{80}(70,11) \centering
\scalebox{1.2}{
\tikz{
\only<3->{\phantom}{\node[latent, draw=white, yshift=0.8cm] (b0) {$1$};}
\only<8->{\phantom}{\node[latent,below=of b0,yshift=0.8cm, xshift=-2cm] (r1) {$r_1$};}
\only<3-7,10-11>{\phantom}{\node[latent,below=of b0,yshift=0.8cm] (r2) {$r_2$};}
\only<3-9>{\phantom}{\node[latent,below=of b0,yshift=0.8cm, xshift=2cm] (r3) {$r_3$};}

\only<8->{\phantom}{\node[latent, below=of r1, draw=white, yshift=0.8cm] (br1) {$\frac{1}{3}$};}
\only<3-7,10-11>{\phantom}{\node[latent, below=of r2, draw=white, yshift=0.8cm] (br2) {$\frac{1}{3}$};}
\only<3-9>{\phantom}{\node[latent, below=of r3, draw=white, yshift=0.8cm] (br3) {$\frac{1}{3}$};}
\only<8->{\phantom}{\node[latent,below=of br1,yshift=0.8cm] (c11) {$c_1$};}
\only<3-7,10-11>{\phantom}{\node[latent,below=of br2,yshift=0.8cm] (c12) {$c_1$};}
\only<3-9>{\phantom}{\node[latent,below=of br3,yshift=0.8cm] (c13) {$c_1$};}

\only<8->{\phantom}{\node[latent, below=of c11, draw=white, yshift=0.8cm] (bc11) {$\frac{1}{3}$};}
\only<3-7,10-11>{\phantom}{\node[latent, below=of c12, draw=white, yshift=0.8cm] (bc12) {$\frac{1}{3}$};}
\only<3-9>{\phantom}{\node[latent, below=of c13, draw=white, yshift=0.8cm] (bc13) {$\frac{1}{3}$};}
\only<8->{\phantom}{\node[latent,below=of bc11,yshift=0.8cm, xshift=-0.7cm] (r1d2) {$s_2$};}
\only<8->{\phantom}{\node[latent,below=of bc11,yshift=0.8cm, xshift=0.7cm] (r1d3) {$s_3$};}
\only<3-7,10-11>{\phantom}{\node[latent,below=of bc12,yshift=0.8cm] (r2d3) {$s_3$};}
\only<3-9>{\phantom}{\node[latent,below=of bc13,yshift=0.8cm] (r3d2) {$s_2$};}

\only<8->{\phantom}{\node[latent,below=of r1d2,yshift=0.8cm,draw=white] (br1d2) {$\frac{1}{3}\frac{1}{2}$};}
\only<8->{\phantom}{\node[latent,below=of r1d3,yshift=0.8cm, draw=white] (br1d3) {$\frac{1}{3}\frac{1}{2}$};}
\only<3-7,10-11>{\phantom}{\node[latent,below=of r2d3,yshift=0.8cm,draw=white] (br2d3) {$\frac{1}{3}$};}
\only<3-9>{\phantom}{\node[latent,below=of r3d2,yshift=0.8cm,draw=white] (br3d2) {$\frac{1}{3}$};}

\only<3->{\phantom}{\edge[-] {b0} {r1};}
\only<3->{\phantom}{\edge[-] {b0} {r2};}
\only<3->{\phantom}{\edge[-] {b0} {r3};}
\only<8->{\phantom}{\edge[-] {r1} {br1};}
\only<3-7,10-11>{\phantom}{\edge[-] {r2} {br2};}
\only<3-9>{\phantom}{\edge[-] {r3} {br3};}
\only<8->{\phantom}{\edge[-] {br1} {c11};}
\only<3-7,10-11>{\phantom}{\edge[-] {br2} {c12};}
\only<3-9>{\phantom}{\edge[-] {br3} {c13};}
\only<8->{\phantom}{\edge[-] {c11} {bc11};}
\only<3-7,10-11>{\phantom}{\edge[-] {c12} {bc12};}
\only<3-9>{\phantom}{\edge[-] {c13} {bc13};}
\only<8->{\phantom}{\edge[-] {bc11} {r1d2,r1d3};}
\only<3-7,10-11>{\phantom}{\edge[-] {bc12} {r2d3};}
\only<3-9>{\phantom}{\edge[-] {bc13} {r3d2};}
\only<8->{\phantom}{\edge[-] {r1d2} {br1d2};}
\only<8->{\phantom}{\edge[-] {r1d3} {br1d3};}
\only<3-7,10-11>{\phantom}{\edge[-] {r2d3} {br2d3};}
\only<3-9>{\phantom}{\edge[-] {r3d2} {br3d2};}
}
}
\end{textblock}


\only<1->{
 \begin{textblock}{80}(0,16)
  \centering
  $P(s_2|r_i)$ \\ \vspace{0.1cm} 
  \onslide<2->{
  \begin{tabular}{c|c|c|c} \setlength\tabcolsep{0.4cm} 
          & \, \only<3-7>{\bm}{$r_1$} \, &  \, \only<8-9>{\bm}{$r_2$} \, & \, \only<10-11>{\bm}{$r_3$} \, \\ \hline 
   $s_2$ & \onslide<7->{$1/2$} & \onslide<9->{$0$} & \onslide<11->{$1$}  \\ \hline
\end{tabular} 
}
\end{textblock}
}

\only<4->{
\begin{textblock}{80}(0,40)
 \begin{equation*}
  P(s|r_{\only<4-7>{1}\only<8-9>{2}\only<10-11>{3}}) = \frac{P(r_{\only<4-7>{1}\only<8-9>{2}\only<10-11>{3}}, s)}{P(r_{\only<4-7>{1}\only<8-9>{2}\only<10-11>{3}})}
 \end{equation*}
\end{textblock}
}


 \only<5>{
 \begin{textblock}{80}(0,57)
  \centering
  $P(r,s)$\\ \vspace{0.1cm}
 \begin{tabular}{c|c|c|c||c} \setlength\tabcolsep{0.4cm} 
          & \, $r_1$ \, &  \, $r_2$ \, & \, $r_3$ \, & \\ \hline 
   $s_2$ & $1/6$ & $0$ & $1/3$     & $1/2$ \\ \hline
   $s_3$ & $1/6$ & $1/3$ & $0$     & $1/2$ \\ \hline \hline
         & $1/3$ &  $1/3$ & $1/3$  & $1$ \\ 
\end{tabular} 
\end{textblock}
}

\only<6->{
\begin{textblock}{80}(0,57)
  \centering
  \only<6>{$P(r_1, s)$}\only<7>{$P(s|r_1)$}\only<8>{$P(r_2, s)$}\only<9>{$P(s|r_2)$}\only<10>{$P(r_3, s)$}\only<11>{$P(s|r_3)$} \\ \vspace{0.1cm}
 \begin{tabular}{c|c|c|c||c} \setlength\tabcolsep{0.4cm} 
          & \, $r_1$ \, &  \, $r_2$ \, & \, $r_3$ \, &  \phantom{$1/2$} \\ \hline 
   \only<7->{\bm}{$s_2$} & \only<6>{$1/6$}\only<7>{\bm{$1/2$}} & \only<8>{$0$}\only<9>{\bm{$0$}} & \only<10>{$1/3$}\only<11>{\bm{$1$}} &  \\ \hline
   $s_3$ & \only<6>{$1/6$}\only<7>{$1/2$} & \only<8>{$1/3$}\only<9>{$1$} & \only<10-11>{$0$}  &  \\ \hline \hline
         & \only<6>{$1/3$}\only<7>{$1$} & \only<8>{$1/3$}\only<9>{$1$} & \only<10>{$1/3$}\only<11>{$1$}  &  \\ 
\end{tabular} 
\end{textblock}
}

\end{frame}

\begin{frame}[plain]
\begin{textblock}{160}(0,4)
 \centering \LARGE 
 \en{Likelihood}
 \es{Verosimilitud: los caminos del modelo causal}
 \end{textblock}

 \begin{textblock}{160}(0,18)
  \centering
  $P(s_2|r_i)$ \\ \vspace{0.1cm} 
  \begin{tabular}{c|c|c|c} \setlength\tabcolsep{0.4cm} 
          & \, $r_1$ \, &  \, $r_2$ \, & \, $r_3$ \, \\ \hline 
   $s_2$ & $1/2$ & $0$ & $1$  \\ \hline
\end{tabular}
\end{textblock}

\only<2>{
 \begin{textblock}{160}(0,49)
 \begin{align*}
  P(s_2|r_i, M) = \frac{\text{\textbf{Caminos que generan} $\bm{s_2}$ dada la hipótesis $r_i$ y el modelo}}{\text{\textbf{Caminos totales} dada la hipótesis $r_i$ y el modelo }}
 \end{align*}
\end{textblock}
}
 
\end{frame}



\begin{frame}[plain]
\begin{textblock}{160}(0,4)
 \centering \LARGE 
 \en{Surprise as a filter of prior beliefs}
 \es{La \textbf{sorpresa} como filtro de las creencias previas}
 \end{textblock}

 \begin{textblock}{160}(0,18)
  \begin{align*}
   P(r_i|s_2) \propto P(r_i) P(s_2|r_i) 
  \end{align*}
 \end{textblock}

 
 \only<2>{
 \begin{textblock}{160}(0,40)
  \centering
   \begin{tabular}{c|c|c|c} \setlength\tabcolsep{0.4cm} 
  \phantom{$P(r_i|s_2) \propto$} & \, $r_1$ \, &  \, $r_2$ \, & \, $r_3$ \, \\ \hline 
   $P(s_2|r_i)$ & $1/2$ & $0$ & $1$  \\ \hline
   \end{tabular}
\end{textblock}
}

 
\only<3>{
 \begin{textblock}{160}(0,40)
  \centering
   \begin{tabular}{c|c|c|c} \setlength\tabcolsep{0.4cm} 
       \phantom{$P(r_i|s_2) \propto$} & \, $r_1$ \, &  \, $r_2$ \, & \, $r_3$ \, \\ \hline 
   $P(s_2|r_i)$ & $1/2$ & $0$ & $1$  \\ \hline
   $P(r_i)$ & $1/3$ & $1/3$ & $1/3$  \\ \hline
   \end{tabular}
\end{textblock}
}

 
\only<4>{
 \begin{textblock}{160}(0,40)
  \centering
  \begin{tabular}{c|c|c|c} \setlength\tabcolsep{0.4cm} 
        \phantom{$P(r_i|s_2) \propto$}  & \, $r_1$ \, &  \, $r_2$ \, & \, $r_3$ \, \\ \hline 
   $P(s_2|r_i)$ & $1/2$ & $0$ & $1$  \\ \hline
   $P(r_i)$ & $1/3$ & $1/3$ & $1/3$  \\ \hline
   $P(r_i|s_2) \propto$ & $1/6$ & $0$ & $1/3$  \\ \hline
\end{tabular}
\end{textblock}
}
 
\only<5>{
 \begin{textblock}{160}(0,40)
  \centering
  \begin{tabular}{c|c|c|c} \setlength\tabcolsep{0.4cm} 
         \phantom{$P(r_i|s_2) \propto$} & \, $r_1$ \, &  \, $r_2$ \, & \, $r_3$ \, \\ \hline 
   $P(s_2|r_i)$ & $1/2$ & $0$ & $1$  \\ \hline
   $P(r_i)$ & $1/3$ & $1/3$ & $1/3$  \\ \hline
   $P(r_i|s_2) \propto$ & $1/6$ & $0$ & $1/3$  \\ \hline \hline
   $P(r_i|s_2) =$ & $1/3$ & $0$ & $2/3$  \\ \hline 
\end{tabular}
\end{textblock}
}
 
 
\end{frame}

\begin{frame}[plain]
\begin{textblock}{160}(0,4)
 \centering \LARGE 
 \es{Posterior}
 \end{textblock}
 \vspace{1cm}
 
 \begin{center}
 $P(r_i|s_2) = $ \Large Creencia inicial no filtrada por la sorpresa
 \end{center}

\end{frame}

\begin{frame}[plain]
\begin{textblock}{160}(0,4)
 \centering \LARGE 
 \es{Evidencia}
 \end{textblock}
 \vspace{1cm}
 
 \begin{equation*}
  P(s_2) = \sum_i P(s_2|r_i) P(r_i) = \frac{1}{2} \frac{1}{3} + 0 \frac{1}{3} + \frac{1}{1} \frac{1}{3} = 1/2 
 \end{equation*}

 \begin{center}
 \Large Predicción a priori (hecha con todas las hipótesis)
 \end{center}

\end{frame}


% 
% 
% \begin{frame}[plain]
% \begin{textblock}{160}(0,4)
%  \centering \LARGE 
%  \es{Sermón}
%  \end{textblock}
% 
% \begin{enumerate}
%  \item s
% \end{enumerate}
% 
% \end{frame}



 
\begin{frame}[plain]
 \begin{textblock}{160}(0,4)
\centering  \Large 
¿Cuáles son las creencia honestas dado este modelo causal?
\end{textblock}
\vspace{0.5cm}

\centering
 
\tikz{         
    \node[det, fill=black!15] (r) {$r_{ij}$} ; 
    \node[const, left=of r, xshift=-1.35cm] (r_name) {\small \en{Result}\es{Ganar/perder}:}; 
    \node[const, right=of r] (dr) {\normalsize $ r_{ij} = (d_{ij}>0)$}; 

    \node[latent, above=of r, yshift=-0.45cm] (d) {$d_{ij}$} ; %
    \node[const, right=of d] (dd) {\normalsize $ d_{ij}=p_i-p_j$}; 
    \node[const, left=of d, xshift=-1.35cm] (s_name) {\small \en{Difference}\es{Diferencia}:};
    
    \node[latent, above=of d, xshift=-0.8cm, yshift=-0.45cm] (p1) {$p_i$} ; %
    \node[latent, above=of d, xshift=0.8cm, yshift=-0.45cm] (p2) {$p_j$} ; %
    \node[const, left=of p1, xshift=-0.55cm] (p_name) {\small \en{Performance}\es{Rendimiento}:}; 

    \node[accion, above=of p1,yshift=0.3cm,fill=white] (s1) {} ; %
    \node[const, right=of s1] (ds1) {$s_i$};
    \node[accion, above=of p2,yshift=0.3cm,fill=white] (s2) {} ; %
    \node[const, right=of s2] (ds2) {$s_j$};
    
    \node[const, right=of p2] (dp2) {\normalsize $p \sim \N(s,\beta^2)$};

    \node[const, left=of s1, xshift=-.85cm] (s_name) {\small \en{Skill}\es{Habilidad}:}; 
    
    \edge {d} {r};
    \edge {p1,p2} {d};
    \edge {s1} {p1};
    \edge {s2} {p2};
}

 
\end{frame}

 
 
\begin{frame}[plain]
\centering
  \includegraphics[width=0.35\textwidth]{../../aux/static/pachacuteckoricancha.jpg}
\end{frame}






\end{document}



