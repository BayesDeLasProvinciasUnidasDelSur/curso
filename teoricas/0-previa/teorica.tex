\documentclass[shownotes,aspectratio=169]{beamer}

\input{../../aux/tex/diapo_encabezado.tex}
\input{../../aux/tex/tikzlibrarybayesnet.code.tex}
 \mode<presentation>
 {
 %   \usetheme{Madrid}      % or try Darmstadt, Madrid, Warsaw, ...
 %   \usecolortheme{default} % or try albatross, beaver, crane, ...
 %   \usefonttheme{serif}  % or try serif, structurebold, ...
  \usetheme{Antibes}
  \setbeamertemplate{navigation symbols}{}
 }
 
\usepackage{todonotes}
\setbeameroption{show notes}

\newif\ifen
\newif\ifes
\newcommand{\en}[1]{\ifen#1\fi}
\newcommand{\es}[1]{\ifes#1\fi}
\estrue

%\title[Bayes del Sur]{}

\begin{document}

\color{black!85}
\large

 
%\setbeamercolor{background canvas}{bg=gray!15}

\begin{frame}[plain,noframenumbering]
 
 \begin{textblock}{90}(00,05)
\begin{center}
 \huge  \textcolor{black!66}{Creencias, datos y sorpresas}
\end{center}
\end{textblock}

 %\vspace{2cm}brown
%\maketitle
\Wider[2cm]{
\includegraphics[width=1\textwidth]{../../aux/static/peligro_predador}
}
\end{frame}

% 
% \begin{frame}[plain]
% \begin{textblock}{160}(0,4)
%  \centering
%  \LARGE \textcolor{black!85}{\en{Today}\es{Hoy}}
% \end{textblock}
% 
% \begin{itemize}
%  \item[$\bullet$] Human dispersal
%  \item[$\bullet$] Biomass (dentro de los vertebrados terrestres)
%  \item[$\bullet$] Empatia
%  \item[$\bullet$] Evolución cultural
%  \item[$\bullet$] Ciencia como intersubjectividad (muto entendimiento)
%  \item[$\bullet$] Base empírica 
%  \item[$\bullet$] Matriz de datos 
%  \item[$\bullet$] Ciencia empírica 
%  \item[$\bullet$] Niveles de conocimiento
%  \item[$\bullet$] Incertidumbre
%  \item[$\bullet$] Creencias honestas
%  \item[$\bullet$] Modelos causales
%  \item[$\bullet$] La lógica de la ciencia empírica
%  \item[$\bullet$] Selección de modelo
% \end{itemize}
% 
% \end{frame}

\begin{frame}[plain]
\begin{textblock}{160}(0,4)
 \centering \LARGE
 Humanos
\end{textblock}
\vspace{1.2cm}
\Wider[1cm]{
\includegraphics[width=1\textwidth]{../../aux/static/mapamundi2.jpg}
}
\end{frame}

\begin{frame}[plain]
\begin{textblock}{160}(0,4)
 \centering \LARGE
 Ocupamos todos los nichos ecológicos
\end{textblock}
\Wider[2cm]{
\includegraphics[width=1\textwidth]{../../aux/static/inuit_igloo_low}
}
\end{frame}



\begin{frame}[plain]
\begin{textblock}{160}(0,4)
 \centering \LARGE
 \en{The cognitive hypotesis}
 \es{La hipótesis cognitiva}
\end{textblock}
\centering \vspace{1cm}
 \includegraphics[width=0.5\textwidth]{../../aux/static/cerebros}  
\end{frame}



\begin{frame}[plain]

 \begin{textblock}{160}(5,30)
\includegraphics[width=0.33\textwidth]{../../aux/static/evolucionLitica.jpg} 
\end{textblock}
\begin{textblock}{160}(0,40)
 \centering \LARGE
 \en{Cultural \\ evolution}
 \es{Evolución \\ cultural}
\end{textblock}
\begin{textblock}{160}(105,1)
\includegraphics[width=0.285\textwidth]{../../aux/static/debian_tree.png}
\end{textblock}
\end{frame}


\begin{frame}[plain]
\begin{textblock}{160}(0,4)
 \centering \LARGE 
 La hipótesis cultural
 \end{textblock}
\vspace{1cm}

\includegraphics[width=1\textwidth]{../../aux/static/evolucionCultural.jpg}

\end{frame}
% 
% \begin{frame}[plain]
% \begin{textblock}{160}(0,4)
%  \centering \LARGE 
%  \en{Evolution of empathy}
%  \es{La evolución de la empatía}
% \end{textblock}
% \vspace{1cm}
%  
% %Wired To mutual understanding
% %Cableados para la comprensión mutua
%  \centering
% \includegraphics[width=0.7\textwidth]{../../aux/static/empatia}
% 
% % Cooperative breeding permited the evolution of extended life span, prolonged childhoods
% 
% % \centering
% % \en{Cooperative breeding permited the evolution of empathy}
% % \es{La crianza cooperative permitió la evolución de la empatía}
%  
% %Humans are often eager to understand others, to be understood, and to cooperate. 
% %Los seres humanos suelen estar ansiosos por comprender a los demás, por ser comprendidos y por cooperar. 
% 
% 
% \end{frame}

 
\begin{frame}[plain]
\centering
  \includegraphics[width=0.35\textwidth]{../../aux/static/pachacuteckoricancha.jpg}
\end{frame}






\end{document}



