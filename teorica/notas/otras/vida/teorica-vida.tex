\documentclass[shownotes,aspectratio=169]{beamer}

\input{../../auxiliar/tex/diapo_encabezado.tex}
% tikzlibrary.code.tex
%
% Copyright 2010-2011 by Laura Dietz
% Copyright 2012 by Jaakko Luttinen
%
% This file may be distributed and/or modified
%
% 1. under the LaTeX Project Public License and/or
% 2. under the GNU General Public License.
%
% See the files LICENSE_LPPL and LICENSE_GPL for more details.

% Load other libraries
\usetikzlibrary{shapes}
\usetikzlibrary{fit}
\usetikzlibrary{chains}
\usetikzlibrary{arrows}

% Latent node
\tikzstyle{latent} = [circle,fill=white,draw=black,inner sep=1pt,
minimum size=20pt, font=\fontsize{10}{10}\selectfont, node distance=1]
% Observed node
\tikzstyle{obs} = [latent,fill=gray!25]
% Invisible node
\tikzstyle{invisible} = [latent,minimum size=0pt,color=white, opacity=0, node distance=0]
% Constant node
\tikzstyle{const} = [rectangle, inner sep=0pt, node distance=0.1]
%state
\tikzstyle{estado} = [latent,minimum size=8pt,node distance=0.4]
%action
\tikzstyle{accion} =[latent,circle,minimum size=5pt,fill=black,node distance=0.4]


% Factor node
\tikzstyle{factor} = [rectangle, fill=black,minimum size=10pt, draw=black, inner
sep=0pt, node distance=1]
% Deterministic node
\tikzstyle{det} = [latent, rectangle]

% Plate node
\tikzstyle{plate} = [draw, rectangle, rounded corners, fit=#1]
% Invisible wrapper node
\tikzstyle{wrap} = [inner sep=0pt, fit=#1]
% Gate
\tikzstyle{gate} = [draw, rectangle, dashed, fit=#1]

% Caption node
\tikzstyle{caption} = [font=\footnotesize, node distance=0] %
\tikzstyle{plate caption} = [caption, node distance=0, inner sep=0pt,
below left=5pt and 0pt of #1.south east] %
\tikzstyle{factor caption} = [caption] %
\tikzstyle{every label} += [caption] %

\tikzset{>={triangle 45}}

%\pgfdeclarelayer{b}
%\pgfdeclarelayer{f}
%\pgfsetlayers{b,main,f}

% \factoredge [options] {inputs} {factors} {outputs}
\newcommand{\factoredge}[4][]{ %
  % Connect all nodes #2 to all nodes #4 via all factors #3.
  \foreach \f in {#3} { %
    \foreach \x in {#2} { %
      \path (\x) edge[-,#1] (\f) ; %
      %\draw[-,#1] (\x) edge[-] (\f) ; %
    } ;
    \foreach \y in {#4} { %
      \path (\f) edge[->,#1] (\y) ; %
      %\draw[->,#1] (\f) -- (\y) ; %
    } ;
  } ;
}

% \edge [options] {inputs} {outputs}
\newcommand{\edge}[3][]{ %
  % Connect all nodes #2 to all nodes #3.
  \foreach \x in {#2} { %
    \foreach \y in {#3} { %
      \path (\x) edge [->,#1] (\y) ;%
      %\draw[->,#1] (\x) -- (\y) ;%
    } ;
  } ;
}

% \factor [options] {name} {caption} {inputs} {outputs}
\newcommand{\factor}[5][]{ %
  % Draw the factor node. Use alias to allow empty names.
  \node[factor, label={[name=#2-caption]#3}, name=#2, #1,
  alias=#2-alias] {} ; %
  % Connect all inputs to outputs via this factor
  \factoredge {#4} {#2-alias} {#5} ; %
}

% \plate [options] {name} {fitlist} {caption}
\newcommand{\plate}[4][]{ %
  \node[wrap=#3] (#2-wrap) {}; %
  \node[plate caption=#2-wrap] (#2-caption) {#4}; %
  \node[plate=(#2-wrap)(#2-caption), #1] (#2) {}; %
}

% \gate [options] {name} {fitlist} {inputs}
\newcommand{\gate}[4][]{ %
  \node[gate=#3, name=#2, #1, alias=#2-alias] {}; %
  \foreach \x in {#4} { %
    \draw [-*,thick] (\x) -- (#2-alias); %
  } ;%
}

% \vgate {name} {fitlist-left} {caption-left} {fitlist-right}
% {caption-right} {inputs}
\newcommand{\vgate}[6]{ %
  % Wrap the left and right parts
  \node[wrap=#2] (#1-left) {}; %
  \node[wrap=#4] (#1-right) {}; %
  % Draw the gate
  \node[gate=(#1-left)(#1-right)] (#1) {}; %
  % Add captions
  \node[caption, below left=of #1.north ] (#1-left-caption)
  {#3}; %
  \node[caption, below right=of #1.north ] (#1-right-caption)
  {#5}; %
  % Draw middle separation
  \draw [-, dashed] (#1.north) -- (#1.south); %
  % Draw inputs
  \foreach \x in {#6} { %
    \draw [-*,thick] (\x) -- (#1); %
  } ;%
}

% \hgate {name} {fitlist-top} {caption-top} {fitlist-bottom}
% {caption-bottom} {inputs}
\newcommand{\hgate}[6]{ %
  % Wrap the left and right parts
  \node[wrap=#2] (#1-top) {}; %
  \node[wrap=#4] (#1-bottom) {}; %
  % Draw the gate
  \node[gate=(#1-top)(#1-bottom)] (#1) {}; %
  % Add captions
  \node[caption, above right=of #1.west ] (#1-top-caption)
  {#3}; %
  \node[caption, below right=of #1.west ] (#1-bottom-caption)
  {#5}; %
  % Draw middle separation
  \draw [-, dashed] (#1.west) -- (#1.east); %
  % Draw inputs
  \foreach \x in {#6} { %
    \draw [-*,thick] (\x) -- (#1); %
  } ;%
}


 \mode<presentation>
 {
 %   \usetheme{Madrid}      % or try Darmstadt, Madrid, Warsaw, ...
 %   \usecolortheme{default} % or try albatross, beaver, crane, ...
 %   \usefonttheme{serif}  % or try serif, structurebold, ...
  \usetheme{Antibes}
  \setbeamertemplate{navigation symbols}{}
 }
 
\usepackage{todonotes}
\setbeameroption{show notes}

\newif\ifen
\newif\ifes
\newcommand{\en}[1]{\ifen#1\fi}
\newcommand{\es}[1]{\ifes#1\fi}
\estrue

%\title[Bayes del Sur]{}

\begin{document}

\color{black!85}
\large

\begin{frame}[plain,noframenumbering]
 
 \begin{textblock}{90}(00,05)
\begin{center}
 \huge  \textcolor{black!66}{Creencias, datos y sorpresas}
\end{center}
\end{textblock}

 %\vspace{2cm}brown
%\maketitle
\Wider[2cm]{
\includegraphics[width=1\textwidth]{../../auxiliar/static/peligro_predador}
}
\end{frame}




\begin{frame}[plain]

\centering \Large

Evoluci\'on

\end{frame}



\begin{frame}[plain]
\begin{textblock}{160}(0,4)
\centering \Large Evolución de estrategias
\end{textblock}
\vspace{0.75cm}

\centering
\tikz{
    \node[latent] (e) {$p$};
    \node[const, right=of e] (ne) {$p\sim \text{Beta}(\alpha,\beta)$};
    
    \node[latent, below=of e] (r) {$m$};
    \node[const, right=of r] (ne) {$m \sim \text{Binomial}(p)$};
    
    \edge {e} {r};
}
 \vspace{1.5cm}
 \pause
 
 Supongamos que se seleccionó la estrategia $p=1.5/2.1$

 \begin{equation}
 P(p) = \delta(p=1.5/2.1)
 \end{equation}
 
\end{frame}


\begin{frame}[plain]
\begin{textblock}{160}(0,4)
\centering \Large Cambio de ambiente
\end{textblock}
\vspace{0.75cm}

Los datos surgen ahora 50\% y 50\% pero la apuesta de las estrategias sigue siendo la misma (ya no hay diversidad).

\end{frame}

\begin{frame}[plain]

Cómo les va jugando individualmente

Cómo les va a la estrategia óptima

\end{frame}

\begin{frame}[plain]
\begin{textblock}{160}(0,4)
\centering \Large Cambio de ambiente
\end{textblock}
\vspace{0.75cm}


\centering
\tikz{

    \node[latent, minimum size=2cm ] (x1_0) {$x_1(t)$} ;
    \node[latent, below=of x1_0, minimum size=2cm ] (x2_0) {$x_2(t)$} ;

    \node[latent, right=of x1_0, minimum size=3cm ] (x1_0g) {$x_1(t)+\Delta x_1(t)$} ;
    \node[latent, right=of x2_0, minimum size=1.8cm, xshift=0.6cm , align=left] (x2_0g) {$x_2(t)+$\\$\Delta x_2(t)$} ;
    
    \node[latent, right=of x1_0g, minimum size=3.8cm, yshift=-1.33cm, align=right] (x_0) {$x_1(t)+\Delta x_1(t)$\\$+x_2(t)+\Delta x_2(t)$ } ;
    
    \node[const, above=of x_0] (nx_0) {$\overbrace{\text{Pool}\hspace{2.5cm}\text{Share}}^{\text{\normalsize Cooperaci\'on}}$} ;
    
    \node[latent, right=of x1_0g, minimum size=2.5cm,  xshift=4.5cm] (x1_1) {$x_1(t+1)$ } ;
    \node[latent, below=of x1_1, minimum size=2.5cm, yshift=0.7cm] (x2_1) {$x_2(t+1)$ } ;
    
    \edge {x1_0} {x1_0g};
    \edge {x2_0} {x2_0g};
    \edge {x1_0g,x2_0g} {x_0};
    \edge {x_0} {x1_1,x2_1};
    
}

\end{frame}

\begin{frame}[plain]
\begin{textblock}{160}(0,4)
\centering \Large Evolución de grupos
\end{textblock}
\vspace{0.75cm}

\centering
\tikz{
    \node[latent] (m) {$M$};
    
    \node[latent, right=of m] (e0) {$e_0$};
    
    \node[latent, right=of e0] (e1) {$e_1$};
    \node[latent, below=of e1] (r1) {$r_1$};
    
    \node[latent, right=of e1] (e2) {$e_2$};
    \node[latent, below=of e2] (r2) {$r_2$};
    
    \node[latent, right=of e2] (e3) {$e_3$};
    
    
    \edge {m} {e0};
    \edge {e0} {e1};
    \edge {e1} {r1,e2};
    \edge {e2} {r2,e3};
}
\end{frame}

\begin{frame}[plain]
\begin{textblock}{160}(0,4)
\centering \Large Evolución de grupos
\end{textblock}
\vspace{0.75cm}

\centering
\tikz{
    \node[latent] (m) {$M$};
    
    \node[latent, right=of m] (e0) {$e_0$};
    
    \node[latent, right=of e0] (e1) {$e_1$};
    \node[latent, below=of e1] (r1) {$r_1$};
    
    \node[latent, right=of e1] (e2) {$e_2$};
    \node[latent, below=of e2] (r2) {$r_2$};
    
    \node[latent, right=of e2] (e3) {$e_3$};
    
    
    \edge {m} {e0};
    \edge {e0} {e1};
    \edge {e1} {r1,e2};
    \edge {e2} {r2,e3};
}
\end{frame}



\begin{frame}[plain]

\centering
  \includegraphics[width=0.55\textwidth]{../../auxiliar/static/pachacuteckoricancha.jpg}
\end{frame}





\end{document}



