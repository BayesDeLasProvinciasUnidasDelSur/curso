\documentclass[shownotes,aspectratio=169]{beamer}

\input{../../auxiliar/tex/diapo_encabezado.tex}
% tikzlibrary.code.tex
%
% Copyright 2010-2011 by Laura Dietz
% Copyright 2012 by Jaakko Luttinen
%
% This file may be distributed and/or modified
%
% 1. under the LaTeX Project Public License and/or
% 2. under the GNU General Public License.
%
% See the files LICENSE_LPPL and LICENSE_GPL for more details.

% Load other libraries
\usetikzlibrary{shapes}
\usetikzlibrary{fit}
\usetikzlibrary{chains}
\usetikzlibrary{arrows}

% Latent node
\tikzstyle{latent} = [circle,fill=white,draw=black,inner sep=1pt,
minimum size=20pt, font=\fontsize{10}{10}\selectfont, node distance=1]
% Observed node
\tikzstyle{obs} = [latent,fill=gray!25]
% Invisible node
\tikzstyle{invisible} = [latent,minimum size=0pt,color=white, opacity=0, node distance=0]
% Constant node
\tikzstyle{const} = [rectangle, inner sep=0pt, node distance=0.1]
%state
\tikzstyle{estado} = [latent,minimum size=8pt,node distance=0.4]
%action
\tikzstyle{accion} =[latent,circle,minimum size=5pt,fill=black,node distance=0.4]


% Factor node
\tikzstyle{factor} = [rectangle, fill=black,minimum size=10pt, draw=black, inner
sep=0pt, node distance=1]
% Deterministic node
\tikzstyle{det} = [latent, rectangle]

% Plate node
\tikzstyle{plate} = [draw, rectangle, rounded corners, fit=#1]
% Invisible wrapper node
\tikzstyle{wrap} = [inner sep=0pt, fit=#1]
% Gate
\tikzstyle{gate} = [draw, rectangle, dashed, fit=#1]

% Caption node
\tikzstyle{caption} = [font=\footnotesize, node distance=0] %
\tikzstyle{plate caption} = [caption, node distance=0, inner sep=0pt,
below left=5pt and 0pt of #1.south east] %
\tikzstyle{factor caption} = [caption] %
\tikzstyle{every label} += [caption] %

\tikzset{>={triangle 45}}

%\pgfdeclarelayer{b}
%\pgfdeclarelayer{f}
%\pgfsetlayers{b,main,f}

% \factoredge [options] {inputs} {factors} {outputs}
\newcommand{\factoredge}[4][]{ %
  % Connect all nodes #2 to all nodes #4 via all factors #3.
  \foreach \f in {#3} { %
    \foreach \x in {#2} { %
      \path (\x) edge[-,#1] (\f) ; %
      %\draw[-,#1] (\x) edge[-] (\f) ; %
    } ;
    \foreach \y in {#4} { %
      \path (\f) edge[->,#1] (\y) ; %
      %\draw[->,#1] (\f) -- (\y) ; %
    } ;
  } ;
}

% \edge [options] {inputs} {outputs}
\newcommand{\edge}[3][]{ %
  % Connect all nodes #2 to all nodes #3.
  \foreach \x in {#2} { %
    \foreach \y in {#3} { %
      \path (\x) edge [->,#1] (\y) ;%
      %\draw[->,#1] (\x) -- (\y) ;%
    } ;
  } ;
}

% \factor [options] {name} {caption} {inputs} {outputs}
\newcommand{\factor}[5][]{ %
  % Draw the factor node. Use alias to allow empty names.
  \node[factor, label={[name=#2-caption]#3}, name=#2, #1,
  alias=#2-alias] {} ; %
  % Connect all inputs to outputs via this factor
  \factoredge {#4} {#2-alias} {#5} ; %
}

% \plate [options] {name} {fitlist} {caption}
\newcommand{\plate}[4][]{ %
  \node[wrap=#3] (#2-wrap) {}; %
  \node[plate caption=#2-wrap] (#2-caption) {#4}; %
  \node[plate=(#2-wrap)(#2-caption), #1] (#2) {}; %
}

% \gate [options] {name} {fitlist} {inputs}
\newcommand{\gate}[4][]{ %
  \node[gate=#3, name=#2, #1, alias=#2-alias] {}; %
  \foreach \x in {#4} { %
    \draw [-*,thick] (\x) -- (#2-alias); %
  } ;%
}

% \vgate {name} {fitlist-left} {caption-left} {fitlist-right}
% {caption-right} {inputs}
\newcommand{\vgate}[6]{ %
  % Wrap the left and right parts
  \node[wrap=#2] (#1-left) {}; %
  \node[wrap=#4] (#1-right) {}; %
  % Draw the gate
  \node[gate=(#1-left)(#1-right)] (#1) {}; %
  % Add captions
  \node[caption, below left=of #1.north ] (#1-left-caption)
  {#3}; %
  \node[caption, below right=of #1.north ] (#1-right-caption)
  {#5}; %
  % Draw middle separation
  \draw [-, dashed] (#1.north) -- (#1.south); %
  % Draw inputs
  \foreach \x in {#6} { %
    \draw [-*,thick] (\x) -- (#1); %
  } ;%
}

% \hgate {name} {fitlist-top} {caption-top} {fitlist-bottom}
% {caption-bottom} {inputs}
\newcommand{\hgate}[6]{ %
  % Wrap the left and right parts
  \node[wrap=#2] (#1-top) {}; %
  \node[wrap=#4] (#1-bottom) {}; %
  % Draw the gate
  \node[gate=(#1-top)(#1-bottom)] (#1) {}; %
  % Add captions
  \node[caption, above right=of #1.west ] (#1-top-caption)
  {#3}; %
  \node[caption, below right=of #1.west ] (#1-bottom-caption)
  {#5}; %
  % Draw middle separation
  \draw [-, dashed] (#1.west) -- (#1.east); %
  % Draw inputs
  \foreach \x in {#6} { %
    \draw [-*,thick] (\x) -- (#1); %
  } ;%
}


\usepackage{listings}
\lstset{
  aboveskip=3mm,
  belowskip=3mm,
  showstringspaces=true,
  columns=flexible,
  basicstyle={\ttfamily},
  breaklines=true,
  breakatwhitespace=true,
  tabsize=4,
  showlines=true
}

\definecolor{all}{rgb}{0.90, 0.90, 0.90}

 \mode<presentation>
 {
 %   \usetheme{Madrid}      % or try Darmstadt, Madrid, Warsaw, ...
 %   \usecolortheme{default} % or try albatross, beaver, crane, ...
 %   \usefonttheme{serif}  % or try serif, structurebold, ...
  \usetheme{Antibes}
  \setbeamertemplate{navigation symbols}{}
 }
 
\usepackage{todonotes}
\setbeameroption{show notes}

\newif\ifen
\newif\ifes
\newcommand{\en}[1]{\ifen#1\fi}
\newcommand{\es}[1]{\ifes#1\fi}
\estrue

%\title[Bayes del Sur]{}

\begin{document}

\color{black!85}
\large

\begin{frame}[plain,noframenumbering]
%
%  \begin{textblock}{90}(00,05)
% \begin{center}
%  \huge  \textcolor{black!66}{Conocimiento empírico}
% \end{center}
% \end{textblock}

 \begin{textblock}{47}(113,74)
\centering \Large  \textcolor{white!55}{Evaluación de modelos}
\end{textblock}

 %\vspace{2cm}brown
%\maketitle
\Wider[2cm]{
\includegraphics[width=1\textwidth]{../../auxiliar/static/peligro_predador}
}
\end{frame}

%
% \begin{frame}[plain]
% \begin{textblock}{160}(0,4)
%  \centering \Large Modelos alternativos \\
% \end{textblock}
% \vspace{1cm}
%
% \centering
%
% Heliocéntrico \hspace{1.2cm} Geocéntrico
%
% \includegraphics[width=0.5\textwidth]{../../auxiliar/static/geocentrico-heliocentrico-fourier.png}
%
% \end{frame}
%
% \begin{frame}[plain]
% \begin{textblock}{160}(0,4)
%  \centering \Large Epiciclos \\
% \end{textblock}
% \vspace{0.75cm}
%
% \centering
%
% \includegraphics[width=0.45\textwidth]{../../auxiliar/static/epiciclos.png}
%
% \end{frame}
%
%
% \begin{frame}[plain]
%
% \centering
% \includegraphics[width=1\textwidth]{../../auxiliar/static/geocentrico-heliocentrico.png}
%
% \vspace{0.5cm}
% Con epiciclos, ambos predicen las mismas distancias relativas a la tierra
%
% \end{frame}

%
% \begin{frame}[plain]
%
% \vspace{1.5cm}
%
%  \begin{center}
%  \Large ¿Cómo decidir entonces entre modelos causales alternativos?
%  \end{center}
%
%  \vspace{0.5cm}
%  \pause
%
%  \begin{description}
%  \item[\textbf{Conocimiento previo:}] sabemos que alguno es imposible \pause
%  \item[\textbf{Predicción:}] cuán bien predicen los datos (distancia, luz, otros) \pause
%  \item[\textbf{Simplicidad:}] Si predicen igual, preferimos el más simple
%  \end{description}
%
%
%
% \end{frame}




\begin{frame}[plain]
\begin{textblock}{160}(0,4)
 \centering \Large Evaluación de modelos \\
\end{textblock}

\begin{textblock}{160}(16,24)
\begin{equation*}
 P(\text{Modelo}|\text{Datos}) = \frac{\only<1-2>{P(\text{Datos}|\text{Modelo})}\only<3->{\overbrace{P(\text{Datos}|\text{Modelo})}^{\text{Evidencia}}}  \only<1>{\,P(\text{Modelo})} \only<2->{\overbrace{P(\text{Modelo})}^{\text{Prior}}} }{ P(\text{Datos})} \phantom{\frac{\overbrace{P(\text{Datos}|\text{Modelo})}^{\text{Evidencia}}}{ P(\text{Datos})}}
\end{equation*}
\end{textblock}

\only<3>{
\begin{textblock}{160}(0,47)
\begin{align*}
P(\text{Datos}|\text{Modelo}) & = \sum_{i} P(\text{Datos}|\text{Hip\'otesis}_i,\text{Modelo}) P(\text{Hip\'otesis}_i|\text{Modelo})
\end{align*}
\end{textblock}
}

\only<4>{
\begin{textblock}{160}(0,47)
\begin{align*}
P(\text{Datos}|\text{Modelo}) & = \sum_{i} P(\text{Datos}|\text{Hip\'otesis}_i,\text{Modelo}) P(\text{Hip\'otesis}_i|\text{Modelo}) \\
& = P(\text{D}_1|\text{M}) P(\text{D}_2|\text{D}_1, \text{M}) \dots P(\text{D}_N|\text{D}_{N-1} \dots \text{D}_1, \text{M})
\end{align*}
\end{textblock}
}

\only<5>{
\begin{textblock}{160}(0,50)
\begin{equation*}
 P(\text{Datos}) = \sum_i P(\text{Datos}|\text{Modelo}_i)P(\text{Modelo}_i)
\end{equation*}
\end{textblock}
}



% \only<7>{
% \begin{textblock}{128}(0,92)
% \centering \tiny Sobre comparaci\'on de modelos ver: \href{http://xyala.cap.ed.ac.uk/teaching/tutorials/phylogenetics/Bayesian_Workshop/PDFs/Kass\%20and\%20Raftery\%201995.pdf}{Kass \& Raftery. Bayes factors. 1995.}
% \end{textblock}
% }
\end{frame}

%
% \begin{frame}[plain]
%
% \vspace{0.75cm}
% \centering
% \includegraphics[width=0.8\textwidth]{figures/monty_hall_selection.pdf} \hspace{1cm}
%
% \end{frame}

\begin{frame}[plain]

\centering

\Large Modelos lineales

\end{frame}

\begin{frame}[plain]
\begin{textblock}{160}(0,4)
\centering \Large Modelos lineales
\end{textblock}
 \vspace{1.25cm}
 
\begin{equation*}
y(\bm{x},\bm{w}) = \sum_{i=0}^{M-1} w_i \phi_i(\bm{x}) = \bm{w}^T \bm{\phi}(\bm{x})
\end{equation*}

\vspace{0.5cm}
\pause
% 
% \begin{equation*}
%  t = y(\vm{x},\vm{w}) + \epsilon  \ \ \ \ \  \text{con} \ \ \  \epsilon \sim \N(0,\beta^{-1})
% \end{equation*}
 
 \begin{equation*}
p(t \, | \, \bm{x}, \bm{w}, \beta) = \N(t \, | \, y(\bm{x},\bm{w}) , \beta^{-1})
\end{equation*}
\vspace{0.025cm}
\pause
 
\begin{equation*}
P(\bm{t} | \bm{x}, \bm{w}, \beta) = \prod_{i=1}^n \N(t_i | \bm{w}^T \bm{\phi}(\bm{x}_i) , \beta^{-1}) = \N(\bm{t}|\bm{w}^T \bm{\Phi}, \beta^{-1} \vm{I})
\end{equation*}
\vspace{0.05cm}
\pause

\begin{equation*}
 \bm{\Phi} =
  \begin{pmatrix}
    \phi_0(\bm{x}_1) & \phi_1(\bm{x}_1) & \dots & \phi_{M-1}(\bm{x}_1)\\
    \vdots & \vdots & \ddots & \vdots \\
    \phi_0(\bm{x}_N) & \phi_1(\bm{x}_N) & \dots & \phi_{M-1}(\bm{x}_N)
  \end{pmatrix}
  = 
  \begin{pmatrix}
   \bm{\phi}(\vm{x}_1)^T \\
   \vdots \\
   \bm{\phi}(\vm{x}_N)^T \\
  \end{pmatrix}
\end{equation*}

 
\end{frame}



\begin{frame}[plain]
\begin{textblock}{160}(0,4)
 \centering \Large Solución fecuentista
\end{textblock}
\vspace{1.25cm}


\begin{equation*}
 \underset{\bm{w}}{\text{ max }} P(\bm{t} | \bm{x}, \bm{w}, \beta) = \underset{\bm{w}}{\text{ min }} \sum_{i=1}^{n}  (t_i - \bm{w}^T\bm{\phi}(\bm{x}_i))^2 
\end{equation*}

\end{frame}

\begin{frame}[plain]
\begin{textblock}{160}(0,4)
 \Large \centering Soluci\'on completa
\end{textblock}
 \vspace{1cm}

\begin{equation*}
 p(\vm{w}) = N(\vm{w}| \vm{0}, \alpha^{-1} \vm{I})
\end{equation*}
 
\begin{figure}[H]
    \centering
    \tikz{

    \node[latent, fill=black!100, minimum size=2pt] (x) {} ; %
    \node[const, right=of x] (c_x) {$\vm{X}$};
    \node[latent, fill=black!20, yshift=-1.5cm] (t) {$\bm{t}$} ; %
    \node[latent, fill=black!100, yshift=-1.5cm , xshift=-2cm,minimum size=2pt] (beta)
    {} ; %
    \node[const, above=of beta] (c_beta) {$\beta$};
    \node[latent, fill=black!0, yshift=-1.5cm, xshift=2cm] (w) {$\vm{w}$};
    \node[latent, fill=black!100, xshift=2cm, minimum size=2pt] (alpha) {} ; %
    \node[const, right=of alpha] (c_alpha) {$\alpha$};
    
    \edge {x,beta,w} {t};
    \edge {alpha} {w};
    
    \node[invisible, fill=black!0, minimum size=0pt, xshift=-0.52cm] (data_inv) {} ; %
      
    }  
\end{figure}
\end{frame}


\begin{frame}[plain]
\begin{textblock}{160}(0,4)
 \Large \centering Soluci\'on completa
\end{textblock}
 \vspace{0.75cm}

La distribuci\'on \textbf{posteriori} sobre $\vm{w}$ tiene como media

\begin{equation}
 \vm{m}_N = \beta  \vm{S}_N\vm{\Phi}^T \vm{t}
\end{equation}

y como covarianzas

\begin{equation}
 \vm{S}_N^{-1} = \alpha \vm{I} + \beta \vm{\Phi}^T\vm{\Phi}
\end{equation}

\pause

Y la \textbf{evidencia} del modelo es

\begin{equation}
 P(t) = \N(\bm{t}| \vm{0}, \beta^{-1} \vm{I} + \alpha^{-1}\bm{\Phi}\bm{\Phi}^T  )
\end{equation}

\vspace{0.75cm}

\small
\Wider[-8cm]{
\begin{mdframed}[backgroundcolor=black!15]\centering
 Cap\'itulo 2 de Bishop
\end{mdframed}
}

\end{frame}

% 
% \begin{frame}
% \begin{textblock}{128}(0,8)
% \begin{center}
%  \normalsize Propiedades importantes para la vida
% \end{center}
% \end{textblock}
% \vspace{0.75cm}
% 
% \tiny
% 
% Dadas las distribuciones
% \begin{align*}
%     P(\vm{x}) &=  \N(\vm{x}|\bm{\mu},\bm{\Lambda}^{-1}) \\
%    P(\vm{y}|\vm{x}) &=  \N(\vm{y}|\vm{A}\vm{x}+\vm{b},\vm{L}^{-1})    
% \end{align*}
% 
% 
%  Luego,  
% \begin{align*}
%     P(\vm{y}) &=  \N(\vm{y}|\vm{A}\bm{\mu}+\vm{b},\, \vm{L}^{-1} + \vm{A}\vm{\Lambda}^{-1}\vm{A}^T) \\
%    P(\vm{x}|\vm{y}) &=  \N(\vm{x}|\vm{\Sigma}[\vm{A}^T \vm{L}(\vm{y}-\vm{b})+ \bm{\Lambda\mu}],\,  \vm{\Sigma})\label{eq:post}
% \end{align*}
% 
% donde, 
% \begin{equation*}
%  \vm{\Sigma} = (\vm{\Lambda} + \vm{A}^T \vm{LA} )^{-1}
% \end{equation*}
% 
% \vspace{0.3cm}
% 
% \small
% \begin{mdframed}[backgroundcolor=black!15]\centering
%  Leer cap\'itulo 2 de Bishop y/o anexo de la pr\'actica
% \end{mdframed}
% 
% \end{frame}


\begin{frame}[plain]

\Wider[-3cm]{
 \begin{figure}
\begin{subfigure}[t]{0.32\textwidth} 
\caption*{Verosimilitud} 
\end{subfigure}
\begin{subfigure}[t]{0.32\textwidth}
\caption*{Priori/Posteriori} 
\includegraphics[width=\textwidth]{figures/pdf/linearRegression_posterior_0.pdf} 
\end{subfigure}
\begin{subfigure}[t]{0.32\textwidth}
\caption*{Data space} 
\includegraphics[width=\textwidth]{figures/pdf/linearRegression_dataSpace_0.pdf} 
\end{subfigure}


\begin{subfigure}[c]{0.32\textwidth}
\onslide<2->{\includegraphics[width=\textwidth]{figures/pdf/linearRegression_likelihood_1.pdf}}
\end{subfigure}
\begin{subfigure}[c]{0.32\textwidth}
\onslide<3->{\includegraphics[width=\textwidth]{figures/pdf/linearRegression_posterior_1.pdf}}
\end{subfigure}
\begin{subfigure}[c]{0.32\textwidth}
\onslide<4->{\includegraphics[width=\textwidth]{figures/pdf/linearRegression_dataSpace_1.pdf}}
\end{subfigure}

\begin{subfigure}[c]{0.32\textwidth}
\onslide<5->{\includegraphics[width=\textwidth]{figures/pdf/linearRegression_likelihood_2.pdf}}
\end{subfigure}
\begin{subfigure}[c]{0.32\textwidth}
\onslide<6->{\includegraphics[width=\textwidth]{figures/pdf/linearRegression_posterior_2.pdf}}
\end{subfigure}
\begin{subfigure}[c]{0.32\textwidth}
\onslide<7->{\includegraphics[width=\textwidth]{figures/pdf/linearRegression_dataSpace_2.pdf}}
\end{subfigure}

\end{figure}
}
\end{frame}


\begin{frame}[plain]
\begin{textblock}{160}(0,4)
\centering  \Large Regresi\'on lineal Bayesiana
\end{textblock}


\begin{textblock}{80}(0,20)
 \centering
 \onslide<1->{Funci\'on objetivo}
\end{textblock}

\begin{textblock}{80}(80,20)
 \centering
 \onslide<2>{Modelos polinomiales}
\end{textblock}

\begin{textblock}{160}(0,26)
     \centering 
       \onslide<1->{\includegraphics[width=0.45\textwidth]{figures/pdf/model_selection_true_and_sample} }
       \onslide<2>{\includegraphics[width=0.45\textwidth]{figures/pdf/model_selection_MAP_non-informative} }
\end{textblock}

\end{frame}

\begin{frame}[plain]
\begin{textblock}{160}(0,4)
 \centering \Large Evidencia vs Verosimilitud
\end{textblock}


\begin{textblock}{80}(0,22)
 \centering
Evidencia conjunta
\end{textblock}

\begin{textblock}{80}(80,22)
 \centering
 M\'axima verosimilitud
\end{textblock}


\begin{textblock}{160}(0,24)
     \centering 
       \begin{figure}[H]     
     \centering 
     \begin{subfigure}[b]{0.47\textwidth}
       \includegraphics[width=1\textwidth]{figures/pdf/model_selection_evidence}
     \end{subfigure}
     \begin{subfigure}[b]{0.49\textwidth}
       \includegraphics[width=1\textwidth]{figures/pdf/model_selection_maxLikelihood}
     \end{subfigure}
\end{figure}
\end{textblock}

\end{frame}


% \begin{frame}
%  \begin{textblock}{108}(10,08)
%  \begin{center}
%   \large Evidencia
%  \end{center}
% \end{textblock}
% 
% \begin{textblock}{108}(10,22)
% \begin{align*}
%  P(\text{C}|\text{D},\text{M}) = \frac{P(\text{D}|\text{C},\text{M})P(\text{C}|\text{M})}{\underbrace{P(\text{D}|\text{M})}_{\text{Evidencia}}}
% \end{align*}
% \end{textblock}
% 
% 
% 
% \only<2>{
% \begin{textblock}{108}(10,44)
% \begin{align*}
%  P(\text{D}|\text{M}) = \sum_C P(\text{D}|\text{C},\text{M})\, P(\text{C}|\text{M})
% \end{align*}
% \end{textblock}
% 
% }
% 
% 
% \only<3>{
% \begin{textblock}{108}(10,44)
% \begin{align*}
%  P(\text{D}|\text{M}) = \sum_C \underbrace{P(\text{D}|\text{C},\text{M})}_{\hfrac{\text{\tiny Predicci\'on}}{\text{\tiny de D dado C}}}\underbrace{P(\text{C}|\text{M})}_{\hfrac{\text{\tiny Creencia de}}{\text{\tiny C a Priori}}} %\text{La predicci\'on de los datos observados a partir de las creencias a priori}
% \end{align*}
% \end{textblock}
% 
% }
% 
% \only<4->{
% \begin{textblock}{108}(10,44)
% \begin{align*}
%  P(\text{D}|\text{M}) = \underbrace{ \sum_C \underbrace{P(\text{D}|\text{C},\text{M})}_{\hfrac{\text{\tiny Predicci\'on}}{\text{\tiny de D dado C}}}\underbrace{P(\text{C}|\text{M})}_{\hfrac{\text{\tiny Creencia de}}{\text{\tiny C a Priori}}}}_{\hfrac{\text{\tiny Predicci\'on de la datos observados}}{\text{\tiny pesando todas las creencias a priori}}}
% \end{align*}
% \end{textblock}
% }
% 
% 
% 
% 
% \end{frame}



\begin{frame}[plain]
\begin{textblock}{160}(0,4)
\centering \Large  Evidencia
\end{textblock}


 \begin{textblock}{120}(20,10)
  \centering
  \includegraphics[width=0.9\textwidth]{figures/pdf/evidencia_de_modelos_alternativos} 
 \end{textblock} 
 
 
 \begin{textblock}{100}(30,83)
  \centering
 \ \ \ \  Balance natural entre complejidad y predicci\'on
  \end{textblock}
\end{frame}

% 
% \begin{frame}[plain]
% 
% \centering
% \includegraphics[width=0.50\textwidth]{../../auxiliar/static/regularizador-meme.jpeg} 
% 
% \end{frame}
% 
% \begin{frame}[plain]
% \begin{textblock}{160}(0,4)
% \centering \Large Regularizador frecuentista L2
% \end{textblock}
% 
% 
% \begin{equation*}
%  \underset{\bm{w}}{\text{ max }} P(\bm{t} | \bm{x}, \bm{w}, \beta) P(\vm{w}) = \underset{\bm{w}}{\text{ min }} \sum_{i=1}^{n}  (t_i - \bm{w}^T\bm{\phi}(\bm{x}_i))^2 + \sum_{i=0}^{M-1} (w_i - 0 )^2
% \end{equation*}
% 
% \todo[inline]{verificar el termino de regularizacion}
% 
% 
% 
% \end{frame}
% 
% 


\begin{frame}[plain]
\begin{textblock}{160}(0,4)
\centering \Large  Prior
\end{textblock}
\vspace{0.75cm}

       \begin{figure}[H]     
     \centering 
     \begin{subfigure}[b]{0.45\textwidth}
    \centering Informativo
    \includegraphics[width=1\textwidth]{figures/pdf/model_selection_MAP_informative} 
     \end{subfigure}
       \begin{subfigure}[b]{0.45\textwidth}
     \centering 
     No informativo
    \includegraphics[width=1\textwidth]{figures/pdf/model_selection_MAP_non-informative} 
   \end{subfigure}
\end{figure}

\end{frame}


\begin{frame}[plain]
\begin{textblock}{160}(0,4)
\centering \Large Prior no-informativo como regularizador
\end{textblock}
\vspace{0.75cm}

\centering
\includegraphics[width=0.66\textwidth]{figures/pdf/regularizador} 


\end{frame}

%
% \begin{frame}[plain]
%
% \centering \Large
%
% Evoluci\'on
%
% \end{frame}
%
%
%
% \begin{frame}[plain]
% \begin{textblock}{160}(0,4)
% \centering \Large Evolución de estrategias
% \end{textblock}
% \vspace{0.75cm}
%
% \centering
% \tikz{
%     \node[latent] (e) {$p$};
%     \node[const, right=of e] (ne) {$p\sim \text{Beta}(\alpha,\beta)$};
%
%     \node[latent, below=of e] (r) {$m$};
%     \node[const, right=of r] (ne) {$m \sim \text{Binomial}(p)$};
%
%     \edge {e} {r};
% }
%  \vspace{1.5cm}
%  \pause
%
%  Supongamos que se seleccionó la estrategia $p=1.5/2.1$
%
%  \begin{equation}
%  P(p) = \delta(p=1.5/2.1)
%  \end{equation}
%
% \end{frame}
%
%
% \begin{frame}[plain]
% \begin{textblock}{160}(0,4)
% \centering \Large Cambio de ambiente
% \end{textblock}
% \vspace{0.75cm}
%
% Los datos surgen ahora 50\% y 50\% pero la apuesta de las estrategias sigue siendo la misma (ya no hay diversidad).
%
% \end{frame}
%
% \begin{frame}[plain]
%
% Cómo les va jugando individualmente
%
% Cómo les va a la estrategia óptima
%
% \end{frame}
%
% \begin{frame}[plain]
% \begin{textblock}{160}(0,4)
% \centering \Large Cambio de ambiente
% \end{textblock}
% \vspace{0.75cm}
%
%
% \centering
% \tikz{
%
%     \node[latent, minimum size=2cm ] (x1_0) {$x_1(t)$} ;
%     \node[latent, below=of x1_0, minimum size=2cm ] (x2_0) {$x_2(t)$} ;
%
%     \node[latent, right=of x1_0, minimum size=3cm ] (x1_0g) {$x_1(t)+\Delta x_1(t)$} ;
%     \node[latent, right=of x2_0, minimum size=1.8cm, xshift=0.6cm , align=left] (x2_0g) {$x_2(t)+$\\$\Delta x_2(t)$} ;
%
%     \node[latent, right=of x1_0g, minimum size=3.8cm, yshift=-1.33cm, align=right] (x_0) {$x_1(t)+\Delta x_1(t)$\\$+x_2(t)+\Delta x_2(t)$ } ;
%
%     \node[const, above=of x_0] (nx_0) {$\overbrace{\text{Pool}\hspace{2.5cm}\text{Share}}^{\text{\normalsize Cooperaci\'on}}$} ;
%
%     \node[latent, right=of x1_0g, minimum size=2.5cm,  xshift=4.5cm] (x1_1) {$x_1(t+1)$ } ;
%     \node[latent, below=of x1_1, minimum size=2.5cm, yshift=0.7cm] (x2_1) {$x_2(t+1)$ } ;
%
%     \edge {x1_0} {x1_0g};
%     \edge {x2_0} {x2_0g};
%     \edge {x1_0g,x2_0g} {x_0};
%     \edge {x_0} {x1_1,x2_1};
%
% }
%
% \end{frame}
%
% \begin{frame}[plain]
% \begin{textblock}{160}(0,4)
% \centering \Large Evolución de grupos
% \end{textblock}
% \vspace{0.75cm}
%
% \centering
% \tikz{
%     \node[latent] (m) {$M$};
%
%     \node[latent, right=of m] (e0) {$e_0$};
%
%     \node[latent, right=of e0] (e1) {$e_1$};
%     \node[latent, below=of e1] (r1) {$r_1$};
%
%     \node[latent, right=of e1] (e2) {$e_2$};
%     \node[latent, below=of e2] (r2) {$r_2$};
%
%     \node[latent, right=of e2] (e3) {$e_3$};
%
%
%     \edge {m} {e0};
%     \edge {e0} {e1};
%     \edge {e1} {r1,e2};
%     \edge {e2} {r2,e3};
% }
% \end{frame}
%
% \begin{frame}[plain]
% \begin{textblock}{160}(0,4)
% \centering \Large Evolución de grupos
% \end{textblock}
% \vspace{0.75cm}
%
% \centering
% \tikz{
%     \node[latent] (m) {$M$};
%
%     \node[latent, right=of m] (e0) {$e_0$};
%
%     \node[latent, right=of e0] (e1) {$e_1$};
%     \node[latent, below=of e1] (r1) {$r_1$};
%
%     \node[latent, right=of e1] (e2) {$e_2$};
%     \node[latent, below=of e2] (r2) {$r_2$};
%
%     \node[latent, right=of e2] (e3) {$e_3$};
%
%
%     \edge {m} {e0};
%     \edge {e0} {e1};
%     \edge {e1} {r1,e2};
%     \edge {e2} {r2,e3};
% }
% \end{frame}
%

\begin{frame}[plain]
\centering
  \includegraphics[width=0.35\textwidth]{../../auxiliar/static/pachacuteckoricancha.jpg}
\end{frame}





\end{document}



