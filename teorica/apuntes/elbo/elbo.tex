
\begin{equation}
\begin{split}
 \log p(x) & = \log p(x) \int q(\theta) \, d\theta =  \int q(\theta) \log p(x) \, d\theta \\
 & = \int q(\theta) \log \frac{p(x,\theta)}{p(\theta \mid x)} \, d\theta =  \int q(\theta) \log \frac{p(x,\theta)q(\theta)}{p(\theta \mid x)q(\theta)} \, d\theta \\
 & = \underbrace{\int q(\theta) \log \frac{p(x,\theta)}{q(\theta)} \, d\theta}_{\text{Evidence lower bound , } \mathcal{L}(q(\theta)) } + \underbrace{\int q(\theta) \log \frac{q(\theta)}{p(\theta \mid x)} \, d\theta}_{\text{KL}(q(\theta),p(\theta|x))} 
\end{split}
\end{equation}

\todo[inline]{why the name ELBO}

\paragraph{Optimization}

\begin{equation}
  \log p(x) =  \mathcal{L}(q(\theta))  + \text{KL}(q(\theta),p(\theta|x))
\end{equation}

The left side does not depend on $q$.
The right side depends on $q$.
We want the $q$ that minimize the KL divergence.
Because de left side is constant on $q$, minimize the KL divergence is the same to maximize the ELBO.

\begin{equation}
 \underset{q(\theta) \in \mathcal{Q}}{\text{argmin}} \ \text{KL}(q(\theta) || p(\theta|x)) \Leftrightarrow \underset{q(\theta) \in \mathcal{Q}}{\text{argmax}} \mathcal{L}(q(\theta)) 
\end{equation}

\paragraph{Final optimization problem}

This is very nice since in $\mathcal{L}(q(\theta)) $ all is computable.
We select $q$ to be the tractable family.
And we always can compute the joint $p(x,\theta)$, because this is our probabilistic model that can be decomposed into de prior $p(\theta)$ and the liklihood $p(x\mid \theta)$.
The likelihood funtion is always given in machine learning problems.
And prior distributions we seted ourselfs.

\vspace{0.3cm}

However we have an integral here. 

\begin{equation}
 \begin{split}
  \mathcal{L}(q(\theta)) & = \int q(\theta) \log \frac{p(x,\theta)}{q(\theta)} \, d\theta = \int q(\theta) \log \frac{p(x\mid \theta)p(\theta)}{q(\theta)} \, d\theta \\
  & = \int q(\theta) \log p(x|\theta) \, d\theta + \int q(\theta) \log \frac{p(\theta)}{q(\theta)} \, d\theta  \\
  & = \mathbb{E}_{q(\theta)} \log p(x \mid \theta) - \text{KL}(q(\theta)||p(\theta))
 \end{split}
\end{equation}

If we have acces to any $q$ and we want to maximize the \textbf{first term} of the right side, we will get de $q(\theta) = \delta(\theta - \theta_{ML})$, because with this distribution we put all the mass over point where likelihood is maximum.

\begin{equation}
  \underset{q(\theta)}{\text{argmax }} \mathbb{E}_{q(\theta)} \log p(x \mid \theta) = \delta(\theta - \theta_{ML})
\end{equation}
 
The first term encourage us to set $q(\theta)$ as the maximum likelihood estimation.

\vspace{0.3cm}

Again, If we have acces to any $q$ and we want to maximize the \textbf{second term} of the right side, we will end up with $q(\theta) = p(\theta)$, because the KL is always non negative and equal to 0 only in this especial case.

\begin{equation}
  \underset{q(\theta)}{\text{argmax }} - \text{KL}(q(\theta)||p(\theta)) = p(\theta)
\end{equation}

The second term encourage us to set $q(\theta)$ as the prior $p(\theta)$.

\vspace{0.3cm}

The two terms together is a compromise between likelihood maximum estimation and our prior distribution.
Indeed, if we have acces to any $q$ to maximize the hole term, we will end up with the posterior distribution $q(\theta \mid x)$.
Because we know that maximization of $\mathcal{L}(q(\theta))$ is equivalent to minimzation of $\text{KL}(q(\theta)||p(\theta\mid x))$, and is zero only when $q(\theta) = p(\theta \mid x)$.

\vspace{0.3cm}

This is not posible because the true posterior is untractable by assumption, so we need restrict the family of distributions $\mathcal{Q}$.

\vspace{0.3cm}

We can see this decompositon as a generalization of Bayes theorem because it stablished the same compromise.

\begin{framed}
 \centering
 How to perform an optimization with respect to (w.r.t) a distribution family?
\end{framed}

\begin{description}
 \item[Mean field approximation]: Factorized familiy
 
 \begin{equation}
  q(\theta) = \prod_{j=1}^m q_j(\theta_j), \ \ \ \ \theta = [\theta_1, \dots, \theta_m] \ , \ \  \forall_{ij} \theta_i \cap \theta_j = \emptyset
 \end{equation}

 \item[Parametric approximation]: Parametric family
 
 \begin{equation}
  q(\theta) = q(\theta \mid \lambda)
 \end{equation}

 
\end{description}












