Queremos resolver la integral

\begin{equation}
 f(x) = \int N(y;\mu_1,\sigma_1^2)\Phi(y+x;\mu_2,\sigma_2^2) dy
\end{equation}

Para ello trabajamos con la drivada $\frac{\partial}{\partial x}f(x) = \theta(x)$,
\begin{equation}
 \theta(x) = \frac{\partial}{\partial x}\int N(y;\mu_1,\sigma_1^2)\Phi(y+x;\mu_2,\sigma_2^2) dy
\end{equation}

Por ``Dominated convergence theorem, integrales y derivadas pueden intercambiar posiciones.
\begin{equation}
 \theta(x) = \int N(y;\mu_1,\sigma_1^2)\frac{\partial}{\partial x}\Phi(y+x;\mu_2,\sigma_2^2) dy
\end{equation}

La derivada de $\Phi$ es justamente una normal,
\begin{equation}
\begin{split}
\theta(x) & = \int N(y;\mu_1,\sigma_1^2)N(y+x;\mu_2,\sigma_2^2) dy \\
& = \int N(y;\mu_1,\sigma_1^2)N(y;\mu_2-x,\sigma_2^2) dy
\end{split}
\end{equation}

Por la demostraci\'on de la secci\'on~\ref{multiplicacion_normales} sabemos
\begin{equation}
 \theta(x) = N(\mu_1; \mu_2 - x, \sigma_1^2 + \sigma_2^2)
\end{equation}

Por simetr\'ia
\begin{equation}
 \theta(x) = N(x; \mu_2 - \mu_1, \sigma_1^2 + \sigma_2^2)
\end{equation}

Retornando a $f(x)$
\begin{equation}
 f(x) = \Phi(x; \mu_2 - \mu_1, \sigma_1^2 + \sigma_2^2)
\end{equation}

