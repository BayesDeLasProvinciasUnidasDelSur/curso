\newif\ifen
\newif\ifes
\newif\iffr
\newcommand{\fr}[1]{\iffr#1 \fi}
\newcommand{\En}[1]{\ifen#1\fi}
\newcommand{\Es}[1]{\ifes#1\fi}
\estrue
\documentclass[shownotes,aspectratio=169]{beamer}

\usepackage{siunitx}
\usepackage{transparent}

\input{../../auxiliar/tex/diapo_encabezado.tex}
\input{../../auxiliar/tex/tikzlibrarybayesnet.code.tex}

\newcommand{\Arrow}[1]{%
\parbox{#1}{\tikz{\draw[->](0,0)--(#1,0);}}
}

\mode<presentation>
 {
 %   \usetheme{Madrid}      % or try Darmstadt, Madrid, Warsaw, ...
 %   \usecolortheme{default} % or try albatross, beaver, crane, ...
 %   \usefonttheme{serif}  % or try serif, structurebold, ...
  \usetheme{Antibes}
  \setbeamertemplate{navigation symbols}{}
 }
\estrue
\usepackage{todonotes}
\setbeameroption{show notes}
%
\newcommand{\gray}{\color{black!55}}
\usepackage{ulem} % sout
\usepackage{mdframed}
\usepackage{listings}
\lstset{
  aboveskip=3mm,
  belowskip=3mm,
  showstringspaces=true,
  columns=flexible,
  basicstyle={\ttfamily},
  breaklines=true,
  breakatwhitespace=true,
  tabsize=4,
  showlines=true
}


\begin{document}

\color{black!85}
\large


\begin{frame}[plain,noframenumbering]

\begin{textblock}{160}(0,0)
\includegraphics[width=1\textwidth]{../../auxiliar/static/peligro_predador}
\end{textblock}

\begin{textblock}{160}(127,67)
\LARGE \textcolor{black!5}{\fontsize{22}{0}\selectfont \textbf{Inferencia  \\[-0.1cm] \hspace{0.5cm} causal}}
\end{textblock}

\begin{textblock}{55}(2,3)
\begin{turn}{0}
\parbox{15cm}{\small
\textcolor{black!95}{Los niveles del razonamiento causal. Flujos de inferencia en}\\
\textcolor{black!95}{modelos causales. Efecto de las intervenciones en modelos} \\
\textcolor{black!95}{causales. Conclusiones causales a partir de observaciones.} \\
\textcolor{black!95}{Identificación de modelo causal mediante intervenciones.} \\
\normalsize\textcolor{black!95}{Unidad 2} \\
}
\end{turn}
\end{textblock}

\end{frame}



%
% \begin{frame}[plain,noframenumbering]
%
% \centering
% \Huge Teorías causales \\
%
% \Large O sistemas de modelos causales
%
% \end{frame}
%

\begin{frame}[plain]
\begin{textblock}{160}(0,4)
 \centering \LARGE Identificación de efecto causal\\
 \large Con observación, sin intervenciones.
 \end{textblock}
 \vspace{0.75cm}


\only<1-2>{
\begin{textblock}{140}(10,27) \centering
\includegraphics[width=0.70\textwidth, page=1]{figuras/simpson.pdf}
\end{textblock}
}
\only<3->{
\begin{textblock}{140}(10,27) \centering
\includegraphics[width=0.70\textwidth, page=2]{figuras/simpson.pdf}
\end{textblock}
}


\only<1-4>{
\begin{textblock}{150}(5,14) \small
\begin{flalign*}
& P(\text{Derrumbes}|do(\text{Intensidad del sismo})) \overset{?}{=}  \onslide<4>{\sum_{\text{Ciudad}}} \only<2>{P(\text{Derrumbes}|\text{Intensidad del sismo})} \only<3-4>{P(\text{Derrumbes}|\text{Intensidad del sismo}, \text{Ciudad})} \onslide<4>{P(\text{Ciudad})}
&&
\end{flalign*}
\end{textblock}
}

\only<5>{
\begin{textblock}{150}(5,18) \centering
\textbf{Para distinguir el efecto causal necesitamos conocer el modelo causal}
\end{textblock}
}




\end{frame}



\begin{frame}[plain]
\begin{textblock}{160}(0,4)
 \centering \LARGE Identificación de modelo causal\\
 \large Con observaciones, sin intervenciones
\end{textblock}
 \centering
 \vspace{0.75cm}

\begin{textblock}{70}(5,18)
\raggedleft
 \tikz{
    \node[det] (a) {$A_{_{\onslide<6->{\phantom}{i}}}$} ; %
    \node[det, below=of a] (b) {$B_{_{\onslide<6->{\phantom}{i}}}$} ; %
    \node[const, left= of a, xshift=-0.3cm, yshift=0.1cm] (pa) { \small
    \begin{tabular}{|c|c|}
          $A=0$  &  $A=1$   \\ \hline
        $0.5$ & $0.5$   \\ \hline
    \end{tabular}
    }; %
    \node[const, above= of pa] (npa) {\small$P(A)$};


    \node[const, left=of b, xshift=-0.3cm, yshift=-0.1cm] (pb) { \small
    \begin{tabular}{c|c|c|}
        &  $B=0$  &  $B=1$   \\ \hline
       $A=0$ & $0.95$ & $0.05$   \\ \hline
       $A=1$ & $0.05$ & $0.95$   \\ \hline
    \end{tabular}
    };
    \node[const, above= of pb] (npb) {\small$P(B|A)$};

    \node[invisible, above=of a, yshift=1cm] (ia) {};

    \onslide<5->{\plate {datos} {(a)(b)} {\tiny$i$: Dato};}

    \edge {a} {b};
    }
\end{textblock}
\only<4->{
\begin{textblock}{70}(85,18)
\raggedright
    \tikz{
    \node[det] (a) {$A_{_{\onslide<6->{\phantom}{i}}}$} ; %
    \node[det, below=of a] (b) {$B_{_{\onslide<6->{\phantom}{i}}}$} ; %
    \node[const, right= of a, xshift=0.3cm, yshift=0.1cm] (pa) { \small
    \begin{tabular}{c|c|c|}
        &  $A=0$  &  $A=1$   \\ \hline
       $B=0$ & $0.95$ & $0.05$   \\ \hline
       $B=1$ & $0.05$ & $0.95$   \\ \hline
    \end{tabular}
    }; %
    \node[const, above= of pa] (npa) {\small$P(A|B)$};


    \node[const, right=of b, xshift=0.3cm, yshift=-0.1cm] (pb) { \small
    \begin{tabular}{|c|c|}
          $B=0$  &  $B=1$   \\ \hline
        $0.5$ & $0.5$   \\ \hline
    \end{tabular}
    };
    \node[const, above= of pb] (npb) {\small$P(B)$};

    \node[invisible, above=of a, yshift=1cm] (ia) {};

    \onslide<5->{\plate {datos} {(a)(b)} {\tiny$i$: Dato};}

    \edge {b} {a};
    }
\end{textblock}
}

\only<2-4>{
\begin{textblock}{140}(44,66)
\begin{flalign*}
& P(A, B) = P(A) P(B|A) \only<3->{= P(B) P(A|B)} &&
\end{flalign*}
\end{textblock}
}

\only<5-6>{
\begin{textblock}{160}(0,60)
\begin{equation*}
P(\text{Modelo} | \text{Datos} = \{(a_1, b_1), (a_2, b_2), \dots \}) = \onslide<5>{?} \onslide<6>{\frac{\overbrace{P(\text{Datos}|\text{Modelo})}^{\text{Predicción}} P(\text{Modelo})}{P(\text{Datos})}}
\end{equation*}
\end{textblock}
}

\only<7->{
\begin{textblock}{160}(0,66)
\begin{equation*}
P(\text{Datos}|\text{Modelo}_{A\rightarrow B}) =  P(\text{Datos}|\text{Modelo}_{B\rightarrow A})
\end{equation*}

\only<8>{Con mismas predicciones, no podemos distinguir el modelo causal!}
\only<9>{\textbf{Para distinguir el modelo causal necesitamos intervenciones do()}}

\end{textblock}
}


\end{frame}




\begin{frame}[plain]
\begin{textblock}{160}(0,4)
 \centering \LARGE Intervenciones

\Large El operador \textit{do}($x$)
\end{textblock}
 \vspace{0.75cm}


\begin{textblock}{160}(30,21)
\begin{flalign*}
\only<1>{& P(\,\,Y_i(t) = y\,\,) = P(Y = y|\textit{do}(T=t)) &&}
\only<2>{& P(\,\underbrace{Y_i(t) = y}_{\hfrac{\text{\small \phantom{p}Potential\phantom{p}}}{\text{\small outcome}}}\,) = P(Y = y|\textit{do}(T=t)) &&}
\only<3>{& P(\,\underbrace{Y_i(t) = y}_{\hfrac{\text{\small \phantom{p}Potential\phantom{p}}}{\text{\small outcome}}}\,) = P(Y = y|\textit{do}(T=t), \text{Modelo Causal}) &&}
\only<4->{& P(\,\underbrace{Y_i(t) = y}_{\hfrac{\text{\small \phantom{p}Potential\phantom{p}}}{\text{\small outcome}}}\,) = \underbrace{P(Y = y|\textit{do}(T=t), \text{Modelo Causal})}_{\hfrac{\text{\small Bayesian}}{\text{\small causal model}}} &&}
\end{flalign*}
\end{textblock}


\only<5->{
\begin{textblock}{160}(45,48)
\begin{flalign*}
& \text{ATE} = \E[\,y\,|\,\textit{do}(t=1)] - \E[\,y\,|\textit{do}(t=0)] &&
\end{flalign*}
\end{textblock}
}

\only<6>{
\begin{textblock}{160}(0,67)
\begin{align*}
& P( y | \textit{do}(t) ) \ \text{ es más general que } \ \E[y|\textit{do}(t)]
\end{align*}
\end{textblock}
}



%
% \only<3>{
% \begin{textblock}{160}(0,24)
% \begin{flalign*}
% & P(\,\underbrace{Y_i(t) = y}_{\hfrac{\text{\small \phantom{p}Potential\phantom{p}}}{\text{\small outcome}}}\,) = \underbrace{P(Y = y|\textit{do}(T=t), \text{Modelo Causal})}_{\hfrac{\text{\small Bayesian}}{\text{\small causal model}}} &&
% \begin{flalign*}
% \end{textblock}
% }

 \end{frame}



\begin{frame}[plain]
\begin{textblock}{160}(0,4)
 \centering \LARGE Inferencia causal \\
 \large Los \textbf{niveles} del razonamiento causal
\end{textblock}

 \vspace{0.75cm}

\only<1-3>{
\begin{textblock}{140}(10,24)
1. \textbf{Asociacional}:  $P(y \, | \, x , \, \text{Modelo Causal})$ \ y \ $P(\text{Modelo Causal} \, | \, x )$  \\[0.05cm] \normalsize
\hspace{0.3cm} Permite evaluar el efecto y el modelo causal sólo si se cumplen ciertas condiciones \\

\vspace{1cm} \large


\only<2-3>{
2. \textbf{Intervencional}:  $P(y \, | \, \text{do}(x), \, \text{Modelo Causal})$ \ y \ $P(\text{Modelo Causal} \, | \, y, \, \text{do}(x) )$ \\[0.05cm] \normalsize
\hspace{0.3cm} Permite evaluar tanto el efecto causal y el modelo causal
}

\vspace{0.7cm} \large

\only<3->{
3. \textbf{Contrafactual}:  $P(\overbrace{y \, | \, \text{do}(x),}^{\text{Contrafactual}} \, \overbrace{y^{\prime}, \, \text{do}(x^{\prime}),}^{\text{Factual}}\, \text{Modelo Causal})$ \\[0.05cm] \normalsize
\hspace{0.3cm} Permite evaluar el efecto causal contrafactual (no permite evaluar el modelo causal)
}
\end{textblock}
}

\only<4>{
\begin{textblock}{140}(10,36) \centering \Large
Estos niveles surgen naturalmente \\

del proceso generativo de lo datos
\end{textblock}
}

\end{frame}

\begin{frame}[plain]
\begin{textblock}{160}(0,4)
 \centering \LARGE Monty Hall Causal \\
 \large \only<1>{Los \textbf{niveles} del razonamiento causal}\only<2>{Asociación}\only<3>{Intervención}\only<4-7>{Contrafáctico}\only<8->{Los \textbf{niveles} del razonamiento causal}
\end{textblock}
 \centering
 \vspace{0.75cm}


\only<2>{
\begin{textblock}{80}(5,22)
\raggedright
%\onslide<2->{Modelo gráfico} \\ \vspace{0.3cm}
 \tikz{
    \node[latent, minimum size=1.4cm] (r) {\includegraphics[width=0.12\textwidth]{../../auxiliar/static/regalo.png}} ;
    \node[const,above=of r] (pr) {\normalsize $P(r)$} ;

    \node[latent, below=of r,minimum size=1.4cm] (d) {\includegraphics[width=0.10\textwidth]{../../auxiliar/static/dedo.png}} ;
    \node[const,left=of d] (pd) {\normalsize $P(s|r)$} ;

\node[invisible,left=of d, xshift=-1.5cm] () {} ;


    \edge {r} {d};
}
\end{textblock}

\begin{textblock}{80}(65,24) \centering \Large
Asociación
\begin{equation*}
P(r, s)
\end{equation*}
\large
\begin{tabular}{c|c|c|c|}
         &  $r1$  &  $r2$ & $r3$   \\ \hline
       $s1$ & $0$ & $1/6$ & $1/6$ \\ \hline
       $s2$ & $1/6$ & $0$ & $1/6$ \\ \hline
       $s3$ & $1/6$ & $1/6$ & $0$ \\ \hline
    \end{tabular}
\end{textblock}
}



\only<3>{
\begin{textblock}{80}(5,22)
\raggedright
%\onslide<2->{Modelo gráfico} \\ \vspace{0.3cm}
 \tikz{

    \node[latent, minimum size=1.4cm] (d) {\includegraphics[width=0.10\textwidth]{../../auxiliar/static/dedo.png}} ;
    \node[const,left=of d] (pd) {\normalsize $P(s|r,c)$} ;

    \node[latent, above=of d, minimum size=1.4cm, xshift=1.2cm] (r) {\includegraphics[width=0.12\textwidth]{../../auxiliar/static/regalo.png}} ;
    \node[const,above=of r] (pr) {\normalsize $P(r)$} ;

    \node[latent, minimum size=1.4cm, fill=black!40, above=of d, xshift=-1.2cm] (c) {\includegraphics[width=0.12\textwidth]{../../auxiliar/static/cerradura.png}} ;
    \node[const,above=of c] (pc) {\normalsize $\text{do}(c1)$} ;


    \node[invisible,left=of d, xshift=-1.5cm] () {} ;



    \edge {r,c} {d};
}
\end{textblock}

\begin{textblock}{80}(65,24) \centering \Large
Intervención
\begin{equation*}
P(r, s| \text{do}(c1) )
\end{equation*}
\large
\begin{tabular}{c|c|c|c|}
         &  $r1$  &  $r2$ & $r3$   \\ \hline
       $s1$ & $0$ & $0$ & $0$ \\ \hline
       $s2$ & $1/6$ & $0$ & $1/3$ \\ \hline
       $s3$ & $1/6$ & $1/3$ & $0$ \\ \hline
    \end{tabular}
\end{textblock}
}


\only<4-7>{
\begin{textblock}{80}(5,22)
\raggedright
%\onslide<2->{Modelo gráfico} \\ \vspace{0.3cm}
 \tikz{

    \only<4-5>{
      \node[latent, minimum size=1.4cm, fill=black!15] (d) {\includegraphics[width=0.10\textwidth]{../../auxiliar/static/dedo.png}} ;
      \node[const,left=of d] (pd) {\normalsize $s=2$} ;

      \node[latent, minimum size=1.4cm, fill=black!40, above=of d, xshift=-1.2cm] (c) {\includegraphics[width=0.12\textwidth]{../../auxiliar/static/cerradura.png}} ;
      \node[const,above=of c] (pc) {\normalsize $\text{do}(c1)$} ;
       \edge {r,c} {d};
    }
    \only<6->{
      {\color{black!15}
      \node[latent, minimum size=1.4cm, fill=black!15, opacity=0.3] (d) {\includegraphics[width=0.10\textwidth]{../../auxiliar/static/dedo.png} } ;
      \node[const,left=of d] (pd) {\normalsize $s=2$} ;


      \node[latent, minimum size=1.4cm, fill=black!40, above=of d, xshift=-1.2cm, opacity=0.3] (c) {\includegraphics[width=0.12\textwidth]{../../auxiliar/static/cerradura.png}} ;
      \node[const,above=of c] (pc) {\normalsize $\text{do}(c1)$} ;
      \edge {r,c} {d};
      }
    }

      \node[latent, above=of d, minimum size=1.4cm, xshift=1.2cm] (r) {\includegraphics[width=0.12\textwidth]{../../auxiliar/static/regalo.png}} ;
      \node[const,above=of r] (pr) {\normalsize $P(r)$} ;

      \only<6->{
        \node[latent,  below=of r, xshift=1.2cm, minimum size=1.4cm] (d_prima) {\includegraphics[width=0.10\textwidth]{../../auxiliar/static/dedo.png} } ;
        \node[const,right=of d_prima] (pd_prima) {\normalsize $P(s^{\prime}|r,c^{\prime})$} ;

        \node[latent, minimum size=1.4cm, fill=black!40, above=of d_prima, xshift=1.2cm] (c_prima) {\includegraphics[width=0.12\textwidth]{../../auxiliar/static/cerradura.png}} ;
        \node[const,above=of c_prima] (pc_prima) {\normalsize $\text{do}(c^{\prime})=2$} ;
        \edge {r,c_prima} {d_prima};
      }

    \node[invisible,left=of d, xshift=-1.5cm] () {} ;



}
\end{textblock}
}

\only<5>{
\begin{textblock}{80}(65,24) \centering \Large
Factual
\begin{equation*}
P(r| \text{do}(c1), s2 )
\end{equation*}
\large
\begin{tabular}{|c|c|c|}
           $r1$  &  $r2$ & $r3$   \\ \hline
        $1/3$ & $0$ & $2/3$ \\ \hline
    \end{tabular}
\end{textblock}
}

\only<6-7>{
\begin{textblock}{80}(75,24) \centering \Large
Contra factual
\begin{equation*}
P(s^{\prime}, r | \text{do}(c1), s2 , \text{do}(c^{\prime}2))
\end{equation*}
\large
\only<7>{
\begin{tabular}{c|c|c|c|}
         &  $r1$  &  $r2$ & $r3$   \\ \hline
       $s^{\prime}1$ & $0$ & $0$ & $2/3$ \\ \hline
       $s^{\prime}2$ & $0$ & $0$ & $0$ \\ \hline
       $s^{\prime}3$ & $1/3$ & $0$ & $0$ \\ \hline
    \end{tabular}
}
\end{textblock}
}


\only<4-7>{
\begin{textblock}{140}(10,73) \centering \Large
¿Que caja hubiera señalado si hubieramos elegido la caja 2, \\
dado que elegimos la caja 1 y señaló la caja 2?
\end{textblock}
}

\only<8->{
\begin{textblock}{50}(4,28) \centering
\textbf{Asociación}
\begin{equation*}
P(r, s)
\end{equation*}
\large
\begin{tabular}{c|c|c|c|}
         &  $r1$  &  $r2$ & $r3$   \\ \hline
       $s1$ & $0$ & $1/6$ & $1/6$ \\ \hline
       $s2$ & $1/6$ & $0$ & $1/6$ \\ \hline
       $s3$ & $1/6$ & $1/6$ & $0$ \\ \hline
    \end{tabular}
\end{textblock}
}

\only<9->{
\begin{textblock}{50}(54,28) \centering
\textbf{Intervención}
\begin{equation*}
P(r, s| \text{do}(c1))
\end{equation*}
\large
\begin{tabular}{c|c|c|c|}
         &  $r1$  &  $r2$ & $r3$   \\ \hline
       $s1$ & $0$ & $0$ & $0$ \\ \hline
       $s2$ & $1/6$ & $0$ & $1/3$ \\ \hline
       $s3$ & $1/6$ & $1/3$ & $0$ \\ \hline
    \end{tabular}
\end{textblock}
}

\only<10->{
\begin{textblock}{50}(104,28) \centering
\textbf{Contra factual}
\begin{equation*}
P(s^{\prime}, r | \text{do}(c^{\prime}2), \text{do}(c1), s2) \ \
\end{equation*}
\large
\begin{tabular}{c|c|c|c|}
         &  $r1$  &  $r2$ & $r3$   \\ \hline
       $s^{\prime}1$ & $0$ & $0$ & $2/3$ \\ \hline
       $s^{\prime}2$ & $0$ & $0$ & $0$ \\ \hline
       $s^{\prime}3$ & $1/3$ & $0$ & $0$ \\ \hline
    \end{tabular}
\end{textblock}
}

\only<11>{
\begin{textblock}{160}(0,72) \centering
\textbf{Efecto causal}\\[-0.8cm]
\begin{align*}
\underbrace{P(r,s|\text{do}(c1), \text{Modelo Causal})}_{\text{Intervención 1}} - \underbrace{P(r,s|\text{do}(c2), \text{Modelo Causal})}_{\text{Intervención 2}}
\end{align*}
\end{textblock}
}

\only<12>{
\begin{textblock}{160}(0,72) \centering
\textbf{Efecto causal}\\ \Large
Suponemos que el modelo causal es correcto!

\large(está en el condicional)
\end{textblock}
}


\end{frame}


\begin{frame}[plain]
\begin{textblock}{160}(0,4)
 \centering \LARGE Identificación de efecto causal\\
 \large Con observación, sin intervenciones.
 \end{textblock}
 \vspace{0.75cm}


\only<1-3>{
\begin{textblock}{160}(0,24)
\begin{equation*}
P(y|\text{do}(T = 1) , \text{Modelo Causal}) - P(y|\text{do}(T = 0) , \text{Modelo Causal})
\end{equation*}
\end{textblock}
}

\only<2-3>{
\begin{textblock}{160}(0,44) \centering
\Large Para estimar el efecto causal sin hacer intervenciones \\
\Large necesitamos \textit{exchangeability}: $(Y \indep T \,|\, W)$

\vspace{0.6cm}

\only<3>{
¿Cómo determinamos las variables de control $W$?
}

\end{textblock}
}


\end{frame}




 \begin{frame}[plain]
\begin{textblock}{160}(0,4)
 \centering \LARGE Flujo de inferencia en modelos causales\\
 \large Independencias condicionales
 \end{textblock}
 \centering
 \vspace{0.75cm}

 \begin{textblock}{68}(0,25)
 \tikz{
    \node[det] (l) {$l$} ; %
    \node[det, above=of l] (a) {$a$} ; %
    \node[det, above=of a,xshift=1.5cm] (t) {$t$} ; %
    \node[det, above=of a,xshift=-1.5cm] (e) {$e$} ; %
    \node[det, below=of t,xshift=1.5cm] (r) {$r$} ; %

    \edge {a} {l};
    \edge {t,e} {a};
    \edge {t} {r};

    \node[const, above= of e, yshift=0.1cm] (cpd_e) {Entradera:};
    \node[const, above= of t, yshift=0.1cm] (cpd_t) {Terremoto:};
    \node[const, left= of a, xshift=-0.1cm] (cpd_a) {Alarma:};
    \node[const, left= of r, xshift=-0.1cm] (cpd_r) {Redes:};
    \node[const, left= of l, xshift=-0.1cm] (cpd_l) {Llamada:};

    }
\end{textblock}


\only<1->{
\begin{textblock}{90}(68,24)
\centering
 \begin{tabular}{c c c}
 & \onslide<2->{$\hfrac{\text{Intermedio}}{\text{no observable}}$} &   \onslide<3->{$\hfrac{\text{Intermedio}}{\text{observable}}}$ \\
 & & \\
 $ e \rightarrow a \rightarrow l $    & \onslide<2->{$P(l) \overset{?}{=} P(l|e)$} & \onslide<3->{$P(l|a) \overset{?}{=} P(l|e,a)$} \\
 %$ l \leftarrow a \leftarrow t $      &  \onslide<4->{$P(t) \overset{?}{=} P(t|l)$}  & \onslide<5->{$P(t|a) \overset{?}{=} P(t|a,l)$} \\
 $ a \leftarrow t \rightarrow r $     & \onslide<4->{$P(r) \overset{?}{=} P(r|a)$} & \onslide<5->{$P(r|t) \overset{?}{=} P(r|t,a)$} \\
 $ e \rightarrow a \leftarrow t $     & \onslide<6->{$P(t) \overset{?}{=} P(t|e)$} & \onslide<7->{$P(t|a) \overset{?}{=} P(t|e,a)$} \\
            $\overset{\downarrow}{l}$  &  & \onslide<8->{$P(t|l) \overset{?}{=} P(t|e,l)$}
 \end{tabular}
 \end{textblock}
 }


 \end{frame}


 \begin{frame}[plain]
\begin{textblock}{160}(0,4)
 \centering \LARGE Flujo de inferencia en modelos causales\\
 \large Independencias condicionales
 \end{textblock}
 \centering
 \vspace{0.75cm}

 \only<1-14>{
 \begin{textblock}{68}(0,25)
 \tikz{
    \node[det] (l) {$l$} ; %
    \node[det, above=of l] (a) {$a$} ; %
    \node[det, above=of a,xshift=1.5cm] (t) {$t$} ; %
    \node[det, above=of a,xshift=-1.5cm] (e) {$e$} ; %
    \node[det, below=of t,xshift=1.5cm] (r) {$r$} ; %

    \edge {a} {l};
    \edge {t,e} {a};
    \edge {t} {r};

    \node[const, above= of e, yshift=0.1cm] (cpd_e) {Entradera:};
    \node[const, above= of t, yshift=0.1cm] (cpd_t) {Terremoto:};
    \node[const, left= of a, xshift=-0.1cm] (cpd_a) {Alarma:};
    \node[const, left= of r, xshift=-0.1cm] (cpd_r) {Redes:};
    \node[const, left= of l, xshift=-0.1cm] (cpd_l) {Llamada:};

    }
\end{textblock}
}

\only<1-14>{
\begin{textblock}{90}(68,24)
\centering
 \begin{tabular}{c c c}
 & \onslide<1->{$\hfrac{\text{Intermedio}}{\text{no observable}}$} &   \onslide<1->{$\hfrac{\text{Intermedio}}{\text{observable}}}$ \\
 & & \\
 $ e \rightarrow a \rightarrow l $    & \only<1>{$P(l) \overset{?}{=} P(l|e)$}\only<2->{$P(l) \neq P(l|e)$} & \only<1-3>{$P(l|a) \overset{?}{=} P(l|e,a)$}\only<4->{$P(l|a) \overset{\phantom{?}}{=} P(l|e,a)$} \\
 $ a \leftarrow t \rightarrow r $     & \only<1-5>{$P(r) \overset{?}{=} P(r|a)$}\only<6->{$P(r) \neq P(r|a)$} & \only<1-7>{$P(r|t) \overset{?}{=} P(r|t,a)$}\only<8->{$P(r|t) \overset{\phantom{?}}{=} P(r|t,a)$} \\
 $ e \rightarrow a \leftarrow t $     & \only<1-9>{$P(t) \overset{?}{=} P(t|e)$}\only<10->{$P(t) \overset{\phantom{?}}{=} P(t|e)$} & \only<1-11>{$P(t|a) \overset{?}{=} P(t|e,a)$}\only<12->{$P(t|a) \neq P(t|e,a)$} \\
            $\overset{\downarrow}{l}$  &  & \only<1-13>{$P(t|l) \overset{?}{=} P(t|e,l)$}\only<14->{$P(t|l) \neq P(t|e,l)$}
 \end{tabular}
\end{textblock}
 }


\only<15-16>{
\begin{textblock}{140}(10,32) \Large
 Hay flujo de inferencia entre los extremos de una cadena si:

 \large (camino \textit{d-conectado}) \\[0.3cm]
 \begin{itemize}
  \item[$\bullet$] Todas las consecuencias comunes (o sus descendientes) son observables
  \item[$\bullet$] Ninguna otra variable es observable
 \end{itemize}

 \vspace{0.8cm}

 \only<16>{
 Se cierra el flujo si está $\underbrace{\text{no \textit{d-conectado}}}_{\text{\small d-separado}}$
  }

\end{textblock}
}

\only<1-2>{
\begin{textblock}{80}(65,72) \Large
$\phantom{\overset{?}{=}} P(l) \only<1>{\overset{?}{=}}\only<2>{\neq} P(l|e)\phantom{\overset{?}{=}}$
\end{textblock}
}

\only<3-4>{
\begin{textblock}{80}(65,72) \Large
$\phantom{\overset{?}{=}} P(l|a) \only<3>{\overset{?}{=}}\only<4>{=} P(l|a,e)\phantom{\overset{?}{=}}$
\end{textblock}
}


\only<5-6>{
\begin{textblock}{80}(65,72) \Large
$\phantom{\overset{?}{=}} P(r) \only<5>{\overset{?}{=}}\only<6>{\neq} P(r|a)\phantom{\overset{?}{=}}$
\end{textblock}
}


\only<7-8>{
\begin{textblock}{80}(65,72) \Large
$\phantom{\overset{?}{=}} P(r|t) \only<7>{\overset{?}{=}}\only<8>{=} P(r|t,a)\phantom{\overset{?}{=}}$
\end{textblock}
}


\only<9-10>{
\begin{textblock}{80}(65,72) \Large
$\phantom{\overset{?}{=}} P(t) \only<9>{\overset{?}{=}}\only<10>{=} P(t|e)\phantom{\overset{?}{=}}$
\end{textblock}
}


\only<11-12>{
\begin{textblock}{80}(65,72) \Large
$\phantom{\overset{?}{=}} P(t|a) \only<11>{\overset{?}{=}}\only<12>{\neq} P(t|a,e)\phantom{\overset{?}{=}}$
\end{textblock}
}


\only<13-14>{
\begin{textblock}{80}(65,72) \Large
$\phantom{\overset{?}{=}} P(t|l) \only<13>{\overset{?}{=}}\only<14>{\neq} P(t|l,e)\phantom{\overset{?}{=}}$
\end{textblock}
}


 \end{frame}


 \begin{frame}[plain]
\begin{textblock}{160}(0,4)
 \centering \LARGE Flujo de inferencia causal \\
 \end{textblock}
 \vspace{0.75cm}


\begin{textblock}{160}(0,14) \centering
 \tikz{
    \node[det] (l) {$l$} ; %
    \node[det, above=of l] (a) {$a$} ; %
    \node[det, above=of a,xshift=1.5cm] (t) {$t$} ; %
    \only<3->{\node[det, fill=black!15, above=of a,xshift=-1.5cm] (e) {$e$};}
    \only<2>{\node[det, double, double distance=0.5mm, fill=black!15, above=of a, xshift=-1.5cm] (e) {$e$} ;}
    \only<1>{\node[det, above=of a,xshift=-1.5cm] (e) {$e$};}

    \node[det, below=of t,xshift=1.5cm] (r) {$r$} ; %
    \only<1-2>{\node[det, above=of t,xshift=-1.5cm] (c) {$c$} ;}
    \only<3->{\node[det, fill=black!15,above=of t,xshift=-1.5cm] (c) {$c$} ;}


    \edge {a} {l};
    \edge {t,e} {a};
    \edge {t} {r};
    \only<2>{\edge {c} {t};}
    \only<1,3>{\edge {c} {e,t};}

    \only<1,3>{\node[const, left= of e, xshift=-0.1cm] (cpd_e) {Entradera:};}
    \only<2>{\node[const, left= of e, xshift=-0.1cm] (cpd_e) {\textit{do}(Entradera)};}
    \node[const, right= of t, xshift=0.1cm] (cpd_t) {:Terremoto};
    \node[const, right= of c, xshift=0.1cm] (cpd_c) {:Ciudad};
    \node[const, left= of a, xshift=-0.1cm] (cpd_a) {Alarma:};
    \node[const, right= of r, xshift=0.1cm] (cpd_r) {:Redes};
    \node[const, left= of l, xshift=-0.1cm] (cpd_l) {Llamada:};

    \node[invisible, left=of e, xshift=-3cm] () {};
    }
\end{textblock}

\only<2>{
\begin{textblock}{150}(5,80) \centering
\Large Intervenimos en el modelo causal
\end{textblock}
}

\only<3>{
\begin{textblock}{150}(5,80) \centering
\Large Cerramos el flujo de inferencia trasero
\end{textblock}
}

\end{frame}

%
%  \begin{frame}[plain]
% \begin{textblock}{160}(0,4)
%  \centering \LARGE  Flujo de inferencia causal
%  \end{textblock}
%
% \begin{textblock}{160}(0,12)
%  \centering
%
%  \tikz{
%     \node[det] (l) {$l$} ; %
%     \node[det, above=of l] (a) {$a$} ; %
%     \node[det, above=of a,xshift=1.5cm] (t) {$t$} ; %
%     \only<1-2>{\node[det, above=of a,xshift=-1.5cm] (e) {$e$} ; }
%     \only<3->{\node[det, above=of a,fill=black!15,xshift=-1.5cm] (e) {$e$} ; }
%     \node[det, below=of t,xshift=1.5cm] (r) {$r$} ; %
%     \only<1-3>{\node[det, above=of e,xshift=1.5cm] (c) {$c$} ; }
%     \only<4->{\node[det, above=of e, fill=black!15, xshift=1.5cm] (c) {$c$} ; }
%
%     \edge {c} {e,t};
%     \edge {a} {l};
%     \edge {t,e} {a};
%     \edge {t} {r};
%
%
%     \only<1-2>{
%     \node[const, left= of c, xshift=-0.1cm] (cpd_c) {
%     \begin{tabular}{|c|c|}
%         \hline
%         $c^0$ & $c^1$ \\ \hline
%         $0.200$ & $0.800$  \\ \hline
%     \end{tabular}
%     };
%     \node[const, above= of cpd_c] (n_c) {$P(\text{Ciudad})$};
%     }
%     \only<3>{
%     \node[const, left= of c, xshift=-0.1cm] (cpd_c) {
%     \begin{tabular}{|c|c|}
%         \hline
%         $c^0$ & $c^1$ \\ \hline
%         $0.208$ & $0.792$  \\ \hline
%     \end{tabular}
%     };
%     \node[const, above= of cpd_c] (n_c) {$P(\text{Ciudad}|\text{Entradera}=0)$};
%     }
%     \only<4>{
%     \node[const, left= of c, xshift=-0.1cm] (cpd_c) {
%     \begin{tabular}{|c|c|}
%         \hline
%         $c^0$ & $c^1$ \\ \hline
%         $1.000$ & $0.000$  \\ \hline
%     \end{tabular}
%     };
%     \node[const, above= of cpd_c] (n_c) {$\phantom{P|}\text{Ciudad}=0\phantom{P}$};
%     }
%
%     \only<1>{
%     \node[const,  left= of e, yshift=-0.3cm, xshift=-0.1cm] (cpd_e) {
%     \begin{tabular}{|c|c|c|}
%         \hline
%        & $e^0$ & $e^1$ \\ \hline
%       $c^0$ & $0.999$ & $0.001$  \\ \hline
%       $c^1$ & $0.95$ & $0.05$  \\ \hline
%     \end{tabular}
%     };
%     \node[const, above= of cpd_e] (n_e) {$P(\text{Entradera}|\text{Ciudad})$};
%     }
%     \only<2>{
%     \node[const,  left= of e, yshift=-0.3cm, xshift=-0.1cm] (cpd_e) {
%     \begin{tabular}{|c|c|}
%         \hline
%         $e^0$ & $e^1$ \\ \hline
%        $0.960$ & $0.040$  \\ \hline
%     \end{tabular}
%     };
%     \node[const, above= of cpd_e] (n_e) {$P(\text{Entradera})$};
%     }
%     \only<3-4>{
%     \node[const,  left= of e, yshift=-0.3cm, xshift=-0.1cm] (cpd_e) {
%     \begin{tabular}{|c|c|}
%         \hline
%         $e^0$ & $e^1$ \\ \hline
%        $1.000$ & $0.000$  \\ \hline
%     \end{tabular}
%     };
%     \node[const, above= of cpd_e] (n_e) {$\text{Entradera}=0$};
%     }
%
%     \only<1>{
%     \node[const, right= of t, yshift=0.9cm, xshift=0.1cm] (cpd_t) {
%     \begin{tabular}{|c|c|c|}
%         \hline
%        & $t^0$ & $t^1$ \\ \hline
%       $c^0$ & $0.99$ & $0.01$  \\ \hline
%      $c^1$  & $0.95$ & $0.05$  \\ \hline
%     \end{tabular}
%     };
%     \node[const, above= of cpd_t] (n_t) {$P(\text{Terremoto}|\text{Ciudad})$};
%     }
%     \only<2>{
%     \node[const, right= of t, yshift=0.6cm, xshift=0.1cm] (cpd_t) {
%     \begin{tabular}{|c|c|}
%         \hline
%         $t^0$ & $t^1$ \\ \hline
%        $0.958$ & $0.042$  \\ \hline
%     \end{tabular}
%     };
%     \node[const, above= of cpd_t] (n_t) {$P(\text{Terremoto})$};
%     }
%     \only<3>{
%     \node[const, right= of t, yshift=0.6cm, xshift=0.1cm] (cpd_t) {
%     \begin{tabular}{|c|c|}
%         \hline
%         $t^0$ & $t^1$ \\ \hline
%        $0.959$ & $0.041$  \\ \hline
%     \end{tabular}
%     };
%     \node[const, above= of cpd_t] (n_t) {$P(\text{Terremoto}|\text{Entradera}=0)$};
%     }
%     \only<4>{
%     \node[const, right= of t, yshift=0.6cm, xshift=0.1cm] (cpd_t) {
%     \begin{tabular}{|c|c|}
%         \hline
%         $t^0$ & $t^1$ \\ \hline
%        $0.990$ & $0.010$  \\ \hline
%     \end{tabular}
%     };
%     \node[const, above= of cpd_t] (n_t) {$P(\text{Terremoto}|\text{Ciudad}=0)$};
%     }
%
%
%     \only<1>{
%     \node[const, left= of a, yshift=-1.6cm, xshift=-0.5cm] (cpd_a) {
%     \begin{tabular}{|c|c|c|}
%         \hline
%         & $a^0$ & $a^1$ \\ \hline
%        ($e^0, t^0$) & $0.99$ & $0.01$  \\ \hline
%        ($e^1, t^0$) & $0.01$ & $0.99$  \\ \hline
%        ($e^0, t^1$) & $0.01$ & $0.99$  \\ \hline
%        ($e^1, t^1$) & $0.0001$ & $0.9999$  \\ \hline
%     \end{tabular}
%     };
%     \node[const, above= of cpd_a] (n_a) {$P(\text{Alarma}|\text{Entradera},\text{Terremoto})$};
%     }
%     \only<2>{
%     \node[const, left= of a, yshift=-0.6cm, xshift=-0.5cm] (cpd_a) {
%     \begin{tabular}{|c|c|}
%         \hline
%          $a^0$ & $a^1$ \\ \hline
%         $0.911$ & $0.089$  \\ \hline
%     \end{tabular}
%     };
%     \node[const, above= of cpd_a] (n_a) {$P(\text{Alarma})$};
%     }
%     \only<3>{
%     \node[const, left= of a, yshift=-0.6cm, xshift=-0.5cm] (cpd_a) {
%     \begin{tabular}{|c|c|}
%         \hline
%          $a^0$ & $a^1$ \\ \hline
%         $0.949$ & $0.051$  \\ \hline
%     \end{tabular}
%     };
%     \node[const, above= of cpd_a] (n_a) {$P(\text{Alarma}|\text{Entradera}=0)$\hspace{1.5cm}\phantom{.}};
%     }
%     \only<4>{
%     \node[const, left= of a, yshift=-0.6cm, xshift=-0.5cm] (cpd_a) {
%     \begin{tabular}{|c|c|}
%         \hline
%          $a^0$ & $a^1$ \\ \hline
%         $0.980$ & $0.020$  \\ \hline
%     \end{tabular}
%     };
%     \node[const, above= of cpd_a] (n_a) {$P(\text{Alarma}|\text{Entradera}=0, \text{Ciudad}=0)$\hspace{3cm}\phantom{.}};
%     }
%
%     \only<1>{
%     \node[const, right= of r, yshift=0.2cm, xshift=0.1cm] (cpd_r) {
%     \begin{tabular}{|c|c|c|}
%         \hline
%         & \, $r^0$ \, & \, $r^1$ \,  \\ \hline
%        $(t^0)$ & $0.99$ & $0.01$   \\ \hline
%        $(t^1)$ & $0.01$ & $0.99$   \\ \hline
%     \end{tabular}
%     };
%     \node[const, above= of cpd_r] (n_r) {$P(\text{Redes}|\text{Terremoto})$};
%     }
%     \only<2>{
%     \node[const, right= of r, yshift=0.2cm, xshift=0.1cm] (cpd_r) {
%     \begin{tabular}{|c|c|}
%         \hline
%         \, $r^0$ \, & \, $r^1$ \,  \\ \hline
%        $0.948$ & $0.052$   \\ \hline
%     \end{tabular}
%     };
%     \node[const, above= of cpd_r] (n_r) {$P(\text{Redes})$};
%     }
%     \only<3>{
%     \node[const, right= of r, yshift=0.2cm, xshift=0.1cm] (cpd_r) {
%     \begin{tabular}{|c|c|}
%         \hline
%         \, $r^0$ \, & \, $r^1$ \,  \\ \hline
%        $0.949$ & $0.051$   \\ \hline
%     \end{tabular}
%     };
%     \node[const, above= of cpd_r] (n_r) {$P(\text{Redes}|\text{Entradera}=0)$};
%     }
%     \only<4>{
%     \node[const, right= of r, yshift=0.2cm, xshift=0.1cm] (cpd_r) {
%     \begin{tabular}{|c|c|}
%         \hline
%         \, $r^0$ \, & \, $r^1$ \,  \\ \hline
%        $0.980$ & $0.020$   \\ \hline
%     \end{tabular}
%     };
%     \node[const, above= of cpd_r] (n_r) {$P(\text{Redes}|\text{Ciudad}=0)$};
%     }
%
%     \only<1>{
%     \node[const, right= of l, yshift=-0.4cm,xshift=0.1cm] (cpd_l) {
%     \begin{tabular}{|c|c|c|}
%         \hline
%         & \, $l^0$ \, & \, $l^1$ \,  \\ \hline
%        $(a^0)$ & $0.99$ & $0.01$   \\ \hline
%        $(a^1)$ & $0.01$ & $0.99$   \\ \hline
%     \end{tabular}
%     };
%     \node[const, above= of cpd_l] (n_l) {$P(\text{Llamada}|\text{Alarma})$};
%     }
%     \only<2>{
%     \node[const, right= of l, yshift=-0.4cm,xshift=0.1cm] (cpd_l) {
%     \begin{tabular}{|c|c|}
%         \hline
%          \, $l^0$ \, & \, $l^1$ \,  \\ \hline
%         $0.903$ & $0.097$   \\ \hline
%     \end{tabular}
%     };
%     \node[const, above= of cpd_l] (n_l) {$P(\text{Llamada})$};
%     }
%     \only<3>{
%     \node[const, right= of l, yshift=-0.4cm,xshift=0.1cm] (cpd_l) {
%     \begin{tabular}{|c|c|}
%         \hline
%          \, $l^0$ \, & \, $l^1$ \,  \\ \hline
%         $0.940$ & $0.060$   \\ \hline
%     \end{tabular}
%     };
%     \node[const, above= of cpd_l] (n_l) {$\phantom{.}\hspace{1.5cm}P(\text{Llamada}|\text{Entradera}=0)$};
%     }
%     \only<4>{
%     \node[const, right= of l, yshift=-0.4cm,xshift=0.1cm] (cpd_l) {
%     \begin{tabular}{|c|c|}
%         \hline
%          \, $l^0$ \, & \, $l^1$ \,  \\ \hline
%         $0.970$ & $0.030$   \\ \hline
%     \end{tabular}
%     };
%     \node[const, above= of cpd_l] (n_l) {$\phantom{.}\hspace{3.6cm}P(\text{Llamada}|\text{Entradera}=0, \text{Ciudad}=0)$};
%     }
%
%     \node[invisible, left=of e, xshift=-5.5cm] (il) {};
%     \node[invisible, right=of r, xshift=4.8cm] (ir) {};
%  }
% \end{textblock}
%
% \end{frame}







\begin{frame}[plain]
\begin{textblock}{160}(0,4)
 \centering \LARGE Identificación del efecto causal\\
 \large Con observables, sin intervenciones
\end{textblock}


\begin{textblock}{160}(10,20)
 \Large Backdoor criterion\\[0.1cm] \large

  Conjunto de variable $W$ tal que:

  \hspace{0.3cm} 1. $W$ cierra todos los caminos traseros de $T$ a $Y$

  \hspace{0.3cm} 2. $W$ no contiene ningún descendiente de $T$
\end{textblock}


\only<2->{
\begin{textblock}{160}(20,42)
 \begin{flalign*}
  P(y|\text{do}(t)) & \only<-5>{= \sum_w P(y|\text{do}(t),w) P(w|\text{do}(t))} \only<5->{= \sum_w P(y|t,w) P(w)}
  &&
 \end{flalign*}
\end{textblock}
}


\only<3-5>{
\begin{textblock}{160}(15,67)
1. Porque $W$ corta el flujo trasero vale: $P(y|\text{do}(t),w) = P(y|t,w)$
\end{textblock}
}

\only<4-5>{
\begin{textblock}{160}(15,75)
2. Porque $W$ no contiene descendientes de $T$ vale: $P(w|\text{do}(t)) = P(w)$
\end{textblock}
}

\only<7->{
\begin{textblock}{160}(15,58)
\begin{flalign*}
& \E[Y|\text{do}(t)]  = \sum_w \E[Y|t,w]P(w) \only<8->{= \E_w \E[Y|t,W] } &&
\end{flalign*}
\end{textblock}
}


\only<9->{
\begin{textblock}{160}(0,76)
\begin{equation*}
 \E[Y|\text{do}(t1)] - \E[Y|\text{do}(t0)] = \E_w \E[Y|t1,W] - \E_w \E[Y|t0,W]
\end{equation*}
\end{textblock}
}




\end{frame}



\begin{frame}[plain]
\begin{textblock}{160}(0,4)
 \centering \LARGE Controles \only<1-6>{buenos}\only<7-12,16>{malos}\only<13-15>{neutrales} \\
 \large \only<12>{Sesgo de selección}\only<13-14>{Mejoran precisión}\only<15>{Reducen precisión}
 \end{textblock}
 \vspace{0.75cm}


\only<1>{
\begin{textblock}{140}(10,30) \centering
\tikz{
    \node[latent, fill=red!30] (z) {$z$} ; %
    \node[latent, fill=black!15 , below=of z, xshift=-1.5cm] (x) {$x$} ; %
    \node[latent, fill=black!15, below=of z, xshift=1.5cm] (y) {$y$} ; %

    \phantom{\node[latent, right=of y] (il) {$l$} ; }
    \phantom{\node[latent, left=of x] (ir) {$r$} ; }

    \edge {z} {x,y};
    \edge {x} {y};
}
\end{textblock}
}


\only<2>{
\begin{textblock}{140}(10,30) \centering
\tikz{
    \node[latent] (u) {$u$} ; %

    \node[latent, fill=red!30,  below=of u, xshift=-1cm,yshift=0.6cm] (z) {$z$} ; %
    \node[latent, fill=black!15 , below=of z, xshift=-1cm,yshift=0.6cm] (x) {$x$} ; %
    \node[latent, fill=black!15, right=of x, xshift=2cm] (y) {$y$} ; %

    \phantom{\node[latent, right=of y] (il) {$l$} ; }
    \phantom{\node[latent, left=of x] (ir) {$r$} ; }


    \edge {u} {z};
    \edge {z} {x};
    \edge {u,x} {y};
}
\end{textblock}
}

\only<3>{
\begin{textblock}{140}(10,30) \centering
\tikz{
    \node[latent] (u) {$u$} ; %

    \node[latent, fill=red!30,  below=of u, xshift=1cm,yshift=0.6cm] (z) {$z$} ; %
    \node[latent, fill=black!15, below=of z, xshift=1cm, yshift=0.6cm] (y) {$y$} ; %
    \node[latent, fill=black!15 , left=of y, xshift=-2cm] (x) {$x$} ; %

    \phantom{\node[latent, right=of y] (il) {$l$} ; }
    \phantom{\node[latent, left=of x] (ir) {$r$} ; }

    \edge {u} {z,x};
    \edge {z,x} {y};
}
\end{textblock}
}

\only<4>{
\begin{textblock}{140}(10,30) \centering
\tikz{
    \node[latent] (u) {$u$} ; %

    \node[latent, fill=red!30,  below=of u, xshift=1cm,yshift=0.6cm] (z) {$z$} ; %
    \node[latent, below=of z, xshift=1cm, yshift=0.6cm] (m) {$m$} ; %
    \node[latent, fill=black!15 , left=of m, xshift=-2cm] (x) {$x$} ; %
    \node[latent, fill=black!15, right=of m] (y) {$y$} ; %

    \phantom{\node[latent, left=of x] (ir) {$r$} ; }

    \edge {u} {z,x};
    \edge {z,x} {m};
    \edge {m} {y};
}
\end{textblock}
}

\only<5>{
\begin{textblock}{140}(10,30) \centering
\tikz{
    \node[latent] (u) {$u$} ; %

    \node[latent, fill=red!30,  below=of u, xshift=-1cm,yshift=0.6cm] (z) {$z$} ; %
    \node[latent, fill=black!15 , below=of z, xshift=-1cm,yshift=0.6cm] (x) {$x$} ; %
    \node[latent, right=of x, xshift=2cm] (m) {$m$} ; %

    \node[latent, fill=black!15, right=of m] (y) {$y$} ;
    \phantom{\node[latent, left=of x] (ir) {$r$} ; }


    \edge {u} {m,z};
    \edge {z} {x};
    \edge {x} {m};
    \edge {m} {y};
}
\end{textblock}
}


\only<6>{
\begin{textblock}{140}(10,30) \centering
\tikz{
    \node[latent, fill=red!30] (z) {$z$} ; %
    \node[latent, fill=black!15 , below=of z, xshift=-1.5cm] (x) {$x$} ; %
    \node[latent, below=of z, xshift=1.5cm] (m) {$m$} ; %

    \node[latent, fill=black!15, right=of m] (y) {$y$} ;
    \phantom{\node[latent, left=of x] (ir) {$r$} ; }

    \edge {x,z} {m};
    \edge {z} {x};
    \edge {m} {y};

}
\end{textblock}
}


\only<7>{
\begin{textblock}{140}(10,30) \centering
\tikz{
    \node[latent, fill=red!30] (z) {$z$} ; %
    \node[latent, fill=black!15 , below=of z, xshift=-1.5cm] (x) {$x$} ; %
    \node[latent, below=of z, xshift=1.5cm] (m) {$m$} ; %

    \node[latent, above=of x] (ux) {$u_1$} ; %
    \node[latent, above=of m] (um) {$u_2$} ; %


    \node[latent, fill=black!15, right=of m] (y) {$y$} ;
    \phantom{\node[latent, left=of x] (ir) {$r$} ; }

    \edge {um,ux} {z};
    \edge {um,x} {m};
    \edge {ux} {x};
    \edge {m} {y};
}
\end{textblock}
}

\only<8>{
\begin{textblock}{140}(10,30) \centering
\tikz{
    \phantom{\node[latent] (ia) {$a$} ; }
    \node[latent, fill=black!15 , below=of ia, xshift=-1.5cm] (x) {$x$} ; %

    \node[latent, fill=black!15 , below=of ia, xshift=-1.5cm] (x) {$x$} ; %
    \node[latent, fill=red!30, below=of ia] (z) {$z$} ;

    \node[latent, fill=black!15, below=of ia, xshift=1.5cm] (y) {$y$} ;

    \phantom{\node[latent, right=of y] (il) {$l$} ; }
    \phantom{\node[latent, left=of x] (ir) {$r$} ; }

    \edge {x} {z};
    \edge {z} {y};
}
\end{textblock}
}


\only<9>{
\begin{textblock}{140}(10,30) \centering
\tikz{
    \phantom{\node[latent] (ia) {$a$} ; }
    \node[latent, fill=black!15 , below=of ia, xshift=-1.5cm] (x) {$x$} ; %

    \node[latent, fill=black!15 , below=of ia, xshift=-1.5cm] (x) {$x$} ; %
    \node[latent, below=of ia] (m) {$m$} ;

    \node[latent, fill=red!30, below=of m] (z) {$z$} ;
    \node[latent, fill=black!15, below=of ia, xshift=1.5cm] (y) {$y$} ;

    \phantom{\node[latent, right=of y] (il) {$l$} ; }
    \phantom{\node[latent, left=of x] (ir) {$r$} ; }

    \edge {x} {m};
    \edge {m} {z,y};

}
\end{textblock}
}

\only<10>{
\begin{textblock}{140}(10,30) \centering
\tikz{
    \phantom{\node[latent] (ia) {$a$} ; }
    \node[latent, fill=black!15 , below=of ia, xshift=-1.5cm] (x) {$x$} ; %

    \node[latent, fill=black!15 , below=of ia, xshift=-1.5cm] (x) {$x$} ; %
    \phantom{\node[latent, below=of ia] (m) {$m$} ;}

    \node[latent, fill=red!30, below=of m] (z) {$z$} ;
    \node[latent, fill=black!15, below=of ia, xshift=1.5cm] (y) {$y$} ;

    \phantom{\node[latent, right=of y] (il) {$l$} ; }
    \phantom{\node[latent, left=of x] (ir) {$r$} ; }

    \edge {x} {y,z};
    \edge {y} {z};

}
\end{textblock}
}

\only<11>{
\begin{textblock}{140}(10,30) \centering
\tikz{
    \node[latent] (ia) {$u$} ;
    \node[latent, fill=black!15 , below=of ia, xshift=-1.5cm] (x) {$x$} ; %

    \node[latent, fill=black!15 , below=of ia, xshift=-1.5cm] (x) {$x$} ; %
    \phantom{\node[latent, below=of ia] (m) {$m$} ;}



    \node[latent, fill=red!30, below=of m] (z) {$z$} ;
    \node[latent, fill=black!15, below=of ia, xshift=1.5cm] (y) {$y$} ;

    \phantom{\node[latent, right=of y] (il) {$l$} ; }
    \phantom{\node[latent, left=of x] (ir) {$r$} ; }

    \edge {x} {y,z};
    \edge {ia} {z,y};

}
\end{textblock}
}

\only<12>{
\begin{textblock}{140}(10,30) \centering
\tikz{
    \phantom{\node[latent] (ia) {$u$} ;}
    \node[latent, fill=black!15 , below=of ia, xshift=-1.5cm] (x) {$x$} ; %

    \node[latent, fill=black!15 , below=of ia, xshift=-1.5cm] (x) {$x$} ; %
    \phantom{\node[latent, below=of ia] (m) {$m$} ;}



    \node[latent, fill=red!30, below=of m] (z) {$z$} ;
    \node[latent, fill=black!15, below=of ia, xshift=1.5cm] (y) {$y$} ;

    \phantom{\node[latent, right=of y] (il) {$l$} ; }
    \phantom{\node[latent, left=of x] (ir) {$r$} ; }

    \edge {x} {y};
    \edge {y} {z};
}
\end{textblock}
}

\only<13>{
\begin{textblock}{140}(10,30) \centering
\tikz{
    \phantom{\node[latent] (ia) {$u$} ;}
    \node[latent, fill=black!15 , below=of ia, xshift=-1.5cm] (x) {$x$} ; %

    \node[latent, fill=black!15 , below=of ia, xshift=-1.5cm] (x) {$x$} ; %
    \phantom{\node[latent, below=of ia] (m) {$m$} ;}



    \node[latent, fill=red!30, right=of ia,xshift=1.5cm] (z) {$z$} ;
    \node[latent, fill=black!15, below=of ia, xshift=1.5cm] (y) {$y$} ;

    \phantom{\node[latent, right=of y] (il) {$l$} ; }
    \phantom{\node[latent, left=of x] (ir) {$r$} ; }

    \edge {x} {y};
    \edge {z} {y};
}
\end{textblock}
}


\only<14>{
\begin{textblock}{140}(10,30) \centering
\tikz{
    \node[latent, fill=red!30] (ia) {$z$} ;
    \node[latent, fill=black!15 , below=of ia, xshift=-1.5cm] (x) {$x$} ; %

    \node[latent, fill=black!15 , below=of ia, xshift=-1.5cm] (x) {$x$} ; %
    \node[latent, below=of ia] (m) {$m$} ;

    \node[latent, fill=black!15, below=of ia, xshift=1.5cm] (y) {$y$} ;

    \phantom{\node[latent, right=of y] (il) {$l$} ; }
    \phantom{\node[latent, left=of x] (ir) {$r$} ; }

    \edge {x} {m};
    \edge {m} {y};
    \edge {ia} {m};
}
\end{textblock}
}


\only<15>{
\begin{textblock}{140}(10,30) \centering
\tikz{
    \phantom{\node[latent] (ia) {$u$} ;}
    \node[latent, fill=black!15 , below=of ia, xshift=-1.5cm] (x) {$x$} ; %

    \node[latent, fill=black!15 , below=of ia, xshift=-1.5cm] (x) {$x$} ; %
    \phantom{\node[latent, below=of ia] (m) {$m$} ;}



    \node[latent, fill=red!30, left=of ia,xshift=-1.5cm] (z) {$z$} ;
    \node[latent, fill=black!15, below=of ia, xshift=1.5cm] (y) {$y$} ;

    \phantom{\node[latent, right=of y] (il) {$l$} ; }
    \phantom{\node[latent, left=of x] (ir) {$r$} ; }

    \edge {x} {y};
    \edge {z} {x};
}
\end{textblock}
}


\only<16>{
\begin{textblock}{140}(10,30) \centering
\tikz{
    \node[latent] (ia) {$u$} ;
    \node[latent, fill=black!15 , below=of ia, xshift=-1.5cm] (x) {$x$} ; %

    \node[latent, fill=black!15 , below=of ia, xshift=-1.5cm] (x) {$x$} ; %
    \phantom{\node[latent, below=of ia] (m) {$m$} ;}



    \node[latent, fill=red!30, left=of ia,xshift=-1.5cm] (z) {$z$} ;
    \node[latent, fill=black!15, below=of ia, xshift=1.5cm] (y) {$y$} ;

    \phantom{\node[latent, right=of y] (il) {$l$} ; }
    \phantom{\node[latent, left=of x] (ir) {$r$} ; }

    \edge {ia} {x,y};
    \edge {x} {y};
    \edge {z} {x};
}
\end{textblock}
}




\end{frame}

\begin{frame}[plain]
\begin{textblock}{160}(0,4)
 \centering \LARGE Estimación de efecto causal\\
 \large \only<3->{Modelos lineales}
 \end{textblock}
 \vspace{0.75cm}

 \only<1-3>{
\begin{textblock}{160}(0,24) \centering
\tikz{
  \node[latent] (z1) {$z_1$} ;
  \node[latent, below=of z1] (w1) {$w_1$} ;
  \only<1>{\node[latent, right=of w1] (z3) {$z_3$} ;}
  \only<2->{\node[latent, fill=red!30, right=of w1] (z3) {$z_3$} ;}
  \only<1>{\node[latent, right=of z3] (w2) {$w_2$} ;}
  \only<2->{\node[latent, fill=red!30, right=of z3] (w2) {$w_2$} ;}
  \node[latent, above=of w2] (z2) {$z_2$} ;
  \node[latent, fill=black!15, below=of w1] (x) {$x$} ;
  \node[latent, below=of z3] (w3) {$w_3$} ;
  \node[latent, fill=black!15, below=of w2] (y) {$y$} ;

  \onslide<3>{
    \node[const, below=of z1, xshift=-0.2cm, yshift=-0.15cm] (z1_w1) {$6$};
    \node[const, below=of z2, xshift=0.2cm, yshift=-0.15cm] (z2_w2) {$5$};
    \node[const, above=of z3, xshift=-0.6cm, yshift=0.4cm] (z3_z1) {$-4$};
    \node[const, above=of z3, xshift=0.63cm, yshift=0.43cm] (z2_z1) {$3$};
    \node[const, below=of z3, xshift=-1cm, yshift=-0.1cm] (z2_x) {$2$};
    \node[const, below=of z3, xshift=1.05cm, yshift=-0.1cm] (z2_y) {$-1$};
    \node[const, below=of w1, xshift=-0.35cm, yshift=-0.15cm] (w1_x) {$-1$};
    \node[const, below=of w2, xshift=0.2cm, yshift=-0.15cm] (w2_y) {$1$};
    \node[const, below=of x, xshift=0.8cm, yshift=0.3cm] (x_w3) {$2$};
    \node[const, below=of w3, xshift=0.8cm, yshift=0.3cm] (w3_y) {$-1$};
  }

  \edge {w3,z3,w2} {y}
  \edge {x} {w3}
  \edge {z1} {w1}
  \edge {z2} {w2}
  \edge {z1,z2} {z3}
  \edge {w1,z3} {x}

}
\end{textblock}
}

\end{frame}


\begin{frame}[plain, fragile]
\begin{textblock}{160}(0,4)
 \centering \LARGE Estimación de efecto causal\\
 \large Modelos lineales
 \end{textblock}
 \vspace{0.75cm}


\begin{textblock}{140}(10,17) \normalsize
\begin{lstlisting}
# https://github.com/glandfried/bayesian-linear-model
from linear_model import BayesianLinearModel
import numpy as np

N = 1000
z1 = np.random.uniform(-3,3, size=N)
w1 = 3*z1 + np.random.normal(size=N,scale=1)
z2 = np.random.uniform(-3,3, size=N)
w2 = 2*z2 + np.random.normal(size=N,scale=1)
z3 = -2*z1 + 2*z2 + np.random.normal(size=N,scale=1)
x = -1*w1 + 2*z3 + np.random.normal(size=N,scale=1)
w3 = 2*x + np.random.normal(size=N,scale=1)
y = 2 - 1*w3 - z3 + w2 + np.random.normal(size=N,scale=1)
\end{lstlisting}
\end{textblock}
\end{frame}


\begin{frame}[plain, fragile]
\begin{textblock}{160}(0,4)
 \centering \LARGE Estimación de efecto causal\\
 \large Modelos lineales
 \end{textblock}
 \vspace{0.75cm}



\begin{textblock}{120}(20,18) \centering
$y \sim c_0 + c_1\,x + c_2\,z_3 + c_3\,w_2$

\includegraphics[width=0.7\textwidth]{figuras/pdf/controles-modeloComplejos.pdf}
\end{textblock}


\end{frame}


\begin{frame}[plain]
\begin{textblock}{160}(0,4)
 \centering \LARGE Identificación de modelo causal \\
\end{textblock}


\begin{textblock}{140}(10,12) \centering \Large
\begin{equation*}
P(\text{Modelo causal}\,|\, \text{Datos},\, \text{Intervenciones}) = \,  ?
\end{equation*}
\end{textblock}

\begin{textblock}{80}(5,34) \centering
 \tikz{
    \node[det] (a) {$A_{_{i}}$} ; %
    \node[det, below=of a] (b) {$B_{_{i}}$} ; %
    \node[const, left= of a, xshift=-0.3cm, yshift=0.1cm] (pa) { \small
    \begin{tabular}{|c|c|}
          $A=0$  &  $A=1$   \\ \hline
        $0.5$ & $0.5$   \\ \hline
    \end{tabular}
    }; %
    \node[const, above= of pa] (npa) {\small$P(A)$};


    \node[const, left=of b, xshift=-0.3cm, yshift=-0.1cm] (pb) { \small
    \begin{tabular}{c|c|c|}
        &  $B=0$  &  $B=1$   \\ \hline
       $A=0$ & $0.95$ & $0.05$   \\ \hline
       $A=1$ & $0.05$ & $0.95$   \\ \hline
    \end{tabular}
    };
    \node[const, above= of pb] (npb) {\small$P(B|A)$};

    \node[invisible, above=of a, yshift=1cm] (ia) {};

    \plate {datos} {(a)(b)} {\tiny$i$: Dato};

    \edge {a} {b};
    }
\end{textblock}
\begin{textblock}{80}(75,34) \centering
    \tikz{
    \node[det] (a) {$A_{_{i}}$} ; %
    \node[det, below=of a] (b) {$B_{_{i}}$} ; %
    \node[const, right= of a, xshift=0.3cm, yshift=0.1cm] (pa) { \small
    \begin{tabular}{c|c|c|}
        &  $A=0$  &  $A=1$   \\ \hline
       $B=0$ & $0.95$ & $0.05$   \\ \hline
       $B=1$ & $0.05$ & $0.95$   \\ \hline
    \end{tabular}
    }; %
    \node[const, above= of pa] (npa) {\small$P(A|B)$};


    \node[const, right=of b, xshift=0.3cm, yshift=-0.1cm] (pb) { \small
    \begin{tabular}{|c|c|}
          $B=0$  &  $B=1$   \\ \hline
        $0.5$ & $0.5$   \\ \hline
    \end{tabular}
    };
    \node[const, above= of pb] (npb) {\small$P(B)$};

    \node[invisible, above=of a, yshift=1cm] (ia) {};

    \plate {datos} {(a)(b)} {\tiny$i$: Dato};

    \edge {b} {a};
    }
\end{textblock}

\end{frame}

\begin{frame}[plain]
\begin{textblock}{160}(0,4)
 \centering \LARGE Teorías causales deterministas \\
 \large Structural (or functional) causal models
\end{textblock}


\begin{textblock}{140}(10,22)
\Large Modelo causal. \large

Conjunto de funciones $F$ y variables internas $X$ y de contorno $U$
\end{textblock}

\only<2->{
\begin{textblock}{80}(0,42) \centering \large
\onslide<4->{Sub}Modelo$_{A\rightarrow B}$ \hspace{0.5cm} \phantom{.}
\end{textblock}

\begin{textblock}{30}(10,50)
\begin{align*}
A & =\only<2-3>{f_A(U_A)}\only<4->{\text{do}(t)} \\
B &= f_B(A,U_B)  \\
\end{align*}
\end{textblock}
}



\only<2->{
\begin{textblock}{30}(40,50) \centering

\tikz{
  \node[latent, minimum size= 0.2cm] (u1) {};
  \node[latent, fill = black, below =of u1, minimum size= 0.2cm, xshift=-0.7cm, yshift=0.3cm] (x1) {};
  \node[latent, fill = black, below =of x1, minimum size= 0.2cm, xshift=0.7cm, yshift=0.3cm] (x2) {};
  \node[latent,above=of x2, minimum size= 0.2cm, xshift=0.7cm, yshift=-0.3cm] (u2) {};
  \only<-3>{\edge {u1} {x1};}
  \edge {u2,x1} {x2};

}
\end{textblock}
}


\only<3->{
\begin{textblock}{80}(80,42) \centering \large
\onslide<4->{Sub}Modelo$_{B\rightarrow A}$ \hspace{0.5cm} \phantom{.}
\end{textblock}

\begin{textblock}{30}(90,50)
\begin{align*}
B &= f_B(U_B)  \\
A & =\only<2-3>{f_A(B,U_A)}\only<4->{\text{do}(t)} \\
\phantom{B} &\phantom{= f_B(X_A,U_B)}  \\
\end{align*}
\end{textblock}
}



\only<3->{
\begin{textblock}{30}(120,52) \centering

\tikz{
  \node[latent, minimum size= 0.2cm] (u1) {};
  \node[latent, fill = black, below =of u1, minimum size= 0.2cm, xshift=-0.7cm, yshift=0.3cm] (x1) {};
  \node[latent, fill = black, below =of x1, minimum size= 0.2cm, xshift=0.7cm, yshift=0.3cm] (x2) {};
  \node[latent,above=of x2, minimum size= 0.2cm, xshift=0.7cm, yshift=-0.3cm] (u2) {};
  \only<-3>{\edge {u1,x2} {x1};}
  \edge {u2} {x2};

}
\end{textblock}
}


\only<5->{
\begin{textblock}{160}(0,78) \centering
\text{\small (Potential outcome) } $Y_t(u) = Y_{M_t}(u)$ \text{\small  \ (Structural Causal Model)}
\end{textblock}
}

\end{frame}

%
% \begin{frame}[plain]
%  \begin{textblock}{160}(0,4)
%  \centering \LARGE
%  Modelos causales deterministas \\
%  \Large no lineales
% \end{textblock}
% \vspace{0.75cm}
%
%
% \begin{align*}
%  \text{Población}(t+1) = r \cdot \text{Población}(t)\cdot (1-\text{Población}(t))
% \end{align*}
%
% \vspace{0.5cm}
%
% \pause
%
% \centering
%
% \tikz{
%     \node[latent] (n1) {$x_n$} ;
%
%     \node[latent, right=of n1, xshift=3cm] (n2) {$x_{n+1}$} ;
%
%      \path[->] (n1) edge node[yshift=0.5cm] {$rx_n(1-x_n)$} (n2);
%
% }
%
% %lorenz
%
% %juego de la vida
%
% \end{frame}
%
% \begin{frame}[plain]
%  \begin{textblock}{160}(0,4)
%  \centering \LARGE
%  Modelos causales deterministas \\  \Large no lineales
% \end{textblock}
% \vspace{0.75cm}
%
% \only<1>{
% \begin{textblock}{160}(0,22)
% \centering
%  \includegraphics[page={1},width=0.6\textwidth]{figuras/poblacion.pdf}
% \end{textblock}
% }
%
% \only<2>{
% \begin{textblock}{160}(0,22)
% \centering
%  \includegraphics[page={2},width=0.6\textwidth]{figuras/poblacion.pdf}
% \end{textblock}
% }
%
% \only<3>{
% \begin{textblock}{160}(0,22)
% \centering
%  \includegraphics[page={3},width=0.6\textwidth]{figuras/poblacion.pdf}
% \end{textblock}
% }
%
% \only<4>{
% \begin{textblock}{160}(0,22)
% \centering
%  \includegraphics[page={4},width=0.6\textwidth]{figuras/poblacion.pdf}
% \end{textblock}
% }
%
% \only<5>{
% \begin{textblock}{160}(0,22)
% \centering
%  \includegraphics[page={5},width=0.6\textwidth]{figuras/poblacion.pdf}
% \end{textblock}
% }
%
% \only<6>{
% \begin{textblock}{160}(0,22)
% \centering
%  \includegraphics[page={6},width=0.6\textwidth]{figuras/poblacion.pdf}
% \end{textblock}
% }
%
% \only<7>{
% \begin{textblock}{160}(0,22)
% \centering
%  \includegraphics[page={7},width=0.6\textwidth]{figuras/poblacion.pdf}
% \end{textblock}
% }
%
% \only<8>{
% \begin{textblock}{160}(0,22)
% \centering
%  \includegraphics[page={8},width=0.6\textwidth]{figuras/poblacion.pdf}
% \end{textblock}
% }
%
%
% \only<9>{
% \begin{textblock}{160}(0,22)
% \centering
%  \includegraphics[page={9},width=0.6\textwidth]{figuras/poblacion.pdf}
% \end{textblock}
% }
%
%
% \only<10>{
% \begin{textblock}{160}(0,22)
% \centering
%  \includegraphics[page={10},width=0.6\textwidth]{figuras/poblacion.pdf}
% \end{textblock}
% }
%
% \only<11>{
% \begin{textblock}{160}(0,22)
% \centering
%  \includegraphics[page={11},width=0.6\textwidth]{figuras/poblacion.pdf}
% \end{textblock}
% }
%
% \only<12>{
% \begin{textblock}{160}(0,22)
% \centering
%  \includegraphics[page={12},width=0.6\textwidth]{figuras/poblacion.pdf}
% \end{textblock}
% }
%
% \only<13>{
% \begin{textblock}{160}(0,22)
% \centering
%  \includegraphics[page={13},width=0.6\textwidth]{figuras/poblacion.pdf}
% \end{textblock}
% }
%
%
%
% \end{frame}


\begin{frame}[plain]
\only<1->{
\begin{textblock}{160}(0,4)
 \centering \LARGE Teorías causales probabilísticas \\
 \large \only<1-6>{Notación extendida para de los modelos gráficos (Factor graph)}\only<7->{Sistemas de modelos causales}
\end{textblock}
}


\only<1>{
\begin{textblock}{80}(5,34)
\raggedleft
 \tikz{
    \node[det] (a) {$A$} ; %
    \node[det, below=of a] (b) {$B$} ; %
    \node[const, left= of a, xshift=-0.3cm, yshift=0.1cm] (pa) { \small
    \begin{tabular}{|c|c|}
          $A=0$  &  $A=1$   \\ \hline
        $0.5$ & $0.5$   \\ \hline
    \end{tabular}
    }; %
    \node[const, above= of pa] (npa) {\small$P(A)$};


    \node[const, left=of b, xshift=-0.3cm, yshift=-0.1cm] (pb) { \small
    \begin{tabular}{c|c|c|}
        &  $B=0$  &  $B=1$   \\ \hline
       $A=0$ & $0.95$ & $0.05$   \\ \hline
       $A=1$ & $0.05$ & $0.95$   \\ \hline
    \end{tabular}
    };
    \node[const, above= of pb] (npb) {\small$P(B|A)$};

    \node[invisible, right=of a, xshift=1.5cm] (ia) {};

    \edge {a} {b};
    }
\end{textblock}
}

\only<2-4>{
\begin{textblock}{80}(5,18)
\raggedleft
 \tikz{

    \node[factor] (fa) {} ; %
    \node[det, below=of fa, yshift=0.3cm] (a) {$A$} ; %
    \node[factor, below=of a, yshift=0.3cm] (fb) {} ; %
    \node[det, below=of fb, yshift=0.3cm] (b) {$B$} ; %


    \node[const, right= of fa, xshift=0cm] (npa) {\small\only<2-3>{$P(A)$}\only<4->{$P(A|\text{do}_A)$}};
    \node[const, left= of fa, xshift=0cm] (pa) { \small
      \only<2-3>{
      \begin{tabular}{|c|c|}
            $A=0$  &  $A=1$   \\ \hline
          $0.5$ & $0.5$   \\ \hline
      \end{tabular}
      }
      \only<4->{
      \begin{tabular}{c|c|c|}
            & $A=0$  &  $A=1$   \\ \hline
         {\scriptsize \text{do}$_A = 0$ } & $0.5$ & $0.5$   \\ \hline
         {\scriptsize \text{do}$_A = 1$ } & $1-\alpha$ & $\alpha$   \\ \hline
      \end{tabular}
      }
    }; %

    \onslide<4->{

      %\node[factor, right=of fa, xshift=-0.7cm] (f2a) {} ;
      %\gate {if} {(fa)(f2a)} {};
      \node[det, above=of fa, yshift=-0.3cm] (doA) {do$_A$} ;
      %\gate {if_Trata} {(f2a)} {};
      %\gate {if_noTrata} {(f2a)} {};

      \node[factor, above=of doA, yshift=-0.3cm] (fdoA) {} ;

      \node[const, right= of fdoA] (npdoA) {\small$P(\text{do}_A)$};
      \node[const, left= of fdoA, xshift=0cm] (pdoA) { \small
        \begin{tabular}{|c|c|}
            do$_A=0$  &  do$_A=1$   \\ \hline
            $1-\delta_{A}$ & $\delta_A$   \\ \hline
        \end{tabular}
    }; %

    }


    \node[const, right= of fb] (npb) {\small$P(B|A)$};
    \node[const, left=of fb, xshift=-0.3cm] (pb) { \small
    \begin{tabular}{c|c|c|}
        &  $B=0$  &  $B=1$   \\ \hline
       $A=0$ & $0.95$ & $0.05$   \\ \hline
       $A=1$ & $0.05$ & $0.95$   \\ \hline
    \end{tabular}
    };

    \node[invisible, right=of a, xshift=1.5cm] (ia) {};


    \edge {fa} {a};
    \onslide<4->{
      \edge {fdoA} {doA};
      \edge {doA} {fa};
    }
    \edge[-] {a} {fb};
    \edge {fb} {b};

    }
\end{textblock}
}

\only<1-2>{
\begin{textblock}{70}(80,42) \centering \Large
\begin{equation*}
 P(A, B | \text{Modelo}_{A \rightarrow B})
\end{equation*}
\end{textblock}
}
\only<3>{
\begin{textblock}{70}(80,42) \centering
\textbf{Nodos}: Variables y Funciones \\[0.6cm]

\textbf{Ejes}: Variable $v$ es parámetro

de la función $f$
\end{textblock}
}
\only<4>{
\begin{textblock}{70}(80,42) \centering \Large
\begin{equation*}
 P(A, B, \text{do}_A | \text{Modelo}_{A \rightarrow B })
\end{equation*}
\end{textblock}
}

\only<5>{
\begin{textblock}{70}(0,42) \centering \Large
\begin{equation*}
 P(A, B, \text{do}_A | \text{Modelo}_{B \rightarrow A })
\end{equation*}
\end{textblock}
}

\begin{textblock}{160}(15,74)
\begin{flalign*}
 \only<8>{& P(A, B, \text{do}_A | \text{Modelo}_{B \rightarrow A}) }
 \only<8>{= P(B) \, P_0(A|B)^{1-\text{do}_A} \, P_1(A)^{\text{do}_A} \, P(\text{do}_A)   \\}
 \only<9>{& P(A, B | \underbrace{\text{do}_A = 1, \text{Modelo}_{B \rightarrow A}}_{\text{Intervención}})= P(B) \, P_1(A)    \\}
 \only<10>{& P(A, B | \underbrace{\text{do}_A = 0, \text{Modelo}_{B \rightarrow A}}_{\text{Sin intervención}})= P(B) \, P_0(A|B)    \\}
 &&
\end{flalign*}
\end{textblock}





\only<5->{
\begin{textblock}{85}(25,18)
\raggedright
 \tikz{

    \node[det] (a) {$A$} ; %
    \only<-8,10->{\node[factor, below=of a] (fa) {} ; }
    \only<9>{\node[factor, below=of a, fill=black!7, draw=black!7] (fa) {} ; }
    \node[det, below=of fa] (b) {$B$} ; %
    \node[factor, below=of b] (fb) {} ; %

    \onslide<6->{
      \only<-8>{\node[det, left=of b] (doA) {do$_A$} ;}
      \only<9->{\node[det, left=of b, fill=black!20] (doA) {do$_A$} ;}
      \only<9>{\node[const, left=of doA] (odoA) {\small do$_A$=1} ;}
      \only<10>{\node[const, left=of doA] (odoA) {\small do$_A$=0} ;}
      \node[factor, below=of doA] (fdoA) {} ; %
      \node[const, left= of fdoA] (ndoA) {\small$P(\text{do}_A)$};
      \node[const, left=of ndoA] (pdoA) { \small
          \begin{tabular}{|c|c|}
                do$_A=0$  &  do$_A=1$   \\ \hline
              $1-\delta_A$ & $\delta_A$   \\ \hline
          \end{tabular}
        };
    }
    \onslide<7->{
      \only<-9>{\node[factor, left=of fa, xshift=0.725cm] (f2a) {} ;}
      \only<10->{\node[factor, left=of fa, draw=black!7, fill=black!7, xshift=0.725cm] (f2a) {} ;}
      \gate {ifA} {(fa)(f2a)} {};
      {\only<10>{\color{black!7}} \gate {ifA1} {(f2a)} {};}
      {\only<9>{\color{black!7}} \gate {ifA0} {(fa)} {};}
    }

    \only<-6>{\node[const, left=of fa] (npa) {\small \only<5>{$P(A|B)$} \only<6>{$P(A|B, \text{do}_A)$}};}
    \node[const, right= of fa, xshift=0.3cm] (pa) { \small
      \only<5>{
        \begin{tabular}{c|c|c|}
            &  $A=0$  &  $A=1$   \\ \hline
          $B=0$ & $0.95$ & $0.05$   \\ \hline
          $B=1$ & $0.05$ & $0.95$   \\ \hline
        \end{tabular}
      }
      \only<6>{
        \begin{tabular}{c|c|c|}
            &  $A=0$  &  $A=1$   \\ \hline
          {\scriptsize $B=0$ do$_A = 0$} & $0.95$ & $0.05$   \\ \hline
          {\scriptsize $B=1$ do$_A = 0$} & $0.05$ & $0.95$   \\ \hline
          {\scriptsize do$_A = 1$} & $1-\alpha$ & $\alpha$   \\ \hline
        \end{tabular}
      }
    }; %
    \only<7->{
      {\only<9>{\color{black!7}}\node[const, right=of ifA] (npa) {\small $P_0(A|B)$};}
      {\only<10>{\color{black!7}} \node[const, left=of ifA] (np2a) {\small $P_1(A)$}; }
      {\only<9>{\color{black!7}}\node[const, right= of npa] (pa) { \small
        \begin{tabular}{c|c|c|}
            &  $A=0$  &  $A=1$   \\ \hline
          $B=0$ & $0.95$ & $0.05$   \\ \hline
          $B=1$ & $0.05$ & $0.95$   \\ \hline
        \end{tabular}
      };}
      {\only<10>{\color{black!7}} \node[const, left= of np2a] (p2a) { \small
        \begin{tabular}{|c|c|}
              $A=0$  &  $A=1$   \\ \hline
           $1-\alpha$ & $\alpha$   \\ \hline
        \end{tabular}
      };}
    }


    \node[const, right= of fb] (npb) {\small$P(B)$};
    \node[const, right=of npb, xshift=0.3cm] (pb) { \small
    \begin{tabular}{|c|c|}
          $B=0$  &  $B=1$   \\ \hline
        $0.5$ & $0.5$   \\ \hline
    \end{tabular}
    };

%     \node[invisible, above=of a, yshift=0.2cm] (ia) {};
%     \node[invisible, left=of a, xshift=0.2cm] (ia) {};


    {\only<9>{\color{black!7}} \edge {fa} {a};}
    \edge {fb} {b};
    {\only<9>{\color{black!7}} \edge[-] {b} {fa}; }
    \onslide<6->{
      \edge {fdoA} {doA};
    }
    \onslide<6>{
      \edge[-] {doA} {fa};
    }
    \onslide<7->{
      \edge[-,dashed] {doA} {ifA0};
      {\only<10>{\color{black!7}} \edge {f2a} {a}; }
    }
    }
\end{textblock}
}

\end{frame}

\begin{frame}[plain]
\begin{textblock}{160}(0,4)
 \centering \LARGE Identificación de modelo causal\\
 \large A través de intervenciones do$(\cdot)$
 \end{textblock}
 \vspace{0.75cm}

\begin{textblock}{140}(3,24)
Datos:

\vspace{0.3cm}
\normalsize
\begin{tabular}{c|c|c|c|}
    $i$ & do$_{Ai}$ &  $A_i$  &  $B_i$   \\ \hline
    \onslide<1-7>{1 & $0$ & $1$ & $1$  \\ \hline
    {\tiny$\dots$} & $0$ & {\tiny$\dots$} & {\tiny$\dots$}   \\ \hline
    10 & $0$ & $0$ & $0$   \\ \hline \hline}11 & $1$ & $0$ & $1$   \\ \hline
    12 & $1$ & $1$ & $0$   \\ \hline
    {\tiny$\dots$} & $1$ & {\tiny$\dots$} & {\tiny$\dots$}  \\ \hline
\end{tabular}
\end{textblock}

\only<2->{
\begin{textblock}{110}(46,15) \normalsize
\begin{flalign*}
&
\only<2>{P(\text{Modelo}_{B\rightarrow A}|\text{Datos})}
\only<3->{\frac{P(\text{Modelo}_{B\rightarrow A}|\text{Datos})}{P(\text{Modelo}_{A\rightarrow B}|\text{Datos})}}
=
\only<2>{\frac{P(\text{Datos}|\text{M}_{B\rightarrow A}) \, P(\text{M}_{B\rightarrow A}) }{P(\text{Datos})}}
\only<3>{\frac{P(\text{Datos}|\text{M}_{B\rightarrow A}) \, P(\text{M}_{B\rightarrow A}) }{P(\text{Datos}|\text{M}_{A\rightarrow B}) \, P(\text{M}_{A\rightarrow B})}}
\only<4->{\frac{P(\text{Datos}|\text{M}_{B\rightarrow A})}{P(\text{Datos}|\text{M}_{A\rightarrow B}) }}
\\[0.5cm] &
\only<5>{ = \frac{\prod_i^n P(B_i, A_i, \text{do}_{A_i}|\text{M}_{BA}) }{\prod_i^n P(B_i, A_i, \text{do}_{A_i}|\text{M}_{AB}) }  }
\only<6>{ = \frac{\prod_i^n P(B_i|\text{M}_{_{BA}}) P_0(A_i|B_i,\text{M}_{_{BA}})^{1-\text{do}_A} P_1(A_i|\text{M}_{_{BA}})^{\text{do}_A}  P(\text{do}_A|\text{M}_{BA}) }{\prod_i^n P(A_i|\text{do}_A,\text{M}_{_{AB}}) P(B_i|A_i,\text{M}_{_{AB}}) P(\text{do}_A|\text{M}_{AB}) }}
\only<7-8>{ = \frac{\prod_i^n  P(B_i|\text{M}_{_{BA}}) P_0(A_i|B_i,\text{M}_{_{BA}})^{1-\text{do}_A} P_1(A_i|\text{M}_{_{BA}})^{\text{do}_A}  \phantom{P(\text{do}_A|\text{M}_{BA})} }{\prod_i^n  P(A_i|\text{do}_A,\text{M}_{_{AB}}) P(B_i|A_i,\text{M}_{_{AB}}) \phantom{P(\text{do}_A|\text{M}_{AB})} }}
\only<9>{ = \prod_{i=11}^n  \frac{P(B_i|\text{M}_{_{BA}}) P_1(A_i|\text{M}_{_{BA}}) }{P(B_i|A_i,\text{M}_{_{AB}}) P(A_i|\text{do}_A = 1,\text{M}_{_{AB}}) }  }
\only<10>{ = \prod_{i=11}^n  \frac{P(B_i|\text{M}_{_{BA}}) \alpha^{A_i} \, (1 - \alpha)^{1-A_i}}{P(B_i|A_i,\text{M}_{_{AB}}) \alpha^{A_i} \, (1 - \alpha)^{1-A_i}  }  }
\only<11->{ = \prod_{i=11}^n \frac{P(B_i|\text{M}_{_{BA}})}{P(B_i|A_i,\text{M}_{_{AB}})  }  }
&&
\end{flalign*}

\centering
\only<12>{
\includegraphics[width=0.7\textwidth]{figuras/identificacion.pdf}
}
\end{textblock}
}

\end{frame}


\begin{frame}[plain]
\begin{textblock}{160}(0,4)
 \centering \LARGE Identificación de modelo causal\\
 \large El conocimiento experto
 \end{textblock}
 \vspace{0.75cm}


 \begin{textblock}{160}(0,42) \Large \centering
 La principal fuente de información para la identificación

 modelos causales alternativos es el conocimiento experto.
 \end{textblock}




\end{frame}



\begin{frame}[plain,noframenumbering]
\centering \vspace{0.5cm}
\includegraphics[width=1\textwidth]{../../auxiliar/static/BP.png}
\end{frame}

%
% \begin{frame}[plain]
% \begin{textblock}{96}(0,6.5)\centering
% {\transparent{0.9}\includegraphics[width=0.8\textwidth]{../../auxiliar/static/inti.png}}
% \end{textblock}
%
% \begin{textblock}{160}(96,5.5)
% \includegraphics[width=0.35\textwidth]{../../auxiliar/static/pachacuteckoricancha}
% \end{textblock}
% \end{frame}





\end{document}



