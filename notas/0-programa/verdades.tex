\documentclass[shownotes,aspectratio=169]{beamer}

\input{../../auxiliar/tex/diapo_encabezado.tex}
\input{../../auxiliar/tex/tikzlibrarybayesnet.code.tex}
 \mode<presentation>
 {
 %   \usetheme{Madrid}      % or try Darmstadt, Madrid, Warsaw, ...
 %   \usecolortheme{default} % or try albatross, beaver, crane, ...
 %   \usefonttheme{serif}  % or try serif, structurebold, ...
  \usetheme{Antibes}
  \setbeamertemplate{navigation symbols}{}
 }
 
\usepackage{todonotes}
\setbeameroption{show notes}

\estrue


%\title[Bayes del Sur]{}

\begin{document}

\color{black!85}
\large

 
%\setbeamercolor{background canvas}{bg=gray!15}


\begin{frame}[plain,noframenumbering]
\begin{textblock}{160}(00,8)
\centering
\LARGE Seminario abierto de inferencia Bayesiana \\
\Large ``Verdades empíricas''
\end{textblock}


\begin{textblock}{160}(00,38) \centering
\Large Taller Pacha Pampas \\
\large La vida como fuente del conocimiento empírico
\end{textblock}



\begin{textblock}{160}(00,62) \centering

Organiza: \href{https://github.com/BayesDeLasProvinciasUnidasDelSur/curso}{Bayes de la Provincias Unidas del Sur}

\vspace{0.5cm}

31 Marzo 2022, Buenos Aires
\end{textblock}

\end{frame}


\begin{frame}[plain]
 \begin{textblock}{160}(0,06) \centering
  \LARGE La ciencia tiene pretensión de verdad
\end{textblock}

 \begin{textblock}{80}(0,24) \centering
 \Large Verdades formales 
\end{textblock}

\begin{textblock}{60}(10,36) \centering
 Validadas dentro de \\ sistemas axiomáticos cerrados \\[0.2cm]
 
 
 Sin incertidumbre
\end{textblock}

\begin{textblock}{80}(80,24) \centering
 \Large Verdades empíricas
\end{textblock}

\begin{textblock}{60}(90,36) \centering
 Validadas dentro de \\ sistemas naturales abiertos\\[0.2cm]
  
 Con incertidumbre
\end{textblock}

\only<2>{
\begin{textblock}{160}(0,66) \centering
\Large ¿Cuáles son entonces la verdades empíricas?
\end{textblock}
}


\end{frame}


\begin{frame}[plain]
\begin{textblock}{80}(34,14)
 \huge \textcolor{black!50}{Sorpresa}
\end{textblock}

\begin{textblock}{47}(113,73.5)
\centering \LARGE  \textcolor{black!5}{Supervivencia}
\end{textblock}

\begin{textblock}{80}(100,27)
\LARGE  \textcolor{black!10}{Creencia}
\end{textblock}

\begin{textblock}{80}(44,61)
\LARGE  \textcolor{black!15}{Dato}
\end{textblock}

 %\vspace{2cm}brown
%\maketitle
\Wider[2cm]{
\includegraphics[width=1\textwidth]{../../auxiliar/static/peligro_predador}
}
\end{frame}
 


\begin{frame}[plain]
 \begin{textblock}{160}(0,4)
 \centering
  \LARGE Acuerdos intersubjetivos
 \end{textblock}

 
 \only<1>{
  \begin{textblock}{80}(-10,20) \centering
 \scalebox{0.8}{
\tikz{ %
         \node[factor, minimum size=1cm] (p1) {} ;
         \node[factor, minimum size=1cm, xshift=1.5cm] (p2) {} ;
         \node[factor, minimum size=1cm, xshift=3cm] (p3) {} ;
         
         
         \node[const, above=of p1, yshift=0.1cm] (np1) {\Large $?$};
         \node[const, above=of p2, yshift=0.1cm] (np2) {\Large $?$};
         \node[const, above=of p3, yshift=0.1cm] (np3) {\Large $?$};
         } 
}
\end{textblock}
% 
}
 
\only<2>{
  \begin{textblock}{80}(-10,20) \centering
 \scalebox{0.8}{
\tikz{ %
         \node[factor, minimum size=1cm] (p1) {} ;
         \node[factor, minimum size=1cm, xshift=1.5cm] (p2) {} ;
         \node[factor, minimum size=1cm, xshift=3cm] (p3) {} ;
         
         
         \node[const, above=of p1, yshift=-0.05cm] (np1) {\Large $1/3$};
         \node[const, above=of p2, yshift=-0.05cm] (np2) {\Large $1/3$};
         \node[const, above=of p3, yshift=-0.05cm] (np3) {\Large $1/3$};
         } 
}
\end{textblock}
% 
}

\only<2->{
\begin{textblock}{100}(55,24) \centering
\Large Principio de indiferencia \\
\normalsize
Primer acuerdo intersubjetivo en contextos de incertidumbre
\end{textblock}
}

 
\only<3>{
 \begin{textblock}{80}(-10,20) \centering
 \scalebox{0.8}{
\tikz{ %
         \node[factor, minimum size=1cm] (p1) {\includegraphics[width=0.05\textwidth]{../../auxiliar/static/cerradura.png}} ;
         \node[det, minimum size=1cm, xshift=1.5cm] (p2) {\includegraphics[width=0.06\textwidth]{../../auxiliar/static/dedo.png}} ;
         \node[factor, minimum size=1cm, xshift=3cm] (p3) {} ;
         
         
         \node[const, above=of p1, yshift=0.1cm] (np1) {\Large $?$};
         \node[const, above=of p2, yshift=0.1cm] (np2) {\Large $0$};
         \node[const, above=of p3, yshift=0.1cm] (np3) {\Large $?$};
         } 
}
\end{textblock}
}

\only<4->{
 \begin{textblock}{80}(-10,20) \centering
 \scalebox{0.8}{
\tikz{ %
         \node[factor, minimum size=1cm] (p1) {\includegraphics[width=0.05\textwidth]{../../auxiliar/static/cerradura.png}} ;
         \node[det, minimum size=1cm, xshift=1.5cm] (p2) {\includegraphics[width=0.06\textwidth]{../../auxiliar/static/dedo.png}} ;
         \node[factor, minimum size=1cm, xshift=3cm] (p3) {} ;
         
         
         \node[const, above=of p1, yshift=-0.05cm] (np1) {\Large $1/3$};
         \node[const, above=of p2, yshift=0.1cm] (np2) {\Large $0$};
         \node[const, above=of p3, yshift=-0.05cm] (np3) {\Large $2/3$};
         } 
}
\end{textblock}
}


\only<3->{
\begin{textblock}{80}(-10,36) \centering
Datos \\[0.5cm]

\only<4->{
 Modelo causal ``Monty Hall'' \\[0.3cm]
\scalebox{0.6}{
\tikz{        
    
    \node[latent] (d) {\includegraphics[width=0.10\textwidth]{../../auxiliar/static/dedo.png}} ;
    \node[const,left=of d] (nd) {\Large $s$} ;
    
    \node[latent, above=of d, xshift=-1.5cm] (r) {\includegraphics[width=0.12\textwidth]{../../auxiliar/static/regalo.png}} ;
    \node[const,left=of r] (nr) {\Large $r$} ;
    
    
    \node[latent, fill=black!30, above=of d, xshift=1.5cm] (c) {\includegraphics[width=0.12\textwidth]{../../auxiliar/static/cerradura.png}} ;
    \node[const,left=of c] (nc) {\Large $c$} ;
         
    \edge {r,c} {d};
}
}
}
\end{textblock}
}


\only<3>{
\begin{textblock}{160}(0,58) \centering \large 
¿Cómo podemos dar continuidad a los acuerdos intersubjetivos?
\end{textblock}
}

\only<4>{
\begin{textblock}{100}(55,54) \centering
\Large Inferencia Bayesiana \\
\normalsize
Creencia previa que sigue siendo compatible

con la evidencia empírica y formal (datos y modelo)
\end{textblock}
}

\end{frame}



\begin{frame}[plain]
\begin{textblock}{160}(0,4)
\centering \LARGE Modelos causales alternativos
\end{textblock}

\begin{textblock}{160}(14,12) 
\begin{equation*}
 P(\text{Modelo}|\text{Datos}) = \frac{\only<1->{\overbrace{P(\text{Datos}|\text{Modelo})}^{\text{\footnotesize Predicción a priori}}} \only<1->{P(\text{Modelo})} }{ P(\text{Datos})} \phantom{\frac{\overbrace{P(\text{Datos}|\text{Modelo})}^{\text{Evidencia}}}{ P(\text{Datos})}}
\end{equation*}
\end{textblock}
% 
% \only<2>{
% \begin{textblock}{160}(0,47) 
% \begin{align*}
% P(\text{Data}|\text{Modelo}) & = \sum_{i} P(\text{Data}|\text{Hypothesis}_i,\text{Model}) P(\text{Hypothesis}_i|\text{Model}) 
% \end{align*}
% \end{textblock}
% }



\only<2>{

\begin{textblock}{140}(10,30) 
\centering
\includegraphics[width=0.66\textwidth]{figuras/monty_hall_selection.pdf} \hspace{2cm}
\end{textblock}

\begin{textblock}{80}(86,30)
\centering
\scalebox{0.5}{
\tikz{        
    
    \node[latent] (d) {\includegraphics[width=0.10\textwidth]{../../auxiliar/static/dedo.png}} ;
    \node[const,left=of d] (nd) {\Large $s$} ;
    
    \node[latent, above=of d, xshift=-1.5cm] (r) {\includegraphics[width=0.12\textwidth]{../../auxiliar/static/regalo.png}} ;
    \node[const,left=of r] (nr) {\Large $r$} ;
    
    
    \node[latent, fill=black!30, above=of d, xshift=1.5cm] (c) {\includegraphics[width=0.12\textwidth]{../../auxiliar/static/cerradura.png}} ;
    \node[const,left=of c] (nc) {\Large $c$} ;
         
    \edge {r,c} {d};
}
}
\end{textblock}


\begin{textblock}{80}(86,64)
\centering
\scalebox{0.5}{
 \tikz{            
    \node[latent,] (r) {\includegraphics[width=0.12\textwidth]{../../auxiliar/static/regalo.png}} ;
    \node[const,left=of r] (nr) {\Large $r$} ;    
    
    
    \node[latent, below=of r] (d) {\includegraphics[width=0.10\textwidth]{../../auxiliar/static/dedo.png}} ;
    \node[const, left=of d] (nd) {\Large $s$} ;

    \edge {r} {d};
    
}
}
\end{textblock}
}

\end{frame}

\begin{frame}[plain]
\begin{textblock}{160}(0,4)
\centering \LARGE Isomorfismo Bayesiano-Evolutivo
\end{textblock}
\vspace{2cm}

\centering
  \includegraphics[width=0.7\textwidth]{../../auxiliar/static/biomass.jpg} 
  
  \vspace{0.2cm}
  
  \footnotesize Distribución de la biomasa en la tierra (Bar-On 2018)
  
  \pause

  \vspace{0.7cm}
  
  \large 
  La fotosíntesis es la fuente de la toda la energía vital
  
\end{frame}


\begin{frame}[plain]
\begin{textblock}{160}(0,4)
\centering \LARGE Agricultura \\
\Large Nuestra cooperación con las plantas
\end{textblock}
\centering


\vspace{1.8cm}

\includegraphics[width=0.24\textwidth]{../../auxiliar/static/output/egipto1.jpeg} 
\includegraphics[width=0.24\textwidth]{../../auxiliar/static/output/egipto2.jpeg} 
\includegraphics[width=0.24\textwidth]{../../auxiliar/static/output/egipto3.jpeg} 
\includegraphics[width=0.24\textwidth]{../../auxiliar/static/output/egipto4.jpeg} 
    
\pause

\vspace{0.4cm}

Las 60.000.000 ha de Pampa Humeda alcanza para 20 veces nuestra población

\end{frame}

\begin{frame}[plain]
\begin{textblock}{160}(0,4)
\centering \LARGE Agronegocio \\
\Large Ruptura de la cooperación y pérdida de biodiversidad
\end{textblock}
\centering \vspace{1.5cm}

\includegraphics[width=0.46\textwidth]{../../auxiliar/static/chaco1984.jpg} 
\includegraphics[width=0.46\textwidth]{../../auxiliar/static/chaco2022.jpg} 
\end{frame}

\begin{frame}[plain]
\begin{textblock}{160}(0,4)
\centering \LARGE Otros temas \\
\Large Estimación de habilidad estado-del-arte
\end{textblock}
\centering \vspace{2cm}

\includegraphics[width=0.95\textwidth]{../../auxiliar/static/atp}

\end{frame}


\begin{frame}[plain]
\centering

\texttt{bayesdelsur@gmail.com}

\vspace{0.5cm}

  \includegraphics[width=0.35\textwidth]{../../auxiliar/static/pachacuteckoricancha.jpg}
\end{frame}



\end{document}



