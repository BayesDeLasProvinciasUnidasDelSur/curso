\documentclass[shownotes,aspectratio=169]{beamer}

\input{../../auxiliar/tex/diapo_encabezado.tex}
\input{../../auxiliar/tex/tikzlibrarybayesnet.code.tex}
 \mode<presentation>
 {
 %   \usetheme{Madrid}      % or try Darmstadt, Madrid, Warsaw, ...
 %   \usecolortheme{default} % or try albatross, beaver, crane, ...
 %   \usefonttheme{serif}  % or try serif, structurebold, ...
  \usetheme{Antibes}
  \setbeamertemplate{navigation symbols}{}
 }
 
\usepackage{todonotes}
\setbeameroption{show notes}

%\title[Bayes del Sur]{}

\begin{document}

\color{black!85}
\large

 \begin{frame}[plain,noframenumbering]


\begin{textblock}{160}(0,0)
\includegraphics[width=1\textwidth]{../../auxiliar/static/deforestacion}
\end{textblock}

\begin{textblock}{80}(18,9)
\textcolor{black!15}{\fontsize{44}{55}\selectfont Verdades}
\end{textblock}

\begin{textblock}{47}(85,70)
\centering \textcolor{black!15}{{\fontsize{52}{65}\selectfont Empíricas}}
\end{textblock}

\begin{textblock}{80}(100,28)
\LARGE  \textcolor{black!15}{\rotatebox[origin=tr]{-3}{\scalebox{9}{\scalebox{1}[-1]{$p$}}}}
\end{textblock}

\begin{textblock}{80}(66,43)
\LARGE  \textcolor{black!15}{\scalebox{6}{$=$}}
\end{textblock}

\begin{textblock}{80}(36,29)
\LARGE  \textcolor{black!15}{\scalebox{9}{$p$}}
\end{textblock}

\vspace{2cm}
\maketitle


%
% \begin{textblock}{160}(01,81)
% \footnotesize \textcolor{black!5}{Congreso Bayesiano Plurinacional 2023} \\}
% \end{textblock}

\end{frame}

%\setbeamercolor{background canvas}{bg=gray!15}
%
% \begin{frame}[plain,noframenumbering]
%
% \begin{textblock}{160}(0,0)
% \includegraphics[width=1\textwidth]{../../auxiliar/static/fuego}
% \end{textblock}
%
% \begin{textblock}{160}(4,26)
% \LARGE \textcolor{black!5}{\fontsize{22}{0}\selectfont \textbf{Sorpresa: el problema}}
% \end{textblock}
% \begin{textblock}{160}(4,34)
% \LARGE \textcolor{black!5}{\fontsize{22}{0}\selectfont \textbf{de la comunicación}}
% \end{textblock}
% \begin{textblock}{160}(4,42)
% \LARGE \textcolor{black!5}{\fontsize{22}{0}\selectfont \textbf{con la realidad}}
% \end{textblock}
% % \begin{textblock}{160}(3,82)
% % \LARGE \textcolor{black!15}{\fontsize{22}{0}\selectfont \textbf{3}}
% % \end{textblock}
%
%
%
% \begin{textblock}{55}[0,0](88,25)
% \begin{turn}{0}
% \parbox{7cm}{\sloppy\setlength\parfillskip{0pt}
% \textcolor{black!0}{Capítulo 3} \\
% \small\textcolor{black!5}{\hspace{0.05cm}La estructura invariante del dato empírico:} \\
% \small\textcolor{black!5}{\hspace{0.1cm}fuente, realidad causal, señal, canal,} \\ \small\textcolor{black!5}{\hspace{0.05cm}percepción, modelo causal, estimación.} \\
% \small\textcolor{black!5}{\hspace{-0.15cm}Base empírica y datos teóricos. Máxima} \\
% \small\textcolor{black!5}{\hspace{-0.35cm}incertidumbre y mínima sorpresa. Información.} \\
% }
% \end{turn}
% \end{textblock}
%
%
% \end{frame}

\begin{frame}[plain,noframenumbering]
\begin{textblock}{160}(0,4) \centering
\Large Unidad 3 \\
\LARGE El problema de la comunicación con la realidad
\end{textblock}
\vspace{1.7cm}


{\Large Temas.} \\[0.2cm]

$\bullet$ Base empírica \\
$\bullet$ Estructura invariante del dato \\
$\bullet$ El lugar de las hipótesis en el dato \\
$\bullet$ La hipótesis indicadora universal \\
$\bullet$ Máxima incertidumbre y mínima sorpresa. \\
$\bullet$ La sorpresa como fuente de información \\
$\bullet$ Información esperada y codificacion óptima \\

\end{frame}

\begin{frame}[plain]
\begin{textblock}{160}(0,4) \centering
\LARGE Base empírica \\
\Large \only<-9>{¿Está dado el dato?}\only<10>{\textbf{El dato se construye}}
\end{textblock}

\only<2>{
\begin{textblock}{160}(0,28) \centering
 \Large ¿Cuál es el conjunto de elementos del mundo \\ que sirven de evidencia indubitable (dato)?
\end{textblock}
}

\only<3-9>{
\begin{textblock}{130}(15,21)
 $\bullet$ BE Filosófica: $\emptyset$ \\
 \only<4->{$\bullet$ BE Epistemológica: Precios\\ }
 \only<5>{$\bullet$ Teoría: Inflación}\only<6->{$\bullet$ BE Metodológica$_1$: Inflación \\}
\only<7>{$\bullet$ Teoría: Producto Bruto Interno}\only<8->{$\bullet$ BE Metodológica$_2$: Producto Bruto Interno\\}
\only<9>{$\bullet$ Teoría: Política pública}
\end{textblock}
}

\only<10>{
\begin{textblock}{130}(15,25) \Large \centering
Depende del conjunto de supuestos \\
que una comunidad no pone en duda!
\end{textblock}
}

\only<4>{
\begin{textblock}{160}(0,48) \centering
\includegraphics[width=0.62\textwidth, page={1}]{figuras/baseEmpirica.pdf}
\end{textblock}}
\only<5>{
\begin{textblock}{160}(0,48) \centering
\includegraphics[width=0.62\textwidth, page={2}]{figuras/baseEmpirica.pdf}
\end{textblock}}
\only<6>{
\begin{textblock}{160}(0,48) \centering
\includegraphics[width=0.62\textwidth, page={3}]{figuras/baseEmpirica.pdf}
\end{textblock}}
\only<7>{
\begin{textblock}{160}(0,48) \centering
\includegraphics[width=0.62\textwidth, page={4}]{figuras/baseEmpirica.pdf}
\end{textblock}}
\only<8>{
\begin{textblock}{160}(0,48) \centering
\includegraphics[width=0.62\textwidth, page={5}]{figuras/baseEmpirica.pdf}
\end{textblock}}
\only<9->{
\begin{textblock}{160}(0,48) \centering
\includegraphics[width=0.62\textwidth, page={6}]{figuras/baseEmpirica.pdf}
\end{textblock}}


\end{frame}

\begin{frame}[plain]
\begin{textblock}{160}(0,4) \centering
\LARGE La posverdad\\
\Large ¿Todo es igualmente válido?
\end{textblock}

\only<2>{\vspace{1.2cm} \centering \Large
Si los datos se construyen entonces no hay una verdad, \\ hay una pluralidad de verdades igualmente válidas
}

\only<6>{\vspace{1.2cm} \centering \Large
El retorno al criterio de autoridad \\[0cm]
\large
La Verdad, como el poder, sale de la boca del fusil
}

\only<3-5>{
\begin{textblock}{50}(3,26) \centering
\includegraphics[width=1\textwidth, page={6}]{../../auxiliar/static/sidewalk_bubblegum_1997_1}
\end{textblock}}
 \only<4-5>{
\begin{textblock}{50}(55,26) \centering
\includegraphics[width=1\textwidth, page={6}]{../../auxiliar/static/sidewalk_bubblegum_1997_2}
\end{textblock}}
\only<5>{
\begin{textblock}{50}(107,26) \centering
\includegraphics[width=1\textwidth, page={6}]{../../auxiliar/static/sidewalk_bubblegum_1997_4}
\end{textblock}}
\end{frame}


\begin{frame}[plain,noframenumbering]

\begin{textblock}{160}(0,0)
\includegraphics[width=1\textwidth]{../../auxiliar/static/fuego}
\end{textblock}

\begin{textblock}{160}(4,26)
\LARGE \textcolor{black!5}{\fontsize{22}{0}\selectfont \textbf{Sorpresa: el problema}}
\end{textblock}
\begin{textblock}{160}(4,34)
\LARGE \textcolor{black!5}{\fontsize{22}{0}\selectfont \textbf{de la comunicación}}
\end{textblock}
\begin{textblock}{160}(4,42)
\LARGE \textcolor{black!5}{\fontsize{22}{0}\selectfont \textbf{con la realidad}}
\end{textblock}
% \begin{textblock}{160}(3,82)
% \LARGE \textcolor{black!15}{\fontsize{22}{0}\selectfont \textbf{3}}
% \end{textblock}



\begin{textblock}{55}[0,0](88,25)
\begin{turn}{0}
\parbox{7cm}{\sloppy\setlength\parfillskip{0pt}
\textcolor{black!0}{Capítulo 3} \\
\small\textcolor{black!5}{\hspace{0.05cm}La estructura invariante del dato empírico:} \\
\small\textcolor{black!5}{\hspace{0.1cm}fuente, realidad causal, señal, canal,} \\ \small\textcolor{black!5}{\hspace{0.05cm}percepción, modelo causal, estimación.} \\
\small\textcolor{black!5}{\hspace{-0.15cm}Base empírica y datos teóricos. Máxima} \\
\small\textcolor{black!5}{\hspace{-0.35cm}incertidumbre y mínima sorpresa. Información.} \\
}
\end{turn}
\end{textblock}


\end{frame}


\begin{frame}[plain]
\begin{textblock}{160}(0,4) \centering
\LARGE La comunicación con la realidad
\end{textblock}




\begin{textblock}{160}(0,25) \centering \Large

El problema. \\

\large

Interpretar correctamente un mensaje recibido por un canal ruidoso.
\end{textblock}

\only<2->{
\begin{textblock}{140}(10,48) \centering
\Large Soluciones. \\

\large

Física: acercarme para escuchar mejor \\
\only<2>{Técnica: la construcción del dato empírico}\only<3->{\textbf{Técnica: la construcción del dato empírico} \\[0.6cm]}
\only<4>{\Large Eliminar el ruido de la comunicación}
\end{textblock}
}

\end{frame}


\begin{frame}[plain]
 \begin{textblock}{160}(0,4)
 \centering \LARGE
 Los datos como funciones proposicionales
\end{textblock}
\vspace{0.75cm}

\end{frame}

\begin{frame}[plain]
 \begin{textblock}{160}(0,4)
 \centering \LARGE
 Los datos como funciones proposicionales
\end{textblock}
\vspace{0.75cm}

\begin{textblock}{160}(0,20)
\begin{equation*}
 f(x) = y
\end{equation*}
\end{textblock}

\begin{textblock}{160}(43,33)
\begin{itemize}
 \item[$x$]
    \textbf{\en{Unit of analysis}\es{Unidad de análisis}} (UA)
 \item[$f$]
   \en{\textbf{Variable} of the unit of analysis}
   \es{\textbf{Variable} de la unidad de análisis} (V)
 \item[$y$]
   \en{\textbf{Value} of the variable}
   \es{\textbf{Resultado} o valor de la variable} (R)
\end{itemize}
\end{textblock}


\only<2>{
\begin{textblock}{160}(0,65) \centering
 \emph{Altura}(Gustavo) = $1.78$m
\end{textblock}
}

\only<3-4>{
\begin{textblock}{160}(0,65) \centering
 \emph{Ideología}(Partido Comunista) = Comunista \\
 \only<4>{P(\emph{Ideología}(Partido Comunista) = Comunista) = 0.8}
\end{textblock}
}

\only<5>{
\begin{textblock}{160}(0,65) \centering
 \emph{Habilidad}(Maradona) $>$ \emph{Habilidad}(Messi)
\end{textblock}
}

\only<5>{
\begin{textblock}{140}(10,60)
\begin{framed} \centering
   \en{The meaning of data is implicit in their \textbf{operationalization}}
   \es{El significado preciso de la función depende de la \textbf{operacionalización}}
   \end{framed}
\end{textblock}
}
\end{frame}



\begin{frame}[plain]
 \begin{textblock}{160}(0,4)
 \centering \LARGE
 Estructura invariante del dato
\end{textblock}
\vspace{0.75cm}

\centering

\only<1>{
\begin{textblock}{160}(0,22)
\begin{tabular}{clcccc}
Resultado (R) & \multicolumn{1}{r|}{} &  & Variable (V) &  &  \multicolumn{1}{|r}{Unidad de análisis (UA) } \\ \hline
 \phantom{Resultado (R)} & & \phantom{=} & \phantom{Procedimientos (P)} &        &    \phantom{Unidad de análisis (UA)}
\end{tabular}
\end{textblock}
}


\only<2>{
\begin{textblock}{160}(0,22)
\begin{tabular}{clcccc}
Resultado (R) & \multicolumn{1}{r|}{} &  & Habilidad (V) &  &  \multicolumn{1}{|c}{Unidad de análisis (UA)} \\ \hline
\phantom{Resultado (R)} & & \phantom{=} & \phantom{Procedimientos (P)} &        &    \phantom{Unidad de análisis (UA)}
\end{tabular}
\end{textblock}
}

\only<3-4>{
\begin{textblock}{160}(0,22)
\begin{tabular}{clcccc}
Resultado (R) & \multicolumn{1}{r|}{} &  & Habilidad (V) &  &  \multicolumn{1}{|c}{Tenistas (UA)} \\ \hline
\phantom{Resultado (R)} & & \phantom{=} & \phantom{Procedimientos (P)} &        &    \phantom{Unidad de análisis (UA)}
\end{tabular}
\end{textblock}
}

\only<4>{
\begin{textblock}{160}(0,36)
 \centering
 ¿Y el Resultado (R)? ¿Cuál es el valor de la variable?
\end{textblock}
}



\only<5-6>{
\begin{textblock}{160}(0,22)
\begin{tabular}{clcccc}
Resultado (R) & \multicolumn{1}{r|}{} &  & Habilidad (V) &  &  \multicolumn{1}{|c}{Tenistas (UA)} \\ \hline
   &  \multicolumn{1}{r|}{}    &  & Dimensiones (D) &  & \multicolumn{1}{|r}{} \\
                 Indicador (I)  &   & =  &  &  &  Fuente de datos (F) \\
 & \multicolumn{1}{r|}{} &  & Procedimientos (P) &        &    \multicolumn{1}{|r}{}   \\
\phantom{Resultado (R)} & & \phantom{=} & \phantom{Procedimientos (P)} &                       &    \phantom{Unidad de análisis (UA)}
\end{tabular}
\end{textblock}
}

 \only<7-8>{
\begin{textblock}{160}(0,22)
\begin{tabular}{clcccc}
Resultado (R) & \multicolumn{1}{r|}{} &  & Habilidad (V) &  &  \multicolumn{1}{|c}{Tenistas (UA)} \\ \hline
   &  \multicolumn{1}{r|}{}    &  & Ganar/perder (D) &  & \multicolumn{1}{|r}{} \\
                 Indicador (I)  &   & =  &  &  & Fuente de datos (F) \\
 & \multicolumn{1}{r|}{} &  & Procedimientos (P) &        &      \multicolumn{1}{|r}{} \\
\phantom{Resultado (R)} & & \phantom{=} & \phantom{Procedimientos (P)} &        &    \phantom{Unidad de análisis (UA)}
\end{tabular}
\end{textblock}
}

\only<9-10>{
\begin{textblock}{160}(0,22)
\begin{tabular}{clcccc}
Resultado (R) & \multicolumn{1}{r|}{} &  & Habilidad (V) &  &  \multicolumn{1}{|c}{Tenistas (UA)} \\ \hline
   &  \multicolumn{1}{r|}{}    &  & Ganar/perder (D) &  & \multicolumn{1}{|r}{} \\
                 Indicador (I)  &   & =  &  &  & \texttt{atptour.com} (F) \\
 & \multicolumn{1}{r|}{} &  & Procedimientos (P) &        &      \multicolumn{1}{|r}{} \\
\phantom{Resultado (R)} & & \phantom{=} & \phantom{Procedimientos (P)} &        &    \phantom{Unidad de análisis (UA)}
\end{tabular}
\end{textblock}
}

\only<11>{
\begin{textblock}{160}(0,22)
\begin{tabular}{clcccc}
Resultado (R) & \multicolumn{1}{r|}{} &  & Habilidad (V) &  &  \multicolumn{1}{|c}{Tenistas (UA)} \\ \hline
   &  \multicolumn{1}{r|}{}    &  & Ganar/perder (D) &  & \multicolumn{1}{|r}{} \\
                 Indicador (I)  &   & =  &  &  & \texttt{atptour.com} (F) \\
 & \multicolumn{1}{r|}{} &  & Scraper (P) &        &      \multicolumn{1}{|r}{} \\
\phantom{Resultado (R)} & & \phantom{=} & \phantom{Procedimientos (P)} &        &    \phantom{Unidad de análisis (UA)}
\end{tabular}
\end{textblock}
}


\only<12->{
\begin{textblock}{160}(0,22)
\begin{tabular}{clcccc}
Resultado (R) & \multicolumn{1}{r|}{} &  & Habilidad (V) &  &  \multicolumn{1}{|c}{Tenistas (UA)} \\ \hline
   &  \multicolumn{1}{r|}{}    &  & Ganar/perder (D) &  & \multicolumn{1}{|r}{} \\
                 True/False (I)  &   & =  &  &  & \texttt{atptour.com} (F) \\
 & \multicolumn{1}{r|}{} &  & Scraper (P) &        &      \multicolumn{1}{|r}{} \\
\phantom{Resultado (R)} & & \phantom{=} & \phantom{Procedimientos (P)} &        &    \phantom{Unidad de análisis (UA)}
\end{tabular}
\end{textblock}
}

\only<13>{
\begin{textblock}{160}(0,75)
 \centering
 ¿Y el Resultado (R)? ¿Cuál es el valor de la variable?
\end{textblock}
}

 \only<6->{
 \begin{textblock}{140}(10,50)
 \begin{description}
  \item[$D:$] Examen de representatividad (especificaciones de la variable)
  \only<8->{\item[$F:$] Examen de viabilidad (accesibilidad, factibilidad)}
  \only<10->{\item[$P:$] Examen de confiabilidad (protocolos, acciones a efectuar)}
  \only<14->{\item[$I \leftrightarrow R:$] \textbf{Hip\'otesis indicadora}}
\end{description}
 \end{textblock}
}

\end{frame}

\begin{frame}[plain]
\begin{textblock}{160}(0,4)
\centering  \LARGE
 \es{¿Cómo definir la hipótesis indicadora?}
 \end{textblock}
\vspace{1cm}

\centering

 \includegraphics[width=0.35\textwidth]{../../auxiliar/static/elo} \\
 \vspace{0.1cm}
 Arpad Elo

\end{frame}



\begin{frame}[plain]
 \begin{textblock}{160}(0,4)
 \centering \LARGE
 Modelos causales deterministas
\end{textblock}
\vspace{0.75cm}

\centering

\tikz{
    \node[det, fill=black!15] (r) {$r$} ;
    \node[const, left=of r, xshift=-2.35cm] (r_name) {\small \en{Result}\es{Ganar/perder}:};
    \node[const, right=of r] (dr) {\normalsize $ r = (d>0)$};

    \node[latent, above=of r, yshift=-0.45cm] (d) {$d$} ; %
    \node[const, right=of d] (dd) {\normalsize $ d=p_i-p_j$};
    \node[const, left=of d, xshift=-2.35cm] (s_name) {\small \en{Difference}\es{Diferencia}:};

    \node[latent, above=of d, xshift=-0.8cm, yshift=-0.45cm] (p1) {$p_i$} ; %
    \node[latent, above=of d, xshift=0.8cm, yshift=-0.45cm] (p2) {$p_j$} ; %
    \node[const, left=of p1, xshift=-1.55cm] (p_name) {\small \en{Performance}\es{Desempeño}:};

    \node[latent, above=of p1,yshift=0.7cm,fill=white] (s1) {$s_i$} ; %
    \node[latent, above=of p2,yshift=0.7cm,fill=white] (s2) {$s_j$} ; %

    \node[latent, above=of p1,xshift=-1cm, yshift=-0.45cm,fill=white] (u1) {$u_i$} ;
    \node[latent, above=of p2,xshift=1cm, yshift=-0.45cm,fill=white] (u2) {$u_j$} ;
    \node[const, left=of u1, xshift=-0.55cm] (u_name) {\small \en{Others factors}\es{Otros factores}:};

    \node[const, right=of p2] (dp2) {\normalsize $p = s + u$};

    \node[const, left=of s1, xshift=-1.55cm] (s_name) {\small \en{Skill}\es{Habilidad}:};

    \edge {d} {r};
    \edge {p1,p2} {d};
    \edge {s1} {p1};
    \edge {s2} {p2};
    \edge {u1} {p1};
    \edge {u2} {p2};
}


\end{frame}




% \begin{frame}[plain,noframenumbering]
%
%  \begin{textblock}{90}(00,05)
% \begin{center}
%  \huge  \textcolor{black!66}{Creencias, datos y sorpresas}
% \end{center}
% \end{textblock}
%
%  %\vspace{2cm}brown
% %\maketitle
% \Wider[2cm]{
% \includegraphics[width=1\textwidth]{../../auxiliar/static/peligro_predador}
% }
% \end{frame}
%
 % if we are told that a highly improbable event has just occurred, we will
% have received more information than if we were told that some very likely event has just occurred, and if we knew that the event was certain to happen we would receive no information. Our measure of information content will therefore depend on the probability distribution p(x), and we therefore look for a quantity h(x) that is a monotonic function of the probability p(x) and that expresses the information content. The form of h(·) can be found by noting that if we have two events x and y that are unrelated, then the information gain from observing both of them should be the sum of the information gained from each of them separately, so that h(x, y) = h(x) + h(y). Two unrelated events will be statistically independent and so p(x, y) = p(x)p(y).



\begin{frame}[plain]
\begin{textblock}{160}(0,4)
 \centering \LARGE
La estructura invariante del dato
\end{textblock}
\vspace{0.75cm}

\begin{textblock}{160}(0,24) \centering
\tikz{
    \node[const] (fuente) {Fuente};
    \node[const, below=of fuente, yshift=-0.6cm] (realidad_causal) {$\hfrac{\text{\normalsize Realidad}}{\text{\normalsize causal}}$};
    \node[const, below=of realidad_causal, yshift=-0.6cm] (senal) {Señal};
    \node[const, right=of senal, xshift=1.4cm] (canal) {Canal};
    \node[const, right=of canal, xshift=1.4cm] (indicador) {Indicador};
    \node[const, above=of indicador, yshift=0.6cm] (modelo) {$\hfrac{\text{\normalsize Modelo}}{\text{\normalsize causal}}$};
    \node[const, above=of modelo, yshift=0.6cm] (estimacion) {Estimación};

    \edge {fuente} {realidad_causal};
    \edge {realidad_causal} {senal};
}
\end{textblock}

\end{frame}



\begin{frame}[plain]
\begin{textblock}{160}(0,4)
 \centering \LARGE 
 \en{Honesty optimizes information}
 \es{La honestidad óptimiza la información}
 \end{textblock}
 \vspace{1.4cm}

 \begin{equation*}
 \underbrace{\text{\en{Entropy}\es{Entropía}}(X)}_{\text{\en{Expected information}\es{Información esperada}}} = \ \sum_{x\in X} \ P(x) \  \cdot \underbrace{(-\log P(x))}_{\hfrac{\text{\scriptsize \en{Information generated}\es{Información generada}}}{\text{\scriptsize \en{by the surprise}\es{por la sorpresa}}}}
\end{equation*}

\pause

\vspace{0.3cm}

\begin{center}
\en{Maximum expected information $\Leftrightarrow$ Maximum uncertainty}
\es{M\'axima información esperada $\Leftrightarrow$ Máxima incertidumbre}
\end{center}

\vspace{0.7cm}

\pause

\Wider[-5cm]{
\begin{mdframed}[backgroundcolor=black!20]
\begin{equation*}
  \text{\en{Honesty}\es{Honestidad}} = \underset{P(X)}{\text{ arg max }} \text{\en{Entropy}\es{Entropía}}(X)
\end{equation*}
\end{mdframed}
}

\end{frame}


 
\begin{frame}[plain]
\centering
  \includegraphics[width=0.35\textwidth]{../../auxiliar/static/pachacuteckoricancha.jpg}
\end{frame}






\end{document}



