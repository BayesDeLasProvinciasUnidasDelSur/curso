\documentclass[shownotes,aspectratio=169]{beamer}

\input{../../auxiliar/tex/diapo_encabezado.tex}
\input{../../auxiliar/tex/tikzlibrarybayesnet.code.tex}
 \mode<presentation>
 {
 %   \usetheme{Madrid}      % or try Darmstadt, Madrid, Warsaw, ...
 %   \usecolortheme{default} % or try albatross, beaver, crane, ...
 %   \usefonttheme{serif}  % or try serif, structurebold, ...
  \usetheme{Antibes}
  \setbeamertemplate{navigation symbols}{}
 }
 
\usepackage{todonotes}
\setbeameroption{show notes}

\newif\ifen
\newif\ifes
\newcommand{\en}[1]{\ifen#1\fi}
\newcommand{\es}[1]{\ifes#1\fi}
\estrue

%\title[Bayes del Sur]{}

\begin{document}

\color{black!85}
\large

\begin{frame}[plain,noframenumbering]
 
 \begin{textblock}{90}(00,05)
\begin{center}
 \huge  \textcolor{black!66}{Creencias, datos y sorpresas}
\end{center}
\end{textblock}

 %\vspace{2cm}brown
%\maketitle
\Wider[2cm]{
\includegraphics[width=1\textwidth]{../../auxiliar/static/peligro_predador}
}
\end{frame}




\begin{frame}[plain]

\centering \Large

Evoluci\'on

\end{frame}



\begin{frame}[plain]
\begin{textblock}{160}(0,4)
\centering \Large Evolución de estrategias
\end{textblock}
\vspace{0.75cm}

\centering
\tikz{
    \node[latent] (e) {$p$};
    \node[const, right=of e] (ne) {$p\sim \text{Beta}(\alpha,\beta)$};
    
    \node[latent, below=of e] (r) {$m$};
    \node[const, right=of r] (ne) {$m \sim \text{Binomial}(p)$};
    
    \edge {e} {r};
}
 \vspace{1.5cm}
 \pause
 
 Supongamos que se seleccionó la estrategia $p=1.5/2.1$

 \begin{equation}
 P(p) = \delta(p=1.5/2.1)
 \end{equation}
 
\end{frame}


\begin{frame}[plain]
\begin{textblock}{160}(0,4)
\centering \Large Cambio de ambiente
\end{textblock}
\vspace{0.75cm}

Los datos surgen ahora 50\% y 50\% pero la apuesta de las estrategias sigue siendo la misma (ya no hay diversidad).

\end{frame}

\begin{frame}[plain]

Cómo les va jugando individualmente

Cómo les va a la estrategia óptima

\end{frame}

\begin{frame}[plain]
\begin{textblock}{160}(0,4)
\centering \Large Cambio de ambiente
\end{textblock}
\vspace{0.75cm}


\centering
\tikz{

    \node[latent, minimum size=2cm ] (x1_0) {$x_1(t)$} ;
    \node[latent, below=of x1_0, minimum size=2cm ] (x2_0) {$x_2(t)$} ;

    \node[latent, right=of x1_0, minimum size=3cm ] (x1_0g) {$x_1(t)+\Delta x_1(t)$} ;
    \node[latent, right=of x2_0, minimum size=1.8cm, xshift=0.6cm , align=left] (x2_0g) {$x_2(t)+$\\$\Delta x_2(t)$} ;
    
    \node[latent, right=of x1_0g, minimum size=3.8cm, yshift=-1.33cm, align=right] (x_0) {$x_1(t)+\Delta x_1(t)$\\$+x_2(t)+\Delta x_2(t)$ } ;
    
    \node[const, above=of x_0] (nx_0) {$\overbrace{\text{Pool}\hspace{2.5cm}\text{Share}}^{\text{\normalsize Cooperaci\'on}}$} ;
    
    \node[latent, right=of x1_0g, minimum size=2.5cm,  xshift=4.5cm] (x1_1) {$x_1(t+1)$ } ;
    \node[latent, below=of x1_1, minimum size=2.5cm, yshift=0.7cm] (x2_1) {$x_2(t+1)$ } ;
    
    \edge {x1_0} {x1_0g};
    \edge {x2_0} {x2_0g};
    \edge {x1_0g,x2_0g} {x_0};
    \edge {x_0} {x1_1,x2_1};
    
}

\end{frame}

\begin{frame}[plain]
\begin{textblock}{160}(0,4)
\centering \Large Evolución de grupos
\end{textblock}
\vspace{0.75cm}

\centering
\tikz{
    \node[latent] (m) {$M$};
    
    \node[latent, right=of m] (e0) {$e_0$};
    
    \node[latent, right=of e0] (e1) {$e_1$};
    \node[latent, below=of e1] (r1) {$r_1$};
    
    \node[latent, right=of e1] (e2) {$e_2$};
    \node[latent, below=of e2] (r2) {$r_2$};
    
    \node[latent, right=of e2] (e3) {$e_3$};
    
    
    \edge {m} {e0};
    \edge {e0} {e1};
    \edge {e1} {r1,e2};
    \edge {e2} {r2,e3};
}
\end{frame}

\begin{frame}[plain]
\begin{textblock}{160}(0,4)
\centering \Large Evolución de grupos
\end{textblock}
\vspace{0.75cm}

\centering
\tikz{
    \node[latent] (m) {$M$};
    
    \node[latent, right=of m] (e0) {$e_0$};
    
    \node[latent, right=of e0] (e1) {$e_1$};
    \node[latent, below=of e1] (r1) {$r_1$};
    
    \node[latent, right=of e1] (e2) {$e_2$};
    \node[latent, below=of e2] (r2) {$r_2$};
    
    \node[latent, right=of e2] (e3) {$e_3$};
    
    
    \edge {m} {e0};
    \edge {e0} {e1};
    \edge {e1} {r1,e2};
    \edge {e2} {r2,e3};
}
\end{frame}



\begin{frame}[plain]

\centering
  \includegraphics[width=0.55\textwidth]{../../auxiliar/static/pachacuteckoricancha.jpg}
\end{frame}





\end{document}



