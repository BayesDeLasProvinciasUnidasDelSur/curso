\documentclass[shownotes,aspectratio=169]{beamer}

\input{../../auxiliar/tex/diapo_encabezado.tex}
% tikzlibrary.code.tex
%
% Copyright 2010-2011 by Laura Dietz
% Copyright 2012 by Jaakko Luttinen
%
% This file may be distributed and/or modified
%
% 1. under the LaTeX Project Public License and/or
% 2. under the GNU General Public License.
%
% See the files LICENSE_LPPL and LICENSE_GPL for more details.

% Load other libraries
\usetikzlibrary{shapes}
\usetikzlibrary{fit}
\usetikzlibrary{chains}
\usetikzlibrary{arrows}

% Latent node
\tikzstyle{latent} = [circle,fill=white,draw=black,inner sep=1pt,
minimum size=20pt, font=\fontsize{10}{10}\selectfont, node distance=1]
% Observed node
\tikzstyle{obs} = [latent,fill=gray!25]
% Invisible node
\tikzstyle{invisible} = [latent,minimum size=0pt,color=white, opacity=0, node distance=0]
% Constant node
\tikzstyle{const} = [rectangle, inner sep=0pt, node distance=0.1]
%state
\tikzstyle{estado} = [latent,minimum size=8pt,node distance=0.4]
%action
\tikzstyle{accion} =[latent,circle,minimum size=5pt,fill=black,node distance=0.4]


% Factor node
\tikzstyle{factor} = [rectangle, fill=black,minimum size=10pt, draw=black, inner
sep=0pt, node distance=1]
% Deterministic node
\tikzstyle{det} = [latent, rectangle]

% Plate node
\tikzstyle{plate} = [draw, rectangle, rounded corners, fit=#1]
% Invisible wrapper node
\tikzstyle{wrap} = [inner sep=0pt, fit=#1]
% Gate
\tikzstyle{gate} = [draw, rectangle, dashed, fit=#1]

% Caption node
\tikzstyle{caption} = [font=\footnotesize, node distance=0] %
\tikzstyle{plate caption} = [caption, node distance=0, inner sep=0pt,
below left=5pt and 0pt of #1.south east] %
\tikzstyle{factor caption} = [caption] %
\tikzstyle{every label} += [caption] %

\tikzset{>={triangle 45}}

%\pgfdeclarelayer{b}
%\pgfdeclarelayer{f}
%\pgfsetlayers{b,main,f}

% \factoredge [options] {inputs} {factors} {outputs}
\newcommand{\factoredge}[4][]{ %
  % Connect all nodes #2 to all nodes #4 via all factors #3.
  \foreach \f in {#3} { %
    \foreach \x in {#2} { %
      \path (\x) edge[-,#1] (\f) ; %
      %\draw[-,#1] (\x) edge[-] (\f) ; %
    } ;
    \foreach \y in {#4} { %
      \path (\f) edge[->,#1] (\y) ; %
      %\draw[->,#1] (\f) -- (\y) ; %
    } ;
  } ;
}

% \edge [options] {inputs} {outputs}
\newcommand{\edge}[3][]{ %
  % Connect all nodes #2 to all nodes #3.
  \foreach \x in {#2} { %
    \foreach \y in {#3} { %
      \path (\x) edge [->,#1] (\y) ;%
      %\draw[->,#1] (\x) -- (\y) ;%
    } ;
  } ;
}

% \factor [options] {name} {caption} {inputs} {outputs}
\newcommand{\factor}[5][]{ %
  % Draw the factor node. Use alias to allow empty names.
  \node[factor, label={[name=#2-caption]#3}, name=#2, #1,
  alias=#2-alias] {} ; %
  % Connect all inputs to outputs via this factor
  \factoredge {#4} {#2-alias} {#5} ; %
}

% \plate [options] {name} {fitlist} {caption}
\newcommand{\plate}[4][]{ %
  \node[wrap=#3] (#2-wrap) {}; %
  \node[plate caption=#2-wrap] (#2-caption) {#4}; %
  \node[plate=(#2-wrap)(#2-caption), #1] (#2) {}; %
}

% \gate [options] {name} {fitlist} {inputs}
\newcommand{\gate}[4][]{ %
  \node[gate=#3, name=#2, #1, alias=#2-alias] {}; %
  \foreach \x in {#4} { %
    \draw [-*,thick] (\x) -- (#2-alias); %
  } ;%
}

% \vgate {name} {fitlist-left} {caption-left} {fitlist-right}
% {caption-right} {inputs}
\newcommand{\vgate}[6]{ %
  % Wrap the left and right parts
  \node[wrap=#2] (#1-left) {}; %
  \node[wrap=#4] (#1-right) {}; %
  % Draw the gate
  \node[gate=(#1-left)(#1-right)] (#1) {}; %
  % Add captions
  \node[caption, below left=of #1.north ] (#1-left-caption)
  {#3}; %
  \node[caption, below right=of #1.north ] (#1-right-caption)
  {#5}; %
  % Draw middle separation
  \draw [-, dashed] (#1.north) -- (#1.south); %
  % Draw inputs
  \foreach \x in {#6} { %
    \draw [-*,thick] (\x) -- (#1); %
  } ;%
}

% \hgate {name} {fitlist-top} {caption-top} {fitlist-bottom}
% {caption-bottom} {inputs}
\newcommand{\hgate}[6]{ %
  % Wrap the left and right parts
  \node[wrap=#2] (#1-top) {}; %
  \node[wrap=#4] (#1-bottom) {}; %
  % Draw the gate
  \node[gate=(#1-top)(#1-bottom)] (#1) {}; %
  % Add captions
  \node[caption, above right=of #1.west ] (#1-top-caption)
  {#3}; %
  \node[caption, below right=of #1.west ] (#1-bottom-caption)
  {#5}; %
  % Draw middle separation
  \draw [-, dashed] (#1.west) -- (#1.east); %
  % Draw inputs
  \foreach \x in {#6} { %
    \draw [-*,thick] (\x) -- (#1); %
  } ;%
}


 \mode<presentation>
 {
 %   \usetheme{Madrid}      % or try Darmstadt, Madrid, Warsaw, ...
 %   \usecolortheme{default} % or try albatross, beaver, crane, ...
 %   \usefonttheme{serif}  % or try serif, structurebold, ...
  \usetheme{Antibes}
  \setbeamertemplate{navigation symbols}{}
 }
 
\usepackage{todonotes}
\setbeameroption{show notes}

%\title[Bayes del Sur]{}

\begin{document}

\color{black!85}
\large
 
%\setbeamercolor{background canvas}{bg=gray!15}


\begin{frame}[plain,noframenumbering]
 
 \begin{textblock}{90}(00,05)
\begin{center}
 \huge  \textcolor{black!66}{Creencias adaptativas}
\end{center}
\end{textblock}

 %\vspace{2cm}brown
%\maketitle
\Wider[2cm]{
\includegraphics[width=1\textwidth]{../../auxiliar/static/peligro_predador}
}

 \begin{textblock}{47}(113,74)
\centering \large  \textcolor{white!55}{Sum-product algorithm}
\end{textblock}

\end{frame}


\begin{frame}[plain]
\begin{textblock}{160}(0,4)
 \centering \LARGE Lógica Booleana 
 \end{textblock}
 \vspace{1.5cm} \centering

 \includegraphics[width=1\textwidth]{../../auxiliar/static/boole.png}

 \normalsize
 \hfill George Boole (1854) \href{https://downloads.tuxfamily.org/openmathdep/logic_ante_1900/Laws_of_Thought-Boole.pdf}{\emph{An Investigation of the Laws of Thought}} 
\end{frame}


\begin{frame}[plain]
\begin{textblock}{160}(0,4)
 \centering \LARGE Lógica de primer orden extendida \\
 \Large (De una práctica suspendida)
 \end{textblock}
 \vspace{1.5cm}
 
Dado \ \ $A \Rightarrow B  \equiv P(B|A) = 1$ 

\pause 

$\bullet$ \ \ $P(B|A) = 1$ (modus ponens) \\
$\bullet$ \ \ $P(\neg A| \neg B) = 1$ (modus tollens) \\
\pause
$\bullet$ \ \ $P(B| \neg A) \leq P(B) $ ($A$ falso implica $B$ menos plausible)
$\bullet$ \ \ $P(A|B) \geq P(A) $ ($B$ verdadero implica $A$ más plausible) 
 
 \vspace{0.6cm}
 
Dado \ \ $P(B|A) \geq P(B)$ ($A$ verdadero, luego $B$ más plausible) \\

\pause 

$\bullet$ \ \ $P(B|\neg A) \leq P(B)$ ($A$ falso, luego $B$ menos plausible) \\
$\bullet$ \ \ $P(A|B) \geq P(A)$ ($B$ verdadero, luego $A$ más plausible) \\
$\bullet$ \ \ $P(\neg A|\neg B) \geq P(\neg A)$ ($B$ falso, luego $A$ menos plausible)
 
 
\end{frame}


\begin{frame}[plain]
\begin{textblock}{160}(0,4)
 \centering \LARGE El problema de la lógica extendida \\
 \Large La complejidad en memoria
 \end{textblock}
 \vspace{1.5cm}
 
 Tenemos $n=27$ variables, las letras del alfabeto \\
 
 \pause
 
 \vspace{0.5cm}
 
 $\bullet$ En lógica binaria asignamos valores a las variables, y con las reglas de la lógica calculamos las combinaciones. \pause Con 27 bits nos alcanza.  \\
 
 \vspace{0.5cm}
 
 \pause
 
 $\bullet$ En lógica extendida debemos asignar una probabilidad a cada posible combinación de todos los posibles combinaciones de valores de verdad.  \\
 
 \vspace{0.2cm}
  \pause
  
  Necesitamos guardar $2^{27}-1$ números reales  \\[0.2cm]
  
 \Wider[-2cm]{
 \begin{itemize}
 \item[$1$:] $P(A, B, \dots, Z) =  \dots$ \\
 \item[$.$:] $\dots$ \\
 \item[$2^{27}-1$:] $P(\neg A, \neg B, \dots, Z) =  \dots$ \\
 \item[$2^{27}$:] $P(\neg A, \neg B, \dots, \neg Z) =  1 - \sum P(\dots)$ 
 \end{itemize}
 }
\end{frame}


\begin{frame}[plain]
\begin{textblock}{160}(0,4)
 \centering \LARGE Objetivo del curso en adelante
 \end{textblock}
 \vspace{1.5cm} \centering
 
\Large
 Ofrecer herramientas para reducir \\ la complejidad espacial y temporal 
 
 
\end{frame}

 
 \begin{frame}[plain]
    \begin{textblock}{160}(0,4)
    \centering \LARGE Modelos gráficos
    \end{textblock}
    \centering
    \vspace{0.75cm}

    
    \begin{itemize}
     \item[$\bullet$] Permite expresar todas las hipótesis de forma intuititva.
     \item[$\bullet$] Representa la probabilidad conjunta de forma completa.
     \item[$\bullet$] Proporciona información para hacer inferencia eficientemente.
    \end{itemize}

     
 \end{frame}

 
 \begin{frame}[plain]
 \begin{textblock}{160}(0,4)
 \centering \Large
 Causalidad
 \end{textblock}

 \centering \normalsize
 Causal relationships are expressed in the form of deterministic, functional equations, and probabilities are introduced through the assumption that certain variables in the equations are unobserved.

\end{frame} 


 \begin{frame}[plain]
\begin{textblock}{160}(0,4)
 \centering \Large
 Modelo Causal
 \end{textblock}
 \centering
 \vspace{0.75cm}
 
 \tikz{
    \node[det] (l) {$l$} ; %
    \node[det, above=of l] (a) {$a$} ; %
    \node[det, above=of a,xshift=1.5cm] (t) {$t$} ; %
    \node[det, above=of a,xshift=-1.5cm] (e) {$e$} ; %
    \node[det, below=of t,xshift=1.5cm] (r) {$r$} ; %
    
    \edge {a} {l};
    \edge {t,e} {a};
    \edge {t} {r};
    
    \node[const, left= of e, xshift=-0.1cm] (cpd_e) {Entradera:};
    \node[const, right= of t, xshift=0.1cm] (cpd_t) {:Terremoto};
    \node[const, left= of a, xshift=-0.1cm] (cpd_a) {Alarma:};
    \node[const, right= of r, xshift=0.1cm] (cpd_r) {:Redes};
    \node[const, left= of l, xshift=-0.1cm] (cpd_l) {Llamada:};
    
    }


 
 \end{frame}
 
 \begin{frame}[plain]
\begin{textblock}{160}(0,4)
 \centering \Large
 Mecanismos Causales
 \end{textblock}
 \vspace{0.75cm}
 
 \begin{align*}
  \onslide<4->{e &= u_e \\ 
  t &= u_t \\}
  \onslide<1->{r &= f(t,u_r) = t \vee u_r \\}
  \onslide<2->{a &= f(e,t,u_a) = e \vee t \vee u_a \\}
  \onslide<3->{l &= f(a,u_l) = a \vee u_l}
 \end{align*} 
 \end{frame}

 \begin{frame}[plain]
\begin{textblock}{160}(0,4)
 \centering \Large
 Proabilidad conjunta
 \end{textblock}
 \vspace{0.75cm}
 
 
\only<1>{
\begin{textblock}{140}(10,34)
\begin{mdframed}[backgroundcolor=black!15]
 \centering 
  Los modelos son \'utiles para definir distribuciones 
  
  de probabilidad en espacios de alta dimensión.
\end{mdframed}
\end{textblock}
}


\only<2>{
\begin{textblock}{130}(18,33)
\begin{flalign*}
 & P(e,t,a,r,l) = && 
\end{flalign*}
\end{textblock}
}

\only<3>{
\begin{textblock}{130}(18,33)
\begin{flalign*}
 & P(e,t,a,r,l) = P(e)\,P(t|e)\,P(a|t,e)\,P(r|a,t,e)\,P(l|a,r,t,e) && 
\end{flalign*}
\end{textblock}
}

\only<4>{
\begin{textblock}{130}(18,33)
\begin{flalign*}
 & P(e,t,a,r,l) = P(e)\,P(t|\cancel{e})\,P(a|t,e)\,P(r|\cancel{a},t,\cancel{e})\,P(l|a,\cancel{r},\cancel{t},\cancel{e}) && 
\end{flalign*}
\end{textblock}
}

\only<5>{
\begin{textblock}{130}(18,33)
\begin{flalign*}
 & P(e,t,a,r,l) = P(e)\,P(t)\,P(a|t,e)\,P(r|t)\,P(l|a) && 
\end{flalign*}
\end{textblock}
}

\only<6>{
\begin{textblock}{130}(17.3,33)
\begin{flalign*}
 & \underbrace{P(e,t,a,r,l)}_{\text{Complejidad } 2^n} = P(e)\,P(t)\,P(a|t,e)\,P(r|t)\,P(l|a) && 
\end{flalign*}
\end{textblock}
}

\only<7>{
\begin{textblock}{130}(17.3,33)
\begin{flalign*}
 & \underbrace{P(e,t,a,r,l)}_{\text{Complejidad } 2^n} = \underbrace{P(e)}_{2}\underbrace{P(t)}_{2}\underbrace{P(a|t,e)}_{2^3}\underbrace{P(r|t)}_{2^2}\underbrace{P(l|a)}_{2^2} && 
\end{flalign*}
\end{textblock}
}


 
%  
%  \only<1->{
%  \begin{textblock}{160}(0,14)
% \begin{align*}
%   \underbrace{P(e,t,r,a,l)}_{\text{Complejidad } 2^n} = \onslide<2->{\underbrace{P(e)P(t)P(r|t)P(a|e,t)P(l|a)}_{\text{Complejidad } 2 + 2 + 2^2 + 2^3 + 2^2}}
%  \end{align*}
%  \end{textblock}
% }
% 
% \begin{textblock}{160}(0,34)
%  \onslide<3>{
%  \begin{align*}
%  P(e|t,r,a,l) & = P(e) \\
%  P(t|e,r,a,l) & = P(t) \\
%  P(r|e,t,a,l) & = p(r|t) \\
%  P(a|e,t,r,l) & = p(a|e,t) \\
%  P(l|e,t,r,a) & = p(l|a) \\
%  \end{align*}
% }
% \end{textblock}
% 
  
 \end{frame}

 
 \begin{frame}[plain]
\begin{textblock}{160}(0,4)
 \centering \Large
 Modelo Causal \\
 \large Prior Entradera
 \end{textblock}
 \vspace{0.75cm}
 
 \centering
 
 \begin{textblock}{160}(0,25)
  $P(e)$ \\[0.1cm]
    \begin{tabular}{|c|c|}
        \hline
        \,$e^0$\, & \,$e^1$\, \\ \hline
        \onslide<3>{$999/1000$} & \onslide<3>{$1/1000$}   \\ \hline
    \end{tabular}
  \end{textblock}

\begin{textblock}{160}(10,54) 
\onslide<2->{ 
 \begin{itemize}
  \item[$\bullet$] Una vez cada 3 años
  \item[$\bullet$] Una casa cada 1000 por día
 \end{itemize}
}
\end{textblock}

 
 \end{frame}
 
 \begin{frame}[plain]
\begin{textblock}{160}(0,4)
 \centering \Large
 Modelo Causal \\
 \large Prior Terremoto
 \end{textblock}
 \vspace{0.75cm}
 
 \centering

 \begin{textblock}{160}(0,25)
  $P(t)$ \\[0.1cm]
    \begin{tabular}{|c|c|}
        \hline
        \,$t^0$\, & \,$t^1$\, \\ \hline
        \onslide<3>{$362/365$} & \onslide<3>{$3/365$}   \\ \hline
    \end{tabular}
\end{textblock}

 
 \begin{textblock}{160}(10,54)
\onslide<2->{ 
 \begin{itemize}
  \item[$\bullet$] Hay 3 terremotos (leves) por años
 \end{itemize}
}
\end{textblock}

 
 \end{frame}
 
 \begin{frame}[plain]
\begin{textblock}{160}(0,4)
 \centering \Large
 Modelo Causal \\
 \large Prior redes sociales
 \end{textblock}
 \vspace{0.75cm}
 
 \centering

 
 \begin{textblock}{160}(0,25)
  $P(r|t)$ \\[0.1cm]
    \begin{tabular}{|c|c|c|}
        \hline
        & \, $r^0$ \, & \, $r^1$ \,  \\ \hline
       $(t^0)$ & \only<3-4>{$1$}\only<5>{$0.99$} & \only<3-4>{$0$}\only<5>{$0.01$}   \\ \hline
       $(t^1)$ & \only<3-4>{$0$}\only<5>{$0.01$} & \only<3-4>{$1$}\only<5>{$0.99$}   \\ \hline
    \end{tabular}
\end{textblock}

 
 \begin{textblock}{140}(10,54)
\onslide<2->{ 
 \begin{itemize}
  \item[$\bullet$] Siempre que hay un terremoto, en alguna de mis redes sociales (whatsapp, facebook, twitter, instagram), se habla del tema.
  \only<4->{\item[$\bullet$] También puedo mirar mal, o por algún momento no haya nada, o que por algún otro motivo nadie pueda comunicarse}
 \end{itemize}
}
\end{textblock}

 
 \end{frame}
 
 
 \begin{frame}[plain]
\begin{textblock}{160}(0,4)
 \centering \Large
 Modelo Causal \\
 \large Prior llamada
 \end{textblock}
 \vspace{0.75cm}
 
 \centering

 
 \begin{textblock}{160}(0,25)
  $P(l|a)$ \\[0.1cm]
    \begin{tabular}{|c|c|c|}
        \hline
        & \, $l^0$ \, & \, $l^1$ \,  \\ \hline
       $(a^0)$ & \onslide<3>{$0.99$} & \onslide<3>{$0.01$}   \\ \hline
       $(a^1)$ & \onslide<3>{$0.01$} & \onslide<3>{$0.99$}   \\ \hline
    \end{tabular}
\end{textblock}

 
 \begin{textblock}{140}(10,54)
\onslide<2->{ 
 \begin{itemize}
  \item[$\bullet$] Siempre que se activa la alarma, me llaman desde el call center de la empresa (si no pasa nada raro)
 \end{itemize}
}
\end{textblock}

 \end{frame}
 
 
 \begin{frame}[plain]
\begin{textblock}{160}(0,4)
 \centering \Large
 Modelo Causal \\
 \large Prior alarma
 \end{textblock}
 \vspace{0.75cm}
 
 \centering

 
 \begin{textblock}{160}(0,18)
  $P(l|a)$ \\[0.1cm]
    \begin{tabular}{|c|c|c|}
        \hline
        & \hspace{1cm} $a^0$ \hspace{1cm} & \hspace{1cm} $a^1$ \hspace{1cm} \\ \hline
       $(e^0, t^0)$ & \only<10->{$0.99$}\only<4-9>{$P(\overline{\alpha})$} & \only<11->{$0.01$}\only<4-10>{$P(\alpha)$} \\ \hline
       $(e^1, t^0)$ & \only<13->{$\approx 0.01$}\only<12>{$0.99 \cdot 0.01$}\only<7-11>{$P(\overline{\alpha} \cap \overline{\varepsilon})$}\only<6>{$P(\overline{\alpha \cup \varepsilon})$} & \only<14->{$\approx 0.99$} \only<5-13>{$P(\alpha \cup \varepsilon)$}  \\ \hline
       $(e^0, t^1)$ & \only<15->{$\approx 0.01$}\only<8-14>{$P(\overline{\alpha} \cap \overline{\tau})$} &\only<15->{$\approx0.99$}\only<8-14>{$P(\alpha \cup \tau)$}  \\ \hline 
       $(e^1, t^1)$ & \only<16>{$\approx 0.0001$} \only<9-15>{$P(\overline{\alpha} \cap \overline{\varepsilon}  \cap \overline{\tau})$} & \only<16>{$\approx 0.9999$} \only<9-15>{$P(\alpha \cup \varepsilon \cup \tau)$}  \\ \hline 
    \end{tabular}
\end{textblock}

 
 \begin{textblock}{140}(10,52)
\onslide<2->{ 
 \begin{itemize}
  \item[$\bullet$] Siempre que entra alguien a la casa o que hay un terremoto se activa la alarma (si no pasa nada raro):
  \onslide<3->{
  \begin{description}
   \item[$\alpha$] Se activa sola: $P(\alpha) = 0.01$
   \item[$\overline{\varepsilon}$] No se activa a pesar de entradera: $P(\overline{\varepsilon}) = 0.01$ 
   \item[$\overline{\tau}$] No se activa a pesar de terremoto: $P(\overline{\tau}) = 0.01$
  \end{description}
}
 \end{itemize}
}
\end{textblock}

 \end{frame}
 
 
 \begin{frame}[plain]
\begin{textblock}{160}(0,4)
 \centering \Large
 Modelo Causal
 \end{textblock}
 
\begin{textblock}{160}(0,12)
 \centering

 \tikz{
    \node[det] (l) {$l$} ; %
    \node[det, above=of l] (a) {$a$} ; %
    \node[det, above=of a,xshift=1.5cm] (t) {$t$} ; %
    \node[det, above=of a,xshift=-1.5cm] (e) {$e$} ; %
    \node[det, below=of t,xshift=1.5cm] (r) {$r$} ; %
    
    \edge {a} {l};
    \edge {t,e} {a};
    \edge {t} {r};
    
    \onslide<2->{
    \node[const, left= of e, xshift=-0.1cm] (cpd_e) {
    \begin{tabular}{|c|c|}
        \hline
        $e^0$ & $e^1$ \\ \hline
        $0.999$ & $0.001$  \\ \hline
    \end{tabular}
    };
    \node[const, above= of cpd_e] (n_e) {Entradera:};
        }
    
    \onslide<3->{
    \node[const, right= of t, xshift=0.1cm] (cpd_t) {
    \begin{tabular}{|c|c|}
        \hline
        $t^0$ & $t^1$ \\ \hline
        $0.992$ & $0.008$  \\ \hline
    \end{tabular}
    };
    \node[const, above= of cpd_t] (n_t) {Terremoto:};
    }
    
    \onslide<4->{
    \node[const, left= of a, yshift=-0.7cm, xshift=-0.3cm] (cpd_a) {
    \begin{tabular}{|c|c|c|}
        \hline
        & $a^0$ & $a^1$ \\ \hline
       ($e^0, t^0$) & $0.99$ & $0.01$  \\ \hline
       ($e^1, t^0$) & $0.01$ & $0.99$  \\ \hline
       ($e^0, t^1$) & $0.01$ & $0.99$  \\ \hline
       ($e^1, t^1$) & $0.0001$ & $0.9999$  \\ \hline
    \end{tabular}
    };
    \node[const, above= of cpd_a] (n_a) {Alarma:};
    }
 
 \onslide<5->{
    \node[const, right= of r, xshift=0.1cm] (cpd_r) {
    \begin{tabular}{|c|c|c|}
        \hline
        & \, $r^0$ \, & \, $r^1$ \,  \\ \hline
       $(t^0)$ & $0.99$ & $0.01$   \\ \hline
       $(t^1)$ & $0.01$ & $0.99$   \\ \hline
    \end{tabular}
    };
    \node[const, above= of cpd_r] (n_r) {Redes:};
    }
    
 
 \onslide<6->{
    \node[const, right= of l, xshift=0.1cm] (cpd_l) {
    \begin{tabular}{|c|c|c|}
        \hline
        & \, $l^0$ \, & \, $l^1$ \,  \\ \hline
       $(a^0)$ & $0.99$ & $0.01$   \\ \hline
       $(a^1)$ & $0.01$ & $0.99$   \\ \hline
    \end{tabular}
    };
    \node[const, above= of cpd_l] (n_l) {Llamada:};
    }
    
 
 }
\end{textblock}
\end{frame}


\begin{frame}[plain]
\begin{textblock}{160}(0,4)
 \centering \Large
 Modelo Causal
 \end{textblock}
 
\begin{textblock}{160}(0,12)
 \centering

 \tikz{
    \node[det] (l) {$l$} ; %
    \node[det, above=of l] (a) {$a$} ; %
    \node[det, above=of a,xshift=1.5cm] (t) {$t$} ; %
    \node[det, above=of a,xshift=-1.5cm] (e) {$e$} ; %
    \node[det, below=of t,xshift=1.5cm] (r) {$r$} ; %
    
    \edge {a} {l};
    \edge {t,e} {a};
    \edge {t} {r};
    
    \node[const, left= of e, xshift=-0.1cm] (cpd_e) {
    \begin{tabular}{|c|c|}
        \hline
        $e^0$ & $e^1$ \\ \hline
        $0.999$ & $0.001$  \\ \hline
    \end{tabular}
    };
    \node[const, above= of cpd_e] (n_e) {Entradera:};
    
    \node[const, right= of t, xshift=0.1cm] (cpd_t) {
    \begin{tabular}{|c|c|}
        \hline
        $t^0$ & $t^1$ \\ \hline
        $0.992$ & $0.008$  \\ \hline
    \end{tabular}
    };
    \node[const, above= of cpd_t] (n_t) {Terremoto:};

    \node[const, left= of a, yshift=-0.7cm, xshift=-0.3cm] (cpd_a) {
    \begin{tabular}{|c|c|c|}
        \hline
        & $a^0$ & $a^1$ \\ \hline
       ($e^0, t^0$) & $0.99$ & $0.01$  \\ \hline
       ($e^1, t^0$) & $0.01$ & $0.99$  \\ \hline
       ($e^0, t^1$) & $0.01$ & $0.99$  \\ \hline
       ($e^1, t^1$) & $0.0001$ & $0.9999$  \\ \hline
    \end{tabular}
    };
    \node[const, above= of cpd_a] (n_a) {Alarma:};

    \node[const, right= of r, xshift=0.1cm] (cpd_r) {
    \begin{tabular}{|c|c|c|}
        \hline
        & \, $r^0$ \, & \, $r^1$ \,  \\ \hline
       $(t^0)$ & $0.99$ & $0.01$   \\ \hline
       $(t^1)$ & $0.01$ & $0.99$   \\ \hline
    \end{tabular}
    };
    \node[const, above= of cpd_r] (n_r) {Redes:};
 
    \node[const, right= of l, xshift=0.1cm] (cpd_l) {
    \begin{tabular}{|c|c|c|}
        \hline
        & \, $l^0$ \, & \, $l^1$ \,  \\ \hline
       $(a^0)$ & $0.99$ & $0.01$   \\ \hline
       $(a^1)$ & $0.01$ & $0.99$   \\ \hline
    \end{tabular}
    };
    \node[const, above= of cpd_l] (n_l) {Llamada:};    
 
 }
\end{textblock}

\only<1>{
\begin{textblock}{130}(15,63)
\begin{flalign*}
 & P(e^0,t^0,a^0,r^0,l^0) = P(e^0)P(t^0)P(a^0|t^0,e^0)P(r^0|t^0)P(l^0|a^0) && 
\end{flalign*}
\end{textblock}
}

\only<2>{
\begin{textblock}{130}(15,63)
\begin{flalign*}
 & P(e^0,t^0,a^0,r^0,l^0) = 0.999 \cdot P(t^0)P(a^0|t^0,e^0)P(r^0|t^0)P(l^0|a^0) && 
\end{flalign*}
\end{textblock}
}

\only<3>{
\begin{textblock}{130}(15,63)
\begin{flalign*}
 & P(e^0,t^0,a^0,r^0,l^0) = 0.999 \cdot 0.992 \cdot 0.99 \cdot 0.99 \cdot 0.99 && 
\end{flalign*}
\end{textblock}
}

\only<4>{
\begin{textblock}{130}(15,63)
\begin{flalign*}
 & P(e^0,t^0,a^0,r^0,l^0) = 0.999 \cdot 0.992 \cdot 0.99 \cdot 0.99 \cdot 0.99 \approx 0.96 && 
\end{flalign*}
\end{textblock}
}

\only<5>{
 \begin{textblock}{130}(15,63)
 \begin{flalign*}
  & P(a^1) =  \phantom{ \sum_{e,t,r,l}} &&
 \end{flalign*}
 \end{textblock}
}

\only<6>{
\begin{textblock}{130}(15,63)
\begin{flalign*}
 & P(a^1) = \sum_e\sum_t\sum_r\sum_l P(e,t,a^1,r,l) && 
\end{flalign*}
\end{textblock}
}

\only<7>{
\begin{textblock}{130}(15,63)
\begin{flalign*}
 & P(a^1) = \sum_{e,t,r,l} P(e,t,a^1,r,l) && 
\end{flalign*}
\end{textblock}
}

\only<8>{
\begin{textblock}{130}(15,63)
\begin{flalign*}
 & P(a^1) = \sum_{e,t,r,l} P(e)P(t)P(a^1|t,e)P(r|t)P(l|a^1) && 
\end{flalign*}
\end{textblock}
}

\only<9>{
\begin{textblock}{130}(15,63)
\begin{flalign*}
 & P(a^1) = \sum_{e,t,r,\textcolor{red}{l}} P(e)P(t)P(a^1|t,e)P(r|t)\textcolor{red}{\bm{P(l|a^1)}} && 
\end{flalign*}
\end{textblock}
}

\only<10>{
\begin{textblock}{130}(15,63)
\begin{flalign*}
  P(a^1) & = P(l^0|a^1) \sum_{e,t,r} P(e)P(t)P(a^1|t,e)P(r|t) \\
 & + P(l^1|a^1) \sum_{e,t,r} P(e)P(t)P(a^1|t,e)P(r|t) &&
\end{flalign*}
\end{textblock}
}

\only<11>{
\begin{textblock}{130}(15,63)
\begin{flalign*}
 & P(a^1) = \Big(\sum_l P(l|a^1) \Big) \Big(\sum_{e,t,r} P(e)P(t)P(a^1|t,e)P(r|t) \Big) && 
\end{flalign*}
\end{textblock}
}

\only<12>{
\begin{textblock}{130}(15,63)
\begin{flalign*}
 & P(a^1) = \Big(\sum_l P(l|a^1) \Big) \Big(\sum_{e,\textcolor{red}{\bm{t}},r} P(e)P(t)P(a^1|t,e)P(r|\textcolor{red}{\bm{t}}) \Big) && 
\end{flalign*}
\end{textblock}
}


\only<13>{
\begin{textblock}{130}(15,63)
\begin{flalign*}
 & P(a^1) = \Big(\sum_l P(l|a^1) \Big) \Big(\sum_{e,t} P(e)P(t)P(a^1|t,e)(\sum_r P(r|t)) \Big) && 
\end{flalign*}
\end{textblock}
}


\only<14>{
\begin{textblock}{130}(15,63)
\begin{flalign*}
 & P(a^1) = \Big(\cancel{\sum_l P(l|a^1)} \Big) \Big(\sum_{e,t} P(e)P(t)P(a^1|t,e)(\sum_r P(r|t)) \Big) && 
\end{flalign*}
\end{textblock}
}


\only<15>{
\begin{textblock}{130}(15,63)
\begin{flalign*}
 & P(a^1) = \Big(\cancel{\sum_l P(l|a^1)} \Big) \Big(\sum_{e,t} P(e)P(t)P(a^1|t,e)(\cancel{\sum_r P(r|t)}) \Big) && 
\end{flalign*}
\end{textblock}
}

\only<16>{
\begin{textblock}{130}(15,63)
\begin{flalign*}
 & P(a^1) = \sum_{e,t} P(e)P(t)P(a^1|t,e) \phantom{\cancel{\sum_h}}&& 
\end{flalign*}
\end{textblock}
}

\only<17>{
\begin{textblock}{130}(15,63)
\begin{flalign*}
 & P(a^1) = \sum_{e,t} P(e)P(t)P(a^1|t,e) = ?  && 
\end{flalign*}
\end{textblock}
}

\only<18>{
\begin{textblock}{130}(15,63)
\begin{flalign*}
 & P(a^1) = \sum_{e,t} P(e)P(t)P(a^1|t,e) \approx 0.019  && 
\end{flalign*}
\end{textblock}
}

\end{frame}


\begin{frame}[plain]
\begin{textblock}{160}(0,4)
 \centering \Large
 Grafo de factorización
\end{textblock}

\begin{textblock}{40}(5,12)
\centering
\begin{figure}[H]
\centering
  \scalebox{0.8}{
\tikz{ %
        
        
        \node[factor] (fa) {} ;
        \node[const,right=of fa] (pa) {$P(a|e,t)$} ;
        \node[det, below=of fa, yshift=0.1cm] (a) {$a$} ; %
        
        \node[factor, below=of a] (fl) {} ;
        \node[const,right=of fl] (pl) {$P(l|a)$} ;
        \node[det, below=of fl,yshift=0.1cm] (l) {$l$} ; %
        
        \node[det, above=of fa, xshift=-1.6cm,yshift=-0.1cm] (e) {$e$} ; %
        \node[factor, above=of e,yshift=-0.1cm] (fe) {} ;
        \node[const,right=of fe] (pe) {$P(e)$} ;
        
        \node[det, above=of fa, xshift=1.6cm,yshift=-0.1cm] (t) {$t$} ; %
        \node[factor, above=of t,yshift=-0.1cm] (ft) {} ;
        \node[const,right=of ft] (pt) {$P(t)$} ;
        \node[factor, below=of t, xshift=1.6cm,yshift=0.1cm] (fr) {} ;
        \node[det,below=of fr,yshift=0.1cm] (r) {$r$} ; %
        \node[const,right=of fr] (pr) {$P(r|t)$} ;
        
        \edge[-] {fa} {a,e,t};
        \edge[-] {fe} {e};
        \edge[-] {ft} {t};
        \edge[-] {fr} {r,t};
        \edge[-] {fl} {l,a};
        } 
}   
\end{figure}
\end{textblock}

\begin{textblock}{115}(45,30)
 
 \centering Nodos: \\ variables y funciones
 
 \vspace{1cm}
 
 \centering Ejes: \\ 
  ``la variable $v$ es argumento de la funci\'on $f$''
 
\end{textblock}
\end{frame}


\begin{frame}[plain]
\begin{textblock}{160}(0,4)
 \centering \Large El truco de la factorización
\end{textblock}
\vspace{1.5cm} \centering

\begin{equation*}
\sum \prod = \prod \sum
\end{equation*}

$a_1 \, b_1 + a_1 \, b_2 + a_2 \, b_1 + a_2 \, b_2  = (a1 + a_2) \, (b_1 + b_2)$

\pause

\begin{equation*}
   P(a^1) = \Big(\sum_l P(l|a^1) \Big) \Big(\sum_{e,t} P(e)P(t)P(a^1|t,e)(\sum_r P(r|t)) \Big)
\end{equation*}

  \includegraphics[width=0.35\textwidth]{../../auxiliar/static/sum_product.png}

\end{frame}



\begin{frame}[plain]
\begin{textblock}{160}(0,4)
 \centering \Large
 Sum-product algorithm
\end{textblock}

\only<2->{
\begin{textblock}{80}(45,30)
\begin{description}
 \onslide<3->{\item[$v(n)$ :] Vecinos del nodo $n$} 
 \item[$m_{x \rightarrow f}(x)$ :] Mensaje de variable $x$ a factor $f$ 
 \item[$m_{f \rightarrow x}(x)$ :] Mensaje de factor $f$ a variable $x$
\end{description}
\end{textblock}
}


\only<1-3>{
\begin{textblock}{80}(45,16)
\centering
Algoritmo para calcular cualquier marginal \\ mediante pasaje de mensajes entre nodos
\end{textblock}
}


\only<4->{
\begin{textblock}{80}(45,16)
\begin{equation*}
P(x) = \prod_{f \in v(x)} m_{f \rightarrow x}(x)
\end{equation*} 
\end{textblock}
}

\only<5->{
\begin{textblock}{80}(45,56)
\begin{equation*}\label{eq:m_v_f}
m_{x \rightarrow f}(x) = \prod_{h \in \only<6>{\textcolor{red}}{v(x) \setminus \{f\}} } m_{h \rightarrow x}(x)
\end{equation*}
\end{textblock}
}

\only<7->{
\begin{textblock}{80}(45,68)
\begin{equation*}\label{eq:m_f_v}
 m_{f \rightarrow x}(x) = \onslide<9->{\sum_{\bm{h}} \Big(} \onslide<8->{f(\bm{h},x)} \prod_{h \in v(f) \setminus \{x\} } m_{h \rightarrow f}(h) \onslide<9->{\Big)} 
\end{equation*}
\end{textblock}
}

\only<1->{
\begin{textblock}{40}(0,20)
\centering
\begin{figure}[H]
\centering
  \scalebox{.8}{
\tikz{ %
        
        
        \node[factor] (fa) {} ;
        \node[det, below=of fa, minimum size=0.55cm, yshift=0.4cm] (a) {$a$} ; %
        
        \node[factor, below=of a, yshift=0.4cm] (fl) {} ;
        \node[det, below=of fl, minimum size=0.55cm,yshift=0.4cm] (l) {$l$} ; %
        
        \node[det, above=of fa, minimum size=0.55cm, xshift=-0.8cm, yshift=-0.4cm] (e) {$e$} ; %
        \node[factor, above=of e, yshift=-0.4cm] (fe) {} ;
        
        \node[det,minimum size=0.55cm, above=of fa, xshift=0.8cm, yshift=-0.4cm] (t) {$t$} ; %
        \node[factor, above=of t, yshift=-0.4cm] (ft) {} ;
        \node[factor, below=of t, xshift=0.8cm, yshift=0.4cm] (fr) {} ;
        \node[det,minimum size=0.55cm, below=of fr, yshift=0.4cm] (r) {$r$} ; %
        
        \edge[-] {fa} {a,e,t};
        \edge[-] {fe} {e};
        \edge[-] {ft} {t};
        \edge[-] {fr} {r,t};
        \edge[-] {fl} {l,a};
        } 
}   
\end{figure}
\end{textblock}
}

% % 
\only<10>{
\begin{textblock}{160}(0,87)
\centering \tiny
\href{https://ieeexplore.ieee.org/stamp/stamp.jsp?arnumber=910572}{Kschischang FR, Frey BJ, Loeliger HA. Factor graphs and the sum-product algorithm. 2001}
\end{textblock}
}
\end{frame}

\begin{frame}[plain]
\begin{textblock}{160}(0,4)
 \centering \Large
 Modelo sin observables
\end{textblock}

\begin{textblock}{40}(5,12)
\centering
\begin{figure}[H]
\centering
  \scalebox{0.8}{
\tikz{ %
        
        
        \node[factor] (fa) {} ;
        %\node[const,left=of fa] (pa) {$P(a|e,t)$} ;
        \node[det, below=of fa, yshift=0.1cm] (a) {$a$} ; %
        
        \node[factor, below=of a] (fl) {} ;
        %\node[const,right=of fl] (pl) {$P(l|a)$} ;
        \node[det, below=of fl,yshift=0.1cm] (l) {$l$} ; %
        
        \node[det, above=of fa, xshift=-1.6cm,yshift=-0.1cm] (e) {$e$} ; %
        \node[factor, above=of e,yshift=-0.1cm] (fe) {} ;
        %\node[const,right=of fe] (pe) {$P(e)$} ;
        
        \node[det, above=of fa, xshift=1.6cm,yshift=-0.1cm] (t) {$t$} ; %
        \node[factor, above=of t,yshift=-0.1cm] (ft) {} ;
        %\node[const,right=of ft] (pt) {$P(t)$} ;
        \node[factor, below=of t, xshift=1.6cm,yshift=0.1cm] (fr) {} ;
        \node[det,below=of fr,yshift=0.1cm] (r) {$r$} ; %
        %\node[const,left=of fr] (pr) {$P(r|t)$} ;
        
        
        \edge[-] {fa} {a,e,t};
        \edge[-] {fe} {e};
        \edge[-] {ft} {t};
        \edge[-] {fr} {r,t};
        \edge[-] {fl} {l,a};
        
        \onslide<1->{\path[draw, ->, fill=black!50] (fe) edge[bend left,draw=black!50] node[right,color=black!75] {\only<2->{$P(e)$}} (e);}
        \onslide<3>{\path[draw, ->, fill=black!50] (e) edge[bend left,draw=black!50] node[above,color=black!75] {$P(e)$} (fa);}
        \onslide<4->{\path[draw, ->, fill=black!50] (ft) edge[bend left,draw=black!50] node[right,color=black!75] {$P(t)$} (t);}
        \onslide<5>{\path[draw, ->, fill=black!50] (r) edge[bend left,draw=black!50] node[left,color=black!75] {$1$} (fr);}
        \onslide<6->{\path[draw, ->, fill=black!50] (fr) edge[bend left,draw=black!50] node[left,color=black!75] {\only<7->{$1$}} (t);}
        \onslide<8->{\path[draw, ->, fill=black!50] (fl) edge[bend left,draw=black!50] node[left,color=black!75] {$1$} (a);}
        \onslide<9->{\path[draw, ->, fill=black!50] (fa) edge[bend left,draw=black!50] node[left,color=black!75] {\only<12->{$1$}} (t);}
        \onslide<13->{\path[draw, ->, fill=black!50] (fa) edge[bend left,draw=black!50] node[left,color=black!75] {$1$} (e);}
        \onslide<15->{\path[draw, ->, fill=black!50] (fr) edge[bend left,draw=black!50] node[right,color=black!75] {\only<16->{$P(r)$}} (r);}
        \onslide<17->{\path[draw, ->, fill=black!50] (fa) edge[bend left,draw=black!50] node[right,color=black!75] {\only<18->{$P(a)$}} (a);}
        \onslide<19->{\path[draw, ->, fill=black!50] (fl) edge[bend left,draw=black!50] node[right,color=black!75] {$P(l)$} (l);}
        }
}   
\end{figure}
\end{textblock}

\begin{textblock}{100}(62,16)
 $m_{f_e\rightarrow e}(e) = \onslide<2->{P(e)} \onslide<3>{= m_{e\rightarrow f_a}(e)}$  \\
 \onslide<4->{$m_{f_t\rightarrow t}(t) = P(t)$ } \\
 \only<5>{$m_{r \rightarrow f_r}(r) = 1$ \\}
 $\onslide<6->{m_{f_r\rightarrow t}(t) = \sum_r \, P(r|t)} \onslide<7->{= 1}$  \\
 $\onslide<8->{m_{f_l\rightarrow a}(a) = \sum_l \, P(l|a) = 1} $  \\
 \onslide<9->{$m_{f_a\rightarrow t}(t) = \sum_{ea} \, P(e) P(a|e,t) \only<10>{= \sum_{e} \, P(e) \sum_{a} P(a|e,t) } \only<11>{= \sum_{e} \, P(e)} \only<12->{= 1 }$  }\\
 \onslide<13->{$m_{f_a\rightarrow e}(e) = \sum_{ta} \, P(t) P(a|e,t) = 1$ } \\
 \onslide<15->{$m_{f_r\rightarrow r}(r) = \sum_{t} \, P(t) P(r|t) \onslide<16->{= P(r)}$ } \\
 \onslide<17->{$m_{f_a\rightarrow a}(a) = \sum_{et} \, P(e)P(t)P(a|e,t)  \onslide<18->{= P(a)}$ } \\
 \onslide<19->{$m_{f_l\rightarrow l}(l) = \sum_{a} \, P(a) P(l|a) = P(l)$ }
 \end{textblock}

 \only<14>{
\begin{textblock}{80}(60,60)
 \centering
 Todos los mensajes \\ que suben son $1$
\end{textblock}
}

 \only<20>{
\begin{textblock}{80}(60,70)
 \centering
 Todos los mensajes que \\ bajan son marginales 
\end{textblock}
}


 \only<21>{
\begin{textblock}{80}(60,65)
 \centering
 \begin{align*}
  P(l=1)\cdot 365 = ?
 \end{align*}
\end{textblock}
}

\end{frame}


\begin{frame}[plain]
\begin{textblock}{160}(0,4)
 \centering \Large
Modelo con observables
\end{textblock}

\begin{textblock}{40}(5,12)
\centering
\begin{figure}[H]
\centering
  \scalebox{0.8}{
\tikz{ %
        
        
        \node[factor] (fa) {} ;
        %\node[const,left=of fa] (pa) {$P(a|e,t)$} ;
        \node[det, below=of fa, yshift=0.1cm] (a) {$a$} ; %
        
        \node[factor, below=of a] (fl) {} ;
        %\node[const,right=of fl] (pl) {$P(l|a)$} ;
        \node[det, below=of fl,yshift=0.1cm] (l) {$l$} ; %
        
        \node[det, fill=black!10,above=of fa, xshift=-1.6cm,yshift=-0.1cm] (e) {$e^*$} ; %
        \node[factor, above=of e,yshift=-0.1cm] (fe) {} ;
        %\node[const,right=of fe] (pe) {$P(e)$} ;
        
        \node[det, above=of fa, xshift=1.6cm,yshift=-0.1cm] (t) {$t$} ; %
        \node[factor, above=of t,yshift=-0.1cm] (ft) {} ;
        %\node[const,right=of ft] (pt) {$P(t)$} ;
        \node[factor, below=of t, xshift=1.6cm,yshift=0.1cm] (fr) {} ;
        \node[det,below=of fr,yshift=0.1cm] (r) {$r$} ; %
        %\node[const,left=of fr] (pr) {$P(r|t)$} ;
        
        
        \edge[-] {fa} {a,e,t};
        \edge[-] {fe} {e};
        \edge[-] {ft} {t};
        \edge[-] {fr} {r,t};
        \edge[-] {fl} {l,a};
        
        \path[draw, ->, fill=black!50] (fe) edge[bend left,draw=black!50] node[right,color=black!75] {$P(e^*)$} (e);
        \path[draw, ->, fill=black!50] (ft) edge[bend left,draw=black!50] node[right,color=black!75] {$P(t)$} (t);
        \path[draw, ->, fill=black!50] (fr) edge[bend left,draw=black!50] node[left,color=black!75] {\only<1->{$1$}} (t);
        \path[draw, ->, fill=black!50] (fl) edge[bend left,draw=black!50] node[left,color=black!75] {$1$} (a);
        \onslide<2->{\path[draw, ->, fill=black!50] (fa) edge[bend left,draw=black!50] node[left,color=black!75] {\only<3->{$P(e^*)$}} (t);}
        \path[draw, ->, fill=black!50] (fa) edge[bend left,draw=black!50] node[left,color=black!75] {$1$} (e);
        \onslide<7->{\path[draw, ->, fill=black!50] (fr) edge[bend left,draw=black!50] node[right,color=black!75] {\only<8->{$P(r,e^*)$}} (r);}
        \onslide<9->{\path[draw, ->, fill=black!50] (fa) edge[bend left,draw=black!50] node[right,color=black!75] {\only<10->{$P(a,e^*)$}} (a);}
        \onslide<11->{\path[draw, ->, fill=black!50] (fl) edge[bend left,draw=black!50] node[right,color=black!75] {\only<12->{$P(l,e^*)$}} (l);}
        }
}   
\end{figure}
\end{textblock}

\begin{textblock}{100}(62,16)
 $m_{f_e\rightarrow e}\phantom{(e)} = P(e^*) $  \\
 $m_{f_t\rightarrow t}(t) = P(t)$  \\
 $m_{f_r\rightarrow t}(t) = \sum_r \, P(r|t) = 1$  \\
 $m_{f_l\rightarrow a}(a) = \sum_l \, P(l|a) = 1 $  \\
 $m_{f_a\rightarrow t}(t) = \onslide<2->{\sum_{a} \, P(e^*) P(a|e^*,t)} \onslide<3->{= P(e^*)}$  \\
 $m_{f_a\rightarrow e}\phantom{(e)} = \sum_{ta} \, P(t) P(a|e^*,t) = 1$  \\
 $m_{f_r\rightarrow r}(r) = \onslide<7->{\sum_{t} \, P(t) P(r|t)} \onslide<8->{= P(r,e^*)}$ \\
 $m_{f_a\rightarrow a}(a) = \onslide<9->{\sum_{t} \, P(e^*)P(t)P(a|e^*,t)}\onslide<10->{  = P(a,e^*)}$  \\
 $m_{f_l\rightarrow l}(l) = \onslide<11->{\sum_{a} \, P(a,e^*) P(l|a)} \onslide<12->{ = P(l,e^*)}$
 \end{textblock}

 
\only<4-6>{
\begin{textblock}{80}(70,60)
\begin{flalign*}
  P(t,e^*) & = P(e^*) P(t) \\
  \onslide<5->{P(t|e^*) &= \frac{P(e^*) P(t) }{P(e^*)}} \onslide<6->{= P(t)}&&
\end{flalign*}
\end{textblock}
}

\only<13-14>{
\begin{textblock}{80}(70,60)
\begin{flalign*}
  P(l|e^*) & \, = P(l,e^*)/P(e^*) \only<14>{\overset{?}= \, P(l)} &&
\end{flalign*}
\end{textblock}
}

\only<15>{
\begin{textblock}{80}(69.15,60)
\begin{flalign*}
  \underbrace{P(l|e^*)}_{\hfrac{\text{que llamen}}{\text{cuando entran}}} & = P(l,e^*)/P(e^*) \overset{?}=  \underbrace{P(l)}_{\hfrac{\text{que llamen}}{\text{}}} &&
\end{flalign*}
\end{textblock}
}

\end{frame}




\begin{frame}[plain]
\begin{textblock}{160}(0,4)
\centering \Large Flujos de inferencia
\end{textblock}

\only<1->{
\begin{textblock}{160}(0,18)
\centering
 \begin{tabular}{c c|c}
 & $\hfrac{\text{Intermedio}}{\text{no observable}}$ &   $\hfrac{\text{Intermedio}}{\text{observable}}$ \\
 & & \\
 $ e \rightarrow a \rightarrow l $    & \onslide<2->{$P(l) \overset{?}{=} P(l|e)$} & \onslide<3->{$P(l|a) \overset{?}{=} P(l|e,a)$} \\ 
 $ l \leftarrow a \leftarrow t $      &  \onslide<4->{$P(t) \overset{?}{=} P(t|l)$}  & \onslide<5->{$P(t|a) \overset{?}{=} P(t|a,l)$} \\ 
 $ a \leftarrow t \rightarrow r $     & \onslide<6->{$P(r) \overset{?}{=} P(r|a)$} & \onslide<7->{$P(r|t) \overset{?}{=} P(r|t,a)$} \\
 $ e \rightarrow a \leftarrow t $     & \onslide<8->{$P(t) \overset{?}{=} P(t|e)$} & \onslide<9->{$P(t|a) \overset{?}{=} P(t|e,a)$} \\
            $\overset{\downarrow}{l}$  &  & \onslide<10->{$P(t|l) \overset{?}{=} P(t|e,l)$}
 \end{tabular} 
 \end{textblock}
 }
 

\end{frame}

\begin{frame}[plain]
\centering
  \includegraphics[width=0.35\textwidth]{../../auxiliar/static/pachacuteckoricancha.jpg}
\end{frame}


\end{document}



