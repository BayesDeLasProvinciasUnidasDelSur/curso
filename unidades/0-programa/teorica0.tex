\documentclass[shownotes,aspectratio=169]{beamer}

\input{../../auxiliar/tex/diapo_encabezado.tex}
\input{../../auxiliar/tex/tikzlibrarybayesnet.code.tex}
 \mode<presentation>
 {
 %   \usetheme{Madrid}      % or try Darmstadt, Madrid, Warsaw, ...
 %   \usecolortheme{default} % or try albatross, beaver, crane, ...
 %   \usefonttheme{serif}  % or try serif, structurebold, ...
  \usetheme{Antibes}
  \setbeamertemplate{navigation symbols}{}
 }
\estrue
\usepackage{todonotes}
\setbeameroption{show notes}

\usepackage{listings}
\lstset{
  aboveskip=3mm,
  belowskip=3mm,
  showstringspaces=true,
  columns=flexible,
  basicstyle={\ttfamily},
  breaklines=true,
  breakatwhitespace=true,
  tabsize=4,
  showlines=true
}


\begin{document}

\color{black!85}
\large


\begin{frame}[plain,noframenumbering]


\begin{textblock}{160}(0,0)
\includegraphics[width=1\textwidth]{../../auxiliar/static/deforestacion}
\end{textblock}

\begin{textblock}{80}(18,9)
\textcolor{black!15}{\fontsize{44}{55}\selectfont Verdades}
\end{textblock}

\begin{textblock}{47}(85,70)
\centering \textcolor{black!15}{{\fontsize{52}{65}\selectfont Empíricas}}
\end{textblock}

\begin{textblock}{80}(100,28)
\LARGE  \textcolor{black!15}{\rotatebox[origin=tr]{-3}{\scalebox{9}{\scalebox{1}[-1]{$p$}}}}
\end{textblock}

\begin{textblock}{80}(66,43)
\LARGE  \textcolor{black!15}{\scalebox{6}{$=$}}
\end{textblock}

\begin{textblock}{80}(36,29)
\LARGE  \textcolor{black!15}{\scalebox{9}{$p$}}
\end{textblock}



\begin{textblock}{160}(01,81)
\footnotesize \textcolor{black!5}{\textbf{Seminario Verdades Empíricas. Hacia el \\
Congreso Bayesiano Plurinacional 2023} \\}
\end{textblock}

\end{frame}



% \begin{frame}[plain,noframenumbering]
% \begin{textblock}{80}(54,14)
%  \huge \textcolor{black!50}{Sorpresa}
% \end{textblock}
%
% \begin{textblock}{47}(113,73.5)
% \centering \LARGE  \textcolor{black!5}{Supervivencia}
% \end{textblock}
%
% \begin{textblock}{80}(100,27)
% \LARGE  \textcolor{black!10}{Creencia}
% \end{textblock}
%
% \begin{textblock}{80}(44,61)
% \LARGE  \textcolor{black!15}{Dato}
% \end{textblock}
%
% \begin{textblock}{160}(01,87)
% \scriptsize \textcolor{black!5}{Bayes de la Provincias Unidas del Sur, 2022.}
% \end{textblock}
%
% \begin{textblock}{160}(01,01)
% \normalsize \textcolor{black!60}{1.\ Introducción}
% \end{textblock}
%
%
%  %\vspace{2cm}brown
% %\maketitle
% \Wider[2cm]{
% \includegraphics[width=1\textwidth]{../../auxiliar/static/peligro_predador}
% }
% \end{frame}


%
% \begin{frame}[plain,noframenumbering]
% \begin{textblock}{160}(00,6)
% \centering
% \LARGE \ Seminario: ``Verdades empíricas'' \\
% \Large 1. Principios interculturales de acuerdos intersubjetivos
% \end{textblock}
%
% \begin{textblock}{160}(00,24) \centering
% Hacia el Congreso Bayesiano Plurinacional  \\
% Apoya \href{https://github.com/BayesDeLasProvinciasUnidasDelSur/curso}{Bayes (de las Provincias Unidas) del Sur.}
% \end{textblock}
%
% \begin{textblock}{140}(10,44)
% \small
% %\Large Los principios de la inferencia Bayesiana \\ \justify \large
%
% $\bullet$ Incertidumbre \\
% $\bullet$ Principio de razón suficiente \\
% $\bullet$ Principio de indiferencia \\
% $\bullet$ Principio de integridad \\
% $\bullet$ Principio de coherencia \\
% $\bullet$ Las reglas del razonamiento bajo incertidumbre \\
% $\bullet$ Teorema de Bayes \\
% $\bullet$ Evaluación de modelos causales alternativos \\
% \end{textblock}
%
% \end{frame}



\begin{frame}[plain]
\begin{textblock}{160}(00,04)
\centering
\LARGE Verdad
\end{textblock}
\vspace{2.5cm} \large

\centering

 La ciencia es una institución humana con pretensión de \textbf{verdad}. \\[0.1cm] \pause

\textbf{Acuerdos intersubjetivos con validez intercultural (o universal)}


\vspace{1cm}
\pause


Para todas las personas. En todas las culturas.

\end{frame}


\begin{frame}[plain]
\begin{textblock}{160}(00,04)
\centering
\LARGE Verdades
\end{textblock}
\vspace{1.5cm} \large

\centering

 \Large Ciencias formales \\
 \normalsize \textcolor{black!50}{(Matemáticas, lógicas)} \\
 \large  Validadas en sistemas cerrados sin incertidumbre\\

 \vspace{0.6cm}

  \pause

 \Large Ciencias empíricas \\
\normalsize  \textcolor{black!50}{(Físicas, Químicas, Biológicas, Sociales)} \\
\large Validadas en sistemas abiertos con incertidumbre

\pause
\vspace{1cm}

¿Cuál es el origen de la \textbf{incertidumbre}?

\end{frame}




\begin{frame}[plain]
\begin{textblock}{160}(00,04)
\centering \LARGE Incertidumbre ontológica \\
\Large que existe en la naturaleza
\end{textblock}
\vspace{2cm} \centering


Dadas mismas condiciones iniciales \\
se genera a veces un resultado y a veces otro

\pause

\vspace{0.7cm}

\includegraphics[width=0.22\textwidth]{../../auxiliar/static/piedra-papel-tijera.jpg} \hspace{0.5cm}
\includegraphics[width=0.22\textwidth]{../../auxiliar/static/dosdados.jpg}

{\small Libre albedrío \ \hspace{2cm} \ Azar \hspace{0.4cm}}


\pause
\vspace{0.5cm}
\textbf{¿Se puede crear?}


\end{frame}



\begin{frame}[plain]
\begin{textblock}{160}(00,04)
\centering
\LARGE Principio de razón suificiente \\
\end{textblock}
\vspace{1.75cm} \Large  \centering



\begin{textblock}{70}(74,20)
\includegraphics[width=0.67\textwidth]{../../auxiliar/static/pachacuteckoricancha.jpg}
\end{textblock}

\begin{textblock}{60}(18,42)
\centering Todo lo que ocurre \\ tiene una causa previa \\ que lo genera
\end{textblock}



\end{frame}



\begin{frame}[plain]
\begin{textblock}{160}(00,04)
\centering
\LARGE Principio de razón suificiente \\
\Large Modelos causales deterministas
\end{textblock}
\vspace{1cm}

\pause

\begin{align*}
 \text{Población}(t+1) = r \cdot \text{Población}(t)\cdot (1-\text{Población}(t))
\end{align*}

\vspace{0.5cm}

\centering \pause

\tikz{
    \node[latent, minimum size=1.25cm] (n1) {$\text{pob}_t$} ;

    \node[latent, right=of n1, xshift=3cm, minimum size=1.25cm] (n2) {$\text{pob}_{t+1}$} ;

     \path[->] (n1) edge node [yshift=0.5cm] {$r \cdot \text{pob}_t \cdot (1-\text{pob}_t)$} (n2);

}

\end{frame}


\begin{frame}[plain]
\begin{textblock}{160}(00,04)
\centering
\LARGE Principio de razón suificiente \\
\Large \only<1-13>{Modelos causales deterministas}\only<14>{Incertidumbre determinista!}
\end{textblock}
\vspace{1cm}


\only<1>{
\begin{textblock}{160}(0,22)
\centering
 \includegraphics[page={1},width=0.6\textwidth]{figuras/poblacion.pdf}
\end{textblock}
}

\only<2>{
\begin{textblock}{160}(0,22)
\centering
 \includegraphics[page={2},width=0.6\textwidth]{figuras/poblacion.pdf}
\end{textblock}
}

\only<3>{
\begin{textblock}{160}(0,22)
\centering
 \includegraphics[page={3},width=0.6\textwidth]{figuras/poblacion.pdf}
\end{textblock}
}

\only<4>{
\begin{textblock}{160}(0,22)
\centering
 \includegraphics[page={4},width=0.6\textwidth]{figuras/poblacion.pdf}
\end{textblock}
}

\only<5>{
\begin{textblock}{160}(0,22)
\centering
 \includegraphics[page={5},width=0.6\textwidth]{figuras/poblacion.pdf}
\end{textblock}
}

\only<6>{
\begin{textblock}{160}(0,22)
\centering
 \includegraphics[page={6},width=0.6\textwidth]{figuras/poblacion.pdf}
\end{textblock}
}

\only<7>{
\begin{textblock}{160}(0,22)
\centering
 \includegraphics[page={7},width=0.6\textwidth]{figuras/poblacion.pdf}
\end{textblock}
}

\only<8>{
\begin{textblock}{160}(0,22)
\centering
 \includegraphics[page={8},width=0.6\textwidth]{figuras/poblacion.pdf}
\end{textblock}
}


\only<9>{
\begin{textblock}{160}(0,22)
\centering
 \includegraphics[page={9},width=0.6\textwidth]{figuras/poblacion.pdf}
\end{textblock}
}


\only<10>{
\begin{textblock}{160}(0,22)
\centering
 \includegraphics[page={10},width=0.6\textwidth]{figuras/poblacion.pdf}
\end{textblock}
}

\only<11>{
\begin{textblock}{160}(0,22)
\centering
 \includegraphics[page={11},width=0.6\textwidth]{figuras/poblacion.pdf}
\end{textblock}
}


\only<12>{
\begin{textblock}{160}(0,22)
\centering
 \includegraphics[page={15},width=0.6\textwidth]{figuras/poblacion.pdf}
\end{textblock}
}


\only<13>{
\begin{textblock}{160}(0,22)
\centering
 \includegraphics[page={12},width=0.59\textwidth]{figuras/poblacion.pdf}
\end{textblock}
}

\only<14>{
\begin{textblock}{160}(0,22)
\centering
 \includegraphics[page={13},width=0.59\textwidth]{figuras/poblacion.pdf}
\end{textblock}
}
\end{frame}


\begin{frame}[plain]
\begin{textblock}{160}(0,4)
\centering \LARGE Creamos (casi) azar! \\
\Large Números pseudo-aleatorios
\end{textblock}

\only<2>{
\begin{textblock}{160}(0,22)
\centering
 \includegraphics[page={14},width=0.59\textwidth]{figuras/poblacion.pdf}
\end{textblock}
}

\end{frame}


\begin{frame}[plain]
 \begin{textblock}{160}(0,4)
 \centering \LARGE Incertidumbre epistémica \\
 \Large que existe en nuestro conocimiento
 \end{textblock}
 \vspace{1.7cm} \centering

 \pause

\textbf{No podemos conocer con infinita precisión}

\vspace{0.6cm}

 La sensibilidad a las condiciones iniciales es suficiente \\ para introducir incertidumbre en sistemas deterministas

\pause \vspace{1.2cm}


Incertidumbre ontológica $\subset$ Incertidumbre epistémica
\end{frame}


%
% \begin{frame}[plain]
%  \begin{textblock}{160}(0,4)
%  \centering \LARGE Incertidumbres \\
%  %\Large y probabilidades
%  \end{textblock}
%  \vspace{1.3cm} \centering
%
%  \Large Visión ontológica \\
%
%  \large La incertidumbre existe en la naturaleza, producto del azar (o libre albedrío)\\[0.2cm]
%
%  %\large \textbf{Variable aleatoria}: Probabilidad como frecuencia
%
%  \vspace{1cm} \pause
%
%  \Large Visión epistémica  \\
%
%  \large La incertidumbre existe en nuestro conocimiento, producto de nuestra ignorancia\\[0.2cm]
%
%  %\large \textbf{Hipótesis}: Probabilidad como grado de creencia
%
%
%  \end{frame}



\begin{frame}[plain]
\begin{textblock}{160}(00,04)
\centering \LARGE Verdades con incertidumbre
\end{textblock}
\vspace{2.5cm} \Large \centering


\Wider[1cm]{
``No hay verdades absolutas, todas las verdades son medias verdades. El mal surge de quererlas tratar como verdades absolutas.''
}
\large

\vspace{0.2cm}

\hfill \textcolor{black!50}{Whitehead (Dialogues 1953)}


\end{frame}




\begin{frame}[plain]
\begin{textblock}{160}(0,4) \centering
\LARGE Verdades con incertidumbre\\
\Large Como relativismo
\end{textblock}
\vspace{1.5cm}
\onslide<1-2>{
\centering
 \Large ``Todos los puntos de vista deben ser considerados y respetados''
}

 \vspace{0.5cm}

 \onslide<2>{
\Large ``No hay una forma correcta de evaluar modelos''
}

\only<3->{
\begin{textblock}{50}(3,20) \centering
\includegraphics[width=1\textwidth, page={6}]{../../auxiliar/static/sidewalk_bubblegum_1997_1}
\end{textblock}}
 \only<4->{
\begin{textblock}{50}(55,20) \centering
\includegraphics[width=1\textwidth, page={6}]{../../auxiliar/static/sidewalk_bubblegum_1997_2}
\end{textblock}}
% \only<4>{
% \begin{textblock}{50}(107,20) \centering
% \includegraphics[width=1\textwidth, page={6}]{../../auxiliar/static/sidewalk_bubblegum_1997_3}
% \end{textblock}}
\only<5->{
\begin{textblock}{50}(107,20) \centering
\includegraphics[width=1\textwidth, page={6}]{../../auxiliar/static/sidewalk_bubblegum_1997_4}
\end{textblock}}


\only<6>{
\begin{textblock}{160}(0,74)
\centering \Large
El retorno al criterio de autoridad \\[0cm]
\large
La verdad, como el poder, sale de la boca del fusil
\end{textblock}}
\end{frame}



\begin{frame}[plain]
\begin{textblock}{160}(0,4) \centering
\LARGE Verdades con incertidumbre \\
\Large \only<2->{Universalizables}
\end{textblock}
\vspace{2cm}

\Large \centering

¿Es posible alcanzar acuerdos intersubjetivos \\ en contextos de incertidumbre?

\vspace{1cm} \large

\pause

Sí, porque \textbf{podemos evitar mentir}!

\vspace{0.2cm} \pause

$\bullet$ No decir más de lo que sabemos

$\bullet$ Incorporando toda la información disponible

\pause \centering \vspace{1cm}

\textbf{¿Cómo exactamente?}


\end{frame}


\begin{frame}[plain,noframenumbering]
\begin{textblock}{170}(-9,0)
\rotatebox[origin=tr]{90}{\includegraphics[width=0.53\textwidth]{../../auxiliar/static/egipto3.jpeg}}
\end{textblock}

\begin{textblock}{160}(16,9)
\LARGE \textcolor{black!5}{\fontsize{22}{0}\selectfont \textbf{Principios interculturales}}
\end{textblock}
\begin{textblock}{160}(22,18)
\LARGE \textcolor{black!5}{\fontsize{22}{0}\selectfont \textbf{de acuerdos intersubjetivos}}
\end{textblock}


\begin{textblock}{55}(71,38)
\begin{turn}{33}
\parbox{6cm}{
\textcolor{black!5}{\hspace{-0.3cm}Capítulo 1} \\
\small\textcolor{black!5}{\hspace{-0.1cm}Principio de indiferencia, de}\\
\small\textcolor{black!5}{integridad, y de coherencia.} \\
\small\textcolor{black!5}{\hspace{0.1cm}Las reglas de razonamiento} \\ \small\textcolor{black!5}{\hspace{0.15cm}bajo incertidumbre. Evaluación} \\
\small\textcolor{black!5}{\hspace{0.36cm}de modelos alternativos.} \\
}
\end{turn}
\end{textblock}

\end{frame}







\end{document}



