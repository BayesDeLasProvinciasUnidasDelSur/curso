\documentclass[shownotes,aspectratio=169]{beamer}

\input{../../auxiliar/tex/diapo_encabezado.tex}
% tikzlibrary.code.tex
%
% Copyright 2010-2011 by Laura Dietz
% Copyright 2012 by Jaakko Luttinen
%
% This file may be distributed and/or modified
%
% 1. under the LaTeX Project Public License and/or
% 2. under the GNU General Public License.
%
% See the files LICENSE_LPPL and LICENSE_GPL for more details.

% Load other libraries
\usetikzlibrary{shapes}
\usetikzlibrary{fit}
\usetikzlibrary{chains}
\usetikzlibrary{arrows}

% Latent node
\tikzstyle{latent} = [circle,fill=white,draw=black,inner sep=1pt,
minimum size=20pt, font=\fontsize{10}{10}\selectfont, node distance=1]
% Observed node
\tikzstyle{obs} = [latent,fill=gray!25]
% Invisible node
\tikzstyle{invisible} = [latent,minimum size=0pt,color=white, opacity=0, node distance=0]
% Constant node
\tikzstyle{const} = [rectangle, inner sep=0pt, node distance=0.1]
%state
\tikzstyle{estado} = [latent,minimum size=8pt,node distance=0.4]
%action
\tikzstyle{accion} =[latent,circle,minimum size=5pt,fill=black,node distance=0.4]


% Factor node
\tikzstyle{factor} = [rectangle, fill=black,minimum size=10pt, draw=black, inner
sep=0pt, node distance=1]
% Deterministic node
\tikzstyle{det} = [latent, rectangle]

% Plate node
\tikzstyle{plate} = [draw, rectangle, rounded corners, fit=#1]
% Invisible wrapper node
\tikzstyle{wrap} = [inner sep=0pt, fit=#1]
% Gate
\tikzstyle{gate} = [draw, rectangle, dashed, fit=#1]

% Caption node
\tikzstyle{caption} = [font=\footnotesize, node distance=0] %
\tikzstyle{plate caption} = [caption, node distance=0, inner sep=0pt,
below left=5pt and 0pt of #1.south east] %
\tikzstyle{factor caption} = [caption] %
\tikzstyle{every label} += [caption] %

\tikzset{>={triangle 45}}

%\pgfdeclarelayer{b}
%\pgfdeclarelayer{f}
%\pgfsetlayers{b,main,f}

% \factoredge [options] {inputs} {factors} {outputs}
\newcommand{\factoredge}[4][]{ %
  % Connect all nodes #2 to all nodes #4 via all factors #3.
  \foreach \f in {#3} { %
    \foreach \x in {#2} { %
      \path (\x) edge[-,#1] (\f) ; %
      %\draw[-,#1] (\x) edge[-] (\f) ; %
    } ;
    \foreach \y in {#4} { %
      \path (\f) edge[->,#1] (\y) ; %
      %\draw[->,#1] (\f) -- (\y) ; %
    } ;
  } ;
}

% \edge [options] {inputs} {outputs}
\newcommand{\edge}[3][]{ %
  % Connect all nodes #2 to all nodes #3.
  \foreach \x in {#2} { %
    \foreach \y in {#3} { %
      \path (\x) edge [->,#1] (\y) ;%
      %\draw[->,#1] (\x) -- (\y) ;%
    } ;
  } ;
}

% \factor [options] {name} {caption} {inputs} {outputs}
\newcommand{\factor}[5][]{ %
  % Draw the factor node. Use alias to allow empty names.
  \node[factor, label={[name=#2-caption]#3}, name=#2, #1,
  alias=#2-alias] {} ; %
  % Connect all inputs to outputs via this factor
  \factoredge {#4} {#2-alias} {#5} ; %
}

% \plate [options] {name} {fitlist} {caption}
\newcommand{\plate}[4][]{ %
  \node[wrap=#3] (#2-wrap) {}; %
  \node[plate caption=#2-wrap] (#2-caption) {#4}; %
  \node[plate=(#2-wrap)(#2-caption), #1] (#2) {}; %
}

% \gate [options] {name} {fitlist} {inputs}
\newcommand{\gate}[4][]{ %
  \node[gate=#3, name=#2, #1, alias=#2-alias] {}; %
  \foreach \x in {#4} { %
    \draw [-*,thick] (\x) -- (#2-alias); %
  } ;%
}

% \vgate {name} {fitlist-left} {caption-left} {fitlist-right}
% {caption-right} {inputs}
\newcommand{\vgate}[6]{ %
  % Wrap the left and right parts
  \node[wrap=#2] (#1-left) {}; %
  \node[wrap=#4] (#1-right) {}; %
  % Draw the gate
  \node[gate=(#1-left)(#1-right)] (#1) {}; %
  % Add captions
  \node[caption, below left=of #1.north ] (#1-left-caption)
  {#3}; %
  \node[caption, below right=of #1.north ] (#1-right-caption)
  {#5}; %
  % Draw middle separation
  \draw [-, dashed] (#1.north) -- (#1.south); %
  % Draw inputs
  \foreach \x in {#6} { %
    \draw [-*,thick] (\x) -- (#1); %
  } ;%
}

% \hgate {name} {fitlist-top} {caption-top} {fitlist-bottom}
% {caption-bottom} {inputs}
\newcommand{\hgate}[6]{ %
  % Wrap the left and right parts
  \node[wrap=#2] (#1-top) {}; %
  \node[wrap=#4] (#1-bottom) {}; %
  % Draw the gate
  \node[gate=(#1-top)(#1-bottom)] (#1) {}; %
  % Add captions
  \node[caption, above right=of #1.west ] (#1-top-caption)
  {#3}; %
  \node[caption, below right=of #1.west ] (#1-bottom-caption)
  {#5}; %
  % Draw middle separation
  \draw [-, dashed] (#1.west) -- (#1.east); %
  % Draw inputs
  \foreach \x in {#6} { %
    \draw [-*,thick] (\x) -- (#1); %
  } ;%
}


 \mode<presentation>
 {
 %   \usetheme{Madrid}      % or try Darmstadt, Madrid, Warsaw, ...
 %   \usecolortheme{default} % or try albatross, beaver, crane, ...
 %   \usefonttheme{serif}  % or try serif, structurebold, ...
  \usetheme{Antibes}
  \setbeamertemplate{navigation symbols}{}
 }
\estrue
\usepackage{todonotes}
\setbeameroption{show notes}

\newcommand{\gray}{\color{black!55}}

\usepackage{listings}
\lstset{
  aboveskip=3mm,
  belowskip=3mm,
  showstringspaces=true,
  columns=flexible,
  basicstyle={\ttfamily},
  breaklines=true,
  breakatwhitespace=true,
  tabsize=4,
  showlines=true
}


\begin{document}

\color{black!85}
\large


\begin{frame}[plain,noframenumbering]
\centering \vspace{0.5cm}
\includegraphics[width=1\textwidth]{../../auxiliar/static/CBP.png}
\end{frame}

\begin{frame}[plain,noframenumbering]


\begin{textblock}{160}(0,0)
\includegraphics[width=1\textwidth]{../../auxiliar/static/deforestacion}
\end{textblock}

\begin{textblock}{80}(18,9)
\textcolor{black!15}{\fontsize{44}{55}\selectfont Verdades}
\end{textblock}

\begin{textblock}{47}(85,70)
\centering \textcolor{black!15}{{\fontsize{52}{65}\selectfont Empíricas}}
\end{textblock}

\begin{textblock}{80}(100,28)
\LARGE  \textcolor{black!15}{\rotatebox[origin=tr]{-3}{\scalebox{9}{\scalebox{1}[-1]{$p$}}}}
\end{textblock}

\begin{textblock}{80}(66,43)
\LARGE  \textcolor{black!15}{\scalebox{6}{$=$}}
\end{textblock}

\begin{textblock}{80}(36,29)
\LARGE  \textcolor{black!15}{\scalebox{9}{$p$}}
\end{textblock}



\begin{textblock}{160}(01,81)
\footnotesize \textcolor{black!5}{\textbf{Seminario ``Acuerdos inte\dots''. Hacia el \\
Congreso Bayesiano Plurinacional 2023} \\}
\end{textblock}

\end{frame}



% \begin{frame}[plain,noframenumbering]
% \begin{textblock}{80}(54,14)
%  \huge \textcolor{black!50}{Sorpresa}
% \end{textblock}
%
% \begin{textblock}{47}(113,73.5)
% \centering \LARGE  \textcolor{black!5}{Supervivencia}
% \end{textblock}
%
% \begin{textblock}{80}(100,27)
% \LARGE  \textcolor{black!10}{Creencia}
% \end{textblock}
%
% \begin{textblock}{80}(44,61)
% \LARGE  \textcolor{black!15}{Dato}
% \end{textblock}
%
% \begin{textblock}{160}(01,87)
% \scriptsize \textcolor{black!5}{Bayes de la Provincias Unidas del Sur, 2022.}
% \end{textblock}
%
% \begin{textblock}{160}(01,01)
% \normalsize \textcolor{black!60}{1.\ Introducción}
% \end{textblock}
%
%
%  %\vspace{2cm}brown
% %\maketitle
% \Wider[2cm]{
% \includegraphics[width=1\textwidth]{../../auxiliar/static/peligro_predador}
% }
% \end{frame}


%
% \begin{frame}[plain,noframenumbering]
% \begin{textblock}{160}(00,6)
% \centering
% \LARGE \ Seminario: ``Verdades empíricas'' \\
% \Large 1. Principios interculturales de acuerdos intersubjetivos
% \end{textblock}
%
% \begin{textblock}{160}(00,24) \centering
% Hacia el Congreso Bayesiano Plurinacional  \\
% Apoya \href{https://github.com/BayesDeLasProvinciasUnidasDelSur/curso}{Bayes (de las Provincias Unidas) del Sur.}
% \end{textblock}
%
% \begin{textblock}{140}(10,44)
% \small
% %\Large Los principios de la inferencia Bayesiana \\ \justify \large
%
% $\bullet$ Incertidumbre \\
% $\bullet$ Principio de razón suficiente \\
% $\bullet$ Principio de indiferencia \\
% $\bullet$ Principio de integridad \\
% $\bullet$ Principio de coherencia \\
% $\bullet$ Las reglas del razonamiento bajo incertidumbre \\
% $\bullet$ Teorema de Bayes \\
% $\bullet$ Evaluación de modelos causales alternativos \\
% \end{textblock}
%
% \end{frame}



\begin{frame}[plain]
\begin{textblock}{160}(00,04)
\centering
\LARGE Verdad
\end{textblock}
\vspace{2cm} \large

\centering

 La ciencia es una institución humana con pretensión de \textbf{verdad}. \\[0.1cm] \pause

\textbf{Acuerdos intersubjetivos con validez intercultural (o universal)}


\vspace{1cm}
\pause

Para todas las personas. En todas las culturas.

\end{frame}


\begin{frame}[plain]
\begin{textblock}{160}(00,04)
\centering
\LARGE Verdades
\end{textblock}
\vspace{1cm} \large

\centering

 \Large Ciencias formales \\
 \normalsize \textcolor{black!50}{(Matemáticas, lógicas)} \\
 \large  Validadas en sistemas cerrados sin incertidumbre\\

 \vspace{0.7cm}

  \pause

 \Large Ciencias empíricas \\
\normalsize  \textcolor{black!50}{(Físicas, Químicas, Biológicas, Sociales)} \\
\large Validadas en sistemas abiertos con incertidumbre

\pause
\vspace{0.6cm}

¿Es posible alcanzar acuerdos intersubjetivos \\ en contextos de incertidumbre?
%
% \pause
% \vspace{0.2cm}
%
%
% Sí. Podemos evitar mentir.

\end{frame}


\begin{frame}[plain]
\begin{textblock}{160}(00,04)
\centering
\LARGE No hay una forma correcta de evaluar hipótesis \\
\Large Todo depende del punto de vista (o función de costo)
\end{textblock}
\vspace{1cm} \large


\only<2->{
\begin{textblock}{50}(3,26) \centering
\includegraphics[width=1\textwidth, page={6}]{../../auxiliar/static/sidewalk_bubblegum_1997_1}
\end{textblock}}
 \only<3->{
\begin{textblock}{50}(55,26) \centering
\includegraphics[width=1\textwidth, page={6}]{../../auxiliar/static/sidewalk_bubblegum_1997_2}
\end{textblock}}
% \only<4>{
% \begin{textblock}{50}(107,20) \centering
% \includegraphics[width=1\textwidth, page={6}]{../../auxiliar/static/sidewalk_bubblegum_1997_3}
% \end{textblock}}
\only<4->{
\begin{textblock}{50}(107,26) \centering
\includegraphics[width=1\textwidth, page={6}]{../../auxiliar/static/sidewalk_bubblegum_1997_4}
\end{textblock}}

\end{frame}

\begin{frame}[plain]
\begin{textblock}{160}(0,4) \centering
\LARGE Verdades con incertidumbre \\
\Large Universalizables
\end{textblock}
\vspace{2cm}

\Large \centering

Podemos alcanzar acuerdos intersubjetivos \\ en contextos de incertidumbre \\
gracias a que \textbf{podemos evitar mentir}:



\vspace{1cm} \pause \large

$\bullet$ No decir más de lo que sabemos

$\bullet$ Incorporando toda la información disponible

\pause \centering \vspace{1cm}

\textbf{¿Cómo exactamente?}


\end{frame}



\begin{frame}[plain,noframenumbering]
\begin{textblock}{170}(-9,0)
\rotatebox[origin=tr]{90}{\includegraphics[width=0.53\textwidth]{../../auxiliar/static/egipto3.jpeg}}
\end{textblock}

\begin{textblock}{160}(16,9)
\LARGE \textcolor{black!5}{\fontsize{22}{0}\selectfont \textbf{Principios interculturales}}
\end{textblock}
\begin{textblock}{160}(22,18)
\LARGE \textcolor{black!5}{\fontsize{22}{0}\selectfont \textbf{de acuerdos intersubjetivos}}
\end{textblock}


\begin{textblock}{55}(71,38)
\begin{turn}{33}
\parbox{6cm}{
\textcolor{black!5}{\hspace{-0.3cm}Capítulo 1} \\
\small\textcolor{black!5}{\hspace{-0.1cm}Principio de indiferencia, de}\\
\small\textcolor{black!5}{integridad, y de coherencia.} \\
\small\textcolor{black!5}{\hspace{0.1cm}Las reglas de razonamiento} \\ \small\textcolor{black!5}{\hspace{0.15cm}bajo incertidumbre. Evaluación} \\
\small\textcolor{black!5}{\hspace{0.36cm}de modelos alternativos.} \\
}
\end{turn}
\end{textblock}

\end{frame}





%
% \begin{frame}[plain]
%  \begin{textblock}{160}(0,4)
%  \centering \LARGE Incertidumbre \\
%  \Large Libre albedrío
% \end{textblock}
% \vspace{1.5cm}
% \centering
%
% \only<2-3>{
% \begin{textblock}{75}(75,28)
% Detrás de una caja hay un regalo \\[0.3cm]
%
%  \scalebox{1}{
% \tikz{ %
%          \node[factor, minimum size=1cm] (p1) {} ;
%          \node[factor, minimum size=1cm, xshift=1.5cm] (p2) {} ;
%          \node[factor, minimum size=1cm, xshift=3cm] (p3) {} ;
%
%
%          \node[const, above=of p1, yshift=0.1cm] (np1) {\Large $?$};
%          \node[const, above=of p2, yshift=0.1cm] (np2) {\Large $?$};
%          \node[const, above=of p3, yshift=0.1cm] (np3) {\Large $?$};
%          }
% }
% \end{textblock}
% }
%
%
% \begin{textblock}{75}(05,18)
%  \centering
%  \begin{figure}[H]
% \centering
%     \scalebox{1.2}{
%  \tikz{ %
%         \node[estado] (s) {};
%         \node[const, above=of s] {$s$};
%         \node[accion, below=of s, xshift=-1cm] (a1) {} ; %
% 	\node[accion, below=of s, xshift=0cm] (a2) {} ; %
% 	\node[accion, below=of s, xshift=1cm] (a3) {} ; %
% 	\node[const, right=of a3] {$a$};
%         \edge[-] {s} {a1,a2,a3};
%
% 	\node[estado, below=of a1,xshift=0cm] (s1a) {}; %
%
% 	\node[estado, below=of a2,xshift=0cm] (s2a) {}; %
%
% 	\node[estado, below=of a3,xshift=0cm] (s3a) {}; %
%
% 	\node[const, right=of s3a] {$s^{\prime}$};
% 	\edge[-] {a1} {s1a};
% 	\edge[-] {a2} {s2a};
% 	\edge[-] {a3} {s3a};
%         }
%
%     }
% \end{figure}
% \normalsize \textbf{Universos paralelos} \ \ \ \   \\ $s$: estados, $a$: acciones \ \ \ \
% \end{textblock}
%
% \only<3>{
% \begin{textblock}{160}(0,70)
% \centering \Large
% Funci\'ones de probabilidad permiten expresar nuestra incertidumbre
%  \begin{align*}
% P(a = 1)  = \, ? \ \ \  P(a = 2) = \,? \ \ \  P(a = 3) = \,?
%  \end{align*}
% \end{textblock}
% }
%
% \end{frame}
%

\begin{frame}[plain]
 \begin{textblock}{160}(0,4)
 \centering \LARGE \onslide<3->{Distribución de creencias \\}
 \Large \only<6>{\textbf{Primer acuerdo intersubjetivo!}}
\end{textblock}
\vspace{1.5cm}
\centering


\only<1>{
\begin{textblock}{160}(0,62)
\Large Detrás de una caja hay un regalo. \\[0.1cm]

\large ¿Dónde está el regalo?
\end{textblock}
}

\only<1>{
\begin{textblock}{160}(0,28)
 \scalebox{1.1}{
\tikz{ %
         \node[factor, minimum size=1cm] (p1) {} ;
         \node[factor, minimum size=1cm, xshift=1.5cm] (p2) {} ;
         \node[factor, minimum size=1cm, xshift=3cm] (p3) {} ;


         \node[const, above=of p1, yshift=0.1cm] (np1) {\Large $?$};
         \node[const, above=of p2, yshift=0.1cm] (np2) {\Large $?$};
         \node[const, above=of p3, yshift=0.1cm] (np3) {\Large $?$};
         }
}
\end{textblock}
}

\only<2>{
\begin{textblock}{160}(0,28)
 \scalebox{1.1}{
\tikz{ %
         \node[factor, minimum size=1cm] (p1) {} ;
         \node[factor, minimum size=1cm, xshift=1.5cm] (p2) {} ;
         \node[factor, minimum size=1cm, xshift=3cm] (p3) {} ;


         \node[const, above=of p1, yshift=0.125cm] (np1) {\Large $0$};
         \node[const, above=of p2, yshift=0.125cm] (np2) {\Large $1$};
         \node[const, above=of p3, yshift=0.125cm] (np3) {\Large $0$};
         }
}
\end{textblock}
}

\only<3>{
\begin{textblock}{160}(0,28)
 \scalebox{1.1}{
\tikz{ %
         \node[factor, minimum size=1cm] (p1) {} ;
         \node[factor, minimum size=1cm, xshift=1.5cm] (p2) {} ;
         \node[factor, minimum size=1cm, xshift=3cm] (p3) {} ;


         \node[const, above=of p1, yshift=-0.05cm] (np1) {\Large $1/10$};
         \node[const, above=of p2, yshift=-0.05cm] (np2) {\Large $8/10$};
         \node[const, above=of p3, yshift=-0.05cm] (np3) {\Large $1/10$};
         }
}
\end{textblock}
}


\only<4->{
\begin{textblock}{160}(0,28)
 \scalebox{1.1}{
\tikz{ %
         \node[factor, minimum size=1cm] (p1) {} ;
         \node[factor, minimum size=1cm, xshift=1.5cm] (p2) {} ;
         \node[factor, minimum size=1cm, xshift=3cm] (p3) {} ;


         \node[const, above=of p1, yshift=-0.05cm] (np1) {\Large $1/3$};
         \node[const, above=of p2, yshift=-0.05cm] (np2) {\Large $1/3$};
         \node[const, above=of p3, yshift=-0.05cm] (np3) {\Large $1/3$};
         }
}
\end{textblock}
}

\only<5->{
\begin{textblock}{140}(10,64)   \centering \Large
Principio de indiferencia \\ \large Máximiza incertidumbre dada información disponible
\end{textblock}
}

\end{frame}
%
%
% \begin{frame}[plain]
% \begin{textblock}{160}(00,04)
% \centering
% \LARGE Principio de indiferencia \\
% \Large Primer \textbf{acuerdo intersubjetivo} en contextos de incertidumbre
% \end{textblock}
% \vspace{1.5cm}
%
% \begin{textblock}{160}(0,28)  \centering
%  \scalebox{1.1}{
% \tikz{ %
%          \node[factor, minimum size=1cm] (p1) {} ;
%          \node[factor, minimum size=1cm, xshift=1.5cm] (p2) {} ;
%          \node[factor, minimum size=1cm, xshift=3cm] (p3) {} ;
%
%
%          \node[const, above=of p1, yshift=-0.05cm] (np1) {\Large $1/3$};
%          \node[const, above=of p2, yshift=-0.05cm] (np2) {\Large $1/3$};
%          \node[const, above=of p3, yshift=-0.05cm] (np3) {\Large $1/3$};
%          }
% }
% \end{textblock}
%
% \begin{textblock}{140}(10,64)   \centering \Large
% Dividir las creencias en partes iguales \\ por los caminos del modelo causal
% \end{textblock}
%
% \end{frame}


\begin{frame}[plain]
\begin{textblock}{160}(0,4)
 \centering \Large ¿Cómo podemos dar continuidad a los acuerdos intersubjetivos?
 \end{textblock}
\vspace{1cm}


\begin{textblock}{160}(20,22)
\onslide<2->{Modelo causal} \\ \vspace{0.3cm}
 \tikz{
    \node[latent,] (r) {\includegraphics[width=0.06\textwidth]{../../auxiliar/static/regalo.png}} ;
    \node[const,left=of r] (nr) {\Large $r$} ;

    \onslide<2->{
    \node[latent, below=of r] (d) {\includegraphics[width=0.05\textwidth]{../../auxiliar/static/dedo.png}} ;
    \node[const, left=of d] (nd) {\Large $s$} ;

    \edge {r} {d};
    }
}
\end{textblock}

\only<1-2>{
\begin{textblock}{160}(65,33)
\scalebox{1.5}{
\tikz{
    \node[factor, minimum size=1cm] (p1) {} ;
    \node[factor, minimum size=1cm, xshift=1.5cm] (p2) {} ;
    \node[factor, minimum size=1cm, xshift=3cm] (p3) {} ;

    \node[const, above=of p1, yshift=.15cm] (fp1) {$1/3$};
    \node[const, above=of p2, yshift=.15cm] (fp2) {$1/3$};
    \node[const, above=of p3, yshift=.15cm] (fp3) {$1/3$};
    \node[const, below=of p2, yshift=-.10cm, xshift=0.3cm] (dedo) {};

    \node[invisible, xshift=4.75cm] (s-dist) {};
    \node[invisible, yshift=-1cm] (s-dist) {};
    \node[invisible, yshift=1.2cm] (s-dist) {};
    }
}
\end{textblock}
}

\only<3>{
\begin{textblock}{160}(65,33)
\scalebox{1.5}{
\tikz{ %

         \node[factor, minimum size=1cm] (p1) {} ;
         \node[det, minimum size=1cm, xshift=1.5cm] (p2) {\includegraphics[width=0.03\textwidth]{../../auxiliar/static/dedo.png}} ;
         \node[factor, minimum size=1cm, xshift=3cm] (p3) {} ;
%
%
         \node[const, above=of p1, yshift=.15cm] (fp1) {$?$};
         \node[const, above=of p2, yshift=.15cm] (fp2) {$0$};
         \node[const, above=of p3, yshift=.15cm] (fp3) {$?$};
         \node[const, below=of p2, yshift=-.10cm, xshift=0.3cm] (dedo) {};

%         \node[const, above=of p2, xshift=.8cm, yshift=.15cm] (fp3) {$66\%$};
%
         \node[invisible, xshift=4.75cm] (s-dist) {};
         \node[invisible, yshift=-1cm] (s-dist) {};
         \node[invisible, yshift=1.2cm] (s-dist) {};
%
%         \plate[color=red] {no} {(p1)} {}; %
%         \plate {si} {(p2)(p3)} {}; %

        }
}
\end{textblock}
}

\end{frame}

\begin{frame}[plain]
\begin{textblock}{160}(0,4)
 \centering \LARGE Principio de indiferencia
 \end{textblock}
\vspace{1cm}


\only<1-3>{
\begin{textblock}{160}(20,22)
Modelo causal \\ \vspace{0.3cm}
 \tikz{
    \node[latent,] (r) {\includegraphics[width=0.06\textwidth]{../../auxiliar/static/regalo.png}} ;
    \node[const,left=of r] (nr) {\Large $r$} ;

    \node[latent, below=of r] (d) {\includegraphics[width=0.05\textwidth]{../../auxiliar/static/dedo.png}} ;
    \node[const, left=of d] (nd) {\Large $s$} ;

    \edge {r} {d};
}
\end{textblock}
}


\only<1->{
\begin{textblock}{80}(60,18) \centering
\scalebox{1.1}{
\tikz{
\onslide<1->{
\node[latent, draw=white, yshift=0.6cm] (b0) {$ 1$};

\node[latent,below=of b0,yshift=0.6cm, xshift=-3cm] (r1) {$r_1$};
\node[latent,below=of b0,yshift=0.6cm] (r2) {$r_2$};
\node[latent,below=of b0,yshift=0.6cm, xshift=3cm] (r3) {$r_3$};

\node[latent, below=of r1, draw=white, yshift=0.6cm] (br1) {$\frac{1}{3}$};
\node[latent, below=of r2, draw=white, yshift=0.6cm] (br2) {$\frac{1}{3}$};
\node[latent, below=of r3, draw=white, yshift=0.6cm] (br3) {$\frac{1}{3}$};
}
\onslide<2->{
\node[latent,below=of br1,yshift=0.6cm, xshift=-0.7cm] (r1d2) {$s_2$};
\node[latent,below=of br1,yshift=0.6cm, xshift=0.7cm] (r1d3) {$s_3$};

\node[latent,below=of r1d2,yshift=0.6cm,draw=white] (br1d2) {$\frac{1}{3}\frac{1}{2}$};
\node[latent,below=of r1d3,yshift=0.6cm, draw=white] (br1d3) {$\frac{1}{3}\frac{1}{2}$};
}
\onslide<3->{
\node[latent,below=of br2,yshift=0.6cm, xshift=-0.7cm] (r2d1) {$s_1$};
\node[latent,below=of br2,yshift=0.6cm, xshift=0.7cm] (r2d3) {$s_3$};
\node[latent,below=of br3,yshift=0.6cm, xshift=-0.7cm] (r3d1) {$s_1$};
\node[latent,below=of br3,yshift=0.6cm, xshift=0.7cm] (r3d2) {$s_2$};

\node[latent,below=of r2d1,yshift=0.6cm, draw=white] (br2d1) {$\frac{1}{3}\frac{1}{2}$};
\node[latent,below=of r2d3,yshift=0.6cm,draw=white] (br2d3) {$\frac{1}{3}\frac{1}{2}$};
\node[latent,below=of r3d1,yshift=0.6cm, draw=white] (br3d1) {$\frac{1}{3}\frac{1}{2}$};
\node[latent,below=of r3d2,yshift=0.6cm,draw=white] (br3d2) {$\frac{1}{3}\frac{1}{2}$};
}
\onslide<1->{
\edge[-] {b0} {r1,r2,r3};
\edge[-] {r1} {br1};
\edge[-] {r2} {br2};
\edge[-] {r3} {br3};
}
\onslide<2->{
\edge[-] {br1} {r1d2,r1d3};
\edge[-] {r1d2} {br1d2};
\edge[-] {r1d3} {br1d3};
}
\onslide<3->{
\edge[-] {br2} {r2d1, r2d3};
\edge[-] {br3} {r3d1,r3d2};
\edge[-] {r2d1} {br2d1};
\edge[-] {r2d3} {br2d3};
\edge[-] {r3d1} {br3d1};
\edge[-] {r3d2} {br3d2};
}
}
}
\end{textblock}
}


\only<4->{
 \begin{textblock}{65}(0,22)
  \centering
  Creencia$(r,s|\text{Modelo})$ \\ \vspace{0.3cm}
 \begin{tabular}{c|c|c|c||c} \setlength\tabcolsep{0.4cm}
        & \, $r_1$ \, &  \, $r_2$ \, & \, $r_3$ \, & \\ \hline
  $s_1$  & \onslide<5->{$0$} & \onslide<6->{$1/6$} & \onslide<6->{$1/6$} & \\ \hline
  $s_2$  & \onslide<7->{$1/6$} & \onslide<7->{$0$} & \onslide<7->{$1/6$} &  \\ \hline
       $s_3$ & \onslide<8->{$1/6$} & \onslide<8->{$1/6$} & \onslide<8->{$0$} &  \\ \hline \hline
              & & &  & \\
\end{tabular}
\end{textblock}
}

\end{frame}

\begin{frame}[plain]
 \begin{textblock}{160}(0,4)
 \centering \LARGE \only<1>{Principio de indiferencia}\only<2->{Principio de integridad}
 \end{textblock}

\vspace{1cm}

 \begin{textblock}{160}(0,22)
  \centering
  Creencia$(r,s|\text{M})$ \\ \vspace{0.3cm}
 \begin{tabular}{c|c|c|c||c} \setlength\tabcolsep{0.4cm}
     $\phantom{\bm{s_2}}$   & \, $r_1$ \, &  \, $r_2$ \, & \, $r_3$ \, &  \phantom{\bm{$1/3$}} \\ \hline
  $\only<6>{\gray}s_1$ & $\only<6>{\gray}0$ & $\only<6>{\gray}1/6$ & $\only<6>{\gray}1/6$ & \onslide<5->{$\only<6>{\gray}1/3$} \\ \hline
  $\only<6>{\bm}{s_2}$ & $1/6$ & $0$ & $1/6$ & \onslide<5->{$1/3$} \\ \hline
  $\only<6>{\gray}s_3$ & $\only<6>{\gray}1/6$ & $\only<6>{\gray}1/6$ & $\only<6>{\gray}0$ & \onslide<5->{$\only<6>{\gray}1/3$} \\ \hline \hline
        & \onslide<3->{$\only<6>{\gray}1/3$} & \onslide<3->{$\only<6>{\gray}1/3$} & \onslide<3->{$\only<6>{\gray}1/3$} &  \\
\end{tabular}

\vspace{0.3cm}

\onslide<2->{
\begin{align*}
 \text{Creencia}(r|\text{M}) = \sum_s \text{Creencia}(r,s|\text{M})
\end{align*}
}
\vspace{-0.5cm}
\onslide<4->{
\begin{align*}
 \text{Creencia}(s|\text{M}) = \sum_r \text{Creencia}(r,s|\text{M})
\end{align*}
}
\end{textblock}

\end{frame}


\begin{frame}[plain]
 \begin{textblock}{160}(0,4)
 \centering \LARGE Principio de coherencia \\[-0.1cm]
 \only<3->{\Large (Nueva creencia)}
 \end{textblock}

\vspace{1cm}

\only<1->{
 \begin{textblock}{160}(0,22)
  \centering
  \only<1-2>{Creencia$(r,s=2|\text{M})$}\only<3->{Creencia$(r|s=2,\text{M})$} \\ \vspace{0.3cm}
 \begin{tabular}{c|c|c|c||c} \setlength\tabcolsep{0.4cm}
        $\phantom{\bm{s_2}}$ & \, $r_1$ \, &  \, $r_2$ \, & \, $r_3$ \, &  \phantom{\bm{$1/3$}} \\ \hline
  &  &  &  & \\ \hline
  $\bm{s_2}$ & \only<1-2>{$1/6$}\only<3>{$\frac{1}{6}/\frac{1}{3}$}\only<4->{$1/2$} & $0$ & \only<1-2>{$1/6$}\only<3>{$\frac{1}{6}/\frac{1}{3}$}\only<4->{$1/2$} & \only<1>{$1/3$}\only<2>{{$\bm{1/3}$}}\only<3>{$\frac{1}{3}/\frac{1}{3}$}\only<4->{$1$} \\ \hline
  &  &  & &  \\
\end{tabular}
\end{textblock}
}

\only<1>{
\begin{textblock}{160}(0,58)
\begin{equation*}
\ \phantom{\underbrace{\text{Creencia}(r|s=2,\text{M})}_{\text{Nueva creencia}} =} \hfrac{\overbrace{\text{Creencia}(r, s=2|\text{M})}^{\text{Creencia que sobrevive}}}{\phantom{\underbrace{\text{Creencia}(s_2|\text{M})}_{\text{Creencia total que sobrevive}}}}
\end{equation*}
\end{textblock}
}

\only<2>{
\begin{textblock}{160}(0,58)
\begin{equation*}
\ \phantom{\underbrace{\text{Creencia}(r|s=2,\text{M})}_{\text{Nueva creencia}} =}\hfrac{\overbrace{\text{Creencia}(r, s=2|\text{M})}^{\text{Creencia que sobrevive}}}{\underbrace{\text{Creencia}(s_2|\text{M})}_{\text{Creencia total que sobrevive}}}
\end{equation*}
\end{textblock}
}


\only<3-4>{
\begin{textblock}{160}(0,58)
\begin{equation*}
\underbrace{\text{Creencia}(r|s=2,\text{M})}_{\text{Nueva creencia}} = \frac{\overbrace{\text{Creencia}(r, s=2|\text{M})}^{\text{Creencia que sobrevive}}}{\underbrace{\text{Creencia}(s_2|\text{M})}_{\text{Creencia total que sobrevive}}}
\end{equation*}
\end{textblock}
}


\only<5->{
\begin{textblock}{160}(7,57)
\centering
\scalebox{1.2}{
\tikz{ %

         \node[factor, minimum size=1cm] (p1) {} ;
         \node[det, minimum size=1cm, xshift=1.5cm] (p2) {\includegraphics[width=0.03\textwidth]{../../auxiliar/static/dedo.png}} ;
         \node[factor, minimum size=1cm, xshift=3cm] (p3) {} ;

         \node[const, above=of p1, yshift=.15cm] (fp1) {$1/2$};
         \node[const, above=of p2, yshift=.15cm] (fp2) {$0$};
         \node[const, above=of p3, yshift=.15cm] (fp3) {$1/2$};
         \node[const, below=of p2, yshift=-.10cm, xshift=0.3cm] (dedo) {};

         \node[invisible, xshift=4.75cm] (s-dist) {};
         \node[invisible, yshift=-1cm] (s-dist) {};
         \node[invisible, yshift=1.2cm] (s-dist) {};

        }
}
\end{textblock}
}

\end{frame}

\begin{frame}[plain]
\begin{textblock}{160}(0,4)
\centering \LARGE Principio de coherencia \\ \Large preservación de los acuerdos intersubjetivos
\end{textblock}
\vspace{1.25cm}

\only<2>{
 \begin{center}
 El principio de coherencia preserva la creencia \\
  previa que sigue siendo compatible con \\
  la evidencia empírica y formal (datos y modelo)
 \end{center}
}

\end{frame}


\begin{frame}[plain]
\begin{textblock}{160}(0,4)
\centering \LARGE  Las reglas de la probabilidad
\end{textblock}

\vspace{0.75cm}



\begin{equation*}
  \text{Marginal}_{i} = \sum_j \text{Conjunta}_{ij}  \ \ \ \ \ \ \ \ \ \ \ \  \ \ \ \ \  \text{Condicional}_{j|i} = \frac{\text{Conjunta}_{ij}}{\text{Marginal}_{i}}
\end{equation*}

\vspace{0.75cm}


\begin{columns}[t]
\begin{column}{0.5\textwidth}
 \centering \textbf{Regla de la suma}


\begin{equation*}
 P(r) = \sum_j P(r,s_j)
\end{equation*}



 \footnotesize
 No perdemos creencia al distribuirla \\
 Si sumamos, la recuperamos.

 \end{column}
 \begin{column}{0.5\textwidth}
\centering  \textbf{Regla del producto}

\begin{equation*}
 P(r|s)  = \frac{P(r,s)}{P(s)}
\end{equation*}

\vspace{0.1cm}

\footnotesize
Preservamos la creencia previa que \\
sigue siendo compatible con el dato

\end{column}
\end{columns}

\end{frame}


\begin{frame}[plain] \centering
\begin{textblock}{160}(00,04)
\centering \LARGE La misma conclusión de siempre \\
\end{textblock}
\vspace{1.2cm} \centering \large

Cox (1946) \emph{Probability, Frequency and Reasonable Expectation}\\[0.1cm]
Las consecuencias del ``sentido común''

\vspace{0.8cm}

Kolmogorov (1933) \emph{Foundations of the Theory of Probability} \\[0.1cm]
Como medida de conjuntos

\vspace{0.8cm}

Ramsey (1926) \emph{Truth and Probability} \\[0.1cm]
Los pagos que aceptarías en una apuesta

\vspace{0.4cm}

% \begin{framed}
% Las reglas de la probabilidad pueden derivarse a partir de sistemas axiomáticos conceptualmente muy distintos, lo cual es uno de los punto fuerte a su favor.
% \end{framed}

\end{frame}


%
% \begin{frame}[plain]
% \begin{textblock}{160}(00,04)
% \centering
% \LARGE Principio de reciprocidad \\
% \large Cartas entre Pascal y Fermat, 1648
% \end{textblock}
% \vspace{1.5cm}
%
%
%
% Tiramos dos veces una moneda: \\
% $\circ$ Si sale seca (S) en la primera y en la segunda, te hago un favor (Color rojo). \\
% $\circ$ Caso contrario (C), me hacés un favor (Color negro). \\
% \begin{center}
% \tikz{
% \node[latent, draw=white, yshift=0.7cm, minimum size=0.1cm] (b0) {};
% \node[latent,below=of b0,yshift=0.7cm, xshift=-1cm] (r1) {$S$};
% \node[latent,below=of b0,yshift=0.7cm, xshift=1cm] (r2) {$C$};
%
% \node[latent, below=of r1, draw=white, yshift=0.8cm, minimum size=0.1cm] (bc11) {};
% \node[accion, below=of r2, draw=white, yshift=0cm] (bc12) {};
% \node[latent,below=of bc11,yshift=0.8cm, xshift=-0.5cm] (r1d2) {$S$};
% \node[latent,below=of bc11,yshift=0.8cm, xshift=0.5cm] (r1d3) {$C$};
%
% \node[accion,below=of r1d2,yshift=0cm, color=red] (br1d2) {};
% \node[accion,below=of r1d3,yshift=0cm] (br1d3) {};
% \edge[-] {b0} {r1,r2};
% \edge[-] {r1} {bc11};
% \edge[-] {r2} {bc12};
% \edge[-] {bc11} {r1d2,r1d3};
% \edge[-] {r1d2} {br1d2};
% \edge[-] {r1d3} {br1d3};
% }
% \end{center}
%
%
% \centering
% ¿Cuál es el valor justo de la reciprocidad en este contexto de incertidumbre?
%
% \pause \justify
%
% $\bullet$ En la práctica vimos que el \emph{inverso de la probabilidad} garantiza la coexistencia.
% $\bullet$ Determinamos la probabilidad de negro como la suma de los dos elementos.
%
% \end{frame}
%

% \begin{frame}[plain] \centering
% \begin{textblock}{160}(00,04)
% \centering \LARGE Axiomas de Kolmogorov
% \end{textblock}
% \vspace{1.5cm}
%
% \includegraphics[page={2},width=0.8\textwidth]{../../auxiliar/static/kolmogorov.png}
%
% \end{frame}
%
% \begin{frame}[plain] \centering
% \begin{textblock}{160}(00,04)
% \centering \LARGE Axiomas de Kolmogorov \\
% \Large Regla de la suma
% \end{textblock}
% \vspace{1.5cm}
%
%
% \only<1>{
% \begin{textblock}{160}(00,24)
% \begin{equation*}
% P(A) + P(\neg A) \overset{5}{=} P(A + \neg A) \overset{1}{=} P(E) \overset{4}{=} 1
% \end{equation*}
% \end{textblock}
% }
%
% \only<2->{
% \begin{textblock}{160}(00,24)
% \begin{equation*}
% P(A) = 1 - P(\neg A)
% \end{equation*}
% \end{textblock}
% }
%
%
% \only<3->{
% \begin{textblock}{140}(10, 44) \justify
% Dado que $A = A \cap (B + \neg B)$
% \begin{equation*}
% P(A) \overset{5}{=} P(A \cap B) + P(A \cap \neg B)
% \end{equation*}
% \end{textblock}
% }
%
% \end{frame}
%
%
% \begin{frame}[plain] \centering
% \begin{textblock}{160}(00,04)
% \centering \LARGE La definición de Kolmogorov \\
% \Large Probabilidad condicional (o regla del producto)
% \end{textblock}
% \vspace{1.5cm}
%
% \begin{textblock}{140}(10, 24) \justify
% Si $P(A) > 0$
% \begin{equation*}
% P(B|A) = \frac{P(A \cap B) }{P(A)}
% \end{equation*}
% \end{textblock}
%
% \only<2->{
% \begin{textblock}{140}(10, 48) \justify
% Se puede mostrar que $P(\cdot|A)$ es una probabilidad \\[0.1cm]
% \ $\bullet$ Axioma 3 $P(B|A) > 0$ \\
% \ $\bullet$ Axioma 4 $P(E|A) = 1$ \\
% \ $\bullet$ Axioma 5 $P(B + C |A) = P(B|A) + P(C|A)$ con $B \cap C = \emptyset$
% \end{textblock}
% }
%
% \end{frame}


\begin{frame}[plain]
\begin{textblock}{160}(0,4)
 \centering \LARGE
 Modelo: Monty Hall
 \end{textblock}
 \vspace{-1cm}

 \only<1>{
 \begin{textblock}{80}(0,22)
 \centering
 \includegraphics[width=0.8\textwidth]{figuras/montyHall_model_0.pdf}
 \end{textblock}
 }

 \only<2-3>{
 \begin{textblock}{80}(0,22)
 \centering
 \includegraphics[width=0.8\textwidth]{figuras/montyHall_model_1.pdf}
 \end{textblock}
 }

\only<4->{
 \begin{textblock}{80}(0,22)
 \centering
 \includegraphics[width=0.8\textwidth]{figuras/montyHall_model_2.pdf}
 \end{textblock}
 }


\only<1-2>{
 \begin{textblock}{80}(70,30)
 \centering
\includegraphics[width=1\textwidth]{figuras/montyHall_2}
 \end{textblock}
}

\only<3-4>{
 \begin{textblock}{80}(70,30)
 \centering
\includegraphics[width=1\textwidth]{figuras/montyHall_6}
 \end{textblock}
}

\only<5>{
 \begin{textblock}{80}(70,30)
 \centering
\includegraphics[width=1\textwidth]{figuras/montyHall_7}
 \end{textblock}
}

\end{frame}


\begin{frame}[plain]
\begin{textblock}{160}(0,4)
 \centering \LARGE Modelo: Monty Hall
 \end{textblock}
 \vspace{-1cm}

 \only<1-3>{
 \begin{textblock}{80}(0,21)
 \centering
 \includegraphics[width=0.8\textwidth]{figuras/montyHall_model.pdf}
 \end{textblock}
 }

  \only<4-12>{
 \begin{textblock}{80}(0,22)
  \centering
  $P(r,s)$ \\ \vspace{0.3cm}
 \begin{tabular}{c|c|c|c||c} \setlength\tabcolsep{0.4cm}
        & \, $r_1$ \, &  \, $r_2$ \, & \, $r_3$ \, & \\ \hline
  { $s_2$}  & \onslide<5->{$1/6$} & \onslide<7->{$0$} & \onslide<9->{$1/3$} & \onslide<12->{$1/2$} \\ \hline
       {$s_3$} & \onslide<6->{$1/6$} & \onslide<8->{$1/3$} & \onslide<10->{$0$} & \onslide<12->{$1/2$} \\ \hline
              & \onslide<12->{$1/3$} &  \onslide<12->{$1/3$} & \onslide<12->{$1/3$}  & \onslide<12->{$1$} \\
\end{tabular}
\end{textblock}
}

\only<13>{
 \begin{textblock}{80}(0,22)
  \centering
  $P(r,s_2)$ \\ \vspace{0.3cm}
 \begin{tabular}{c|c|c|c||c} \setlength\tabcolsep{0.4cm}
        & \, $r_1$ \, &  \, $r_2$ \, & \, $r_3$ \, & \\ \hline
        { $s_2$}  & \onslide<6->{$1/6$} & \onslide<8->{$0$} & \onslide<10->{$1/3$} & \onslide<13->{$1/2$} \\ \hline
\end{tabular}
\end{textblock}
}


\only<14->{
 \begin{textblock}{80}(0,22)
  \centering
  $P(r|s_2)$ \\ \vspace{0.3cm}
 \begin{tabular}{c|c|c|c||c} \setlength\tabcolsep{0.4cm}
        & \, $r_1$ \, &  \, $r_2$ \, & \, $r_3$ \, & \phantom{$1/2$}\\ \hline
  { $s_2$}  & \onslide<6->{$1/3$} & \onslide<8->{$0$} & \onslide<10->{$2/3$} & \onslide<13->{$1$} \\ \hline
\end{tabular}
\end{textblock}
}


\only<11-12>{
\begin{textblock}{80}(0,58)
 \centering
\begin{center}
 Regla de la suma
 \end{center}

 $P(s_i) = \sum_{j} P(r_j,s_i)$
 \\

\end{textblock}
}

\only<13-14>{
\begin{textblock}{80}(0,58)
 \centering
\begin{center}
 Regla del producto
 \end{center}
 \begin{equation*}
P(r_i|s_2) = \frac{P(r_i,s_2)}{P(s_2)}
 \end{equation*}

\end{textblock}
}


\only<15>{
\begin{textblock}{80}(0,53)
\centering
\includegraphics[width=0.8\textwidth]{figuras/montyHall_8}
 \end{textblock}
}

 \only<2-12>{
\begin{textblock}{80}(70,14) \centering
\scalebox{1.2}{
 \tikz{
 \onslide<2->{
\node[latent, draw=white, yshift=0.8cm] (b0) {$1$};
\node[latent,below=of b0,yshift=0.8cm, xshift=-2cm] (r1) {$r_1$};
{\node[latent,below=of b0,yshift=0.8cm] (r2) {$r_2$}; }
\node[latent,below=of b0,yshift=0.8cm, xshift=2cm] (r3) {$r_3$};
\node[latent, below=of r1, draw=white, yshift=0.7cm] (bc11) {$\frac{1}{3}$};
{\node[latent, below=of r2, draw=white, yshift=0.7cm] (bc12) {$\frac{1}{3}$};}
\node[latent, below=of r3, draw=white, yshift=0.7cm] (bc13) {$\frac{1}{3}$};
}
\onslide<3->{
\node[latent,below=of bc11,yshift=0.7cm, xshift=-0.5cm] (r1d2) {$s_2$};
{\node[latent,below=of bc11,yshift=0.7cm, xshift=0.5cm] (r1d3) {$s_3$};}
{\node[latent,below=of bc12,yshift=0.7cm] (r2d3) {$s_3$};}
\node[latent,below=of bc13,yshift=0.7cm] (r3d2) {$s_2$};
\node[latent,below=of r1d2,yshift=0.7cm,draw=white] (br1d2) {$\only<5>{\bm}{\frac{1}{3}\frac{1}{2}}$};
{\node[latent,below=of r1d3,yshift=0.7cm, draw=white] (br1d3) {$\only<6>{\bm}{\frac{1}{3}\frac{1}{2}}$};}
{\node[latent,below=of r2d3,yshift=0.7cm,draw=white] (br2d3) {$\only<8>{\bm}{\frac{1}{3}}$};}
\node[latent,below=of r3d2,yshift=0.7cm,draw=white] (br3d2) {$\only<9>{\bm}{\frac{1}{3}}$};
}

\node[invisible, left=of r1d2,xshift=-0.1cm] (il) {};
\node[invisible, right=of br3d2,xshift=0.1cm] (il) {};

\onslide<2->{
\edge[-] {b0} {r1,r2,r3};
\edge[-] {r1} {bc11};
\edge[-] {r2} {bc12};
\edge[-] {r3} {bc13};
}
\onslide<3->{
\edge[-] {bc11} {r1d2,r1d3};
\edge[-] {bc12} {r2d3};
\edge[-] {bc13} {r3d2};
\edge[-] {r1d2} {br1d2};
\edge[-] {r1d3} {br1d3};
\edge[-] {r2d3} {br2d3};
\edge[-] {r3d2} {br3d2};
}
}
}
\end{textblock}
}


\only<13->{
\begin{textblock}{80}(70,14) \centering
\scalebox{1.2}{
 \tikz{
\node[latent, draw=white, yshift=0.8cm] (b0) {$1$};
\node[latent,below=of b0,yshift=0.8cm, xshift=-2cm] (r1) {$r_1$};
{\color{gray}\node[latent,draw=gray,below=of b0,yshift=0.8cm] (r2) {$r_2$}; }
\node[latent,below=of b0,yshift=0.8cm, xshift=2cm] (r3) {$r_3$};

% \node[latent, below=of r1, draw=white, yshift=0.8cm] (br1) {$\frac{1}{3}$};
% \node[latent, below=of r2, draw=white, yshift=0.8cm] (br2) {$\frac{1}{3}$};
% \node[latent, below=of r3, draw=white, yshift=0.8cm] (br3) {$\frac{1}{3}$};
% \node[latent,below=of br1,yshift=0.8cm] (c11) {$c_1$};
% \node[latent,below=of br2,yshift=0.8cm] (c12) {$c_1$};
% \node[latent,below=of br3,yshift=0.8cm] (c13) {$c_1$};

\node[latent, below=of r1, draw=white, yshift=0.7cm] (bc11) {$\frac{1}{3}$};
{\color{gray}\node[latent, below=of r2, draw=white, yshift=0.7cm] (bc12) {$\frac{1}{3}$};}
\node[latent, below=of r3, draw=white, yshift=0.7cm] (bc13) {$\frac{1}{3}$};
\node[latent,below=of bc11,yshift=0.7cm, xshift=-0.5cm] (r1d2) {$s_2$};
{\color{gray}\node[latent,draw=gray,below=of bc11,yshift=0.7cm, xshift=0.5cm] (r1d3) {$s_3$};}
{\color{gray}\node[latent, draw=gray,below=of bc12,yshift=0.7cm] (r2d3) {$s_3$};}
\node[latent,below=of bc13,yshift=0.7cm] (r3d2) {$s_2$};

\node[latent,below=of r1d2,yshift=0.7cm,draw=white] (br1d2) {$\frac{1}{3}\frac{1}{2}$};
{\color{gray}\node[latent,below=of r1d3,yshift=0.7cm, draw=white] (br1d3) {$\frac{1}{3}\frac{1}{2}$};}
{\color{gray}\node[latent,below=of r2d3,yshift=0.7cm,draw=white] (br2d3) {$\frac{1}{3}$};}
\node[latent,below=of r3d2,yshift=0.7cm,draw=white] (br3d2) {$\frac{1}{3}$};
\edge[-] {b0} {r1,r3};
\edge[-,draw=gray] {b0} {r2};
% \edge[-] {r1} {br1};
% \edge[-] {r2} {br2};
% \edge[-] {r3} {br3};
% \edge[-] {br1} {c11};
% \edge[-] {br2} {c12};
% \edge[-] {br3} {c13};
\edge[-] {r1} {bc11};
\edge[-,draw=gray] {r2} {bc12};
\edge[-] {r3} {bc13};
\edge[-] {bc11} {r1d2};
\edge[-,draw=gray] {bc11} {r1d3};
\edge[-,draw=gray] {bc12} {r2d3};
\edge[-] {bc13} {r3d2};
\edge[-] {r1d2} {br1d2};
\edge[-,draw=gray] {r1d3} {br1d3};
\edge[-,draw=gray] {r2d3} {br2d3};
\edge[-] {r3d2} {br3d2};
}
}
\end{textblock}
}

\end{frame}

\begin{frame}[plain]
\begin{textblock}{160}(0,4)
\centering \LARGE Teorema de Bayes \\
\Large {\gray El corolario de Laplace}
\end{textblock}

\only<1-4>{
\begin{textblock}{160}(0,34)
 \begin{align*}
  \phantom{P(r_i)} P(r_i|s_2) & = \only<-3>{\frac{\bm{P(r_i, s_2)}}{P(s_2)}} \only<4>{\frac{P(s_2|r_i)P(r_i)}{P(s_2)}}
  \only<2>{\\ P(s_2 | r_i) &= \frac{P(r_i, s_2)}{P(r_i)}}
  \only<3>{\\[0.27cm]P(r_i) P(s_2 | r_i)  &= P(r_i, s_2)}
 \end{align*}
\end{textblock}
}

\only<5>{
\begin{textblock}{160}(0,43)
\begin{equation*}
P(\text{Hip\'otesis}_i |\,\text{Datos}) = \frac{P(\text{Datos}\,|\,\text{Hip\'otesis}_i) \, P(\text{Hip\'otesis}_i)}{P(\text{Datos})}
\end{equation*}
\end{textblock}
}


\only<6>{
\begin{textblock}{160}(0,37)
\begin{equation*}
\underbrace{P(\text{Hip\'otesis}_i|\,\text{Datos})}_{\text{\scriptsize Posteriori}} = \frac{\overbrace{P(\text{Datos}\,|\,\text{Hip\'otesis}_i)}^{\text{\scriptsize Verosimilitud}} \overbrace{P(\text{Hip\'otesis}_i)}^{\text{\scriptsize Priori}} }{\underbrace{P(\text{Datos})}_{\text{\scriptsize Evidencia}}}
\end{equation*}
\end{textblock}
}

\vspace{0.2cm}

\only<7->{
%\vspace{0.3cm}
\Wider[2cm]{
\begin{textblock}{160}(0,34.25)
\begin{equation*}
\underbrace{P(\text{Hip\'otesis}_i|\,\text{Datos, Modelo})}_{\text{\scriptsize Posteriori}} = \frac{\overbrace{P(\text{Datos}\,|\,\text{Hip\'otesis$_i$, Modelo})}^{\text{\scriptsize Verosimilitud}} \overbrace{P(\text{Hip\'otesis}_i|\text{ Modelo})}^{\text{\scriptsize Priori}} }{\underbrace{P(\text{Datos }|\text{ Modelo})}_{\text{\scriptsize Evidencia}}}
\end{equation*}
\end{textblock}
}
}

\only<8->{
\begin{textblock}{100}(30,65)
\centering \vspace{0.05cm}
El \textbf{modelo} es lo que permite relacionar

los \textbf{datos} con nuestras \textbf{hipótesis}!
\vspace{0.1cm}
\end{textblock}
}





\end{frame}

%
% \begin{frame}[plain]
% \begin{textblock}{160}(0,4)
% \centering \LARGE El teorema de Bayes \\ \Large como generalización de la hipótesis indicadora
% \end{textblock}
% \vspace{0.75cm}
%
% \only<2>{
%  \begin{center} \Large
%  Principio de validez universal para determinar las creencias \\ honestas dadas las evidencias empíricas y formales
%  \end{center}
% }
%
% \end{frame}
%
%




\begin{frame}[plain]
\begin{textblock}{160}(0,4)
 \centering \LARGE
 \en{Likelihood}
 \es{Verosimilutd: predicción del dato dada la hipótesis}
 \end{textblock}
 \vspace{-1cm}


\begin{textblock}{80}(70,11) \centering
\scalebox{1.2}{
\tikz{
\only<3->{\phantom}{\node[latent, draw=white, yshift=0.8cm] (b0) {$1$};}
\only<8->{\phantom}{\node[latent,below=of b0,yshift=0.8cm, xshift=-2cm] (r1) {$r_1$};}
\only<3-7,10-11>{\phantom}{\node[latent,below=of b0,yshift=0.8cm] (r2) {$r_2$};}
\only<3-9>{\phantom}{\node[latent,below=of b0,yshift=0.8cm, xshift=2cm] (r3) {$r_3$};}

\only<8->{\phantom}{\node[latent, below=of r1, draw=white, yshift=0.8cm] (br1) {$\frac{1}{3}$};}
\only<3-7,10-11>{\phantom}{\node[latent, below=of r2, draw=white, yshift=0.8cm] (br2) {$\frac{1}{3}$};}
\only<3-9>{\phantom}{\node[latent, below=of r3, draw=white, yshift=0.8cm] (br3) {$\frac{1}{3}$};}
\only<8->{\phantom}{\node[latent,below=of br1,yshift=0.8cm] (c11) {$c_1$};}
\only<3-7,10-11>{\phantom}{\node[latent,below=of br2,yshift=0.8cm] (c12) {$c_1$};}
\only<3-9>{\phantom}{\node[latent,below=of br3,yshift=0.8cm] (c13) {$c_1$};}

\only<8->{\phantom}{\node[latent, below=of c11, draw=white, yshift=0.8cm] (bc11) {$\frac{1}{3}$};}
\only<3-7,10-11>{\phantom}{\node[latent, below=of c12, draw=white, yshift=0.8cm] (bc12) {$\frac{1}{3}$};}
\only<3-9>{\phantom}{\node[latent, below=of c13, draw=white, yshift=0.8cm] (bc13) {$\frac{1}{3}$};}
\only<8->{\phantom}{\node[latent,below=of bc11,yshift=0.8cm, xshift=-0.7cm] (r1d2) {$s_2$};}
\only<8->{\phantom}{\node[latent,below=of bc11,yshift=0.8cm, xshift=0.7cm] (r1d3) {$s_3$};}
\only<3-7,10-11>{\phantom}{\node[latent,below=of bc12,yshift=0.8cm] (r2d3) {$s_3$};}
\only<3-9>{\phantom}{\node[latent,below=of bc13,yshift=0.8cm] (r3d2) {$s_2$};}

\only<8->{\phantom}{\node[latent,below=of r1d2,yshift=0.8cm,draw=white] (br1d2) {$\frac{1}{3}\frac{1}{2}$};}
\only<8->{\phantom}{\node[latent,below=of r1d3,yshift=0.8cm, draw=white] (br1d3) {$\frac{1}{3}\frac{1}{2}$};}
\only<3-7,10-11>{\phantom}{\node[latent,below=of r2d3,yshift=0.8cm,draw=white] (br2d3) {$\frac{1}{3}$};}
\only<3-9>{\phantom}{\node[latent,below=of r3d2,yshift=0.8cm,draw=white] (br3d2) {$\frac{1}{3}$};}

\only<3->{\phantom}{\edge[-] {b0} {r1};}
\only<3->{\phantom}{\edge[-] {b0} {r2};}
\only<3->{\phantom}{\edge[-] {b0} {r3};}
\only<8->{\phantom}{\edge[-] {r1} {br1};}
\only<3-7,10-11>{\phantom}{\edge[-] {r2} {br2};}
\only<3-9>{\phantom}{\edge[-] {r3} {br3};}
\only<8->{\phantom}{\edge[-] {br1} {c11};}
\only<3-7,10-11>{\phantom}{\edge[-] {br2} {c12};}
\only<3-9>{\phantom}{\edge[-] {br3} {c13};}
\only<8->{\phantom}{\edge[-] {c11} {bc11};}
\only<3-7,10-11>{\phantom}{\edge[-] {c12} {bc12};}
\only<3-9>{\phantom}{\edge[-] {c13} {bc13};}
\only<8->{\phantom}{\edge[-] {bc11} {r1d2,r1d3};}
\only<3-7,10-11>{\phantom}{\edge[-] {bc12} {r2d3};}
\only<3-9>{\phantom}{\edge[-] {bc13} {r3d2};}
\only<8->{\phantom}{\edge[-] {r1d2} {br1d2};}
\only<8->{\phantom}{\edge[-] {r1d3} {br1d3};}
\only<3-7,10-11>{\phantom}{\edge[-] {r2d3} {br2d3};}
\only<3-9>{\phantom}{\edge[-] {r3d2} {br3d2};}
}
}
\end{textblock}


\only<1->{
 \begin{textblock}{80}(0,16)
  \centering
  $P(s_2|r_i)$ \\ \vspace{0.1cm}
  \onslide<2->{
  \begin{tabular}{c|c|c|c} \setlength\tabcolsep{0.4cm}
          & \, \only<3-7>{\bm}{$r_1$} \, &  \, \only<8-9>{\bm}{$r_2$} \, & \, \only<10-11>{\bm}{$r_3$} \, \\ \hline
   $s_2$ & \onslide<7->{$1/2$} & \onslide<9->{$0$} & \onslide<11->{$1$}  \\ \hline
\end{tabular}
}
\end{textblock}
}

\only<4->{
\begin{textblock}{80}(0,40)
 \begin{equation*}
  P(s|r_{\only<4-7>{1}\only<8-9>{2}\only<10-11>{3}}) = \frac{P(r_{\only<4-7>{1}\only<8-9>{2}\only<10-11>{3}}, s)}{P(r_{\only<4-7>{1}\only<8-9>{2}\only<10-11>{3}})}
 \end{equation*}
\end{textblock}
}


 \only<5>{
 \begin{textblock}{80}(0,57)
  \centering
  $P(r,s)$\\ \vspace{0.1cm}
 \begin{tabular}{c|c|c|c||c} \setlength\tabcolsep{0.4cm}
          & \, $r_1$ \, &  \, $r_2$ \, & \, $r_3$ \, & \\ \hline
   $s_2$ & $1/6$ & $0$ & $1/3$     & $1/2$ \\ \hline
   $s_3$ & $1/6$ & $1/3$ & $0$     & $1/2$ \\ \hline \hline
         & $1/3$ &  $1/3$ & $1/3$  & $1$ \\
\end{tabular}
\end{textblock}
}

\only<6->{
\begin{textblock}{80}(0,57)
  \centering
  \only<6>{$P(r_1, s)$}\only<7>{$P(s|r_1)$}\only<8>{$P(r_2, s)$}\only<9>{$P(s|r_2)$}\only<10>{$P(r_3, s)$}\only<11>{$P(s|r_3)$} \\ \vspace{0.1cm}
 \begin{tabular}{c|c|c|c||c} \setlength\tabcolsep{0.4cm}
          & \, $r_1$ \, &  \, $r_2$ \, & \, $r_3$ \, &  \phantom{$1/2$} \\ \hline
   \only<7->{\bm}{$s_2$} & \only<6>{$1/6$}\only<7>{\bm{$1/2$}} & \only<8>{$0$}\only<9>{\bm{$0$}} & \only<10>{$1/3$}\only<11>{\bm{$1$}} &  \\ \hline
   $s_3$ & \only<6>{$1/6$}\only<7>{$1/2$} & \only<8>{$1/3$}\only<9>{$1$} & \only<10-11>{$0$}  &  \\ \hline \hline
         & \only<6>{$1/3$}\only<7>{$1$} & \only<8>{$1/3$}\only<9>{$1$} & \only<10>{$1/3$}\only<11>{$1$}  &  \\
\end{tabular}
\end{textblock}
}

\end{frame}

% \begin{frame}[plain]
% \begin{textblock}{160}(0,4)
%  \centering \LARGE
%  \en{Likelihood}
%  \es{Verosimilitud: los caminos del modelo causal}
%  \end{textblock}
%
%  \begin{textblock}{160}(0,18)
%   \centering
%   $P(s_2|r_i)$ \\ \vspace{0.1cm}
%   \begin{tabular}{c|c|c|c} \setlength\tabcolsep{0.4cm}
%           & \, $r_1$ \, &  \, $r_2$ \, & \, $r_3$ \, \\ \hline
%    $s_2$ & $1/2$ & $0$ & $1$  \\ \hline
% \end{tabular}
% \end{textblock}
%
% \only<2>{
%  \begin{textblock}{160}(0,49)
%  \begin{align*}
%   P(s_2|r_i, M) = \frac{\text{\textbf{Caminos que generan} $\bm{s_2}$ dada la hipótesis $r_i$ y el modelo}}{\text{\textbf{Caminos totales} dada la hipótesis $r_i$ y el modelo }}
%  \end{align*}
% \end{textblock}
% }
%
% \end{frame}





\begin{frame}[plain]
\begin{textblock}{160}(0,4)
 \centering \LARGE Posterior \\
 \Large La \textbf{sorpresa} como filtro de las creencias previas
 \end{textblock}

 \begin{textblock}{160}(0,18)
  \begin{align*}
   P(r_i|s_2) \propto \underbrace{P(r_i) P(s_2|r_i)}_{P(r_i, s_2)}
  \end{align*}
 \end{textblock}


 \only<2>{
 \begin{textblock}{160}(0,40)
  \centering
   \begin{tabular}{c|c|c|c} \setlength\tabcolsep{0.4cm}
  \phantom{$P(r_i|s_2) \propto$} & \, $r_1$ \, &  \, $r_2$ \, & \, $r_3$ \, \\ \hline
     $P(r_i)$ & $1/3$ & $1/3$ & $1/3$  \\ \hline
   \end{tabular}
\end{textblock}
}


\only<3>{
 \begin{textblock}{160}(0,40)
  \centering
   \begin{tabular}{c|c|c|c} \setlength\tabcolsep{0.4cm}
       \phantom{$P(r_i|s_2) \propto$} & \, $r_1$ \, &  \, $r_2$ \, & \, $r_3$ \, \\ \hline
   $P(r_i)$ & $1/3$ & $1/3$ & $1/3$  \\ \hline
   $P(s_2|r_i)$ & $1/2$ & $0$ & $1$  \\ \hline
   \end{tabular}
\end{textblock}
}


\only<4>{
 \begin{textblock}{160}(0,40)
  \centering
  \begin{tabular}{c|c|c|c} \setlength\tabcolsep{0.4cm}
        \phantom{$P(r_i|s_2) \propto$}  & \, $r_1$ \, &  \, $r_2$ \, & \, $r_3$ \, \\ \hline
   $P(r_i)$ & $1/3$ & $1/3$ & $1/3$  \\ \hline
   $P(s_2|r_i)$ & $1/2$ & $0$ & $1$  \\ \hline
   $P(r_i,s_2)$ & $1/6$ & $0$ & $1/3$  \\ \hline
\end{tabular}
\end{textblock}
}

\only<5>{
 \begin{textblock}{160}(0,40)
  \centering
  \begin{tabular}{c|c|c|c} \setlength\tabcolsep{0.4cm}
         \phantom{$P(r_i|s_2) \propto$} & \, $r_1$ \, &  \, $r_2$ \, & \, $r_3$ \, \\ \hline
   $P(r_i)$ & $1/3$ & $1/3$ & $1/3$  \\ \hline
   $P(s_2|r_i)$ & $1/2$ & $0$ & $1$  \\ \hline
   $P(r_i,s_2) $ & $1/6$ & $0$ & $1/3$  \\ \hline \hline
   $P(r_i|s_2) $ & $1/3$ & $0$ & $2/3$  \\ \hline
\end{tabular}
\end{textblock}
}


\end{frame}
%
% \begin{frame}[plain]
% \begin{textblock}{160}(0,4)
%  \centering \LARGE
%  \es{Posterior}
%  \end{textblock}
%  \vspace{1cm}
%
%  \begin{center}
%  $P(r_i|s_2) = $ \Large Creencia inicial no filtrada por la sorpresa
%  \end{center}
%
% \end{frame}


%
%
% \begin{frame}[plain]
% \begin{textblock}{160}(0,4)
%  \centering \LARGE Posterior \\
%  \Large La \textbf{sorpresa} como filtro de las creencias previas
%  \end{textblock}
%
%
%  \begin{textblock}{160}(0,22) \centering
%  Ejemplo de estimación de habilidad (TrueSkill)
%  \end{textblock}
%
%
%
% \begin{textblock}{160}(0,32)
% \centering
% \only<2>{\includegraphics[width=0.49\textwidth]{figuras/posterior_win}
% }
% \end{textblock}
%
% \begin{textblock}{160}(0,32)
% \centering
% \only<3>{\includegraphics[page=2,width=0.49\textwidth]{figuras/posterior_win}
% }
% \end{textblock}
%
% \begin{textblock}{160}(0,32)
% \centering
% \only<4>{\includegraphics[page=3,width=0.49\textwidth]{figuras/posterior_win}
% }
% \end{textblock}
%
% \begin{textblock}{160}(0,32)
% \centering
% \only<5>{\includegraphics[page=4,width=0.49\textwidth]{figuras/posterior_win}
% }
% \end{textblock}
%
%
% \begin{textblock}{160}(0,32)
% \centering
% \only<6>{\includegraphics[page=5,width=0.49\textwidth]{figuras/posterior_win}
% }
% \end{textblock}
%
%
%
% \end{frame}

\begin{frame}[plain]
\begin{textblock}{160}(0,4)
 \centering \LARGE Evidencia \\
 \Large Predicción a priori, con la contribución de todas las hipótesis.
 \end{textblock}
 \vspace{1cm}

  \begin{textblock}{160}(0,22)
  \begin{align*}
   P(r_i|s_2) = \frac{P(s_2|r_i)P(r_i)}{\underbrace{P(s_2)}_{\text{Evidencia}}}
  \end{align*}
 \end{textblock}



 \only<2->{
 \begin{textblock}{160}(0,50)
 \begin{equation*}
  P(s_2) = \sum_i P(s_2|r_i) P(r_i) \onslide<3->{= \frac{1}{2} \frac{1}{3} + 0 \frac{1}{3} + 1 \frac{1}{3} } \onslide<4>{= 1/2}
 \end{equation*}
 \end{textblock}
}
\end{frame}

\begin{frame}[plain]
\begin{textblock}{160}(0,4)
\centering \LARGE Modelos causales alternativos
\end{textblock}
 \vspace{1.25cm}

 \Large \centering

 ¿Y el acuerdo intersubjetivo respecto de los modelos?

 \pause

 \begin{equation*}
P(\text{Modelo}_i|\text{Datos})
 \end{equation*}

 \end{frame}


\begin{frame}[plain]
\begin{textblock}{160}(0,4)
\centering \LARGE Modelos causales alternativos
\end{textblock}
 \vspace{1.25cm}

\begin{textblock}{80}(80,18)
\centering
 \tikz{
    \node[latent,] (r) {\includegraphics[width=0.12\textwidth]{../../auxiliar/static/regalo.png}} ;
    \node[const,left=of r] (nr) {\Large $r$} ;


    \node[latent, below=of r] (d) {\includegraphics[width=0.10\textwidth]{../../auxiliar/static/dedo.png}} ;
    \node[const, left=of d] (nd) {\Large $s$} ;

    \edge {r} {d};

}

\vspace{0.75cm}
\onslide<1>{
\tikz{
         \node[factor, minimum size=1cm] (p1) {\includegraphics[width=0.07\textwidth]{../../auxiliar/static/cerradura.png}} ;
         \node[det, minimum size=1cm, xshift=1.5cm] (p2) {\includegraphics[width=0.07\textwidth]{../../auxiliar/static/dedo.png}} ;
         \node[factor, minimum size=1cm, xshift=3cm] (p3) {} ;

         \node[const, above=of p1, yshift=.15cm] (fp1) {$1/2$};
         \node[const, above=of p2, yshift=.15cm] (fp2) {$0$};
         \node[const, above=of p3, yshift=.15cm] (fp3) {$1/2$};
         \node[const, below=of p2, yshift=-.10cm, xshift=0.3cm] (dedo) {};

        }
}

\end{textblock}



\begin{textblock}{80}(0,18)
\centering
\tikz{

    \node[latent] (d) {\includegraphics[width=0.10\textwidth]{../../auxiliar/static/dedo.png}} ;
    \node[const,left=of d] (nd) {\Large $s$} ;

    \node[latent, above=of d, xshift=-1.5cm] (r) {\includegraphics[width=0.12\textwidth]{../../auxiliar/static/regalo.png}} ;
    \node[const,left=of r] (nr) {\Large $r$} ;


    \node[latent, fill=black!30, above=of d, xshift=1.5cm] (c) {\includegraphics[width=0.12\textwidth]{../../auxiliar/static/cerradura.png}} ;
    \node[const,left=of c] (nc) {\Large $c$} ;

    \edge {r,c} {d};
}

\vspace{0.75cm}
\onslide<1>{
\tikz{
         \node[factor, minimum size=1cm] (p1) {\includegraphics[width=0.07\textwidth]{../../auxiliar/static/cerradura.png}} ;
         \node[det, minimum size=1cm, xshift=1.5cm] (p2) {\includegraphics[width=0.07\textwidth]{../../auxiliar/static/dedo.png}} ;
         \node[factor, minimum size=1cm, xshift=3cm] (p3) {} ;

         \node[const, above=of p1, yshift=.15cm] (fp1) {$1/3$};
         \node[const, above=of p2, yshift=.15cm] (fp2) {$0$};
         \node[const, above=of p3, yshift=.15cm] (fp3) {$2/3$};
         \node[const, below=of p2, yshift=-.10cm, xshift=0.3cm] (dedo) {};

        }
}

\end{textblock}

\end{frame}



\begin{frame}[plain,fragile]
\begin{textblock}{160}(0,4)
\centering \LARGE Modelos causales alternativos
\end{textblock}
\vspace{1cm}

\only<-5>{
\begin{textblock}{160}(0,18)
\begin{equation*}
 P(\text{Modelo}_i|\text{Datos}) = \frac{\overbrace{P(\text{Datos}|\text{Modelo}_i)}^{\hfrac{\text{\footnotesize Predicción a priori}}{\text{\footnotesize o evidencia}} } P(\text{Modelo}_i)}{ P(\text{Datos})}
\end{equation*}
\end{textblock}
}

\only<2>{
\begin{textblock}{160}(0,44)
\begin{equation*}
P(\text{Hip\'otesis}_i|\,\text{Datos, Modelo}) = \frac{P(\text{Datos}\,|\,\text{Hip\'otesis$_i$, Modelo}) P(\text{Hip\'otesis}_i|\text{ Modelo})} {\underbrace{P(\text{Datos }|\text{ Modelo})}_{\hfrac{\text{\footnotesize Predicción a priori}}{\text{\footnotesize o evidencia}}} }
\end{equation*}
\end{textblock}
}



\only<3-4>{
\begin{textblock}{160}(0,42)
 \begin{equation*}
\begin{split}
 \frac{P(\text{Modelo}_A|\text{Datos})}{P(\text{Modelo}_B|\text{Datos})} = \frac{P(\text{Datos}|\text{Modelo}_A)} {P(\text{Datos}|\text{Modelo}_B)} \only<4>{\phantom}{\frac{P(\text{Modelo}_A)}{P(\text{Modelo}_B)}}
\end{split}
\end{equation*}
\end{textblock}
}

\only<5->{
\begin{textblock}{160}(0,38)
 \begin{equation*}
\begin{split}
P(\text{Dat\en{a}\es{os}}|\text{Model\es{o}}) & = P(d_1|\text{Model\es{o}})P(d_2|d_1,\text{Model\es{o}}) \dots
%\\ \onslide<6->{& = \left( \sum^{\text{Hipótesis}}_h P(d_1|h,\text{M}) P(h|\text{M}) \right) \left( \sum^{\text{Hipótesis}}_h P(d_2|d_1,h,\text{M}) P(h|d_1,\text{M}) \right)  \dots \\}%\onslide<3->{& =  \prod_i^{|\text{Datos}|} \sum_h P(d_i|d_1, \dots, d_{i-1}, h, \text{M}) P(h|d_1, \dots, d_{i-1},\text{M}) }
\end{split}
\end{equation*}
\end{textblock}
}


\only<6,10>{
\begin{textblock}{80}(60,22)
\tikz{
    \node[factor, minimum size=1cm] (p1) {\includegraphics[width=0.07\textwidth]{../../auxiliar/static/cerradura.png}} ;
    \node[factor, minimum size=1cm, xshift=1.5cm] (p2) {} ;
    \node[factor, minimum size=1cm, xshift=3cm] (p3) {} ;
}
\end{textblock}
}
\only<7-8>{
\begin{textblock}{80}(60,22)
\tikz{
    \node[factor, minimum size=1cm] (p1) {\includegraphics[width=0.07\textwidth]{../../auxiliar/static/cerradura.png}} ;
    \node[det, minimum size=1cm, xshift=1.5cm] (p2) {\includegraphics[width=0.07\textwidth]{../../auxiliar/static/dedo.png}} ;
    \node[factor, minimum size=1cm, xshift=3cm] (p3) {} ;
}
\end{textblock}
}
\only<9>{
\begin{textblock}{80}(60,22)
\tikz{
    \node[det, minimum size=1cm] (p1) {\includegraphics[width=0.07\textwidth]{../../auxiliar/static/regalo.png}} ;
    \node[det, minimum size=1cm, xshift=1.5cm] (p2) {} ;
    \node[det, minimum size=1cm, xshift=3cm] (p3) {} ;
}
\end{textblock}
}
\only<11-12>{
\begin{textblock}{80}(60,22)
\tikz{
    \node[factor, minimum size=1cm] (p1) {\includegraphics[width=0.07\textwidth]{../../auxiliar/static/cerradura.png}} ;
    \node[factor, minimum size=1cm, xshift=1.5cm] (p2) {} ;
    \node[det, minimum size=1cm, xshift=3cm] (p3) {\includegraphics[width=0.07\textwidth]{../../auxiliar/static/dedo.png}} ;
}
\end{textblock}
}
\only<13>{
\begin{textblock}{80}(60,22)
\tikz{
    \node[det, minimum size=1cm] (p1) {} ;
    \node[det, minimum size=1cm, xshift=1.5cm] (p2) {\includegraphics[width=0.07\textwidth]{../../auxiliar/static/regalo.png}} ;
    \node[det, minimum size=1cm, xshift=3cm] (p3) {} ;
}
\end{textblock}
}


\only<6>{
\begin{textblock}{80}(0,52) \centering
\begin{tabular}{|c|c|c|c||c|} \hline  \setlength\tabcolsep{0.4cm}
\phantom{$\bm{s_2}$} & \, $r_1$ \, &  \, $r_2$ \, & \, $r_3$ \, & \phantom{$\bm{1/2}$} \\ \hline
  $s_1$ & $0$ & $0$ & $0$ &   $0$ \\ \hline
  $s_2$ & $1/6$ & $0$ & $1/3$ &  $1/2$ \\  \hline
  $s_3$ & $1/6$ & $1/3$ & $0$ & $1/2$ \\ \hline
  \end{tabular}
\end{textblock}
}
\only<7>{
\begin{textblock}{80}(0,52) \centering
\begin{tabular}{|c|c|c|c||c|} \hline  \setlength\tabcolsep{0.4cm}
\phantom{$\bm{s_2}$} & \, $r_1$ \, &  \, $r_2$ \, & \, $r_3$ \, &  \phantom{$\bm{1/2}$}  \\ \hline
  $\gray s_1$ & $\gray0$ & $\gray0$ & $\gray0$ &   $\gray 0$ \\ \hline
  $\bm{s_2}$ & $1/6$ & $0$ & $1/3$ &  $\bm{1/2}$ \\  \hline
  $\gray s_3$ & $\gray1/6$ & $\gray1/3$ & $\gray0$ & $\gray1/2$ \\ \hline
  \end{tabular}
\end{textblock}
}
\only<8>{
\begin{textblock}{80}(0,52) \centering
\begin{tabular}{|c|c|c|c||c|} \hline  \setlength\tabcolsep{0.4cm}
\phantom{$\bm{s_2}$} & \, $r_1$ \, &  \, $r_2$ \, & \, $r_3$ \, & \phantom{$\bm{1/2}$}  \\ \hline
            & & &  &  \\ \hline
  $s_2$ & $1/3$ & $0$ & $2/3$ &  1 \\  \hline
 & & & &\\ \hline
  \end{tabular}
\end{textblock}
}
\only<9>{
\begin{textblock}{80}(0,52) \centering
\begin{tabular}{|c|c|c|c||c|} \hline  \setlength\tabcolsep{0.4cm}
\phantom{$\bm{s_2}$} & \, $\bm{r_1}$ \, &  \, $r_2$ \, & \, $r_3$ \, & \phantom{$\bm{1/2}$}  \\ \hline
            & & &  &  \\ \hline
  $s_2$ & $\bm{1/3}$ & $0$ & $2/3$ &  1 \\  \hline
 & & & &\\ \hline
  \end{tabular}
\end{textblock}
}
\only<10>{
\begin{textblock}{80}(0,52) \centering
\begin{tabular}{|c|c|c|c||c|} \hline  \setlength\tabcolsep{0.4cm}
\phantom{$\bm{s_2}$} & \, $r_1$ \, &  \, $r_2$ \, & \, $r_3$ \, & \phantom{$\bm{1/2}$} \\ \hline
  $s_1$ & $0$ & $0$ & $0$ &   $0$ \\ \hline
  $s_2$ & $1/6$ & $0$ & $1/3$ &  $1/2$ \\  \hline
  $s_3$ & $1/6$ & $1/3$ & $0$ & $1/2$ \\ \hline
  \end{tabular}
\end{textblock}
}
\only<11>{
\begin{textblock}{80}(0,52) \centering
\begin{tabular}{|c|c|c|c||c|} \hline  \setlength\tabcolsep{0.4cm}
\phantom{$\bm{s_2}$} & \, $r_1$ \, &  \, $r_2$ \, & \, $r_3$ \, &  \phantom{$\bm{1/2}$}  \\ \hline
  $\gray s_1$ & $\gray0$ & $\gray0$ & $\gray0$ &   $\gray 0$ \\ \hline
  $\gray s_2$ & $\gray1/6$ & $\gray0$ & $\gray1/3$ &  $\gray1/2$ \\  \hline
  $\bm{s_3}$ & $1/6$ & $1/3$ & $0$ & $\bm{1/2}$ \\ \hline
  \end{tabular}
\end{textblock}
}
\only<12>{
\begin{textblock}{80}(0,52) \centering
\begin{tabular}{|c|c|c|c||c|} \hline  \setlength\tabcolsep{0.4cm}
\phantom{$\bm{s_2}$} & \, $r_1$ \, &  \, $r_2$ \, & \, $r_3$ \, & \phantom{$\bm{1/2}$}  \\ \hline
            & & &  &  \\ \hline
 & & & &\\ \hline
 $s_3$ & $1/3$ & $2/3$ & $0$ &  1 \\  \hline
  \end{tabular}
\end{textblock}
}
\only<13>{
\begin{textblock}{80}(0,52) \centering
\begin{tabular}{|c|c|c|c||c|} \hline  \setlength\tabcolsep{0.4cm}
\phantom{$\bm{s_2}$} & \, $r_1$ \, &  \, $\bm{r_2}$ \, & \, $r_3$ \, & \phantom{$\bm{1/2}$}  \\ \hline
            & & &  &  \\ \hline
 & & & &\\ \hline
 $s_3$ & $1/3$ & $\bm{2/3}$ & $0$ &  1 \\  \hline
  \end{tabular}
\end{textblock}
}
\only<6>{
\begin{textblock}{80}(80,52) \centering
\begin{tabular}{|c|c|c|c||c|} \hline  \setlength\tabcolsep{0.4cm}
\phantom{$\bm{s_2}$} & \, $r_1$ \, &  \, $r_2$ \, & \, $r_3$ \, & \phantom{$\bm{1/3}$}  \\ \hline
  $s_1$ & $0$ & $1/6$ & $1/6$ &   $1/3$ \\ \hline
  $s_2$ & $1/6$ & $0$ & $1/6$ &  $1/3$ \\  \hline
  $s_3$ & $1/6$ & $1/6$ & $0$ & $1/3$ \\ \hline
  \end{tabular}
\end{textblock}
}
\only<7>{
\begin{textblock}{80}(80,52) \centering
\begin{tabular}{|c|c|c|c||c|} \hline  \setlength\tabcolsep{0.4cm}
 \phantom{$\bm{s_2}$} & \, $r_1$ \, &  \, $r_2$ \, & \, $r_3$ \, & \phantom{$\bm{1/3}$} \\ \hline
  $\gray s_1$ & $\gray0$ & $\gray1/6$ & $\gray1/6$ &   $\gray 1/3$ \\ \hline
  $\bm{s_2}$ & $1/6$ & $0$ & $1/6$ &  $\bm{1/3}$ \\  \hline
  $\gray s_3$ & $\gray1/6$ & $\gray1/6$ & $\gray0$ & $\gray1/3$ \\ \hline
  \end{tabular}
\end{textblock}
}
\only<8>{
\begin{textblock}{80}(80,52) \centering
\begin{tabular}{|c|c|c|c||c|} \hline  \setlength\tabcolsep{0.4cm}
\phantom{$\bm{s_2}$} & \, $r_1$ \, &  \, $r_2$ \, & \, $r_3$ \, & \phantom{$\bm{1/2}$}  \\ \hline
            & & &  &  \\ \hline
  $s_2$ & $1/2$ & $0$ & $1/2$ &  1 \\  \hline
 & & & &\\ \hline
  \end{tabular}
\end{textblock}
}
\only<9>{
\begin{textblock}{80}(80,52) \centering
\begin{tabular}{|c|c|c|c||c|} \hline  \setlength\tabcolsep{0.4cm}
\phantom{$\bm{s_2}$} & \, $\bm{r_1}$ \, &  \, $r_2$ \, & \, $r_3$ \, & \phantom{$\bm{1/2}$}  \\ \hline
            & & &  &  \\ \hline
  $s_2$ & $\bm{1/2}$ & $0$ & $1/2$ &  1 \\  \hline
 & & & &\\ \hline
  \end{tabular}
\end{textblock}
}
\only<10>{
\begin{textblock}{80}(80,52) \centering
\begin{tabular}{|c|c|c|c||c|} \hline  \setlength\tabcolsep{0.4cm}
\phantom{$\bm{s_2}$} & \, $r_1$ \, &  \, $r_2$ \, & \, $r_3$ \, & \phantom{$\bm{1/3}$}  \\ \hline
  $s_1$ & $0$ & $1/6$ & $1/6$ &   $1/3$ \\ \hline
  $s_2$ & $1/6$ & $0$ & $1/6$ &  $1/3$ \\  \hline
  $s_3$ & $1/6$ & $1/6$ & $0$ & $1/3$ \\ \hline
  \end{tabular}
\end{textblock}
}
\only<11>{
\begin{textblock}{80}(80,52) \centering
\begin{tabular}{|c|c|c|c||c|} \hline  \setlength\tabcolsep{0.4cm}
 \phantom{$\bm{s_2}$} & \, $r_1$ \, &  \, $r_2$ \, & \, $r_3$ \, & \phantom{$\bm{1/3}$} \\ \hline
  $\gray s_1$ & $\gray0$ & $\gray1/6$ & $\gray1/6$ &   $\gray 1/3$ \\ \hline
  $\gray s_2$ & $\gray1/6$ & $\gray0$ & $\gray1/6$ &  $\gray1/3$ \\  \hline
  $\bm{s_3}$ & $1/6$ & $1/6$ & $0$ & $\bm{1/3}$ \\ \hline
  \end{tabular}
\end{textblock}
}
\only<12>{
\begin{textblock}{80}(80,52) \centering
\begin{tabular}{|c|c|c|c||c|} \hline  \setlength\tabcolsep{0.4cm}
\phantom{$\bm{s_2}$} & \, $r_1$ \, &  \, $r_2$ \, & \, $r_3$ \, & \phantom{$\bm{1/2}$}  \\ \hline
            & & &  &  \\ \hline
 & & & &\\ \hline
 $s_3$ & $1/2$ & $1/2$ & $0$ &  1 \\  \hline
  \end{tabular}
\end{textblock}
}
\only<13>{
\begin{textblock}{80}(80,52) \centering
\begin{tabular}{|c|c|c|c||c|} \hline  \setlength\tabcolsep{0.4cm}
\phantom{$\bm{s_2}$} & \, $r_1$ \, &  \, $\bm{r_2}$ \, & \, $r_3$ \, & \phantom{$\bm{1/2}$}  \\ \hline
            & & &  &  \\ \hline
 & & & &\\ \hline
 $s_3$ & $1/2$ & $\bm{1/2}$ & $0$ &  1 \\  \hline
  \end{tabular}
\end{textblock}
}



\begin{textblock}{80}(0,73) \centering
 \begin{equation*}
 \onslide<6->{ P(\text{D}|\text{M}) }  \onslide<7->{= \frac{1}{2}} \, \onslide<9->{\frac{1}{3}} \, \onslide<11->{\frac{1}{2}} \, \onslide<13>{\frac{2}{3}}
 \end{equation*}
\end{textblock}
\begin{textblock}{80}(80,73) \centering
\begin{equation*}
 \onslide<6->{ P(\text{D}|\text{M})}  \onslide<7->{= \frac{1}{3}} \, \onslide<9->{\frac{1}{2}} \, \onslide<11->{\frac{1}{3}} \, \onslide<13>{\frac{1}{2}}
 \end{equation*}
\end{textblock}



\end{frame}


% \begin{frame}[plain,fragile]
% \begin{textblock}{160}(0,4)
% \centering \LARGE Modelos causales alternativos \\
% \Large Datos generados con el modelo Monty Hall
% \end{textblock}
% \vspace{1cm}
%
%
% \begin{lstlisting}[belowskip=-0.6 \baselineskip]
% datos = [ rand(1:3) == 1 ? 0 : 1  for i in 1:16]
% \end{lstlisting}
% \pause
% \begin{lstlisting}[belowskip=-0.6 \baselineskip]
%
% prior = 1/2
% \end{lstlisting}
% \pause
% \begin{lstlisting}[belowskip=-0.6 \baselineskip]
% evidencia_B = cumprod([1/2 for d in datos].*1/3)
% \end{lstlisting}
% \pause
% \begin{lstlisting}[belowskip=-0.6 \baselineskip]
% evidencia_A = cumprod([(1/3)^(1-d)*(2/3)^d  for d in datos].*1/2)
% \end{lstlisting}
% \pause
% \begin{lstlisting}[belowskip=-0.6 \baselineskip]
%
% p_datos = (evidencia_A * prior) + (evidencia_B * prior)
% \end{lstlisting}
% \pause
% \begin{lstlisting}[belowskip=-0.6 \baselineskip]
%
% p_modelo_A = evidencia_A * prior / p_datos
% \end{lstlisting}
% \pause
% \begin{lstlisting}
% p_modelo_B = evidencia_B * prior / p_datos
% \end{lstlisting}
% \end{frame}



\begin{frame}[plain]
\begin{textblock}{160}(0,4)
\centering \LARGE Modelos causales alternativos
%\\ \Large Datos generados con el modelo Monty Hall
\end{textblock}
%
% \begin{textblock}{160}(14,12)
% \begin{equation*}
%  P(\text{Modelo}|\text{Datos}) = \frac{\only<1->{\overbrace{P(\text{Data}|\text{Modelo})}^{\text{\footnotesize Predicción a priori}}} \only<1->{P(\text{Modelo})} }{ P(\text{Data})} \phantom{\frac{\overbrace{P(\text{Datos}|\text{Modelo})}^{\text{Evidencia}}}{ P(\text{Datos})}}
% \end{equation*}
% \end{textblock}
% %
% \only<2>{
% \begin{textblock}{160}(0,47)
% \begin{align*}
% P(\text{Data}|\text{Modelo}) & = \sum_{i} P(\text{Data}|\text{Hypothesis}_i,\text{Model}) P(\text{Hypothesis}_i|\text{Model})
% \end{align*}
% \end{textblock}
% }



\only<1>{

\begin{textblock}{140}(10,26)
\centering
\includegraphics[width=0.66\textwidth]{figuras/monty_hall_selection.pdf} \hspace{2cm}
\end{textblock}

\begin{textblock}{80}(86,26)
\centering
\scalebox{0.5}{
\tikz{

    \node[latent] (d) {\includegraphics[width=0.10\textwidth]{../../auxiliar/static/dedo.png}} ;
    \node[const,left=of d] (nd) {\Large $s$} ;

    \node[latent, above=of d, xshift=-1.5cm] (r) {\includegraphics[width=0.12\textwidth]{../../auxiliar/static/regalo.png}} ;
    \node[const,left=of r] (nr) {\Large $r$} ;


    \node[latent, fill=black!30, above=of d, xshift=1.5cm] (c) {\includegraphics[width=0.12\textwidth]{../../auxiliar/static/cerradura.png}} ;
    \node[const,left=of c] (nc) {\Large $c$} ;

    \edge {r,c} {d};
}
}
\end{textblock}


\begin{textblock}{80}(86,60)
\centering
\scalebox{0.5}{
 \tikz{
    \node[latent,] (r) {\includegraphics[width=0.12\textwidth]{../../auxiliar/static/regalo.png}} ;
    \node[const,left=of r] (nr) {\Large $r$} ;


    \node[latent, below=of r] (d) {\includegraphics[width=0.10\textwidth]{../../auxiliar/static/dedo.png}} ;
    \node[const, left=of d] (nd) {\Large $s$} ;

    \edge {r} {d};

}
}
\end{textblock}
}

\end{frame}

%
% \begin{frame}[plain]
% \begin{textblock}{160}(0,4)
% \centering \LARGE Comparación de modelos \\
% \Large Predicción a priori
% \end{textblock}
% \vspace{1cm}
%
%  \begin{equation*}
% \begin{split}
% P(\text{Dat\en{a}\es{os}}|\text{Model\es{o}}) & = P(d_1|\text{Model\es{o}})P(d_2|d_1,\text{Model\es{o}}) \dots \\
% & = \text{geometric mean}(P(\text{Dat\en{a}\es{os}}|\text{Model\es{o}}))^{|\text{Dat\en{a}\es{os}}|}
% \end{split}
% \end{equation*}
%
% \vspace{0.4cm}
% \pause
%
% \begin{equation*}
% \begin{split}
% P(\text{Dat\en{a}\es{os}}|\text{Model\es{o}}) & = \sum^{\text{Hipótesis}}_h P(\text{Dat\en{a}\es{os}}|h,\text{Model\es{o}}) P(h|\text{Model\es{o}}) \\
% & = \text{arithmetic mean}_h(P(\text{Dat\en{a}\es{os}}|h,\text{Model\es{o}}))
% \end{split}
% \end{equation*}
%
% \end{frame}
%

% \begin{frame}[plain]
% \begin{textblock}{160}(0,4)
% \centering \LARGE Predicción a priori \\
% \Large Contribución de todas las hipótesis
% \end{textblock}
% \vspace{1cm}
%
% \begin{textblock}{160}(0,28)
%  \begin{equation*}
% \begin{split}
% P(\text{Dat\en{a}\es{os}}|\text{Model\es{o}}) & = P(d_1|\text{Model\es{o}})P(d_2|d_1,\text{Model\es{o}}) \dots \\
% \onslide<2->{& = \left( \sum^{\text{Hipótesis}}_h P(d_1|h,\text{M}) P(h|\text{M}) \right) \left( \sum^{\text{Hipótesis}}_h P(d_2|d_1,h,\text{M}) P(h|d_1,\text{M}) \right)  \dots \\}%\onslide<3->{& =  \prod_i^{|\text{Datos}|} \sum_h P(d_i|d_1, \dots, d_{i-1}, h, \text{M}) P(h|d_1, \dots, d_{i-1},\text{M}) }
% \end{split}
% \end{equation*}
% \end{textblock}
%
% \end{frame}
%


\begin{frame}[plain]
\begin{textblock}{160}(0,4)
\centering \LARGE Modelos causales alternativos \\
\Large Preguntas
\end{textblock}
\vspace{1.25cm}

\pause

1. ¿Qué hubiera pasado con la evaluación de modelos si una de las pistas hubiera señalado la caja que habíamos elegido previamente?

\pause
\vspace{0.4cm}

2. ¿Cómo podemos corregir ese modelo sabiendo que a veces, muy rara vez, la persona que señala se puede equivocar?


\end{frame}

\begin{frame}[plain]
\begin{textblock}{96}(0,6.5)\centering
{\transparent{0.9}\includegraphics[width=0.8\textwidth]{../../auxiliar/static/inti.png}}
\end{textblock}

\begin{textblock}{160}(96,5.5)
\includegraphics[width=0.35\textwidth]{../../auxiliar/static/pachacuteckoricancha}
\end{textblock}
\end{frame}






\end{document}



